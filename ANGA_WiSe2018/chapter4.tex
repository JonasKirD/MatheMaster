% This work is licensed under the Creative Commons
% Attribution-NonCommercial-ShareAlike 4.0 International License. To view a copy
% of this license, visit http://creativecommons.org/licenses/by-nc-sa/4.0/ or
% send a letter to Creative Commons, PO Box 1866, Mountain View, CA 94042, USA.

\chapter{Die Riemannsche Vermutung}
Sei $K =\F_q$ ein endlicher Körper, $F|K$ Funktionenkörper vom Geschlecht $g$.

\section{Endlichkeit der Klassenzahl}
\begin{erinnerungnr}
    Der endliche Körper $\F_q$ ist vollkommen und hat für jedes $d \in \N$ genau eine Erweiterung $L \subseteq \overline{\F_q}$ vom Grad $d$.
    Diese ist galoissch mit zyklischer Galoisgruppe, erzeugt vom relativen Frobenius $x \mapsto x^q$.
\end{erinnerungnr}

\begin{definition}
    Für $d \in \Z$ seien
    \begin{align*}
        \Div^d(F|K) &:= \{A \in \Div(F|K) \mid \deg A = d\} \\
        \Div^d_{\geq 0}(F|K) & := \{A \in \Div^d(F|K) \mid A \geq 0\}\\
        \Cs^d(F|K) & := \{[A] \in \Cs(F|K) \mid A \in \Div^d(F|K)\}.
    \end{align*}
\end{definition}

\begin{lemma}
    $$ |S^d(F|K)| < \infty \quad\forall~d$$
\end{lemma}
\begin{proof}
    Für $F = K(t)$ klar aus II.3.4. Für $F$ allgemein wähle $t \in F$ transzendent.
    Für $P \in S^d(F|K)$ ist $P|_{K(t)} \in \bigcup\limits_{i=1}^d S^i(K(t)|K)$ 
    und die Abbildung $P \mapsto P|_{K(t)}$ hat endliche Fasern (II.3.5).
\end{proof}

\begin{satz}
    $$ |\Div^d_{\geq 0} (F|K) < \infty \quad \forall~d$$
\end{satz}
\begin{proof}
    $D \in \Div^d_{\geq 0}(F|K) \implies D$ ist Summe von höchstens $d$ vielen
    $$ P \in \bigcup_{i=0}^d S^i(F|K), $$
    denn $D = \sum n_iP_i \implies \deg D = \sum n_i \deg P_i$. 
\end{proof}

\begin{lemma}
    $$|\Cs^d(F|K)| < \infty \quad \forall~d \geq g$$
\end{lemma}
\begin{proof}
    \begin{align*}
        \deg A = d \geq g  & \implies \dim A \geq 1\\
        & \implies A \sim A' \geq 0, [A] = [A'], \deg A' = d
    \end{align*}
    Nach 1.4 existieren nur endlich viele solche $A'$, somit auch nur endlich viele $[A]$.
\end{proof}

\begin{satz}
    $\Cs^0(F|K)$ ist eine endliche abelsche Gruppe.
\end{satz}
\begin{proof}
    Wähle $D \in \Div(F|K)$ mit $d:=\deg D \geq g$.
    $$ \text{Die Abbildung } \begin{cases}
        \Cs^0(F|K) \to \Cs^d(F|K)\\
        [A] \mapsto [A +D]
    \end{cases} \text{ ist eine Bijektion.}$$
\end{proof}

\begin{definition}
    \begin{align*}
        h := h_{F|K} := |\Cs^0(F|K)|, \text{ ist die \textbf{Klassenzahl} von } F|K
    \end{align*}
\end{definition}

\begin{beispiel}
    $$ h_{K(T)|K} = 1 \text{ (II.4.6)}$$
    Allgemeiner gilt: $g=0 \implies h = 1$, denn:
    \begin{align*}
        \deg [A] = 0 & \implies \dim [A] = \deg [A] + 1 -g = 1 > 0 \\
        & \implies A \sim A' \geq 0, \deg A' = 0 \\
        & \implies A' = 0, \text{ d.h. } [A] = [0].
    \end{align*}
\end{beispiel}

\begin{definition}
    $$ \delta := \min\{d > 0 \mid \Div^d(F|K) \ne \emptyset\}$$
\end{definition}

\begin{bemerkungnr}
    Da $\im(\deg) \leq \Z$ ist:
    $$ \Div^d(F|K) \ne \emptyset \iff \delta|d.$$
    In diesem Fall ist $|\Cs^d(F|K)|=h_{F|K}$.    
\end{bemerkungnr}

\begin{lemma}
    Für $[C] \in \Cs(F|K)$ ist
    $$ | \{A \in [C] \mid A \geq 0\}| = \frac{1}{q-1}(q^{\dim C}-1).$$
\end{lemma}
\begin{proof}
    \begin{align*}
        &\begin{cases}
            \mathcal{L}(C)\setminus \{0\} \to \{A \in [C] \mid A \geq 0\}\\
            x \mapsto C + (x)
        \end{cases} \text{ ist surjektiv mit } C + (x) = C + (y)\\
        &\iff \frac{x}{y} \in K^\times\\
        &\implies |\{A \in [C] \mid A \geq 0\}| = |\P(\mathcal{L}(C))| = \frac{|\mathcal{L}(C) - 1}{|K| -1} 
        = \frac{q^{\dim C}-1}{q-1}
    \end{align*}
\end{proof}

\begin{satz}
    Für $d > 2g - 2$ mit $\delta|d$ ist
    $$ |\Div^d_{\geq 0}(F|K)| = \frac{h}{q-1}(q^{d+1-g}-1).$$
\end{satz}
\begin{proof}
    \begin{align*}
        \delta|d & \implies |\Cs^d(F|K)| = h\\
        [C] \in \Cs^d(F|K) & \stackrel{d > 2g - 2}{\implies} \dim C = \deg C + 1 -g = d + 1 -g\\
        & \implies |\Div^d_{\geq 0}(F|K)| = \sum_{[C]\in \Cs^d(F|K)} |\{A \in [C] \mid A \geq 0\}|\\
        & \qquad = h \frac{1}{q-1}(q^{\dim C} -1) = \frac{h}{q-1}(q^{d +1 -g}-1)
    \end{align*}
\end{proof}

\begin{definition}
    Die \textbf{Norm} von $D \in \Div(F|K)$ ist
    $$ N(D) := q^{\deg D}. $$
\end{definition}

\begin{beispiel}
    Für $P \in S(F|K)$ ist
    $$ N(P) = |F_P|.$$
\end{beispiel}

\section{Die Zeta-Funktion}
\begin{definition}
    Schreibe $a_n := |\Div^n(F|K)|$. Die \textbf{Zeta-Funktion} von $F|K$ ist
    $$ Z(t):=Z_{F|K}(t):= \sum_{n \geq 0} a_n t^n \in \C\llbracket t\rrbracket.$$
\end{definition}

\begin{satz}
    \begin{enumerate}[label=(\alph*)]
        \item $g=0 \implies$ 
        $$ Z(t) = \frac{1}{q-1}\left(\frac{q}{1-(qt)^\delta} - \frac{1}{1-t^\delta}\right)$$
        \item $g > 0 \implies$
        $$ Z(t) = \frac{1}{q-1}\left(\sum_{0\leq\deg[C]\leq 2g-2}q^{\dim [C]}t^{\deg [C]}+hq^{1-g}(qt)^{2g-2+\delta}\frac{1}{1-(qt)^{\delta}}
        - h\frac{1}{1-t^\delta}
        \right)$$
    \end{enumerate}
\end{satz}
\begin{proof}
    \begin{align*}
        \text{(a) } \sum_{n=0}^\infty a_n t^n & = \sum_{n=0}^\infty a_{\delta n}t^{\delta n}\\
        & \stackeq{1.12~h=1} \sum_{n=0}^\infty \frac{1}{q-1}\left(q^{\delta n +1}-1\right)t^{\delta n}\\
        & = \frac{1}{q-1} \left(\frac{1}{1-q^\delta t^\delta} - \frac{1}{1 - t^\delta}\right)\\
        \text{(b) } \sum_{n=0}^\infty a_n t^n & = \sum_{\deg [C] \geq 0} |\{A \in [C] \mid A \geq 0\}|t^{\deg [C]}\\
        & = \sum_{\deg [C] \geq 0} \frac{q^{\dim [C]}-1}{q-1} t^{\deg [C]}\\
        & = \frac{1}{q-1}\left(\sum_{0 \leq \deg [C] \leq 2g - 2} q^{\dim [C]} t^{\deg [C]}\right.
            + \sum_{\deg [C] > 2g -2} q^{\deg [C] + 1 - g} t^{\deg [C]}\\
            & \left. \qquad - \sum_{\deg [C] \geq 0} t^{\deg [C]}\right)\\
        & = \frac{1}{q-1}\left(\sum_{0 \leq \deg [C] \leq 2g - 2} q^{\dim [C]} t^{\deg [C]} 
            + \sum_{n > \underbrace{\frac{2g -2}{\delta}}_{\in \Z}} hq^{\delta n + 1-g}t^{\delta n}
            - \sum_{n=0}^\infty h t^{\delta n} \right)
    \end{align*}
\end{proof}

\begin{korollar}
    Die Potenzreihe $Z(t)$ konvergiert für $|t| < q^{-1}$, und diese Funktion setzt sich zu einer rationalen Funktion
    auf $\C$ mit einfachem Pol bei $t=1$ fort.
\end{korollar}

\begin{bemerkungnr}
    Setzt man $t = q^{-s}$, so erhält man die (vollständige) Dedekindsche Zetafunktion von $F|K$
    $$ \zeta(s) := \zeta_{F|K}(s) := Z(q^{-s}), s \in \C .$$ 
    Für $\Re(s) > 1$ ist $|q^{-s}| < q^{-1}$, somit 
    $$\zeta(s) = \sum_{i=1}^\infty\sum_{\substack{\deg D=n,\\D\geq 0}} (q^n)^{-s} = \sum_{D\geq 0} N(D)^{-s}.$$ 
\end{bemerkungnr}

\begin{satz}
    Für $|t| < q^{-1}$ ist
    $$ Z(t) = \prod_{P \in S(F|K)}(1-t^{\deg P})^{-1}, $$
    wobei das Produkt in diesem Bereich absolut konvergiert.
\end{satz}
\begin{proof}
    Für $|t| < q^{-1}$ ist 
    \begin{align*}
        &\sum_{P\in S(F|K)} |t|^{\deg P} = \sum_{n=1}^\infty |S^n(F|K)|\cdot|t|^n \leq \sum_{n=0}^\infty a_n |t|^n \stackrel{2.3}{<} \infty\\
        &\implies \prod_{P\in S(F|K)} (1-t^{\deg P})^{-1} = \prod_{p \in S(F|K)}\sum_{n=0}^\infty t^{\deg nP} = \sum_{A \geq 0} t^{\deg A}
        = \sum_{n=0}^\infty a_nt^n
    \end{align*}
\end{proof}

\begin{korollar}
    $Z(t)$ hat keine Nullstellen mit $|t| < q^{-1}$.
\end{korollar}

\begin{definition}
    Für $r \in \N$ sei
    \begin{align*}
        & K_r := \F_{q^r}\\
        & F_r := FK_r \\
        & N_r := |S^1(F_r|K_r)| \\
        & Z_r := Z_{F_r|K_r}
    \end{align*}
\end{definition}

\begin{lemma}
    $$ P\in S^d(F|K) \implies \Con_{F_r|F}(P)=P_1 + \ldots + P_m, $$
    wobei $m = \ggT(d,r)$ paarweise verschieden, $P_i \in S^{\frac{d}{m}}(F_r|K_r)$.
\end{lemma}
\begin{proof}
    Seien $P_1, \ldots, P_m$ die Primstellen von $F_r$ über $P$
    $$ \implies \Con_{F_r|F}(P) = \sum_{i=1}^m e_{P_i|P}P_i. $$
    \begin{itemize}
        \item \textit{$e_{P_i|P}=1$:} III.7.3(a)
        \item \textit{$\deg P_i = \frac{d}{\ggT(d,r)}$:} $(F_r)_{P_i} \stackeq{\text{III.7.3(d)}} F_P\cdot\F_{q^r} = \F_{q^d}\cdot\F_{q^r} = \F_{q^{\kgV(d,r)}}$
        $$ \implies \deg P_i = [(F_r)_{P_i}:K_r] = \frac{\kgV(d,r)}{r} = \frac{d}{\ggT(d,r)}$$
        \item \textit{$M=\ggT(d,r)$:} $d = \deg P \stackeq{III.7.3(c)} \deg \Con(P) = m \cdot \deg P_i$
    \end{itemize}
\end{proof}

\begin{lemma}
    Mit $r,d \in \N, m = \ggT(d,r)$ ist 
    $$ (1-t^{\frac{rd}{m}})^m = \prod_{\zeta \in \mu_r}(1-(\zeta t)^d) \quad \text{in } \C[t],$$
    wobei $\mu_r = \{\zeta \in \C \mid \zeta^r = 1\}$.
\end{lemma}
\begin{proof}
    $(t^{\frac{r}{m}}-1)^m = \prod\limits_{\zeta\in\mu_r}(t-\zeta^d)$, denn beide Seiten sind normiert vom Grad $r$, und jedes $\zeta \in \mu_{\frac{r}{m}}$ ist eine $m$-fache Nullstelle.
\end{proof}

\begin{satz}
    $$ Z_r(t^r) = \prod_{\zeta \in \mu_r} Z(\zeta t) $$
\end{satz}
\begin{proof}
    Es genügt dies für $|t| < q^{-1}$ zu zeigen:
    \begin{align*}
        \implies Z_r(t^r) &= \prod_{P' \in S(F_r|K_r)} (1 - t^{r\cdot \deg P'})^{-1}\\
            &= \prod_{P \in S(F|K)}\prod_{P'|P} (1 - t^{r\cdot \deg P'})^{-1}
    \end{align*}
    Für $P \in S(F|K)$ sei $d = \deg P, m=\ggT(d,r)$
    \begin{align*}
        & \implies \prod_{P'|P} (1 - t^{r\cdot \deg P'}) \stackeq{2.8} (1 - t^{\frac{r\cdot d}{m}})^m \stackeq{2.9} \prod_{\zeta \in \mu_r}(1-(\zeta t)^d)\\
        & \implies Z_r(t^r) = \prod_{P \in S(F|K)}\prod_{\zeta \in \mu_r}(1-(\zeta t)^{\deg P})^{-1} \stackeq{2.5} \prod_{\zeta \in \mu_r} Z(\zeta t)
    \end{align*}
\end{proof}

\begin{theorem}[F.K. Schmidt]
    $$ \delta = 1 $$
\end{theorem}
\begin{proof}
    Für $r = \delta$ ist
    \begin{align*}
        Z_r(t^r) &= \prod_{\zeta \in \mu_r} Z(\zeta t) = \prod_{\zeta \in \mu_r} \prod_{P \in S(F|K)} (1-(\zeta t)^{\deg P})^{-1}
            \stackeq{\zeta^d=1} \prod_{\zeta \in \mu_r}\prod_P (1-t^{\deg P})^{-1} \\
            &= Z(t)^r\\
        &\left.\begin{array}{l}
            \ord_{t=1}Z(t) = -1 \implies \ord_{t=1}Z(t)^r = -r \\
            \ord_{t=1}Z_r(t) = -1 \implies \ord_{t=1} Z(t^r) = -1
        \end{array}\right\} \implies \delta = r = 1.
    \end{align*}
\end{proof}

\begin{beispiel}
    Ist \emph{$g=0$}, so ist $F|K$ rational (\#23), und 
    $$Z(t) = \frac{1}{(1-t)(1-qt)}.$$
    Ist \emph{$g=1$}, so ist $F|K$ elliptisch (d.h. $S^1(F|K) \ne \emptyset$):
    $$ \deg D = 1 \implies \dim D = \deg D + 1 -g = 1 \implies D \sim P \in S^1(F|K). $$
\end{beispiel}

\begin{satz}[Funktionalgleichung]
    $$ Z(t) = q^{g-1}t^{2g -2}Z\left(\frac{1}{q^t}\right) $$
\end{satz}
\begin{proof}
    \begin{align*}
        g=0:& \text{ aus 2.12}\\
        g \geq 1:&~ q^{g -1}t^{2g-2}\left(hq^{1-g}\left(\frac{1}{t}\right)^{2g -1}\frac{1}{1-\frac{1}{t}}-h\frac{1}{1-\frac{1}{qt}}\right)\\
        & = h \frac{1}{t-1} - \underbrace{q^{g-1}t^{2g-2}h\frac{1}{1-\frac{1}{qt}}}_{= q^gt^{2g-1}h\frac{1}{qt -1}}
    \end{align*}
    Sei $W$ ein kanonischer Divisor von $F|K$.
    \begin{align*}
        \sum_{0 \leq \deg [C] \leq 2g -2} q^{\dim[C]}t^{\deg [C]} &= \sum_{0 \leq \deg [C] \leq 2g -2} q^{\deg[C] + 1 - g + \dim[W-C]}t^{\deg [C]}\\
            &= q^{g -1}t^{2g-2} \sum_{0 \leq \deg [C] \leq 2g -2} q^{\deg[C-W]+\dim[W-C]}t^{\deg [C-W]}\\
            &= q^{g-1}t^{2g-2} \sum_{0 \leq \deg[D]\leq2,-2} q^{-\deg[D]+\dim[D]}t^{-\deg[D]}\\
            &= q^{g-1}t^{2g-2} \sum_{0 \leq \deg[D]\leq2,-2} q^{\dim [D]}\left(\frac{1}{qt}\right)^{\deg[D]}
    \end{align*}
\end{proof}

\section{Der Satz von Hasse-Weil}

\begin{definition}
    $$ L(t) := L_{F|K}(t): = (1-t)(1-qt)Z_{F|K}(t) \in \C[t] $$  
\end{definition}

\begin{beispiel}
    $g=0 \implies L=1$
\end{beispiel}

\begin{lemma}
    \begin{enumerate}[label=(\alph*)]
        \item $L \in \Z[t]$
        \item $\deg L = 2g$
        \item $L(1) = h$
        \item $L(t) = q^gt^{2g}L\left(\frac{1}{qt}\right)$
    \end{enumerate}
\end{lemma}
\begin{proof}
    o.E. $g\geq 1$.
    \emph{(a):} $L \in \C[t] \cap \Z[t] = \Z[t]$\nl
    \emph{(b):} 2.2\nl
    \emph{(c):} $L(1) = \frac{1}{q-1} \left(0 + hq^{1-g}q^{2g-1}\cdot 0 - h(1-q)\right) = h$\nl
    \emph{(d):} 2.13
\end{proof}

\begin{satz}
    Für die Koeffizienten von $ L = \sum\limits_{i=0}^{2g} c_it^i$ gelten:
    \begin{enumerate}[label=(\alph*)]
        \item $c_0=1, c_{2g}=q^g$
        \item $c_{2g-i} = q^{g-i}c_i$ für $0 \leq i \leq g$
        \item $c_1 = N_1 - (q + 1)$, erinnere $ N_r = |S^1(F_r|K_r)|.$
    \end{enumerate}
\end{satz}
\begin{proof}
    \emph{$g=0$:} $L=1, |S^1(K(T)|K)|=q+1$\\
    \emph{$g \geq 1$:} Aus 2.3(d) folgt
    \begin{align*}
        &c_0 + \ldots + c_{2g}t^{2g} = q^g c_0 t^{2g} + q^{g-1} c_1 t^{2g-1} + \ldots + c_{2g-1} q^{1-g}t + c_{2g} q^{-g}\\
        &\implies \text{(b)}\\
        & L = (1-t)(1-qt)\sum_{i=0}^\infty a_i t^i\\
        & \implies c_0 = a_0 = 1 \implies c_{2g} = q^g c_0 = q\\
        & \implies c_1 = a_1 - a_0 - qa_0 = |S^1(F|K)| - (q+1)
    \end{align*}
\end{proof}

\begin{satz}
    $L$ zerfällt in $\C[t]$ zu
    $$ L(t) = \prod_{i=1}^{2g} (1-\alpha_it), $$
    wobei $\alpha_1, \ldots, \alpha_{2g}$ ganze algebraische Zahlen mit 
    $\alpha_i \alpha_{i+g} = q$ für $i=1,\ldots,g$ sind. Es gilt
    $$ L_r (t) := L_{F_r|K_r}(t) := \prod_{i=1}^{2g}(1-\alpha_i^r t).$$ 
\end{satz}
\begin{proof}
    $$ L^\ast(t) := t^{2g}L\left(\frac{1}{t}\right) = t^{2g} + c_1 t^{2g -1} + \ldots + q^g = \prod (t - \alpha_i)$$
    mit $\alpha_1,\ldots,\alpha_{2g} \in \C$ ganz über $\Z$.
    \begin{align*}
        \prod (t - \alpha_i) &= L^\ast(t) = t^{2g}L\left(\frac{1}{t}\right) \stackeq{t=qu} q^{2g}u^{2g}L\left(\frac{1}{qu}\right)
            \stackeq{\text{2.3(d)}} q^g L(u) \\
            & = q^g \prod (1- \alpha_i u) = q^g \prod \left(1-\alpha_i\frac{t}{q}\right)\\
            & = \underbrace{q^g\prod_{i=1}^{2g}\frac{\alpha_i}{q}}_{= 1}\prod_{i=1}^{2g} \left(t-\frac{q}{\alpha_i}\right)\\
    \end{align*}
    \begin{align*}
        & \implies \text{ Die Nullstellen sind o.E. } \alpha_1, \frac{q}{\alpha_1}, \alpha_2, \frac{q}{\alpha_2}, \ldots, \alpha_k,
            \frac{q}{\alpha_k}, \underbrace{q^{\frac{1}{2}}, \ldots, q^{\frac{1}{2}}}_{m}, \underbrace{-q^{\frac{1}{2}}, \ldots, -q^{\frac{1}{2}}}_{n}\\
        & \implies g = k + \frac{m}{2} + \frac{n}{2} \stackrel{\prod \alpha_i = q^g}{\implies} n \text{ gerade } \implies m \text{ gerade}
    \end{align*}
    \begin{align*}
        Z_r(t^r) &= \prod_{\zeta \in \mu_r} Z(\zeta t)\\
        L_r(t^r) &= (1-t^r)(1-q^rt^r)Z_r(t^r)\\
            &= (1-t^r)(1-q^rt^r)\prod_\zeta Z(\zeta t)\\
            &= (1-t^r)(1-q^rt^r)\prod_{\zeta} \frac{L(\zeta t)}{(1-\zeta t)(1 - q^\zeta t)}\\
            & \stackeq{\text{2.9}} \prod_\zeta L(\zeta t)\\
            &= \prod_\zeta \prod_i (1-\alpha_i \zeta t)\\
            &= \prod_{i=1}^{2g} (1 - (\alpha_i t)^r)\\
        & \implies L_r(t) = \prod_{i=1}^{2g} (1 - \alpha_i^r t)
    \end{align*}
\end{proof}

\begin{korollar}
    Für jedes $r \geq 1$ ist
    $$ N_r = q^r + 1 - \sum_{i=1}^{2g} \alpha_i^r.$$
\end{korollar}
\begin{proof}
    Mit $L_r(t) = \sum\limits_{i=0}^{2g} d_i t^i = \prod\limits_{i=1}^{2g} (1-\alpha_i^r t) \implies
    - \sum \alpha_i^r = d_1 \stackeq{\text{3.4}} N_r - (q+1)$.
\end{proof}

\begin{theorem}[Weil 1948]
    Für alle $i$ ist $|\alpha_i| = q^\frac{1}{2}$.
\end{theorem}

\begin{bemerkungnr}
    Dies bedeutet, dass alle Nullstellen von $\zeta_F(s) = Z(q^{-s})$ auf der Geraden $\Re(s) = \frac{1}{2}$ liegen (erweiterte Riemannsche Vermutung für $F|K$).
\end{bemerkungnr}

\begin{korollar}[Hasse-Weil-Schranke]
    $\left| |S^1(F|K)| - (q+1) \right| \leq 2g \sqrt{q}$
\end{korollar}
\begin{proof}
    $|N_1 - (q+1)| = |\sum\alpha_i| \leq \sum\limits_{i=1}^{2g}|\alpha_i| \stackeq{\text{3.7}} 2g \sqrt{q}$
\end{proof}

\begin{beispiel}
    Es folgt $q \gg g$, so ist $S^1(F|K) \ne \emptyset$.
\end{beispiel}

\begin{bemerkungnr}
    Ist $f \in K[X,Y]$ irreduzibel in $\overline{K}[X,Y]$ vom Grad $d$ und $F = \Quot(\sfrac{K[X,Y]}{(f)})$, so gilt
    \begin{equation}\tag{\text{Plücker-Formel}}
        g_F \leq \frac{(d-1)(d-2)}{2}.
    \end{equation}
    Damit folgt aus 3.9 die Aussage I.4.14 übe ebene Kurven.
\end{bemerkungnr}

\section{Beweis von 3.7 nach Bombien}
\begin{satz}
    Sei $r \geq 1$. Genau dann gilt 3.7 für $F$, wenn 3.7 für $F_r$ gilt.
\end{satz}
\begin{proof}
    Nach 3.6 gilt
    \begin{align*}
        & L(t) = \prod_{i=1}^{2g} (1-\alpha_it),\\
        & L_r(t) = \prod_{i=1}^{2g} (1-\alpha_i^rt)
    \end{align*}
    und $|\alpha_i| = q^\frac{1}{2} \iff |\alpha_i^r| = (q^r)^\frac{1}{2}$.
\end{proof}

\begin{bemerkungnr}
    Um 3.7 zu zeigen, können wir daher ab jetzt annehmen, dass
    \begin{enumerate}[label=(\roman*)]
        \item $g > 0$
        \item $q = q_0^2$ für ein $q_0 \in \N$
        \item $q > (g+1)^4$
    \end{enumerate}
\end{bemerkungnr}

\begin{satz}
    Gibt es $c_1, c_2$ mit
    $$ q^r + 1 - c_2 q^\frac{r}{2} \leq N_r \leq q^r + 1 + c_1 q^\frac{r}{2}$$
    für alle $r \geq 1$, so gilt 3.7 für $F$.
\end{satz}
\begin{proof}
    o.E. $c_1 = c_2 =: c$\\
    Nach 3.6 folgt $\left|\sum \alpha_i^r \right| = \left| N_r - (q^r + 1)\right| \leq c q^\frac{r}{2}$. Definiere
    $$ f(z) := \sum_{i=1}^{2g} \frac{\alpha_it}{1-\alpha_iz} = \sum_{r=1}^\infty\left(\sum_{i=1}^{2g} \alpha_i^r\right)z^r \in \C\llbracket z\rrbracket \cap \C(z).$$
    \begin{align*}
        &\implies \text{ Konvergenzradius } \rho := \rho(f) = \min\{|\alpha_i|^{-1} \mid i = 1,\ldots,2g\}\\
        & \left|\sum \alpha_i^r\right| \leq c q^\frac{r}{2} \implies \rho \geq q^{-\frac{1}{2}}\\
        & \implies |\alpha_i| \leq q^\frac{1}{2}
    \end{align*}
    Da $\prod\limits_{i=1}^{2g}\alpha_i = q^g$ folgt 
    $$ |\alpha_i| = q^\frac{1}{2} \forall i. $$
\end{proof}

\begin{satz}
    Unter (i)-(iii) ist 
    $$ N_1 < q +1 + (2g +1)q^\frac{1}{2}. $$
\end{satz}
\begin{proof}
    \emph{$N_1=0$:} $\checkmark$\nl
    \emph{$N_1 > 0$:} Wähle $P_0 \in S^1(F|K)$. Setze 
    \begin{align*}
        m&:= q_0 -1,\\
        n&:= 2g + q_0,\\
        l&:=m+nq_0=q-1 + (2g-1)q^\frac{1}{2}.
    \end{align*}
    \emph{Ziel:} es gibt $0 \ne x \in \L(lP_0)$ mit 
    $$v_p(x)>0 \forall P \in S^1(F|K)\setminus \{P_0\},$$
    denn dann ist
    $$ N_1 -1 = |S^1(F|K)\setminus \{P_0\}| \leq \deg(x)_0 = \deg (x)_\infty \leq \deg (lP_0) = l.$$
    Definiere
    \begin{align*}
        \L_0 &:= \Span_K \{xy^{q_0} \mid x \in \L(mP_0), y \in \L(nP_0)\} \subseteq \L(lP_0) \\
        M_0 &:= \{0 \leq i \leq m \mid \exists x \in F: (x)_\infty = i P_o\}
    \end{align*}
    für $i \in M_0$ wähle $u_i \in F$ mit $(u_i)_\infty = i P_0$.\nl\nl
    \emph{Behauptung 1:} $(u_i)_{i \in M_0}$ ist Basis von $\L(mP_0)$

    \emph{Beweis.} Die lineare Unabhängigkeit ist klar, da $$ v_{P_0}(\sum \lambda_i u_i) = \infty.$$
    Es gilt offenbar
    $$ \dim_K \L(mP_0) = q_0 -g.$$
    Desweiteren gilt
    $$ i \in M_0 \iff \L(iP_0) \supsetneqq \L((i-1)P_0).$$
    Da $$\dim_K(iP_0) - \dim_K ((i-1)P_0) \stackrel{II.5.4}{\leq} \deg P_0 = 1$$ 
    sind $$ \L(0) \subseteq \L (P_0) \subseteq \ldots \subseteq \L(mP_0)$$
    genau $\dim(mP_0) - \dim (0) = q_0 - g -1 = m-g$ viele Inkulsionen sind echt.
    Somit $|M_0| = 1 + m -g = \dim \L(mP_0)$.\nl\nl
    \emph{Behauptung 2:} Jedes $z \in \L_0$ kann eindeutig geschrieben werden als 
    $$ z = \sum_{i \in M_0} u_i y_i, y_i \in \L(nP_0).$$
    \emph{Beweis.} Die Existenz folgt aus Behauptung 1, mit $(a+b)^{q_0} = a^{q_0} + b^{q_0}$. 
    Es gilt
    \begin{align*}
        &\sum u_i y_i^{q_0} = 0, y_i \ne 0 \text{ für ein } i_0\\
        \implies & v_{P_0} (u_iy_i^{q_0}) \equiv -i \mod q_0 \text{ falls } y_i \ne 0\\
        \implies & \text{ alle verschieden modulo } q_0 \stackrel{m < q_0}{\implies} \text{ alle verschieden}.
    \end{align*}
    Und somit folgt die Eindeutigkeit.\nl\nl
    Definiere nun
    $$ \lambda: \begin{cases}
        \L_0 \to \L((q_0m+n)P_0)\\
        \sum_{i\in M_0}u_iy_i^{q_0} \mapsto \sum_{i\in M_0}u_i^{q_0}y_i.
    \end{cases}$$
    Aus Behauptung 2 folgt $\lambda$ ist wohldefiniert. Offenbar ist $\lambda$ auch ein Gruppenhomomorphismus.
    Um zu zeigen, dass $\ker (\lambda) \ne \{0\}$ genügt es zu zeigen, dass
    $$ \dim_{\F_q} \L_0 > \dim (q_0m+n) \implies \lambda \text{ nicht injektiv}.$$
    Es gilt 
    $$ \dim_{\F_q} \L_0 = (m+1-g)(n+1-g) > q_0m + n + 1 -g = \dim (q_0m + n). $$
    Für $0 \ne x \in \ker (\lambda)$ ist $v_P (x) > 0 \forall P\in S^1(F|K)\setminus \{P_0\}$.
    Daraus folgt
    \begin{align*}
        &x \in \L(lP_0) \implies v_P(x) \geq 0, \text{ analog } v_P(u_i) \geq 0 \forall i\\
        &P \in S^1(F|K) \implies F_p = \F_q \implies x(P)^q = x(P).
    \end{align*}
    Schreibe $x = \sum\limits_{i \in M_0} u_i y_i^{q_0}$
    \begin{align*}
        \implies x(P)^{q_0} &= \left(\sum_{i \in M_0} u_i(P) y_i(P)^{q_0}\right)^{q_0} \\
        &= \sum_{i \in M_0} u_i(P)^{q_0} y_i(P) = \left(\sum_{i \in M_0} u_i^{q_0} y_i\right)(P) = (\lambda(x))(P)\\
        \implies x(P) &= 0 \implies v_P(x) > 0.
    \end{align*}
\end{proof}

\begin{lemma}
    Sei $G' = C_r \times G, |G| = m| r$.\\
    Ist $H \leq G'$ mit $|H| = re, |H \cap G| = e$ \\
    so existieren genau $e$ viele $U \leq H$ mit $U \cong C_r$ und $U\cap G=1$. 
\end{lemma}

\begin{satz}
    Ist $E|K$ eine geometrische galoissche Erweiterung von $F|K$ vom Grad $m$, und $r = mn, E' = E\F_{q^r}$,
    dann gibt es genau $m$ Zwischenkörper $E_1, \ldots, E_m$ von $E'|F$ mit $\Gal(E'|E_i)$ zyklisch der Ordnung $r$, und
    $E_i\F_{q^r} = E'$.
    Für jedes $i$ gilt:
    \begin{enumerate}[label=(\alph*)]
        \item Der Konstantenkörper von $E_i$ ist $\F_q$.
        \item $g_{E_i}=g_E$
        \item $m |S^1(F|K)| = \sum\limits_{i=1}^m |S^(E_i|K)|$
    \end{enumerate}
\end{satz}
\begin{proof}
    $$\begin{tikzcd}[cramped]
        E \ar[rr, dash]{C_r} \ar[dd, dash, "G"', "m"] & & E' \ar[dd, dash]{G}\\
        & E_i \ar[ur, dash]{H_i} &\\
        F \ar[ur, dash] \ar[rr, dash]{C_r} \dar[dash]& & F_r \dar[dash] \\
        \F_q \ar[rr, dash]{C_r}& & \F_{q^r}
    \end{tikzcd}$$
    \begin{align*}
        &\Gamma := \Gal(E'|F) = \underbrace{\Gal(F_r|F)}_{\langle\sigma\rangle \cong C-r} \times \underbrace{\Gal(E|F)}_{=:G}\\
        &\stackrel{4.6}{\implies}\text{ genau $m$ viele } H_i \leq r \text{ mit } H_i \cong C_r, H_i \cap H = 1\\
        & \implies \text{ genau $m$ viele } E_i \text{ mit } \Gal(E'|E_i) \cong C_r, E_iF_r =E
    \end{align*}
    \emph{(a):}
    \begin{align*}
        &E_i \cap \F_{q^r} = (E')^{H_i} \cap \F_{q^r} = \left(\F_{q^r}^{\res_{E'|\F_{q^r}}}\right)^{H_i} = \F_q \text{, denn}\\
        &\ker (\res_{E'|\F_{q^r}}: \Gal(E'|F) \to \Gal(\F_{q^r}|\F_q)) = \Gal(E'|\F_r) = G \\
        &\implies \res_{E'|\F_{q}}(H_i) \cong H_i \cong C_r
    \end{align*}
    \emph{(b):} $$ E_i\F_{q^r} = E_iF_r = E' \implies g_{E_i}=g_{E'}=g_E$$
    $$\begin{tikzcd}
        E \dar[dash] \drar[dash]{\text{galoissch}}\\
        F \dar[dash] \rar[dash]& F_r \dar[dash]\\
        \F_q(t) \dar[dash] \rar[dash]& \F_{q^r}(t)\dar[dash] \\
        \F_q \rar[dash]& \F_{q^r}
    \end{tikzcd}$$
\end{proof}

\begin{bemerkungnr}
    Wegen 4.1 können wir zusätzlich folgende Situation annehmen:
    \begin{itemize}
        \item[(iv)] $F_0 = K(t) \subseteq F \subseteq E$ mit $E|F_0$ galoissch und geometrisch
        \item[(iii')] $q > (g_E +1)^4$ 
    \end{itemize}
\end{bemerkungnr}

\begin{satz}
    Unter (i)-(iv),(iii') gilt $N_r \geq q^r +1 -c_2q^\frac{1}{2}$, für ein $c_2$.
\end{satz}
\begin{proof}
    Sei $m = [E:F]$, $n = [F:F_0]$, $r=mn$, $E'=E\F_{q^r}$, $F'=F\F_{q^r}$, $F_0'=F_0\F_{q^r}$.
    $$\begin{tikzcd}
        E \rar[dash]{r} \dar[dash]{m}& E' \dar[dash]{m}\\
        F \rar[dash]{r} \dar[dash]{n}& F' \dar[dash]{n}\\
        F_0 \rar[dash]{r}\dar[dash]& F_0'\dar[dash]\\
        \F_q \rar[dash]{r}& \F_{q^r}
    \end{tikzcd}$$
    Nach 4.6 gibt es:
    \begin{itemize}
        \item genau $m$ viele $V_1, \ldots, V_m \subseteq \Gal(E'|F)$ mit $V_i \cong C_r$ und $V_i \cap \Gal(E'|F) = 1$,
        \item genau $r$ viele $U_1, \ldots, U_r \leq \Gal(E'|F_0)$ mit $ U_i \cong C_r$ und $U_i \cap \Gal(E'|F_0) = 1$.
    \end{itemize}
    Da $E'^{V_i}F_0' \supseteq E'^{V_i}FF_0' = E'^{V_i}F'=E'$ ist $V_i \cap \Gal(E'|F_0')=1$,
    somit o.E. $U_i = V_i$ für $i=1, \ldots,m$.
    Sei $E_i = E'^{U_i}$. Nach 4.7(c) ist
    \begin{itemize}
        \item $m |S^1(F|K)| = \sum\limits_{i=1}^m |S^1(E_i|K)|$
        \item $r|S^1(F_0|K)| = \sum\limits_{i=1}^r |S^1(E_i|K)|$
    \end{itemize}
    Wir wissen
    \begin{itemize}[label=-]
        \item $|S^1(F_0|K)|=q+1$
        \item $|S^1(E_i|K)| \stackrel{4.4}{\leq} q+1 + (2g_E + 1)q^\frac{1}{2}$
    \end{itemize}
    \begin{align*}
        \implies & m |S^1(F|K)| = r (q+1) - \sum_{i=m+1}^r |S^1(E_i|K)|\\
        & \geq r(q+1) + (rm)(q+1+(2g_E + 1)q^\frac{1}{2})\\
        & = m(q+1) + (r-m)(2g_E+1)q^\frac{1}{2}\\
        \implies & |S^1(F|K)| \geq q +1 - \frac{mn -m}{m}(2g_E + 1)q^\frac{1}{2}
    \end{align*}
\end{proof}
