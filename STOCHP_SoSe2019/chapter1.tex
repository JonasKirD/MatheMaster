% This work is licensed under the Creative Commons
% Attribution-NonCommercial-ShareAlike 4.0 International License. To view a copy
% of this license, visit http://creativecommons.org/licenses/by-nc-sa/4.0/ or
% send a letter to Creative Commons, PO Box 1866, Mountain View, CA 94042, USA.

\chapter{Grundlegende Begriffe}
\section{Bezeichnungen}

\begin{notationnr}[Grundlegende Bezeichnungen]
	\begin{align*}
		&\N &=&&&\set{1,2,3,\ldots}\\
		&\N_0&=&&&\set{0,1,2,\ldots}\\
		&\Z&=&&&\set{0,\pm 1,\pm2,\ldots}\\
		&\Q &&&&\text{rationale Zahlen}\\
		&\R &&&&\text{reelle Zahlen}\\
		&\C &&&&\text{komplexe Zahlen}\\
		&x_+ &=&&&\max\set{x,0},\quad x\in\R\\
		&\indi_A &&&&\text{Indikatorfunktion einer Menge }A\\
		&\delta_x&&&&\text{Einpunkt- oder Dirac-Maß, das in $x$ konzentriert ist}\\
		&\delta_{x,y}&&&&\text{die Kroneker'sche Deltafuntktion}\\
		&\tilde{f}(x)&=&&&\overline{f(-x)}\\
		&\E[X]&&&&\text{Erwartungswert der Zufallsgröße/des Zufallsvektors }X\\
		&\L_p(\R^d)&&&&\text{Menge der komplex-wertigen Lebesguemessbaren Funktionen $f$ auf $\R^d$,}\\
		&&&&&\text{für die $|f|^p$ Lebesgue-integrierbar ist}\\
		&\L_p(\Z^d)&&&&\text{Menge der komplex-wertigen Funktionen $f$ auf $\Z^d$}\\
		&&&&&\text{für die $|f|^p$ Lebesgue-integrierbar sind}\\
		&\norm{\cdot}_p&&&&\L_p\text{-Norm}\\
		&L_p(\R^d)&&&&L_p\text{-Raum bezüglich des Lebesgue-Maßes auf $\R^d$}\\
		&L_p(\Z^d)&&&&L_p\text{-Raum bezüglich des Zählmaßes auf $\Z^d$} 
	\end{align*}
	$L_p^r$ und $\L_p^r$ bezeichnen Räume, die von reellwertigen Funktionen erzeugt werden.
\end{notationnr}

\begin{notationnr}[Mehrdimensionale Bezeichnungen]\enter
	Ist $x\in\R^d$ ($C^d,~\N_0^d$, usw.) so bezeichnet $x_i$ $(1\leq i\leq d)$ die $i$-te Koordinate von $x$ und wir schreiben $x$ als $x=(x_1,\ldots,x_d)$.
	In Ausdrücken wo Matrix-Operationen vorkommen, z.B. $A\mal x$ wobei $A$ eine Matrix ist, betrachten wir $x$ als einen Spalten-Vektor.
	Das Nullelement von $\R^d$ wird auch mit 0 bezeichnet.
	Sei $x\in\R^d$ oder $\C^d$ und $\alpha\in\N_0^d$.
	Dann schreiben wir:
	\begin{align*}
		x^\alpha&=x_1^{\alpha_1}\mal x_2^{\alpha_2}\mal\ldots\mal x_d^{\alpha_d}=\prod\limits_{i=1}^d x_i^{\alpha_i}\\
		\norm{x}&=\sqrt{\abs{x_1}^2+\ldots+\abs{x_d}^2}=\sqrt{\sum\limits_{i=1}^d\abs{x_i}^2}\\
		\abs{\alpha}&=\alpha_1+\ldots+\alpha_d=\sum\limits_{i=1}^d\alpha_i\\
		\alpha!&=\alpha_1!\mal\ldots\mal\alpha_d!=\prod\limits_{i=1}^d\alpha_i!\\
		\begin{pmatrix}
			\alpha\\
			\beta
		\end{pmatrix}&=\frac{\alpha!}{(\alpha-\beta)!\mal\beta!}=
		\begin{pmatrix}
			\alpha_1\\
			\beta_1
		\end{pmatrix}\mal\ldots\mal
		\begin{pmatrix}
			\alpha_d\\
			\beta_d
		\end{pmatrix}&\forall& \beta\leq\alpha\\
		D^\alpha&=\frac{\partial^{|\alpha|}}{\partial x_1^{\alpha_1}\cdots\partial x_d^{\alpha_d}}\\
		\abs{k}&=\abs{k_1}+\ldots+\abs{k_d}=\sum\limits_{i=1}^d\abs{k_1}&\forall& k\in\Z^d
	\end{align*}
\end{notationnr}

\begin{bemerkung}\
	\begin{itemize}
		\item Wenn $\alpha=(0,\ldots,0)$ dann $x^\alpha:=1$.
		\item Es gilt
		\begin{align*}
			D^\alpha D^\beta g=D^{\alpha+\beta} g
		\end{align*}				
		wobei $g$ eine beliebige reell- oder komplex-wertige Funktion auf $\R^d$ ist, für die die partiellen Ableitungen $D^{\alpha+\beta}g$ existieren.
		\item $D^\alpha g:= g$ wenn $\alpha=(0,\ldots,0)$.
		\item Sei $\beta=(\beta_1,\ldots,\beta_d)$ ein anderes $d$-Tupel von nichtnegativen ganzen Zahlen.
		Wir schreiben
		\begin{align*}
			\beta\leq\alpha:\iff\beta_j\leq\alpha_j\qquad\forall j\in\set{1,\ldots,d}
		\end{align*}
		\item Mit $\d x$, $\d\lambda(x)$ oder $\d\lambda_d(x)$ bezeichnen wir Integration bezüglich des Lebesgue'schen Maßes $\lambda=\lambda_d$ auf $\R^d$, und $(x,y):=\scaProd{x}{y}$ ist das Skalarprodukt von $x,y\in\R^d$ oder $\C^d$.
	\end{itemize}
\end{bemerkung}

\section{Historische Bemerkungen}
\begin{itemize}
	\item Zeitliche Einordnungen: Anfang des 20. Jahrhunderts
	\item Anwendungen in Physik, Technik, Finanzwelt, Wetter, $\ldots$
	\item Untersuchung von Vorgängen, die mit der Zeit ablaufen und bei deinen der Zufall eine Rolle abspielt.
	\item Die Wahrscheinlichkeitstheorie besaß damals keine allgemeinen Verfahren für die Untersuchung von solchen Erscheinungen.
	\item Notwendigkeit eine allgemeine Theorie der zufällige Prozesse auszuarbeiten.
\end{itemize}

\begin{beisp}\
	\begin{itemize}
		\item Zwei Gase / Flüssigkeiten werden in Berührung gebracht; die Moleküle der einen Flüssigkeit dringen in die andere ein: \define{Diffusion}\\
		\betone{Fragen:}
		\begin{itemize}
			\item Nach welchen Gesetzen geht der Diffusionsvorgang vor sich?
			\item Wie schnell?
			\item Wann stellt sich ein Gleichgewicht ein (d.h. wann haben sich die Flüssigkeiten / Gase vollständig vermischt)?
		\end{itemize}
		\item radioaktiver Zerfall: instabile Atomkerne senden Strahlung aus und ändern ihren Zustand; der Zeitpunkt ist zufällig
		\item Brownsche Bewegung: Moleküle stoßen zufällig gegen viele Atome. 
		Dies nutze man dann als Beweis für Richtigkeit des Atom-Teilchenmodells.
		\item Anzahl der Anrufe, die in einer Telefonzentrale während eines bestimmten Zeitintervalls erfolgen $\leadsto$ \undefine{Poisson-Prozess}
		\item Rauschen bei elektrischen Signalen
		\item $\ldots$
	\end{itemize}
\end{beisp}

\begin{beisp}[Herleitung der \define{Fokker-Planck-Diffusionsgleichung} der Diffusionstheorie (Anfang des 20. Jhd.)]\enter
	\index{Fokker-Planck-Diffusionsgleichung}
	Ein Teilchen erleidet zu dem Zeitpunkten $n\cdot\tau$, $n\in\set{1,2,\ldots}$ unabhängige zufällige Zusammenstöße, 
	die jedes Mal eine Verschiebung um $h\in\R$ nach rechts mit Wahrscheinlichkeit $p:=:p(\tau)$ (Wahrscheinlichkeit hängt von Feinheit $\tau$ ab), 
	oder nach links mit Wahrscheinlichkeit $q:=1-p$ verursachen.
	
	\begin{notation}
		\begin{align*}
			f(x,t):=:f(x,n\cdot\tau)
		\end{align*}
		sei die Wahrscheinlichkeit, dass das Teilchen nach $n$ Stößen, ausgehend von $x=0$ zur Zeit $t=0$, im Punkt $x\in\R$ auftritt.
	\end{notation}
	Diskretisierung: Offenbar gilt
	\begin{align*}
		f(x,t)=f(k\mal h,n\mal\tau)=0
	\end{align*}
	im Spezialfall, dass  $n$ gerade und $k$ ungerade oder $n$ ungerade und $k$ gerade.
	Sei nun $m\in\N$ die Anzahl der Schritte nach rechts.
	Dann ist
	\begin{align*}
		m-(n-m)=\frac{x}{h}=k
	\end{align*}
	und 
	\begin{align*}
		f(k\mal h,n\mal\tau)=
		\begin{pmatrix}
			n\\
			m
		\end{pmatrix}
		\mal p^m\mal p^{n-m}
		=n\mal\tau
	\end{align*}
	\betone{Kurze Rechnung zur Herleitung der Differentialgleichung:}\\
	$f$ erfüllt die Differenzengleichung
	\begin{align}\label{eq:BeispEinfuehrung1}\tag{1}
		f(x,t+\tau)=p\mal f(x-h,t)+q\mal f(x+h,t)
	\end{align}	 
	mit Anfangsbedingungen
	\begin{align*}
		f(0,0)=1,\qquad
		f(x,0)=0\quad
		\forall x\neq0.
	\end{align*}
	\begin{aufgabe}[1.2 Teil 1]\enter
		Rechnen Sie nach, dass \eqref{eq:BeispEinfuehrung1} die Differenzengleichung mit Anfangsbedingungen erfüllt.
	\end{aufgabe}
	\begin{lösung}
		ToDo %TODO
	\end{lösung}
	Wir bilden den Grenzwert $h,\tau\longrightarrow 0$.
	Aus physikalischen Überlegungen folgt, dass dabei $h,\tau$ und $p$ (geht gegen $\frac{1}{2}$) gewisse Bedingungen erfüllen müssen:
	\begin{align}\label{eq:BeispEinfuehrung2}\tag{2}
		\frac{h^2}{\tau}\longrightarrow 2\mal D,\qquad
		\frac{p-q}{h}\longrightarrow\frac{c}{D}
	\end{align}
	wobei $x=n\mal h$, $t=n\mal\tau$; $c$ und $D$ sind Konstanten:
	\begin{itemize}
		\item $c$ ist die Strömungsgeschwindigkeit
		\item $D$ ist der Diffusionskoeffizient
	\end{itemize}
	Wir ziehen von beiden Seiten von \eqref{eq:BeispEinfuehrung1} den Term $f(x,t)$ ab:
	\begin{align}
			f(x,t+\tau)-f(x,t)\nonumber
			&=p\mal f(x-h,t)+q\mal f(x+h,t)-f(x,t)\mal\underbrace{p-q}_{=1}\\
			&=p\mal\klammern[\big]{f(x-h,t)-f(x,t)}\label{eq:BeispEinfuehrung3}\tag{3}
			+q\mal\klammern[\big]{f(x+h,t)-f(x,t)}
	\end{align}
	Sei $f$ einmal nach $t$ und zweimal nach $x$ differenzierbar.
	\begin{align*}
		f(x,t+\tau)-f(x,t)
		&=\tau\mal\frac{\partial f(x,t)}{\partial t}+o(\tau)\\
		f(x-h,t)-f(x,t)
		&=-h\mal\frac{\partial f(x,t)}{\partial x}
		+\frac{1}{2}\mal h^2\mal\frac{\partial^2 f(x,t)}{\partial x^2}+o(h^2)\\
		f(x+h,t)-f(t,x)
		&=h\mal\frac{\partial f(x,t)}{\partial x}+\frac{1}{2}\mal h^2\mal\frac{\partial^2 f(x,t)}{\partial x^2}+o(h^2)
	\end{align*}
	Die Gleichungen entsteht durch Taylorentwicklung (bis zum ersten bzw. bis zum zweiten Term).
	Wir setzen dies in \eqref{eq:BeispEinfuehrung3} ein und verwenden dabei \eqref{eq:BeispEinfuehrung2}:
	\begin{align*}
		\frac{\partial f(x,t)}{\partial t}
		=-2\mal c\mal\frac{\partial f(x,t)}{\partial x}+D\mal\frac{\partial^2 f(x,t)}{\partial x^2}
	\end{align*}
	Dies ist die \define{Fokker-Planck-Differentialgleichung}.
	\begin{aufgabe}[1.2 Teiil 2]
		Leiten Sie die Differentialgleichung ausführlich her.
	\end{aufgabe}
	\begin{lösung}
		To Do %TODO
	\end{lösung}
\end{beisp}

Anfang der 30er Jahre (20. Jhd.) wurden die 
Grundsteine der allgemeinen Theorie von stochastischen Prozessen gelegt:
\begin{itemize}
	\item A.N. Kolmogorov: \undefine{Prozesse ohne Nachwirkung} / \undefine{Markovsche Prozesse}
	\item A.J. Khinchin: stationäre Prozesse 
\end{itemize}

\begin{beisp}[Poisson-Prozess]\enter
	\index{Poisson-Prozess}
	Der \undefine{Poisson-Prozess} wurde vorher schon von Physikern wie Einstein und Smoluchowski im Zusammenhang mit der Brownschen Bewegung untersucht.\\
	In zufälligen Zeitpunkten tritt ein gewisses Ereignis $A$ ein.\\
	$X(t)$ sei die Anzahl des Eintretens von $A$ im Zeitintervall $(0,t)\subseteq\R$.
	Setze weiterhin
	\begin{align*}
		P_k(t):=\P\big[X(t)=k\big],\qquad\forall k\in\N_{\geq0}
	\end{align*}
	\betone{Voraussetzungen an die Folge $(P_k)_{k\in\N_0}$:}
	\begin{enumerate}
		\item $(P_k)_{k\in\N_0}$ sei \define{stationär}, d.h. für das $k$-fache Auftreten im Intervall $(T,T+t)$ hängt die Wahrscheinlichkeit $P_k(t)$ nicht von $T$ ab.
		\index{stationär}
		\item $(P_k)_{k\in\N_0}$ sei \define{ohne Nachwirkung (Markoveigenschaft)}, d.h. die obige Wahrscheinlichkeit $P_k(t)$ ist unabhängig davon, wie viele Male und wann $A$ \betone{vorher} eintrat
		\index{ohne Nachwirkung}
		\item $(P_k)_{k\in\N_0}$ sei \define{ordinär}, d.h. das mehrfache Auftreten von $A$ in einem sehr kleinen Intervall $\Delta t$ kann praktisch nicht vorkommen.\\
		\index{ordinär}
		Genauer:
		Die Wahrscheinlichkeit dafür ist $o(\Delta t)$.
	\end{enumerate}
	 Aus diesen Annahmen folgt (ohne Beweis):
	 \begin{align*}
	 	P_k(t)&=\frac{(\lambda\mal t)^k}{k!}\mal\exp(-\lambda\mal t)
	 \end{align*}
	 (Poisson-Verteilung) mit einer Konstanten $\lambda>0$, die sogenannte \define{Intensität} des Prozesses.\nl
	 \betone{Beispiele für $A$:}
	 \begin{itemize}
	 	\item Auftreten eines kosmischen Teilchens auf eine  bestimmte Fläche
	 	\item Zerfall eines Atoms eines radioaktiven Stoffes
	 	(falls nicht zu groß oder zu klein)
	 \end{itemize}
\end{beisp}

Bis jetzt war $t$ in den Beispielen die Zeit, sie kann aber z.B. auch eine Ortsangabe sein.
Beispiel: Geostatistik (Modellierung von Gesteinsschichten)

\section{Stochastische Prozesse}
In diesem Abschnitt seien:
\begin{itemize}
	\item $T\neq\emptyset$: beliebige Menge, die \define{Parametermenge}
	\item $(\Omega,\A,\P)$: Wahrscheinlichkeitsraum
	\item $(S,\B)$: Messraum ($S$ ist beliebige Menge und $\B$ eine $\sigma$-Algebra auf $S$)
\end{itemize}

\begin{definition}\label{def1.3.1}
	Ein \define{stochastischer Prozess (SP) / Zufallsfeld}
	\index{Zufallsfeld}
	\index{stochastischer Prozess|see{Zufallsfeld}}
	$Z$ auf $T$ mit \define{Grundraum}\index{Grundraum}
	$(\Omega,\A,\P)$ ist eine Abbildung, die jedem Element $t\in T$ eine $S$-wertige Zufallsvariable zuordnet.
	\begin{align*}
		Z\colon T\to(\Omega\to S),\qquad t\mapsto\big(\omega\mapsto X(\omega)\big)
	\end{align*}
\end{definition}

\begin{bemerkungnr}\label{bemerkung1.3.2}\
	\begin{enumerate}[label=(\arabic*)]
		\item Anstelle von $Z(t)$ wird auch $Z_t$ geschrieben.
		Die Schreibweise $Z(t)(\omega)$, $\omega\in\Omega$ wäre umständlich; man schreibt
		\begin{align*}
			Z_t(\omega):=Z(t,\omega):=Z(t)(\omega)\qquad\forall \omega\in\Omega,\forall t\in T
		\end{align*}
		Das Feld $Z$ wird auch mit  %TODO add Erklärung
		\begin{align*}
			\set{Z(t)}\qquad\oder\qquad\set{Z_t}_{t\in T}\qquad\oder\qquad\set{Z_t:t\in T}
		\end{align*}
		bezeichnet.
		\item Ist $S$ ein metrischer Raum oder topologischer Raum, z.B. eine Teilmenge von $\R^d$ oder $\C^d$, so nehmen wir für $\B$ die \define{Borel-$\sigma$-Algebra}, d.h. die kleinste $\sigma$-Algebra, die alle offenen Mengen enthält.
		\index{Borel-$\sigma$-Algebra}
		\item Sind die Zufallsgrößen reell- oder komplexwertig, so heißt $Z$ ein \define{reelles} bzw. \define{komplexes Zufallsfeld}.
		\item Falls $T\in\set[\big]{\R,[0,\infty),[0,1]}$, dann heißt $Z$ \define{zeitstetiger Prozess}. \index{zeitsteiger Prozess}
		\item Falls $T\in\set[\big]{\N,\N_0,\Z}$, dann heißt $Z$ \define{zeitdiskreter Prozess / Zeitreihe}. \index{zeitdiskreter Prozess}\index{Zeitreihe|see{zeitdiskreter Prozess}}
		\item Die Funktionen
		\begin{align*}
			T\to S,\qquad		
			t\mapsto Z(t,\omega)
		\end{align*}
		heißt \define{Trajektorie / Realisierung / Pfad} für jedes feste $\omega\in\Omega$.
		\index{Trajektorie}\index{Realisierung|see{Trajektorie}}\index{Pfad|see{Trajektorie}}\index{Pfadraum}\\
		Der \define{Pfadraum} ist die Gesamtheit aller Pfade.\\
		Man kann das Feld $Z$ auch als eine Abbildung auffassen, die jedem $\omega\in\Omega$ eine $S$-wertige Funktion auf $T$ zuordnet, nämlich
		die Realisierung
		\begin{align*}
			Z(\cdot,\omega)\colon T\to S,\qquad t\mapsto Z(t,\omega)\qquad\forall \omega\in\Omega
		\end{align*}
		Eine weitere Sichtweise ist, das Feld als eine Abbildung
		\begin{align*}
			Z:T\times\Omega\to S,\qquad (t,\omega)\mapsto Z(t,\omega)
		\end{align*}
		zu betrachten, die für jedes feste $t\in T$ messbar bezüglich $(\A,\B)$ ist.
		\item Sei $(\Omega,\A_0,\P_0)$ die \define{Vervollständigung} des Grundraumes, d.h.: \index{Vervollständigung}\\
		$\A_0$ ist die $\sigma$-Algebra, aller Mengen der Form
		\begin{align*}
			A_0&=(A\cup N)\setminus M
		\end{align*}
		wobei $A\in\A$ und $M,N\subseteq\Omega$ Teilmengen von $\P$-Nullmengen sind; %TODO Formaler machen
		\begin{align*}
			\P_0(A_0):=\P(A)\qquad\forall A_0\in\A_0.
		\end{align*}
		Die Erweiterung hat die günstige Eigenschaft, dass beliebige Teilmengen von Nullmengen Ereignisse sind.
		Wegen $\A\subseteq\A_0$ können wir ein Feld mit Grundraum $(\Omega,\A,\P)$ auch als Feld mit Grundraum $(\Omega,\A_0,\P_0)$ betrachten.
		Da $\P_0|_\A=\P$ bleiben die Verteilungen der Zufallsvariablen $Z_t$ unverändert.
	\end{enumerate}
\end{bemerkungnr}

\begin{aufgabe}[1.3.2.7]
	Zeigen Sie:
	\begin{enumerate}
		\item $\A_0$ ist eine $\sigma$-Algebra.
		\item $\P_0$ ist wohldefiniert (also unabhängig von der Darstellung).
		\item $\P_0$ ist ein Wahrscheinlichkeitsmaß.
	\end{enumerate}
\end{aufgabe}

\begin{lösung}
	To Do %TODO
\end{lösung}

\begin{beisp}
	Der Wahrscheinlichkeitsraum $(\Omega,\A,\P)$ mit
	\begin{align*}
		\Omega&=[0,1]\\
		\A&=\set{\emptyset,\Omega,\intervall{0}{\frac{1}{2}},\intervallOH{\frac{1}{2}}{1}}\\
		\P\klammern{\intervall{0}{\frac{1}{2}}}&=0,1\qquad\P\klammern{\intervall{\frac{1}{2}}{1}}=0,9.
	\end{align*}		
	ist \betone{nicht} vollständig.
	 Dann ist
	 \begin{align*}
	 	\A_0=?\qquad\P_0=?
	 \end{align*}
	 \betone{oder}:
	 \begin{align*}
	 	\P\klammern{\intervall{0}{\frac{1}{2}}}&=0\qquad\P\klammern{\intervall{\frac{1}{2}}{1}}=1
	 \end{align*}
	  Dann ist
	 \begin{align*}
	 	\A_0=?\qquad\P_0=?
	 \end{align*}

	\begin{aufgabe}[1.3.2.7]	 
	 	Zeige, dass $(\Omega,\A,\P)$ nicht vollständig ist und gibt in beiden Fällen $\A_0$ und $\P_0$ an.
	\end{aufgabe}
	\begin{lösung}
		To Do %TODO
	\end{lösung}
\end{beisp}

\begin{beispiel}\label{beisp1.3.3}\
	\begin{enumerate}[label=(\alph*)]
		\item Sind $X_j,j\in\N_0$ beliebige Zufallsgrößen, so sind 
		\begin{align*}
			\set{X_j:j\in\N_0}
			\qquad\und\qquad
			S_n:=\sum\limits_{j=0}^n X_j\qquad\forall n\in\N_0
		\end{align*}
		Zeitreihen ($T=\N_0$ und $S=\R$).
		\item Seien $a$ und $\varphi$ zwei Zufallsgrößen und $u\in\R$.
		Dann ist \index{harmonische Schwingung}
		\begin{align*}
			X_t:=a\mal\cos(u\mal t+\varphi)\qquad\forall t\in\R
		\end{align*}
		ein stochastischer Prozess auf $\R$, eine sogenannte \define{harmonische Schwingung} mit zufälligen Parametern
		\begin{itemize}
			\item $a$: \define{Amplitude}
			\item $\varphi$: \define{Phase}
		\end{itemize}
		und \define{Frequenz} $u$ gegeben.
		(Für festes $t\in T$ ist $X_t$ eine Zufallsgröße, da $\cos$ stetig und damit messbar als Kombination messbarer Abbildungen.)\\
		Viele Prozesse in Anwendungen lassen sich als Summe von solchen harmonischen Schwingungen modellieren / annähern.
		\begin{align*}
			X_t(\omega)=\sum\limits_{j=1}^n a_j(\omega)\mal\cos\big(u_j\mal t+\varphi_j(\omega)\big)\qquad\forall t\in\R,\forall\omega\in\Omega
		\end{align*}
		Die Menge $\set{u_1,\ldots,u_n}$ der Frequenzen heißt das \define{Spektrum} von $X$. \index{Spektrum}\\
		Wichtige Aufgabe in Anwendungen ist die Bestimmung des Spektrums mit Hilfe von endlich vielen Beobachtungen 
		\begin{align*}
			X_{t_1}(\omega),\ldots,X_{t_m}(\omega)
		\end{align*}
		z.B. die Anzahl der Sonnenflecke.\\
		Oft ist es vorteilhaft komplexwertige Schwingungen zu betrachten:
		\begin{align}\label{eq:beispiel1.3.3HarmonischeSchwingungKomplex}
			Z_t:=\sum\limits_{j=1}^n c_j\mal\exp(\ii\mal u_j\mal t)\qquad\forall t\in\R
		\end{align}
	\end{enumerate}
\end{beispiel}




