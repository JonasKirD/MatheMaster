% This work is licensed under the Creative Commons
% Attribution-NonCommercial-ShareAlike 4.0 International License. To view a copy
% of this license, visit http://creativecommons.org/licenses/by-nc-sa/4.0/ or
% send a letter to Creative Commons, PO Box 1866, Mountain View, CA 94042, USA.

\chapter{Lösungen der Übungsaufgaben}

\section{Lösung von 
	\texorpdfstring{\hyperref[aufg:1]{Aufgabe 1}}{}
}\label{loes:1}

%TODO

\section{Lösung von 
	\texorpdfstring{\hyperref[aufg:2]{Aufgabe 2}}{}
}\label{loes:2}

%TODO

\section{Lösung von 
	\texorpdfstring{\hyperref[aufg:3]{Aufgabe 3}}{}
}\label{loes:3}

Wir zeigen, dass $\A_0$ ein $\sigma$-Algebra ist.\\
$\Omega\in\A_0$, weil $\Omega\subseteq\A$ und somit $\Omega\cup\emptyset\setminus\emptyset\in\A_0$.\nl
Sei $A_0\in\A_0$. Wir zeigen $A_0^C\in\A_0$.
Da $A_0\in\A_0$ existieren $A\in\A$ und $\P$-Nullmengen so, dass $A_0=(A\cup N)\setminus M\in\A_0$.
\begin{align*}
	A_0^C
	&=\Omega\setminus\big((A\cup N)\setminus M\big)\\
	&=\Omega\setminus\big((A\cup N)\cap M^C\big)\\
	&=\Omega\cap\big((A\cup N)\cap M^C\big)\\
	&=(A\cup N)\cap M^C\\
	\overset{\text{DM}}&=
	(A\cup N)^C\cup M\\
	&=(A^C\cap N^C)\cup M\\
	&=(A^C\cup M)\cap (N^C\cup M)\\
	&=(A^C\cup M)\setminus(N^C\cup M)^C\\
	&=(A^C\cup M)\setminus(N\cap M^C)\\
	&=(\tilde{A}\cap\tilde{N})\setminus\tilde{M}\qquad\mit\tilde{A}:=A^C\in\A,~\tilde{N}:=M,~\tilde{M}:=N\cap M^C~\P\text{-Nullmengen}\\
	&\implies A_0^C\in\A_0
\end{align*}
Sei $(A_{0,n})_{n\in\N}\subseteq\A_0$. %TODO


Wir zeigen nun die Wohldefiniertheit von $\P_0$:
%TODO

\section{Lösung von 
	\texorpdfstring{\hyperref[aufg:4]{Aufgabe 4}}{}
}\label{loes:4}

\section{Lösung von 
	\texorpdfstring{\hyperref[aufg:5]{Aufgabe 5}}{}
}\label{loes:5}

\section{Lösung von 
	\texorpdfstring{\hyperref[aufg:6]{Aufgabe 6}}{}
}\label{loes:6}

\section{Lösung von 
	\texorpdfstring{\hyperref[aufg:7]{Aufgabe 7}}{}
}\label{loes:7}

\section{Lösung von 
	\texorpdfstring{\hyperref[aufg:8]{Aufgabe 8}}{}
}\label{loes:8}

\section{Lösung von 
	\texorpdfstring{\hyperref[aufg:9]{Aufgabe 9}}{}
}\label{loes:9}

\section{Lösung von 
	\texorpdfstring{\hyperref[aufg:10]{Aufgabe 10}}{}
}\label{loes:10}

Nein, ist nicht stationär im engeren Sinne. Gegenbeispiel:





