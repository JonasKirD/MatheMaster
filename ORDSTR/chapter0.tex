% !TEX root = ORDSTR.tex
% This work is licensed under the Creative Commons
% Attribution-NonCommercial-ShareAlike 4.0 International License. To view a copy
% of this license, visit http://creativecommons.org/licenses/by-nc-sa/4.0/ or
% send a letter to Creative Commons, PO Box 1866, Mountain View, CA 94042, USA.

\chapter{Ordnungsstrukturen}
\section{Relationen und Ordnungen}

\renewcommand{\|}{~|~}
\begin{definition}
    Seien $P,R$ Mengen. Wir nennen $\mathscr{R} = (P,R)$ ein \textbf{binäres Relat}, 
    wenn $R \subseteq P \times P$ (also $R$ binäre Relation ist).

    Beachte, dass das Kartesische Produkt wie folgt definiert ist:
    $$ P \times Q := \{(p,q) \| p \in P \textrm{ und } q \in Q\}.$$
\end{definition}

\begin{definition}
    Sei $\mathscr{R} = (P,R)$ ein binäres Relat. Es seien $p,t,q \in P$ beliebig.
    Wir nennen $\mathscr{R}$:
    \begin{enumerate}[label=(\arabic*)]
        \item \textbf{reflexiv}, falls $(pRp)$,
        \item \textbf{transitiv}, falls $pRt \land tRq \implies pRq$,
        \item \textbf{antisymmetrisch}, falls $pRq \land qRp \implies p=q$,
        \item \textbf{symmetrisch}, falls $pRq \implies qRp$ und
        \item \textbf{linear / total}, falls $pRq\vee qRp$.
    \end{enumerate}

    Beachte, dass $pRq$ für $(p,q) \in R$ steht.
\end{definition}

\textit{Linearisierung} partieller Ordnung geht immer!

\begin{definition}\
\begin{itemize}
\item Eine \textbf{Präordnung} ist ein reflexives und transitives binäres Relat.
\item Eine \textbf{partielle Ordnung} ist eine antisymmetrische Präordnung.
\item Eine partielle Ordnung $\mathscr{R} = (P,R)$ heißt \textbf{lineare} Ordnung (oder \textbf{Totalordnung})
     $$:\Longleftrightarrow pRq \textrm{ oder } qRp\qquad\forall p,q \in P$$
\end{itemize}
\end{definition}

\begin{notation}
    Wenn $A,B$ Mengen, so bezeichnet $B^A$ die Menge aller Abbildungen von $A$ nach $B$.
\end{notation}

\begin{definition}
Eine \textbf{Multimenge} $\mathcal{M}$ (von $M$) ist ein Element $\mathcal{M}\in\N^M$, also eine Abbildung $\mathcal{M}:M\to\N$.\\
Anschaulich ist eine Multimenge eine Menge, in der die Elemente mehrfach vorkommen können.
\end{definition}

\begin{lemma}
Sei $(M,\leq)$ eine partielle Ordnung. Dann ist die Multimenge $\mathcal{M}$ per Dominanzordnung geordnet.
\end{lemma}

\begin{beispiel}
\begin{align*}
(\R^2,\leq)&=(\R,\leq)^2\\
a=(a_1,a_2)&\leq(b_1,b_2)=b:\Longleftrightarrow a_1\leq b_1\wedge a_2\leq b_2
\end{align*}
$(R^2,\sqsubseteq)$ ist \textbf{lineare Erweiterung} von $(\R,\leq)^2$ via 
\begin{align*}
a\sqsubseteq b:\Longleftrightarrow a_1<b_1\vee a_1=b_1\wedge a_2\leq b_2
\end{align*}
Sei $(S,\leq)$ lineare Ordnung (d.h. reflexiv, transitiv, antisymmetretisch und linear. Dann ist $(S,\leq)^\Omega:=(S^\Omega,\leq)$ Ordnung via
\begin{align*}
\alpha\leq\beta:\Longleftrightarrow\forall\alpha,\beta\in S^\Omega,\forall x\in\Omega:\alpha x\leq\beta x
\end{align*}
\end{beispiel}

\begin{notation}
Setze
\begin{align*}
\underline{n}&:=\big\lbrace 0,\ldots,n-1\big\rbrace\subseteq\N\\
[n]&:=\big\lbrace1,\ldots,n\big\rbrace\subseteq\N
\end{align*}
\end{notation}

\begin{definition}
Sei $\Omega$ beliebige Menge und $A\subseteq\Omega$. Dann ist die \textbf{charakteristische Funktion / Indikatorfunktion} definiert als
\begin{align*}
\chi_A^\Omega:\Omega\to\lbrace0,1\rbrace,\qquad\omega\mapsto\left\lbrace\begin{array}{cl}
0, & \falls \omega\not\in A\\
1, &\falls\omega\in A
\end{array}\right.
\end{align*}
\end{definition}

Somit können Teilmengen als Multimengen aufgefasst werden, denn $\chi_A$ ist eine Multimenge.

\begin{beispiel}\
\begin{itemize}
\item $\underline{2}^\Omega$ ist die Potenzmenge
\item $\N^\Omega$ sind die Multimengen
\item $\Omega^{\N}$ sind die Folgen in $\Omega$
\item $\R_{\geq0}^\Omega$ sind die reellen Multimengen über $\Omega$
\end{itemize}
\end{beispiel}

Allgemein: Seien $S,\Omega$ Mengen. Dann ist $S^\Omega$ die $S$-wertige Multimenge über $\Omega$.

\begin{lemma}
Ist $(S,\leq)$ Präordnung (transitiv + reflexiv), dann ist die Menge $S^\Omega$ auf naheliegende Weise durch die \textbf{Dominanzordnung}
\begin{align*}
\forall\alpha,\beta\in S^\Omega:\alpha\leqq\beta:\Longleftrightarrow\forall\omega\in\Omega:\alpha(\omega)\leq\beta(\omega)
\end{align*}
geordnet. Notation: $(S,\leq)^\Omega:=\big(S^\Omega,\leqq\big)$\\
Ist $\alpha\leqq\beta$, so \textbf{dominiert} $\beta$ das $\alpha$.
\end{lemma}

\begin{aufgabe}
    Sei $N$ eine endliche Menge und 
    sei $\R^N$ mit der \textbf{Dominanzordnung}, d.h. für alle $u,v \in \R^N$ gilt
    $$u \leq v :\iff \forall i \in N (u_i \leq v_i),$$
    gegeben.
    
    Wie lässt sich $\R^N$ linear erweitern?
\end{aufgabe}

\begin{lösung}
    Zunächst wähle eine Totalordnung auf $ N $ und benenne die Elemente so um, dass
    $N = \{1,2,\ldots,n\}$, wobei $n \in \N$.
    Definiere für alle $u,v \in \R^N$:
    $$ I^v_u := \{i \in N \| u_i \leq v_i\}.$$

    Nun definiere die lineare Ordnung wie folgt:
    $$ u \leq_l v :\iff \min I^v_u \leq \min I^u_v.$$
\end{lösung}

\textbf{Frage:} Seien $(\Omega,\leq)$ und $(S,\leq)$ lineare Ordnungen. Existiert dann auch eine lineare Ordnung $(S^\Omega,\sqsubseteq)$, die $(S,\leq)^\Omega$ erweitert, d.h.
\begin{align*}
\forall\alpha,\beta\in S^\Omega:\alpha\leq\beta\implies\alpha\sqsubseteq\beta
\end{align*}

\begin{definition}
Eine Ordnung $(\Omega,\leq)$ heißt \textbf{Wohlordnung} $:\gdw(\Omega,\leq)$ eine Ordnung derart ist, dass jede nichtleere Teilmenge ein (bzgl. der Ordnung) kleinstes Element enthält.
\end{definition}

\begin{beispiel}
$(\N,\leq)$ ist Wohlordnung.
\end{beispiel}

\begin{bemerkung}
$(\Omega,\leq)$ Wohlordnung $\implies(\Omega,\leq)$ lineare Ordnung.
\end{bemerkung}
\begin{proof}
Sei $x,y\in\Omega$. Dann hat $\lbrace x,y\rbrace$ ein kleinstes Element bzgl. $(\Omega,\leq)$. Also ist $(\Omega,\leq)$ linear.
\end{proof}

\textbf{Mitteilung.} Aus dem Auswahlaxiom bzw. dem Zornschen Lemma folgt, dass jede Menge eine Wohlordnung besitzt (d.h. zu jeder Menge $\Omega$ existiert $\leq$ derart, dass $(\Omega,\leq)$ Wohlordnung ist).

\begin{satz}
Sei $(\Omega,\leq)$ Wohlordnung und sei $(S,\leq)$ lineare Ordnung.\\
Dann ist $(S^\Omega,\sqsubseteq)$ lineare Ordnung, die $(S,\leq)\Omega$ erweitert, vermöge
\begin{align*}
\Omega(\alpha<\beta)&:=\lbrace x\in\Omega:\alpha x<\beta x\rbrace &\forall \alpha,\beta\in S^\Omega\\
\Omega(\alpha=\beta)&:=\lbrace x\in\Omega\mid\alpha x=\beta x\rbrace &\forall \alpha,\beta\in S^\Omega\\
\Omega(\alpha>\beta)&:=\lbrace x\in\Omega:\beta x<\alpha x\rbrace &\forall\alpha,\beta\in S^\Omega
\end{align*}
Setze dann
$\alpha\sqsubseteq\beta:\gdw\Omega(\beta>\alpha)=\emptyset$ oder $\alpha=\beta$ oder $\Omega(\alpha<\beta)$ und $\Omega(\beta>\alpha)$ sind nichtleer und das kleinste Element von $\big(\Omega(\alpha<\beta),\leq\big)$ ist kleiner als das kleinste Element von $\big(\Omega(\beta<\alpha),\leq\big)$.
\end{satz}

\begin{satz}
Sei $(\Omega,\leq_\Omega)$ Wohlordnung und sei $(S,\leq_S)$ lineare Ordnung.\\
Dann ist $(S^\Omega,\sqsubseteq)$ via
\begin{align*}
\alpha\sqsubseteq\beta:\Longleftrightarrow\alpha=\beta\vee\exists x\in\Omega:\Big(\alpha x<_S\beta x\wedge\forall t\in\Omega:\big(t<_\Omega x\implies \alpha t=\beta t\big)\Big)
\\&\forall\alpha,\beta\in S^\Omega
\end{align*}
eine lineare Ordnung definiert, die $(S,\leq_S)^\Omega$ erweitert.
\end{satz}

\begin{definition}
Die \textbf{Lexikografische Ordnung} ist definiert als
\begin{align*}
\text{LEX}\big((\Omega,\leq),(S,\leq)\big):=\big(S^\Omega,\sqsubseteq\big)
\end{align*}
\end{definition}

\begin{bemerkung}
$(S^\Omega,\sqsubseteq)$ ist die \textbf{lexikografische} Ordnungserweiterung von $(S,\leq_S)^\Omega$ via $(\Omega,\leq_\Omega)$.
\end{bemerkung}

Ist $(S,\leq)$ nicht linear, dann ist $(S^\Omega,\sqsubseteq)$ immer noch Erweiterung von $(S,\leq)^\Omega$ Es gilt:
\begin{align*}
\big(S^\Omega,\sqsubseteq\big)\text{ linear}\Longleftrightarrow(S,\leq)\text{ linear}
\end{align*}
und außerdem
\begin{align*}
(\Omega,\leq)\text{ Wohlord. und }(S,\leq)\text{ Ord. }\implies\big(S^\Omega,\sqsubseteq\big)\text{ Ord.}
\end{align*}

\section{Ascending / Descending chain condition (ACC / DCC)}
\begin{definition}
Eine Ordnung $(P,\leq)$ heißt
\begin{itemize}
\item \textbf{ACC / noethersch}
\begin{align*}
:\Longleftrightarrow\forall(a_1\leq a_2\leq\ldots):\exists n\in\N:m\geq n\implies a_m=a_n
\end{align*}
\item \textbf{DCC / artinsch}
\begin{align*}
:\Longleftrightarrow\forall(a_1\geq a_2\geq\ldots):\exists n\in\N:m\geq n\implies a_m=a_n
\end{align*}
\end{itemize}
\end{definition}

\begin{lemma}
Es gilt:
\begin{enumerate}
\item $(P,\leq)$ Wohlordnung $\Longleftrightarrow (P,\leq)$ DCC
\item $(P,\leq)$ ACC und DCC $\Longleftrightarrow P$ endlich
\end{enumerate}
\end{lemma}

\begin{bemerkung}\
\begin{itemize}
\item ACC: ``Jede aufsteigende Kette terminiert'' $\gdw$ jede Teilmenge hat ein maximales Element (aufsteigende Kettenbedingung)
\item DCC: ``Jede absteigende Kette terminiert'' $\gdw$ jede Teilmenge hat ein minimales Element.
\end{itemize}
\end{bemerkung}

\begin{beispiel}[Duale Verbände]\
\begin{itemize}
\item ACC: $\text{Id}(\Z)=\big\lbrace n\cdot\Z:n\in\N\big\rbrace$ mit $n\Z\leq m\Z:\Longleftrightarrow n\Z\subseteq m\Z$
\item DCC: \textbf{Teilerordnung} $(\N,\leq_\tau)$ mit $n\leq_ \tau m:\Longleftrightarrow\exists t\in\N:n\cdot t=m$
\end{itemize}
Anti-Isomorphismus:
\begin{align*}
\varphi:(\N,\leq_\tau)\to\text{Id}(\Z),\qquad n\mapsto n\Z\\
n\leq_\tau m\Longleftrightarrow n\Z\supseteq m\Z\\
\implies(\N,\leq_\tau)^{\text{op}}\cong\text{Id}(\Z)
\end{align*}
\end{beispiel}

\begin{definition}
Ist $f:A\to B$ eine Abbildung, dann ist der \textbf{Kern} von $f$ definiert als
\begin{align*}
\ker(f):=\big\lbrace(a_1,a_2)\in A^2:f(a_1)=f(a_2)\big\rbrace
\end{align*}
\end{definition}

\section{Das Delta}
\begin{definition}
\begin{align*}
\delta:\big\lbrace(x,y)\in[0,\infty]^2:x\leq y\big\rbrace\to[0,\infty],\qquad (x,y)\mapsto\left\lbrace\begin{array}{cl}
y-x,&\falls y\neq\infty\\
\infty, &\falls x\neq\infty,y=\infty\\
0, &\falls x=\infty
\end{array}\right.
\end{align*}
\end{definition}

\begin{lemma}
Eine mögliche Erweiterung von $\delta$ ist:
\begin{align*}
\overline{\delta}:[0,\infty]^2\to[0,\infty],\qquad
(x,y)\mapsto\inf\big\lbrace t\in[0,\infty]:y\leq x+t\big\rbrace
\end{align*}
\end{lemma}

\begin{bemerkung}
Für $x,y\in[0,\infty]$ mit $x\leq y$ ist also $\delta(x,y)$ das kleinste Element $t$ in $\big([0,\infty],\leq\big)$ mit $x+t=y$, d.h. $t$ ist die kleinste ``Ergänzung'' von $x$ zu $y$ im natürlich geordneten Monoid $\big([0,\infty],+,0\big)$\\
Man erkennt hieraus leicht, dass $\delta$ \textit{funktoriell} ist:
\begin{align*}
\delta(x,x)=0,\qquad\delta(x,y)+\delta(y,z)=\delta (x,z)\qquad\forall x\leq y\leq z
\end{align*}
hingegen $\overline{\delta}$ nicht. Weiterhin ist $\delta$ modular:
\begin{align*}
\delta(x,x\vee y)=\delta(x\wedge y,y)
\end{align*}
genauso wie $\overline{\delta}$!\\
Aber wenn $\overline{\delta}$ schon nicht funktoriell ist, dann gilt zumindest die Dreiecksungleichung
\begin{align*}
\overline{\delta}(x,z)\leq\overline{\delta}(x,y)+\overline{\delta}(y,z)
\end{align*}
Definieren wir $\overline{\delta}^{\text{op}}$ als $\overline{\delta}^{\text{op}}(y,x):=\overline{\delta}(x,y)$. Dann ist
\begin{align*}
d:[0,\infty]^2\to[0,\infty],\qquad (x,y)\mapsto\overline{\delta}(x,y)+\overline{\delta}^{\text{op}}(x,y)
\end{align*}
eine Metrik mit
\begin{align*}
d(x,y)=\left\lbrace\begin{array}{cl}
|x-y|, &\falls x\neq\infty,y\neq0\\
0, &\falls x=y=\infty\\
\infty, &\sonst
\end{array}\right.
\end{align*}
\end{bemerkung}

\begin{definition}
Sei $\P=(P,R)$ eine Präordnung (d.h. reflexiv + transitiv) und sei $\mathbb{M}=(M,\ast,\varepsilon)$ ein Monoid. Dann heiße eine Abbildung
\begin{align*}
\Delta:R\to M
\end{align*}
\textbf{funktoriell} bzgl. $(\P,\mathbb{M}):\gdw\forall p~R~t,t~R~q\text{ in }P:$
\begin{enumerate}
\item $\Delta(p,p)=\varepsilon$
\item $\Delta(p,t)\ast\Delta(t,q)=\Delta(p,q)$
\end{enumerate}
\end{definition}
 
\begin{beispiel}
\begin{align*}
&\left\lbrace\begin{matrix}
\P=\big([0,\infty],\leq\big)\\
\mathbb{M}=\big([0,\infty],+,0\big)
\end{matrix}\right.\implies \mit\Delta:=\delta\\
&\left\lbrace\begin{array}{l}
\P=(P,R)\mit P:=[0,\infty]^{\N}\\
\mathbb{M}=\big([0,\infty],+,0\big)\\
\Delta:(\alpha,\beta)\mapsto\sum\limits_{n\in\N}\delta(\alpha n,\beta n)\cdot 2^{-n}
\end{array}\right.
\end{align*}
Was ist mit Modularität? Definition ist zu allgemein. Join $(\vee)$ und meet $(\wedge)$ fehlen!
\end{beispiel}
