% This work is licensed under the Creative Commons
% Attribution-NonCommercial-ShareAlike 4.0 International License. To view a copy
% of this license, visit http://creativecommons.org/licenses/by-nc-sa/4.0/ or
% send a letter to Creative Commons, PO Box 1866, Mountain View, CA 94042, USA.

\section{Aufgabenblatt 6}
\subsection{Aufgabe 6.1}
Sei $B$ eine Brownsche Bewegung.
Dann sind die folgenden Prozesse ebenfalls Brownsche Bewegungen:
\begin{enumerate}[label=\alph*)]
	\item $\begin{aligned}
		T_t:=c\cdot B_{\frac{t}{c^2}}\mit c\neq0
	\end{aligned}$
	\item $\begin{aligned}
		U_t:=-B_t
	\end{aligned}$
	\item $\begin{aligned}
		V_t:=\big(B_{t+s}-B_s\big)_{t\geq0}\mit s>0
	\end{aligned}$
	\item $\begin{aligned}
		W_t:=\left\lbrace\begin{array}{cl}
			t\cdot B_{\frac{1}{t}} ,&\falls t>0\\
			0, &\falls t=0
		\end{array}\right.
	\end{aligned}$
\end{enumerate}

\begin{proof}
	\underline{Zeige a):}
	%TODO

	\underline{Zeige b):}
	%TODO

	\underline{Zeige c):}
	%TODO

	\underline{Zeige d):}
	%TODO

\end{proof}

\subsection{Aufgabe 6.2}
Sei $B$ eine Brownsche Bewegung.
Die folgenden Prozesse sind Martingale bzgl. der von $B$ erzeugten Filtration:
\begin{enumerate}[label=\alph*)]
	\item $\begin{aligned}
		\big(B_t\big)_{t\geq0}
	\end{aligned}$
	\item $\begin{aligned}
		\big(B_t^2-t\big)_{t\geq0}
	\end{aligned}$
	\item $\begin{aligned}
		\left(\exp\left(c\cdot B_t-\frac{c^2}{2}\cdot t\right)\right)_{t\geq0}\mit c\in\R
	\end{aligned}$
\end{enumerate}

\begin{proof}
	\underline{Zeige a):}
	%TODO

	\underline{Zeige b):}
	%TODO

	\underline{Zeige c):}
	%TODO
\end{proof}

\subsection{Aufgabe 6.3}
Sei $B$ eine Brownsche Bewegung und $a,b\in(0,\infty)$.
Sei
\begin{align*}
	\tau:=\Big\lbrace t>0:B_t\in\lbrace+a,-b\rbrace\Big\rbrace.
\end{align*}
Zeige:
\begin{enumerate}[label=\alph*)]
	\item $\begin{aligned}
		\P\big(\tau=+a\big)=\frac{b}{a+b}
	\end{aligned}$
	\item $\begin{aligned}
		\E[\tau]=a\cdot b
	\end{aligned}$
\end{enumerate}

\begin{proof}
	Wir nutzen die Sätze  von optionalen Stoppen 8die für stetige Martingale mit $t\geq0$ weiterhin gelten).\nl
	\underline{Zeige a):}
	%TODO

	\underline{Zeige b):}
	%TODO

\end{proof}

\subsection{Aufgabe 6.4}
Sei $B$ eine Brownsche Bewegung, $a>0$ und
\begin{align*}
	\tau_a:=\inf\big\lbrace t>0:B_t=a\big\rbrace.
\end{align*}
Dann gilt:
\begin{enumerate}[label=\alph*)]
	\item $\begin{aligned}
		\E\big[\exp(-\lambda\cdot x)\big]=\exp\big(-a\cdot\sqrt{2\cdot\lambda}\big)\qquad\forall \lambda\geq0
	\end{aligned}$
	\item Die Verteilung von $\tau_a$ ist stabil (siehe Blatt 5, Aufgabe 4) mit Parameter $\alpha=\frac{1}{2}$.
	\item Die Verteilung von $\tau_a$ besitzt die Dichte
	\begin{align*}
		f(x)=\frac{a}{\sqrt{2\cdot\pi}}\cdot\exp\left(-\frac{a^2}{2\cdot x}\right)\cdot x^{-\frac{3}{2}}\qquad\forall x\geq0.
	\end{align*}
\end{enumerate}

\begin{proof}
	\underline{Zeige a):}
	%TODO

	\underline{Zeige b):}
	%TODO

	\underline{Zeige c):}\\
	Eine Dichte wird durch ihre \textbf{Laplace-Transformation}
	\begin{align*}
		\L[f](\lambda):=\int\limits_0^\infty\exp(-\lambda\cdot x)\cdot f(x)\d x\qquad\forall\lambda\geq0
	\end{align*}
	eindeutig bestimmt.

	%TODO
\end{proof}