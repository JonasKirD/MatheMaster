% This work is licensed under the Creative Commons
% Attribution-NonCommercial-ShareAlike 4.0 International License. To view a copy
% of this license, visit http://creativecommons.org/licenses/by-nc-sa/4.0/ or
% send a letter to Creative Commons, PO Box 1866, Mountain View, CA 94042, USA.

\section{Übungsblatt 1}

\subsection{Aufgabe 1.1 (Syntaktisch korrekt?)}
Grammatik für arithmetische Ausdrücke in BNF:
\begin{align*}
	\alpha::=|(\alpha_1+\alpha_2)|(\alpha_1-\alpha_2)|(\alpha_1\div\alpha_2)|(\alpha_1\ast\alpha_2)\mit q\in\Q
\end{align*}

\subsubsection{Aufgabe 1.1 (a)}
Sind die folgenden Zeichenreihen arithmetische Ausdrücke?
\begin{enumerate}[label=(\arabic*)]
	\item Nein, da, Klammern fehlen. Richtig wäre: $(((3-2)+(1-(3\times 4)))$ oder $(((3-2)+1)-(3\times 4))$.
	\item Nein, da rationale Zahlen nicht geklammert werden dürfen. Und ein unäres Minus ist auch nicht definiert. 
	(Anmerkung: Ist ein bisschen komisch, da man negative rationale Zahlen auch darstellen können muss.)
	\item Nein, da das Gleichheitszeichen nicht Teil des Alphabets ist.
	\item Nein, da es kein unäres Minuszeichen gibt.
	\item Nein, da $p,q$ aussagenlogische Variable. Wäre korrekt für $p,q\in\Q$.
\end{enumerate}

\subsubsection{Aufgabe 1.1 (b)}
Grammatik für aussagenlogische Formeln in BNF:
\begin{align*}
	\varphi::=p|\neg \varphi_1|(\varphi_1\wedge\varphi_2)|(\varphi_1\vee\varphi_2)|(\varphi_1\to\varphi_2)|(\varphi_1\leftrightarrow\varphi_2)\mit p\in\mathcal{R}
\end{align*}
Sind folgende Zeichenreihen aussagenlogische Formeln lauf Definition \ref{def3.5Formel}?
\begin{enumerate}[label=(\arabic*)]
	\item Nein, da die Zeichen ``1'' und ``2'' auftauchen.
	\item Nein, da die Zeichen ``2'' und ``3'' auftauchen.
	\item Ja.
	\item Nein, weil Klammern fehlen. Richtig wäre $((p\wedge p)\wedge(p\wedge p))$
	\item Nein, weil Klammern zu viel sind "$(\neg p)$" und fehlen und $\neg$ erwartet Ausdruck.
	\item Ja.
\end{enumerate}

\subsection{Aufgabe 1.2 (Induktionsbeweis \texorpdfstring{$n\leq n^2$}{n<=n²})}
Zeigen Sie mittels vollständiger Induktion, dass jede natürliche Zahl kleiner oder gleich
ihrem Quadrat ist.
\begin{proof}
	Induktionsanfang: $n=0:~0\leq0=0^2$\\
	Induktionsvoraussetzung: Gelte $n\leq n^2$ für beliebiges aber festes $n\in\N$.\\
	Induktionsschritt: 
	\begin{align*}
		n+1\leq
		n+\underbrace{2\cdot n}_{\geq0}+1
		\overset{\text{IV}}{\leq} n^2+2\cdot n+1=(n+1)^2\qquad\square
	\end{align*}
\end{proof}

\subsection{Aufgabe 1.3 (Unendliche Formeln?)}
Beweisen Sie folgende Aussagen.
Wir bezeichnen mit $|F|$ die Länge der Formel $F\in\L(\RR)$.

\subsubsection{Aufgabe 1.3 (a)}
Für jede natürliche Zahl $n > 0$ gibt es eine aussagenlogische Formel, die aus $n$ Zeichen des Alphabets der Aussagenlogik besteht.\nl
\betone{Idee:} Für alle $n\in\N$ ist $\underbrace{\neg\ldots\neg}_{(n-1)\text{-mal}}p$ eine Aussagenlogische der Formel der Länge $n$ falls $p\in\RR$.

\begin{proof}
	Beweis durch Induktion:\\
	Induktionsanfang: $n=1:$ Wähle $p\in\mathcal{R}$. 
	Dann gilt $p\in\mathcal{L}(\mathcal{R})$.\\
	Induktionsvoraussetzung: Für ein beliebiges aber festes $n\in\N$ gibt es eine aussagenlogische Formel der Länge $n$, in Zeichen 
	$\exists\varphi\in\mathcal{L}(\mathcal{R}):|\varphi|=n$.\\
	Induktionsschritt: Nach IV ist $\varphi$ aussagenlogische Formel der Länge $n$. 
	Dann ist $\neg\varphi$ nach Definition \ref{def3.5Formel} 2. auch aussagenlogische Formel, also 
	$\neg\varphi\in\mathcal{L}(\mathcal{R})$ und es gilt:
	\begin{align*}
		|\neg\varphi|=|\neg|+|\varphi|=1+|\varphi|=1+n	
	\end{align*}
\end{proof}

\subsubsection{Aufgabe 1.3 (b)}
Jede aussagenlogische Formel ist eine endliche Zeichenreihe,
d.h. es gibt also keine unendlich langen aussagenlogischen Formeln.

\begin{proof}
	Beweis durch strukturelle Induktion über $\varphi\in\mathcal{L}(\mathcal{R})$ (Menge der aussagenlogischen Formeln).\\
	Induktionsanfang: Sei $\varphi=p$ für ein $p\in\mathcal{R}$. Dann $|\varphi|=|p|=1\in\N$\\
	Induktionshypothese: für aussagenlogische Formeln $\varphi_1,\varphi_2\in\mathcal{L}(\mathcal{R})$ gilt $|\varphi_1|,|\varphi_2|\in\N$.\\
	Induktionsschritt: 
	\begin{align*}
		\varphi=\neg\varphi_1&\implies|\varphi|=|\neg\varphi_1|=|\neg|+|\varphi_1|=1+|\varphi_1|\stackrel{\text{IH}}{\in}\N\\
		\varphi=(\varphi_1\circ\varphi_2)\mit\circ\in\lbrace\wedge,\vee,\to,\leftrightarrow\rbrace&\implies|\varphi|=|(\varphi_1\circ\varphi_2)|=3+|\varphi_1|+|\varphi_2|\stackrel{\text{IH}}{\in}\N
	\end{align*}
\end{proof}

\subsection{Aufgabe 1.4 (Induktion: Klammer, Variablen und Junktoren)}
Gelten die folgenden Aussagen? Beweisen Sie Ihre Antworten. 
Verwenden Sie dazu, falls
möglich, das Prinzip der strukturellen Induktion.

\subsubsection{Aufgabe 1.4 (a)}
In jeder aussagenlogischen Formel ist die Anzahl der öffnenden Klammern gleich
die Anzahl der schließenden Klammern.

\begin{proof}
	Die Aussage stimmt, ja.\\
	Betrachte die Funktionen, die aufgehende bzw. schließende Klammern zählen:
	\begin{align*}
		\sk\equiv\ok:\mathcal{L}(\mathcal{R})\to\N,\qquad
		x\mapsto\left\lbrace\begin{array}{cl}
			0, & \falls x\in\mathcal{R}\text{ (ist Atom)}\\
			\ok(y), & \falls x=\neg y\\
			1+\ok(y)+\ok(z), & \falls x=(y\circ z)
		\end{array}\right.
	\end{align*}
	Man sieht schon, dass beide Funktionen für alle $x\in\mathcal{L}(\mathcal{R})$ übereinstimmen. 
	Aber man kann das Offensichtliche natürlich noch induktiv zeigen:\nl
	Zu zeigen: $\forall\varphi\in\mathcal{L}(\mathcal{R}):\ok(\varphi)=\sk(\varphi)$.\\
	Beweis über strukturelle Induktion:\\
	Induktionsanfang: Sei $\varphi=p\in\mathcal{R}$ Atom. Dann gilt: $\ok(p)=0=\sk(p)$.\\
	Induktionsvoraussetzung: Gelte für beliebiges aber festes 
	$\varphi_1,\varphi_2\in\mathcal{L}(\mathcal{R})$\\ $\ok(\varphi_1)=\sk(\varphi_1)$ und $\ok(\varphi_2)=\sk(\varphi_2)$ .\\ 
	Induktionsschritt:
	\begin{align*}
		\ok(\neg \varphi_1)
		&\stackeq{\text{Def}}\ok(\varphi_1)
		\stackeq{\text{IV}}\sk(\varphi_1)
		\stackeq{\text{Def}}\sk(\neg \varphi_1)\\
		\ok((\varphi_1\circ \varphi_2))
		&\stackeq{\text{Def}}1+\ok(\varphi_1)+\ok(\varphi_2)
		\stackeq{\text{IV}}1+\sk(\varphi_1)+\sk(\varphi_2)
		\stackeq{\text{Def}}\sk((\varphi_1\circ \varphi_2))
	\end{align*}
\end{proof}

\subsubsection{Aufgabe 1.4 (b)}
In jeder aussagenlogischen Formel sind doppelt so viele Klammern enthalten wie
binäre Junktoren.

\begin{proof}
	Ja, diese Aussagen gilt.\\
	Betrachte die Funktion, die Klammern bzw. binäre Junktoren zählt:
	\begin{align*}
		\k:\mathcal{L}(\mathcal{R})\to\N,\quad
			x\mapsto\left\lbrace\begin{array}{cl}
			0, & \falls x\in\mathcal{R}\text{ (ist Atom)}\\
			\k(y), & \falls x=\neg y\\
			2+\k(y)+\k(z), & \falls x=(y\circ z)
		\end{array}\right.
		\quad \bj:\equiv \ok\equiv\sk
	\end{align*}

	Beweis durch strukturelle Induktion über $\varphi\in\mathcal{L}(\mathcal{R})$.\\
	Induktionsanfang: Sei $\varphi=p\in\mathcal{R}$ Atom. 
	Dann gilt: $\k(\varphi)=0=2\cdot 0=\bj(\varphi)$.\\
	Induktionsvoraussetzung: Gelte für beliebige aber feste $\varphi_1,\varphi_2\in\mathcal{L}(\mathcal{R})$\\
	$\k(\varphi_1)=2\cdot\bj(\varphi_1)$ und $\k(\varphi_2)=2\cdot\bj(\varphi_2)$.\\ 
	Induktionsschritt: betrachte $\varphi=\neg\varphi_1$ und $\varphi=(\varphi_1\circ\varphi_2)\mit\circ\in\lbrace\wedge,\vee,\to,\leftrightarrow\rbrace$: 
	\begin{align*}
		\k(\neg \varphi_1)
		\overset{\text{Def}}&=
		\k(\varphi_1)
		\overset{\text{IV}}=
		2\cdot\bj(\varphi_1)
		\overset{\text{Def}}=
		\bj(\neg \varphi_1)\\
		\k((\varphi_1\circ \varphi_2))
		\overset{\text{Def}}&=
		2+\k(\varphi_1)+\k(\varphi_2)\\
		\overset{\text{IV}}&=
		2+2\cdot\bj(\varphi_1)+2\cdot\bj(\varphi_2)\\
		&=2\cdot(1+\bj(\varphi_1)+\bj(\varphi_2))\\
		\overset{\text{Def}}&=
		2\cdot\bj((\varphi_1\circ \varphi_2))
	\end{align*}
\end{proof}

\subsubsection{Aufgabe 1.4 (c)}
In jeder aussagenlogischen Formel ist die Anzahl der aussagenlogischen Variablen
gr¨oßer oder gleich der Anzahl der einstelligen Junktoren. (Mehrfach vorkommende
Variablen oder Junktoren sollen hierbei mehrfach gez¨ahlt werden.)

\begin{lösung}
	Nein, Gegenbeispiel: $\neg\neg\neg p\in\mathcal{L}(\mathcal{R})$ wobei $p\in\mathcal{R}$ aussagenlogische Variable ist und $\neg$ unärer Junktor.
\end{lösung}

\subsection{Aufgabe 1.5 (Funktionen über Formelmengen)}
\subsubsection{Aufgabe 1.5 (a)}
Definieren Sie für eine gegebene aussagenlogische Variable $A$ durch strukturelle Rekursion eine einstellige Funktion $h$ über der Menge der aussagenlogischen Formeln,
die berechnet, wie oft $A$ in einer Formel $F$ vorkommt.

\begin{lösung}
	Sei $A\in\mathcal{R}$ aussagenlogische Variable. Setze
	\begin{align*}
		h_A:\mathcal{L}(\mathcal{R})\to\N_0,\qquad x\mapsto \left\lbrace\begin{array}{cl}
			0, & \falls x\in\mathcal{R}\wedge x\neq A\\
			1, & \falls x\in\mathcal{R}\wedge x=A\\
			h_A(y), & \falls x=\neg y\\
			h_A(y)+h_A(z), & \falls x=(y\circ z)
		\end{array}\right.
	\end{align*}
	
	(unnötig) ausführlicher Zwischenschritt:
	\begin{align*}
		&h_{AR}:\mathcal{R}\to\N_0,\qquad x\mapsto \left\lbrace\begin{array}{cl}
			1, & \falls x= A\\
			0, & \sonst
		\end{array}\right.\\
		&h_{A\neg}:\N\to\N,\qquad n\mapsto n\\
		&h_{A\circ}:\N\times\N\to\N,\qquad(m,n)\mapsto m+n\\
		&\implies h_A:\mathcal{L}(\mathcal{R})\to\N_0,\qquad\varphi\mapsto
		\left\lbrace\begin{array}{cl}
			h_{AR}(p), & \falls \varphi=P\\
			h_{A\neg}(h_A(\varphi_1)), & \falls \varphi=\neg\varphi_1\\
			h_{A\circ}(h_A(\varphi_1),h_A(\varphi_2)), &\falls \varphi=(\varphi_1,\varphi_2)
		\end{array}\right.
	\end{align*}
\end{lösung}

\subsubsection{Aufgabe 1.5 (b)}
Definieren Sie durch strukturelle Rekursion eine Funktion \texttt{laenge}, welche die Anzahl der Zeichen bestimmt, aus denen eine Formel besteht (Leerzeichen nicht mitgezählt). 
Bestimmten Sie sodann \texttt{laenge}$((p \vee (q \wedge \neg p)))$.

\begin{lösung}
	\begin{align*}
		\text{laenge}:\mathcal{L}(\mathcal{R})\to\N_0,\qquad x\mapsto \left\lbrace\begin{array}{cl}
			1, & \falls x\in\mathcal{R}\\
			1+\text{laenge}(y), & \falls x=\neg y\\
			3+\text{laenge}(y)+\text{laenge}(z), & \falls x=(y\circ z)
		\end{array}\right.
	\end{align*}
	Und damit:
	\begin{align*}
		\text{laenge}((p\vee(q\wedge\neg p))) 
		&=3+\text{laenge}(p)+\text{laenge}((q\wedge\neg p))\\
		&=3+1+3+\text{laenge}(q)+\text{laenge}(\neg p)\\
		&=7 + 1 + 1+ \text{laenge}(p)\\
		&=9+1=10
	\end{align*}
\end{lösung}

\subsubsection{Aufgabe 1.5 (c)}
Definieren Sie mit struktureller Rekursion eine Funktion du, die angewandt auf
eine aussagenlogische Formel $F$, folgendes leistet:
\begin{enumerate}[label=(\roman*)]
	\item  Alle aussagenlogischen Variablen $A$ in $F$ werden durch $\neg A$ ersetzt.
	\item Alle $\wedge$ in $F$ werden durch $\vee$ ersetzt.
	\item Alle $\vee$ in $F$ werden durch $\wedge$ ersetzt.
	\item Strukturen der Form $(H \to G)$ in $F$ werden durch $(H \wedge\neg G)$ ersetzt.
	\item Strukturen der Form $(H \leftrightarrow G)$ in $F$ werden durch $((H \wedge\neg G)\vee (G\wedge\neg H))$ ersetzt.
\end{enumerate}

Wenden Sie sodann die Funktion du auf die Formel $(p\to (p\wedge\neg q))$ an.

\begin{lösung}
	\begin{align*}
		&\text{du}:\mathcal{L}(\mathcal{R})\to\mathcal{L}(\mathcal{R})\\
		&x\mapsto \left\lbrace\begin{array}{cl}
			\neg x, & \falls x=p\in\mathcal{R}\\
			\neg\text{du}(G), & \falls x=\neg G\\
			(\text{du}(G)\vee\text{du}(H)), & \falls x=(G\wedge H)\\
			(\text{du}(G)\wedge\text{du}(H)), & \falls x=(G\vee H)\\
			(\text{du}(H)\wedge\neg\text{du}(G)), & \falls x=(H\to G)\\
			((\text{du}(H)\wedge\neg \text{du}(G))\vee(\text{du}(G)\wedge\neg\text{du}(H))), & \falls x=(H\leftrightarrow G)\\
		\end{array}\right.
	\end{align*}

	Ausführlicher:
	\begin{align*}
		&\text{du}_R:\mathcal{R}\to\mathcal{L}(\mathcal{R}), 
			&\varphi&\mapsto\neg \varphi\\
		&\text{du}_{\neg}:\mathcal{L}(\mathcal{R})\to\mathcal{L}(\mathcal{R}),
			&\varphi&\mapsto\neg\varphi\\
		&\text{du}_{\wedge}:\mathcal{L}(\mathcal{R})\times\mathcal{L}(\mathcal{R})\to\mathcal{L}(\mathcal{R}),
			&(\varphi_1,\varphi_2)&\mapsto(\varphi_1\vee\varphi_2)\\
		&\text{du}_{\vee}:\mathcal{L}(\mathcal{R})\times\mathcal{L}(\mathcal{R})\to\mathcal{L}(\mathcal{R}),
			&(\varphi_1,\varphi_2)&\mapsto(\varphi_1\wedge\varphi_2)\\
		&\text{du}_{\to}:\mathcal{L}(\mathcal{R})\times\mathcal{L}(\mathcal{R})\to\mathcal{L}(\mathcal{R}),
			&(\varphi_1,\varphi_2)&\mapsto(\varphi_1\wedge\neg\varphi_2)\\
		&\text{du}_{\leftrightarrow}:\mathcal{L}(\mathcal{R})\times\mathcal{L}(\mathcal{R})\to\mathcal{L}(\mathcal{R}),
			&(\varphi_1,\varphi_2)&\mapsto((\varphi_1\wedge\varphi_2)\vee(\varphi_2\wedge\neg\varphi_1)
	\end{align*}

	Und damit:
	\begin{align*}
		\text{du}((p\to(p\wedge\neg q)))
		&=(\text{du}(p)\wedge\neg\text{du}((p\wedge\neg q)))\\
		&=(\neg p\wedge\neg(\text{du}(p)\vee\text{du}(\neg q)))\\
		&=(\neg p\wedge\neg(\neg p\vee\neg\text{du}(q)))\\
		&=(\neg p\wedge\neg(\neg p\vee\neg\neg q))\\
	\end{align*}
\end{lösung}
