% This work is licensed under the Creative Commons
% Attribution-NonCommercial-ShareAlike 4.0 International License. To view a copy
% of this license, visit http://creativecommons.org/licenses/by-nc-sa/4.0/ or
% send a letter to Creative Commons, PO Box 1866, Mountain View, CA 94042, USA.

\section{Aufgabenblatt 12}
\subsection*{Aufgabe $\ast$)}
Mir ist hierfür kein Algorithmus bekannt, deshalb nutze ich \textit{Methode des scharfen Hinsehens}.
Die dritte Bedingung in $L$ sagt, dass die Wörter in $L$ nicht mit $a$ beginnen dürfen.
Kurz geschrieben (Achtung, keine offizielle Notation!)
\begin{align*}
	L=\Big\lbrace w=u_1 babc u_2,w=u_3 ccc u_4,w\neq a u_5:u_1,\ldots,u_5\in\Sigma^\ast\Big\rbrace
\end{align*}
Somit erhält man den Regulären Ausdruck 
\begin{align*}
	r=(b+c)^\ast\cdot(a+b+c)^\ast\cdot(b\cdot a\cdot b\cdot c+c\cdot c\cdot c)\cdot(a+b+c)^\ast
\end{align*}

\subsection*{Aufgabe $\ast\ast$)}
Hierfür nutzen wir einfach Definition 5.3:

\begin{lösung}
	\underline{Zu (a):}
	\begin{align*}
		L(r_1)
		&=L(((b(b)^\ast)+(bb)^\ast a))\\
		&=L((b(b)^\ast)\cup L((bb)^\ast a)\\
		&=L(b)\cdot L(b^\ast)\cup L((bb)^\ast)\cdot L(a)\\
		&=\lbrace b\rbrace\cdot (L(b))^\ast\cup(L(bb))^\ast\cdot L(a)\\
		&=\lbrace b\rbrace\cdot\lbrace b\rbrace^\ast\cup\lbrace bb\rbrace^\ast\cdot\lbrace a\rbrace\\
		&=\Big\lbrace b^m, b^{2\cdot n}a:m\in\N_{\geq1},n\in\N_{\geq0}\Big\rbrace
	\end{align*}
	
	Letztlich kann man sich die aufwendigen Umformungen sparen: Man ersetzt einfach + durch $\cup$ und, packt die Zeichen in Mengenklammern und fertig
		
	\underline{Zu (b):}
	\begin{align*}
		L(r_2)&=L(((a)^\ast b(a(a)^\ast)b)^\ast b(a+b)^\ast))\\
		&=\lbrace a\rbrace^\ast\cdot\lbrace b\rbrace\cdot(\lbrace a\rbrace\cdot\lbrace a\rbrace^\ast\cdot\lbrace b \rbrace)^\ast\cdot\lbrace b\rbrace\cdot(\underbrace{\lbrace a\rbrace\cup\lbrace b\rbrace}_{=\lbrace a,b\rbrace})^\ast
	\end{align*}
	
	\underline{Zu (c):}
	\begin{align*}
		L(r_3)&=L(((a)^\ast+((((a)^\ast\cdot(b+b\cdot b))(a(a)^\ast(b+b\cdot b))^\ast)(a)^\ast))\\
		&=\lbrace a\rbrace^\ast\cup\Big(\lbrace a\rbrace^\ast\cdot\lbrace b,bb\rbrace\cdot\big(\lbrace a\rbrace\cdot\lbrace a\rbrace^\ast\cdot\lbrace b,bb\rbrace\big)^\ast\cdot\lbrace a\rbrace^\ast\Big)
	\end{align*}
\end{lösung}

\subsection{Aufgabe 1}

\begin{lösung}
	Außer a) sind alle aufgeführten Quadupel Grammatiken, denn $\varepsilon\to...$ ist nicht zulässige nach Definition 6.2.
	Somit sind b),c),d),e),f) Grammatiken vom Typ 0 nach Definiton 6.6.\nl
		
	\underline{Zu b):}
	$G_2$ ist nicht vom Typ 1, da $S\to\varepsilon$ und $S$ auf rechter Seite vorkommt und nicht vom Typ 3, aber vom Typ 2.\nl
	\underline{Zu c):}
	$G_2$ ist offenbar nicht vom Typ 2 oder 3, da $XY\not\in N$.\\
	$G_2$ ist auch nicht vom Typ 1, denn:\\
	\underline{Fall 1:} $X=A\implies u_1=\varepsilon,u_2=Y\implies w=\varepsilon\implies |w|=0<1$\\
	\underline{Fall 2:} $Y=A\implies u_1=X$, was auch nicht möglich ist.\nl
	\underline{Zu d):}
	$G_4$ ist vom Typ 0,1,2,3.\nl
	\underline{Zu e):}
	$G_5$ ist nicht vom Typ $2$ oder Typ 3, aber vom Typ 1.\nl
	\underline{Zu f):}
	$G_6$ ist nicht vom Typ 1, da $S\to\varepsilon$ und $S$ auf rechter Seite vorkommt.\\
	$G_6$ ist auch nicht vom Typ 2 oder 2, da $ZY\not\in N$.
\end{lösung}

\subsection{Aufgabe 2}

\begin{lösung}
	\underline{Zu $G_1$:}	\\
	$G_1$ ist offenbar nicht vom Typ 1, aber vom Typ 2.
	 \begin{align*}
	 	L(G_1)=\big\lbrace a^n b^n:n\in\N_{\geq0}\big\rbrace
	 \end{align*}
	 Da $G_1$ Grammatik vom Typ 2 ist, ist $L(G_1)$ eine $\L_2$-Sprache. 
	 Eine $\L_3$-Sprache kann es nicht sein, da es keinen DEA gibt, der $L(G_1)$ erzeugt.\nl
	 \underline{Zu $G_2$:}\\
	 $G_2$ ist nicht vom Typ 1, aber vom Typ 2 und nicht vom Typ 3. 
	 \begin{align*}
	 	L(G_2)=\big\lbrace a,bb\rbrace^\ast
	 \end{align*}
	 Es gibt einen NEA mit Wortübergängen, der $L(G_2)$ akzeptiert.
	 Deshalb liegt $L(G_2)$ in der Sprachklasse $\L_3$.
	 

\begin{tikzpicture}[shorten >=1pt,node distance=2.7cm,on grid]
  \node[state,initial, accepting]   (q_0)                {$q_0$};
  \path[->] (q_0) edge [loop right] node [right] {bb} ()
                  edge [loop above] node [above] {a} ();
\end{tikzpicture}

	\underline{Zu $G_3$:}\\

	\underline{Zu $G_4$:}
	$G_4$ ist Typ 0, nicht Typ 1,2,3.
\end{lösung} 

\subsection{Aufgabe 3}
Betrachte die Grammatik
\begin{align*}
	G_0&=\Big(\lbrace S,T,U,V,R\rbrace,\lbrace a,b\rbrace,P_0,S\Big)\\
	P_0&=\left\lbrace
		\begin{array}{c}
			 S\to\varepsilon,S\to aSb,S\to T,S\to R,\\
			 T\to bbT, T\to U\\
			 U\to aa U,U\to bbT\\
			 V\to bSa\\
			 R\to\varepsilon\\
			 R\to bSa
		\end{array}\right\rbrace		
\end{align*}
	
\subsubsection{Aufgabe 3 a)}
Geben Sie zu $G_0$ alle nicht-terminierenden Symbole und nicht-erreichbaren Symbole an und geben Sie eine zu $G_0$ äquivalente reduzierte Grammatik $G_1$ an.

\begin{lösung}
	%TODO
\end{lösung}

\subsubsection{Aufgabe 3 b)}
Konstruieren Sie eine Grammatik $G_2$ mit $L(G_2)=L(G_1)\setminus\lbrace\varepsilon\rbrace$, die keine Regeln der Form $A\to\varepsilon$ für $A\in N$ enthält.

\begin{lösung}
	%TODO
\end{lösung}

\subsubsection{Aufgabe 3 c)}
Geben Sie ein zu $G_1$ äquivalente $\varepsilon$-freie Grammatik $G_3$ an.
Erweitern Sie dazu, wenn nötig, die Grammatik $G_2$ um ein neues Startsymbol $S_3$ und entsprechende Regeln.

\begin{lösung}
	%TODO
\end{lösung}

\subsubsection{Aufgabe 3 d)}
Geben Sie eine zu $G_3$ äquivalente Grammatik $G_4$ an, die keine Produktionen der Form $A\to B$ mit Nichtterminalsymbolen $A,B$ enthält.

\begin{lösung}
	%TODO
\end{lösung}

\subsubsection{Aufgabe 3 e)}
Geben Sie eine zu $G_4$ äquivalente Grammatik $G_5$ in Chomsky-Normalform an.

\begin{lösung}
	%TODO
\end{lösung}