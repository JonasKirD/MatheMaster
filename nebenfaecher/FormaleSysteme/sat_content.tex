% This work is licensed under the Creative Commons
% Attribution-NonCommercial-ShareAlike 4.0 International License. To view a copy
% of this license, visit http://creativecommons.org/licenses/by-nc-sa/4.0/ or
% send a letter to Creative Commons, PO Box 1866, Mountain View, CA 94042, USA.

\section{Das DIMACS CNF Dateiformat}
\begin{itemize}
	\item mit Kommentarzeilen beginnen mit ``\texttt{c}'' und dürfen nur am Anfang der Datei stehen
	\item die erste Nicht-Kommentarzeile:\\ \texttt{p cnf <Variablen> <Klauseln>}\\ wobei man die Anzahl der Variablen und Klauseln eintragen muss
	\item jede weitere Zeile beschreibt eine Klausel, bei $n$ Variabeln $\RR=\lbrace p_1,\ldots, p_n\rbrace$ schreibe durch Leerzeichen getrennt die Literale:
	\begin{itemize}
		\item ``\texttt{i}'' für $i\in\lbrace1,\ldots, n\rbrace$
		\item ``\texttt{-i}'' für $\neg p_i$
	\end{itemize}
	\item jede Klauselzeile muss mit ``\texttt{0}'' beendet werden.
	\item Beispiel:
	\begin{align*}
		(((( p_1\vee p_3)\vee\neg p_4)\wedge p_4)\wedge (p_2\vee\neg p_3))\\
		\rightsquigarrow
		\big\langle [p_1, p_3,\neg p_4],[p_4],[p_2,\neg p_3]\big\rangle
	\end{align*}
	\texttt{c Beispiel\\
		p cnf 4 3\\
		1 3 -4 3\\
		1 4 -4 0\\
		4 0\\
		2 -3 0}
	\item Programmaufruf: \texttt{./glucose -model <Dateiname>}, konkret heißt das:\\
	\texttt{glucose\_x64.exe -model sat1.txt}
\end{itemize}

\subsection{Aufgabenstellung}
ungelöstes Sudoku $\to$ Encoder $\to$ SAT-Solver $\to$ Decoder $\to$ gelöstes Sudoku
\begin{itemize}
	\item Encoder konvertiert Sudoku in das DIMACS-Format (Beispiel auf Website vorhanden)
	\item SAT-Solver liefert SAT oder SAT + Model in DIMACS
	\item Decoder soll in vollständig gelöstes Sudoku ausgeben im selben Format wie die Eingabedatei
	\item Nutze Pipes und Standardein- und -ausgabe:\\
		\texttt{\$./encoder.py bsp-input.txt |./glucose -model |./deocoder.py bsp-input.txt}
\end{itemize}
