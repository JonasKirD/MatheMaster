% This work is licensed under the Creative Commons
% Attribution-NonCommercial-ShareAlike 4.0 International License. To view a copy
% of this license, visit http://creativecommons.org/licenses/by-nc-sa/4.0/ or
% send a letter to Creative Commons, PO Box 1866, Mountain View, CA 94042, USA.

\section{Aufgabenblatt 3}
\subsection{Aufgabe 3.1}
\subsubsection{Aufgabe 3.1 (a)}
Die Aussage stimmt, denn:
\begin{align*}
	\F_1\cup\F_2\text{ erfüllbar }
	\overset{\text{Def}}&{\Longleftrightarrow}
	\exists\text{ Modell für }\F_1\cup\F_2\\
	\overset{\text{Def}}&{\Longleftrightarrow}
	\exists\text{ Interpretation }I:\forall F\in\F_1\cup\F_2:I\text{ ist Modell für }F\\
	&\implies
	\exists\text{ Interpretation }I:\forall F\in\F_1:I\text{ ist Modell für }F\text{ und}\\
	&\qquad~~\exists\text{ Interpretation }I:\forall F\in\F_2:I\text{ ist Modell für }F\\
	\overset{\text{Def}}&{\Longleftrightarrow}
	\exists\text{ Modelle für }\F_1,\F_2\\
	\overset{\text{Def}}&{\Longleftrightarrow}
	\F_1,\F_2\text{ sind erfüllbar}
\end{align*}

\subsubsection{Aufgabe 3.1 (b)}
Die Aussage stimmt nicht, Gegenbeispiel:
$\F_1:=\lbrace p\rbrace$ und $\F_2:=\lbrace\neg p\rbrace$ sind erfüllbar, aber $\F_1\cup\F_2=\lbrace p ,\neg p\rbrace$ ist nicht erfüllbar.

\subsubsection{Aufgabe 3.1 (c)}
Nein, denn die Teilmenge $\lbrace p,\neg p\rbrace \subseteq\lbrace p,\neg p\rbrace$ ist nicht erfüllbar, da sich die Elemente $p$ und $\neg p$ dieser Teilmenge sich gegenseitig ausschließen.

\subsection{Aufgabe 3.2}
\subsubsection{Aufgabe 3.2 (a)}
Die Aussage stimmt, denn:
\begin{align*}
	F\text{ unerfüllbar }
	\overset{\text{Def}}&{\Longleftrightarrow}
	\forall\text{ Interpretationen } I:[F]^I=\bot\\
	\overset{\text{}}&{\Longleftrightarrow}
	\forall\text{ Interpretationen } I:\neg^\ast([F]^I)=\neg^\ast(\top)\\
	\overset{\text{}}&{\Longleftrightarrow}
	\forall\text{ Interpretationen } I:[\neg F]^I=\top\\
	\overset{\text{Def}}&{\Longleftrightarrow}
	\forall\text{ Interpretationen } I:I\text{ ist Modell für } \neg F
\end{align*}

\subsubsection{Aufgabe 3.2 (b)}
Die Aussage stimmt nicht, denn:
\begin{align*}
	\F\text{ nicht erfüllbar }
	\overset{\text{Def}}&{\Longleftrightarrow}
	\nexists\text{ Modell für }\F\\
	\overset{\text{Def}}&{\Longleftrightarrow}
	\nexists\text{ Interpretation }I:\forall F\in\F:I\text{ ist Modell für }F\\
	\overset{\text{}}&{\Longleftrightarrow}
	\forall\text{ Interpretation }I:\exists F\in\F:I\text{ ist kein Modell für } F\\
	\overset{\text{}}&{\Longleftrightarrow}
	\forall\text{ Interpretation }I:\exists F\in\F:[F]^I=\bot\\
	\overset{\text{}}&{\Longleftrightarrow}
	\forall\text{ Interpretation }I:\exists F\in\F:[\neg F]^I=\top\\
	\overset{\text{}}&{\Longleftrightarrow}
	\forall\text{ Interpretation }I:\exists F\in\F:I\text{ ist Modell für }\neg F\\
	&\Longleftarrow
	\forall\text{ Interpretation } I,\forall F\in\F:I\text{ ist Modell für }\neg F
\end{align*}
Der letzte Schritt ist aber keine Äquivalenz und somit gilt die Behauptung nicht. 
Gegenbeispiel:\\
$\F:=\lbrace\neg p,p\rbrace$ ist nicht erfüllbar und $\neg p$ bzw. $p$ ist nicht allgemeingültig.

\subsection{Aufgabe 3.3  (Kurze Fragen zu \texorpdfstring{$I\models F$}{I |= F} und \texorpdfstring{$\F\models G$}{F |= G})}
Erfüllungsrelation: $I\models F:\Longleftrightarrow F^I=\top$ "Interpretation $I$ erfüllt Formel $F$"\\
Logische Konsequenzrelation: $\mathcal{G}\models F:\Longleftrightarrow\forall I:(I\models\mathcal{G}\implies I\models F)$

\subsubsection{Aufgabe 3.3 (a)}
\begin{enumerate}[label=(\arabic*)]
	\item Nein, denn: $F$ ist erfüllbar $\Longleftarrow \tilde{I}\models F$, die Umkehrung gilt nicht.
	\item Ja, denn: $F^I=\top\stackrel{\text{Def}}{\Longleftrightarrow} I\models F$
	\item Nein, denn: $\big(\forall\text{ Interpreation }I:F^I=\top\big)\implies I\models F$
	\item Nein, denn: $\Big(\exists\text{ Modell für }F \Big)
		\stackrel{\text{Def}}{\Longleftrightarrow}
		\Big(\exists\text{ Interpretation }\tilde{I}:F^{\tilde{I}}=T\big)\not\Longleftrightarrow I\models F$\\
		(gilt g.d.w. $I=\tilde{I}$)
	\item Nein, denn: $\begin{aligned}
		\lbrace F\rbrace\models F
		\overset{\text{Def}}&{\Longleftrightarrow}
		\big(\forall\text{ Interpretation }I:I\models\lbrace F\rbrace\Rightarrow I\models F\big)\\
		\overset{\text{Def}}&{\Longleftrightarrow}
		\big(\forall\text{ Interpretation }I:I\models F\Rightarrow I\models F\big)\\
		&\Longleftrightarrow\top
	\end{aligned}$\\
	Die Aussage ist also nur äquivalent, g.d.w. $I\models F$ wahr ist.
	\item Ja, denn: $F$ ist unter $I$ wahr $\Longleftrightarrow F^I=\top\stackrel{\text{Def}}{\Longleftrightarrow} I\models F$
\end{enumerate}

\subsubsection{Aufgabe 3.3 (b)}
\begin{enumerate}[label=(\arabic*)]
	\item Ja, denn:
	\begin{align*}
		\Big(\forall I\mit (\forall F\in\F:F^I=\top):G^I=\top\Big)
		&\Longleftrightarrow
		\Big(\forall I:(\forall F\in\F:F^I=\top)\Rightarrow G^I=\top\Big)\\
		\overset{\text{Def}}&{\Longleftrightarrow}
		\Big(\forall I:(\forall F\in\F:I\models F)\Rightarrow I\models G\Big)\\
		\overset{\text{Def}}&{\Longleftrightarrow}
		\Big(\forall I:I\models\F\Rightarrow I\models G\Big)\\
		\overset{\text{Def}}&{\Longleftrightarrow}
		\F\models G
	\end{align*}
	\item Nein, denn: 
	\begin{align*}
		\F,G\text{ haben die gleichen Modelle}
		&\Longleftrightarrow\Big(\forall I:I\models\F\gdw I\models G\Big)\\
		&\Longleftarrow\Big(\forall I:I\models\F\implies I\models G\Big)\\
		\overset{\Def}&{\Longleftrightarrow} \F\models G
	\end{align*}
	\item Nein, denn:
	\begin{align*}
		\Big(\big(\F\text{ erfüllbar }\implies G\text{ erfüllbar }\big)
		\overset{\text{Def}}&{\Longleftrightarrow}
		\Big(\big(\exists I:I\models\F\big)\implies\exists\tilde{I}:\tilde{I}\models G\Big)\\
		&\Longleftarrow
		\Big(\forall I:I\models\F\Rightarrow I\models G\Big)\\
		\overset{\text{Def}}&{\Longleftrightarrow}
		\F\models G
	\end{align*}
	\item Ja, denn:
	\begin{align*}
		\text{Modelle von $\F$ sind Modelle von $G$ }
		\overset{\text{Def}}&{\Longleftrightarrow}
		\big(I\models\F\implies I\models G\big)\\
		\overset{\text{Def}}&{\Longleftrightarrow}
		\F\models G
	\end{align*}
	\item Ja, denn: Betrachte die logische Äquivalenz
		\begin{align}\label{eqLogicIndirect}
			(A\implies B)\Longleftrightarrow(\neg B\implies\neg A)
		\end{align}
		für logische Aussagen $A,B$. Damit gilt:
		\begin{align*}
			&\qquad\Big(\forall I:\big(G^I=\bot\implies\exists F\in\F:F^I=\bot\big)\Big)\\
			\overset{\text{Def}}&{\Longleftrightarrow}
			\Big(\forall I:\big(I\not\models G\implies\exists F\in\F:I\not\models F\big)\Big)\\
			\overset{\text{}}&{\Longleftrightarrow}
			\Big(\forall I:\big(\neg(I\models G)\implies\neg(\forall F\in\F:I\models F)\big)\Big)\\
			\overset{\eqref{eqLogicIndirect}}&{\Longleftrightarrow}
			\Big(\forall I:\big((\forall F\in\F:I\models F)\implies I\models G\big)\Big)\\
			\overset{\text{Def}}&{\Longleftrightarrow}
			\Big(\forall I:\big(I\models \F\implies I\models G\big)\Big)\\
			\overset{\text{Def}}&{\Longleftrightarrow}
			\F\models G
		\end{align*}
		Somit ist (5)$\Longleftrightarrow\F\models G$.
	\item Nein, denn:
	\begin{align*}
		\Big(\forall\text{ Modell $I$ von }G:\exists F\in\F:F^I=\top\Big)
		\overset{}&{\Longrightarrow}
		\Big(\forall I:\big(I\models G\implies \exists F\in\F:I\models F\big)\Big)
	\end{align*}
\end{enumerate}

\subsection{Aufgabe 3.4 (Deduktionstheorem)}
Seien $F,F_1,\ldots F_n\in\mathcal{L}(\mathcal{R}),n\in\N$ aussagenlogische Formeln. 
Dann gilt:
\begin{align*}
	\lbrace F_1,\ldots, F_n\rbrace\models F\Longleftrightarrow\models(((\ldots(F_1\wedge F_2)\ldots)\wedge F_n)\to F)
\end{align*}

\begin{proof}
	\underline{Zeige ``$\implies$'':}\\
	Sei also 
	\begin{align}\label{3.4Hinrichtung}
		\lbrace F_1,\ldots, F_n\rbrace\models F
	\end{align}
	und sei $I$ eine beliebige, aber feste, Interpretation für Formeln aus $\mathcal{L}(\mathcal{R})$. 
	Dann gilt:\nl
	\underline{Fall 1: $I\not\models\lbrace F_1,\ldots,F_n\rbrace$}\\
	Diese Aussage ist äquivalent zu
	\begin{align*}
		I\not\models((\ldots(F_1\wedge F_2)\ldots)\wedge F_n)
	\end{align*}
	(wird hier angenommen, nicht bewiesen). 
	Dann ist dies auch äquivalent zu 
	\begin{align*}
		[((\ldots(F_1\wedge F_2)\ldots)\wedge F_n)]^I=\bot
	\end{align*}
	Aus der Definition von $\to^\ast$,\\
	\begin{tabular}{c|c||c}
		$p$ & $q$ & $p\to^\ast q$\\ \hline
		$\bot$ & $\bot$ & $\top$\\
		$\bot$ & $\top$ & $\top$\\
		$\top$ & $\bot$ & $\bot$\\
		$\top$ & $\top$ & $\top$
	\end{tabular}, 
	folgt damit
	\begin{align*}
		(((\ldots(F_1\wedge F_2)\ldots)\wedge F_n)^I\to^\ast F^I)=\top
	\end{align*}
	und damit ebenso
	\begin{align*}
		I\models(((\ldots(F_1\wedge F_2)\ldots)\wedge F_n)\to F)
	\end{align*}

	\underline{Fall 2: $I\models\lbrace F_1,\ldots,F_n\rbrace$}\\
	Das ist äquivalent zu $I\models((\ldots(F_1\wedge F_2)\ldots)\wedge F_n)$ und damit auch zu $((\ldots (F_1^I\wedge^\ast F_2^I)\ldots)\wedge^\ast F_n^I)=\top$. 
	Nach \eqref{3.4Hinrichtung} gilt auch $F^I=\top$. 
	Aus der Definition von $\to^\ast$ folgt dann 
	\begin{align*}
		(((\ldots(F_1\wedge F_2)\ldots)\wedge F_n)^I\to^\ast F^I)=\top
	\end{align*}
	und damit
	\begin{align*}
		I\models(((\ldots(F_1\wedge F_2)\ldots)\wedge F_n)\to F).
	\end{align*}

	\underline{Zeige ``$\Longleftarrow$'':} Entfällt.
\end{proof}

\subsection{Aufgabe 3.5 (Logische Folgerung und Unerfüllbarkeit)}
\begin{align*}
	F\models G\Longleftrightarrow F\cup\lbrace\neg G\rbrace\text{ unerfüllbar}
\end{align*}

\begin{proof}
	Mit
	\begin{align}\label{3.5}
		(A\implies B)\Longleftrightarrow \neg A\vee B
	\end{align}
	gilt
	\begin{align*}
		\F\models G 
		\overset{\text{Def}}&{\Longleftrightarrow}
		\forall\text{ Interpretation }I:(I\models \F\implies I\models G)\\
		\overset{\eqref{3.5}}&{\Longleftrightarrow}
		\forall\text{ Interpretation }I:
		\neg(I\models \F)\vee I\models G\\
		\overset{\text{}}&{\Longleftrightarrow}
		\forall\text{ Interpretation }I:
		I\not\models \F\vee I\not\models \lbrace\neg G\rbrace\\
		\overset{\text{}}&{\Longleftrightarrow}
		\forall\text{ Interpretation }I:
		I\not\models \F\cup\lbrace\neg G\rbrace\\
		\overset{\text{}}&{\Longleftrightarrow}
		\F\cup\lbrace\neg G\rbrace\text{ ist unerfüllbar}
	\end{align*}
\end{proof}

\subsection{Aufgabe 3.6 (Ex falso quodlibet seguitur)}
\textit{``Aus Falschem folgt beliebiges.''}\nl
Seien $F$ und $G$ aussagenlogische Formeln und sei $\F$ eine Menge aussagenlogischer Formeln. 
Dann gilt:
\begin{align*}
	F,\neg F\in\F\implies\forall G\in\L(\RR): F\models G
\end{align*}

\begin{proof}
	Da $F,\neg F\in\F$ ist $\F$ unerfüllbar. 
	Somit gibt es keine Modell für $\F$. 
	Somit sind alle Modelle von $\F$ (da es keine gibt) auch Modelle von $G$. 
	Damit folgt $F\models G$.
\end{proof}
