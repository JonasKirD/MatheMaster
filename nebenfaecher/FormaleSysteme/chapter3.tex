% This work is licensed under the Creative Commons
% Attribution-NonCommercial-ShareAlike 4.0 International License. To view a copy
% of this license, visit http://creativecommons.org/licenses/by-nc-sa/4.0/ or
% send a letter to Creative Commons, PO Box 1866, Mountain View, CA 94042, USA.

\chapter{Grammatiken, kontextfreie Sprachen und Kellerautomaten}

\section{Die Chomsky-Hierarchie}

\begin{definition}\label{def6.2}
	Eine \textbf{Grammatik} ist von der Form $G(N,\Sigma,P,S)$ wobei
	\begin{itemize}
		\item $N$ und $\Sigma$ endliche disjunkte Alphabete sind.
		\item $\Sigma$ ist die Menge der \textbf{Terminalsymbole} (meist klein geschrieben).
		\item $N$ ist die Menge der \textbf{Nichtterminalsymbole} (meist groß geschrieben).
		\item $S\in N$ ist das \textbf{Startsymbol}
		\item $P\subseteq(N\cup\Sigma)^+\times(N\cup\Sigma)^\ast$ eine endliche Menge von \textbf{Ersetzungsregeln / Produktionen}
	\end{itemize}
	Produktionen $(u,v)\in P$ schreibt man gewöhnlich als $u\to v$.\\
	Beachte: Typ 2 $\not\implies$ Typ 1.
\end{definition}

\begin{definition}\label{def6.4}
	Sei $G=(N,\Sigma,P,S)$ eine Grammatik und seien $x,y\in(N\cup\Sigma)^\ast$ Wörter. Definiere
	\begin{enumerate}[label=\arabic*)]
		\item \textbf{$y$ ist aus $x$ direkt ableitbar}, i.Z.:
		\begin{align*}
			x\vdash_G y:\Longleftrightarrow\exists x_1,x_2\in (N\cup\Sigma)^\ast,\exists u\to v\in P: x=x_1 ux_2\wedge y=x_1 vx_2
		\end{align*}
		\item \textbf{$y$ aus $x$ in $n$ Schritte ableitbar}, i.Z.
		\begin{align*}
			x\vdash_G^n y:\Longleftrightarrow\exists x_0,\ldots,x_n\in (N\cup\Sigma)^\ast:\forall 0\leq i<n:x_0=x\wedge x_n=y\wedge x_i\vdash_G x_{i+1}
		\end{align*}
		\item \textbf{$y$ aus $x$ ableitbar}, i.Z.
		\begin{align*}
			x\vdash_G^\ast y:\Longleftrightarrow\exists n\geq 0:x\vdash_G^n y
		\end{align*}
		\item Die \textbf{durch $G$ erzeugte Sprache} ist
		\begin{align*}
			L(G):=\big\lbrace w\in\Sigma^\ast:S\vdash_G^\ast w\big\rbrace
		\end{align*}
	\end{enumerate}
\end{definition}

\begin{definition}[Chomsky-Hierarchie]\label{def6.6ChomskyHierarchie}\enter
	Sei $G=(N,\Sigma,P,S)$ eine Grammatik.
	\begin{itemize}
		\item Jede Grammatik $G$ heißt Grammatik vom \textbf{Typ 0}.
		\item $G$ heißt Grammatik vom \textbf{Typ 1 (kontextsensitiv} $:\Longleftrightarrow$ jede Produktion in $P$ die Form
		\begin{itemize}
			\item $S\to\varepsilon$ oder
			\item $u_1 Au_2\to u_1 wu_2$ mit $A\in N$, $u_1,u_2,w\in(N\cup\Sigma^\ast)$ und $|w|\geq1$
		\end{itemize}
		hat. Ist $S\to\varepsilon\in P$, so kommt $S$ nicht auf der rechten Seite einer Produktion vor.
		\item $G$ heißt Grammatik vom \textbf{Typ 2 (kontextfrei)} $:\Longleftrightarrow$ jede Regel in $P$ die Form $A\to w$ hat mit $A\in N$, $w\in(\Sigma\cup N)^\ast$.
		\item $G$ heißt Grammatik vom \textbf{Typ 3 (rechtslinear)} $:\Longleftrightarrow$ jede Regel in $P$ die Form $A\to uB$ oder $A\to u$ hat mit $A,B\in N,u\in\Sigma^\ast$.
	\end{itemize}
\end{definition}

\begin{definition}\label{def6.7Sprachklassen}
	Für $i\in\lbrace0,1,2,3\rbrace$ ist die \textbf{Klasse der Typ-$i$-Sprachen} definiert als
	\begin{align*}
		\L_i:=\big\lbrace L(G):G\text{ ist Grammatik vom Typ }i\big\rbrace
	\end{align*}
\end{definition}

\begin{satz}[Hierarchie der Sprachklassen]
	\begin{align*}
		\L_3\varsubsetneqq\L_2\varsubsetneqq\L_2\varsubsetneqq\L_0
	\end{align*}
\end{satz}

\begin{proof}
	Nach Definition der Grammatiktypen \ref{def6.6ChomskyHierarchie} gilt offenbar $\L_3\subseteq \L_2$ und $\L_1\subseteq\L_0$.\nl
	\underline{Zeige $\L_3\varsubsetneqq\L_2$:} 
	$L:=\big\lbrace a^nb^n:n\geq0\big\rbrace\in\L^2$ und da es keinen NEA dafür gibt, nicht in $\L_3$.\nl
	\underline{Zeige $\L_2\subseteq\L_1$:} Korollar \ref{korollar8.10}.\nl
	\underline{Zeige $\L_2\varsubsetneqq\L_1$:} Sehr technisch
\end{proof}

\section{Rechtslineare Grammatiken und reguläre Sprachen}

\begin{satz}\label{satz7.1}
	Die Typ-3-Sprachen sind genau die regulären / erkennbaren Sprachen.%, d.h. $\L_3=\lbrace L:L\text{ regulär}\rbrace$.
\end{satz}

\section{Normalformen kontextfreier Grammatiken}

Zwei Grammatiken heißen \textbf{äquivalent}, wenn sie dieselbe Sprache erzeugen.

\begin{definition}\label{def8.1}
	Sei $G=(N,\Sigma,P,S)$ eine kontextfreie Grammatik (Typ 2).
	\begin{enumerate}[label=\arabic*)]
		\item $A\in N$ heißt \textbf{terminierend}
		\begin{align*}
			:\Longleftrightarrow\exists w\in \Sigma^\ast:A\vdash_G^\ast w
		\end{align*}
		\item $A\in N$ heißt \textbf{erreichbar}
		\begin{align*}
			:\Longleftrightarrow\exists u,v\in(\Sigma\cup N)^\ast:S\vdash_G^\ast uAv
		\end{align*}
		\item $G$ heißt \textbf{reduziert}, falls alle Elemente von $N$ erreichbar und terminierend sind.
	\end{enumerate}
\end{definition}

\begin{satz}\label{satz8.5}
	Das Leerheitsproblem "$L=\emptyset$?" ist für kontextfreie Sprachen entscheidbar:
	\begin{align*}
		L(G)\neq\emptyset\Longrightarrow\exists w\in\Sigma^\ast:S\vdash_G^\ast w
		\Longleftrightarrow S\text{ terminierend}
	\end{align*}
\end{satz}

\begin{lemma}\label{lemma8.6}
	Zu jeder kontextfreien Grammatik $G=(N,\Sigma,P,S)$ kann man effektiv eine äquivalente reduzierte Grammatik $G''$ erzeugen.
\end{lemma}

\begin{proof}
	\textbf{Erster Schritt:} Eliminieren nicht terminierender Symbole:
	\begin{align*}
		G'&:=\big(N',\Sigma,P',S\big)\\
		N'&:=\big\lbrace A\in N:A\text{ terminierend in }G\big\rbrace\qquad L(G)\neq\emptyset\implies S\in N'\\
		P'&:=\big\lbrace A\to w\in P:A\in N',w\in(N'\cup\Sigma)^\ast\big\rbrace
	\end{align*}
	\textbf{Zweiter Schritt:} Eliminieren von unerreichbaren Symbolen:
	\begin{align*}
		G''&:=\big(N'',\Sigma,P'',S\big)\\
		N''&:=\big\lbrace A\in N':A\text{ erreichbar in }G'\big\rbrace\\
		P''&:=\big\lbrace A\to w\in P':A\in N''\big\rbrace
	\end{align*}
	Beachte:
	\begin{align*}
		A\in n''\wedge A\to u_1 Bu_2\in P'\implies B\in N''
	\end{align*}
\end{proof}

\begin{lemma}\label{lemma8.7}
	Zu jeder kontextfreien Grammatik $G$ lässt sich eine "äquivalente" Grammatik $G'$ ohne Regeln der Form $A\to\varepsilon$ konstruieren mit $L(G')=L(G)\setminus\lbrace\varepsilon\rbrace$.
\end{lemma}

\begin{proof}
	\begin{enumerate}
		\item Finde alle $A\in N$ mit $A\vdash_G^\ast\varepsilon$:
		\begin{align*}
			N_1&:=\big\lbrace A\in N:A\to\varepsilon\in P\big\rbrace\\
			N_{i+1}&:=N_i\cup\big\lbrace A\in N:A\to B_1\ldots B_n\in P\text{ für }B_j\in N_j\big\rbrace
		\end{align*}
		Es gibt ein $k\in\N$ mit $N_k=N_{k+1}=\bigcup\limits_{i\geq1}N_i$. Für dieses $k$ gilt:
		\begin{align*}
			A\in N_k\Longleftrightarrow A\vdash_G^\ast\varepsilon
		\end{align*}
		\item Eliminiere in $G$ alle Regeln $A\to\varepsilon$.
		Um dies auszugleichen, nimmt man für alle Regeln
		\begin{align*}
			A\to u_1B_1\ldots u_nB_nu_{n+1}\mit B_i\in N_k\text{ und }u_i\in(\Sigma\cup N\setminus N_i)^\ast
		\end{align*}
		die Regeln
		\begin{align*}
			A\to u_1\beta_1 u_2\ldots u_n\beta_n u_{n+1}\mit\beta\in\lbrace B_i,\varepsilon\rbrace\text{ und }u_1\beta_1 u_2\ldots u_n\beta_n u_{n+1}\neq\varepsilon
		\end{align*}
		hinzu.
	\end{enumerate}
\end{proof}

\begin{definition}\label{def8.8}
	Eine kontextfreie Grammatik heißt $\varepsilon$\textbf{-frei} $:\Longleftrightarrow$
	\begin{enumerate}[label=\arabic*)]
		\item Sie enthält keine Regeln $A\to\varepsilon$ für $A\neq S$.
		\item Ist $S\to\varepsilon$ enthalten, so kommt $S$ nicht auf der rechten Seite einer Regel vor.
	\end{enumerate}
\end{definition}

\begin{satz}\label{satz8.9}
	Zu jeder kontextfreien Grammatik $G$ kann man effektiv eine äquivalente $\varepsilon$-freie Grammatik konstruieren.
\end{satz}

\begin{proof}
	Konstruiere $G'$ wie im Beweis von Lemma \ref{lemma8.7}.\nl
	\underline{Falls $\varepsilon\not\in L(G)$:} Dann ist $G'$ die gesuchte $\varepsilon$-freie Grammatik.\nl
	\underline{Ansonsten:} Erweitere $G'$ um ein neues Startsymbol $S_0$ und die Produktionen $S_0\to S$, $S_0\to\varepsilon$.
\end{proof}

\begin{korollar}\label{korollar8.10}
	$\L_2\subseteq \L_1$
\end{korollar}

\begin{proof}
	Offenbar ist jede $\varepsilon$-freie kontextfrei Grammatik eine Typ 1 Grammatik:
	\begin{itemize}
		\item Produktionen: $u_1 A u_2\to u_1 wu_2\mit |w|\geq 1$
		\item kontextfrei: $u_i=\varepsilon$, $S\to\varepsilon$ und $S$ nicht auf rechter Seite.
	\end{itemize}
\end{proof}

Man kann auch auf Regeln der Form $A\to B$ verzichten:

\begin{satz}\label{satz8.11}
	Zu jeder kontextfreien Grammatik kann man effektiv eine äquivalente kontextfreie Grammatik konstruieren, die keine Regeln der Form $A\to B$ mit $A,B\in N$ enthält.
\end{satz}

\begin{proof}[Beweisskizze]\
	\begin{enumerate}
		\item Bestimme zu jedem $A\in N$ die Menge $N(A):=\big\lbrace B\in N:A\vdash_G^\ast B\big\rbrace$.
		\begin{align*}
			N_0(A)&:=\lbrace A\rbrace\\
			N_{i+1}(A)&:=N_i(A)\cup\Big\lbrace B\in N\mid\exists B'\in N_i(A):B'\to B\in P\Big\rbrace
		\end{align*}
		Es gibt ein $k$ mit $N_k(A)=N_{k+1}(A)$ und für dieses $k$ ist $N(A)=N_k(A)$.
		\item Definiere $\begin{aligned}
			P':=\Big\lbrace A\to w:B\to w\in P,B\in N(A)\und w\not\in N\Big\rbrace
		\end{aligned}$
	\end{enumerate}
\end{proof}

\begin{satz}[Chomsky-Normalform]\label{atz8.12ChomskyNormalform}\enter
	Jede kontextfreie Grammtik lässt sich umformen in eine äquivalente Grammatik in \textbf{Chomsky-Normalform}, d.h. eine Grammatik mit Regeln in folgender Form:
	\begin{itemize}
		\item $A\to a,A\to BC$ mit $A,B,C\in N,a\in\Sigma$
		\item und eventuell $S\to\varepsilon$ wobei $S$ nicht rechts vorkommt.
	\end{itemize}
\end{satz}

\begin{proof}\
	\begin{enumerate}
		\item Konstruiere zu der gegebenen Grammatik eine äquivalente $\varepsilon$-freie ohne Regeln der Form $A\to B$.
		\item Führe für jedes $a\in\Sigma$ ein neues Nichtterminalsymbol $X_a$ und die Produktion $X_a\to a$ ein.
		\item Ersetze in jeder Produktion $A\to w$ mit $w\not\in\Sigma$ alle Terminalsymbole $a$ durch die zugehörigen $X_a$.
		\item Produktionen 
		\begin{align*}
			A\to B_1\ldots B_2\mit n>2
		\end{align*}
		werden ersetzt durch
		\begin{align*}
			A\to B_1C_1,\qquad C_1\to B_2C_2\ldots C_{n-2}\to B_{n-1}B_n
		\end{align*}
		wobei die $C_i$ jeweils neue Symbole sind.
	\end{enumerate}
\end{proof}

\section{Abschlusseigenschaften kontextfreier Sprachen} %mehr als der Satz ist aus dem Abschnitt wirklich nicht interessant

\begin{satz}\label{satz9.1}
	$\L_2$ ist unter Vereinigung, Konkatenation und Kleene-Stern abgeschlossen. 
	Abschluss unter Durchschnitt und Komplement gilt \underline{nicht}.
\end{satz}

\section{Kellerautomaten}
NEAs sind nicht in der Lage, alle kontextfreien Sprachen zu akzeptieren.
Dafür brauchen wir eine Speicherkomponente \textbf{Keller}.

\begin{definition}\label{def10.1Kellerautomat}
	Ein \textbf{Kellerautomat (PDA)} hat die Form $\A=(Q,\Sigma,\Gamma,q_0,Z_0,\Delta,F)$ wobei
	\begin{itemize}
		\item $Q$ eine endliche Menge von \textbf{Zuständen}
		\item $\Sigma$ das \textbf{Eingabealphabet}
		\item $\Gamma$ das \textbf{Kelleralphabet}
		\item $q_0\in Q$ der \textbf{Anfangszustand}
		\item $Z_0\in\Gamma$ das \textbf{Kellerstartsymbol}
		\item $\Delta\subseteq Q\times\big(\Sigma\cup\lbrace\varepsilon\rbrace\big)\times\Gamma\times\Gamma^\ast\times Q$ eine endliche \textbf{Übergangsrelation}
		\item $F\subseteq Q$ eine Menge von \textbf{Endzuständen}
	\end{itemize}
\end{definition}

Bedeutung der Übergangsrelation $(q,a,Z,\gamma,q')$:
Im Zustand $q$ mit aktuellem Eingabesymbol $a$ und oberstem Kellersymbol $Z$ darf der Automat $Z$ durch $\gamma$ ersetzen und in den Zustand $q'$ und zum nächsten Eingabesymbol übergehen.

\begin{definition}\label{def10.2}
	Eine \textbf{Konfiguration} von $\A$ hat die Form
	\begin{align*}
		\K=(q,w,\gamma)\in Q\times\Sigma^\ast\times\Gamma^\ast
	\end{align*}
	wobei
	\begin{itemize}
		\item $q\in Q$ der aktuelle Zustand ist
		\item $w\in\Sigma^\ast$ der \textbf{noch zu lesende Rest} der Eingabe ist (Lesekopf steht auf dem ersten Symbol von $w$
		\item $\gamma\in\Gamma^\ast$ der \textbf{Kellerinhalt} ist (Schreiblesekopf steht auf dem ersten Symbol von $\gamma$)
	\end{itemize}
	Die Übergangsrelation ermöglicht die folgenden \textbf{Konfigurationsübergänge}:
	\begin{itemize}
		\item $(q,aw,Z\gamma)\vdash(q',w,\beta\gamma)$ falls $(q,a,Z,\beta,q')\in\Delta$
		\item $(q,w,Z\gamma)~\vdash(q',w,\beta\gamma)$ falls $(q,\varepsilon,Z,\beta,q')\in\Delta$
		\item $\begin{aligned}
			\K\vdash^\ast\K':\Longleftrightarrow\exists n\in\N_{\geq0}:\exists\K_0,\ldots,\K_n:\K=\K_0\vdash\K_1\vdash\ldots\vdash\K_n=\K'
		\end{aligned}$
	\end{itemize}
	Der Automat $\A$ \textbf{akzeptiert} das Wort $w\in\Sigma^\ast$
	\begin{align*}
		:\Longleftrightarrow\exists q\in F,\gamma\in\Gamma^\ast:\big(q_0,w,Z_0\big)\vdash^\ast(q,\varepsilon,\gamma)
	\end{align*}
	(also Eingabewort ganz gelesen und Endzustand erreicht).
	Die von $\A$ \textbf{akzeptierte Sprache} ist
	\begin{align*}
		L(\A):=\big\lbrace w\in\Sigma^\ast:\A\text{ akzeptiert }w\big\rbrace
	\end{align*}
\end{definition}

\begin{definition}\label{def10.5}
	Ein PDA $\A=(Q,\Sigma,\Gamma,q_0,Z_0,\Delta)$ heißt \textbf{deterministisch} $:\Longleftrightarrow$
	\begin{enumerate}
		\item $\forall q\in Q,\forall a\in\Sigma\cup\lbrace\varepsilon,\forall Z\in\Gamma$ existiert \underline{höchstens ein} Übergang der Form\\ $(q,a,Z,\ldots,\ldots)\in\Delta$.
		\item Existiert ein Tupel $(q,\varepsilon,\Sigma,\ldots,\ldots)\in\Delta$, so existiert \underline{kein} Tupel der Form\\ $(q,a,Z,\ldots,\ldots)\in\Delta$ mit $a\in\Sigma$.
	\end{enumerate}
\end{definition}

\begin{satz}\label{satz10.8}
	Für eine formale Sprache $L$ sind äquivalent:
	\begin{enumerate}
		\item $L=L(G)$ für eine kontextfreie Grammatik $G$.
		\item $L=N(\A)$ für einen PDA $\A$.
	\end{enumerate}
\end{satz}

\begin{korollar}\label{korollar10.9}
	Sei $L\subseteq\Sigma^\ast$ kontextfrei und $R\subseteq\Sigma^\ast$ regulär.
	Dann ist $L\cap R$ kontextfrei.
\end{korollar}

