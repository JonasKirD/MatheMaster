% This work is licensed under the Creative Commons
% Attribution-NonCommercial-ShareAlike 4.0 International License. To view a copy
% of this license, visit http://creativecommons.org/licenses/by-nc-sa/4.0/ or
% send a letter to Creative Commons, PO Box 1866, Mountain View, CA 94042, USA.

\section{Aufgabenblatt 5}
\subsection{Aufgabe 5.1 (Beispiele zur NNF und KF Transformation}
\textbf{Negationsnormalform:} ``Negation immer nur vor Variablen''\\
\begin{align}
	&\frac{\neg\neg H}{H}\label{Neg1}\\
	&\frac{\neg(G_1\wedge G_2)}{(\neg G_1\vee G_2)}\label{Neg2}\\
	&\frac{\neg(G_1\vee G_2)}{(\neg G_1\wedge G_2)}\label{Neg3}
\end{align}
Die Symbole $\to, \leftrightarrow$ müssen zuerst umgeschrieben werden.\\

\textbf{Klauselform / konjunktive Normalform /CNF:}\\
\begin{align}
	&\frac{\neg\neg D}{D}\label{KNF1}\tag{KNF1}\\
	&\frac{(D_1\wedge D_2)}{D_1\mid D_2}\label{KNF2}\tag{KNF2}\\
	&\frac{\neg(D_1\wedge D_2)}{\neg D_1,\neg D_2}\label{KNF3}\tag{KNF3}\\
	&\frac{(D_1\vee D_2)}{D_1, D_2}\label{KNF4}\tag{KNF4}\\
	&\frac{\neg(D_1\vee D_2)}{\neg D_1\mid\neg D_2}\label{KNF5}\tag{KNF5}\\
\end{align}

\subsubsection{Aufgabe 5.1 (a)}
Negationsnormalform:
\begin{align*}
	(\underline{\neg (( p\wedge q)\vee p)}\vee p)
	\overset{\ref{Neg3}}&{\equiv}
	((\underline{\neg(p\wedge q)}\wedge\neg p)\vee q)\\
	\overset{\ref{Neg2}}&{\equiv}
	(((\neg p\vee \neg q)\wedge\neg p)\vee p)
\end{align*}
Disjunktive Normalform:
\begin{align*}
	\langle[(\neg (( p\wedge q)\vee p)\underline{\vee} p)]\rangle
	\overset{\ref{KNF4}}&{\equiv}
	\langle[(\underline{\neg} (( p\wedge q)\vee p), p)]\rangle\\
	\overset{\ref{KNF5}}&{\equiv}
	\langle[(\neg (( p\wedge q)\vee p)],[\neg p,p)]\rangle\\
	\overset{\ref{KNF3}}&{\equiv}
	\langle[(\neg p,\neg, p)],[\neg p,p)]\rangle\\
\end{align*}
Dies entspricht der Formel
\begin{align*}
	(((\neg p\vee\neg q)\vee p)\wedge(\neg p\vee p)))
\end{align*}

\subsubsection{Aufgabe 5.1 (b)}
Negationsnormalform:
\begin{align*}
	&~~~~\underline{\neg}(((p\vee q)\wedge(q\wedge r))\vee\neg((r\wedge q)\wedge(q\vee p)))\\
	\overset{\ref{Neg3}}&{\equiv}
	(\underline{\neg}((p\vee q)\wedge(q\wedge r))\wedge\neg\neg((r\wedge q)\wedge(q\vee p)))\\
	\overset{\ref{Neg2}}&{\equiv}
	((\underline{\neg}(p\vee q)\vee\underline{\neg}(q\wedge r))\wedge\neg\neg((r\wedge q)\wedge(q\vee p)))\\
	\overset{\ref{Neg2},\ref{Neg3}}&{\equiv}
	(((\neg p\wedge\neg q)\vee(\neg q\vee\neg r))\wedge\underline{\neg\neg}((r\wedge q)\wedge(q\vee p)))\\
	\overset{\ref{Neg1}}&{\equiv}
	(((\neg p\wedge\neg q)\vee(\neg q\vee\neg r))\wedge((r\wedge q)\wedge(q\vee p)))
\end{align*}

Disjunktive Normalform:
\begin{align*}
	&~~~~\langle[\underline{\neg}(((p\vee q)\wedge(q\wedge r))\underline{\vee}\neg((r\wedge q)\wedge(q\vee p)))]\rangle\\
	\overset{\ref{KNF5}}&{\equiv}
	\langle[\neg((p\vee q)\wedge(q\wedge r))],[\underline{\neg\neg}((r\wedge q)\wedge(q\vee p))]\rangle\\
	\overset{\ref{KNF1}}&{\equiv}
	\langle[\underline{\neg}((p\vee q)\underline{\wedge}(q\wedge r))],[((r\wedge q)\wedge(q\vee p))]\rangle\\
	\overset{\ref{KNF3}}&{\equiv}
	\langle[\neg(p\vee q),\underline{\neg}(q\wedge r)],[((r\wedge q)\wedge(q\vee p))]\rangle\\
	\overset{\ref{KNF3}}&{\equiv}
	\langle[\neg(p\vee q),\neg q, \neg r)],[((r\wedge q)\underline{\wedge}(q\vee p))]\rangle\\
	\overset{\ref{KNF2}}&{\equiv}
	\langle[\neg(p\vee q),\neg q, \neg r)],[((r\wedge q)],[(q\vee p)]\rangle\\
	\overset{\ref{KNF5},\ref{KNF2},\ref{KNF4}}&{\equiv}
	\langle[\neg p,\neg q,\neg r],[\neg q,,\neg q, \neg r],[r],[q],[q,p]\rangle\\
\end{align*}

\subsection{Aufgabe 5.2 ((Un-)erfüllbarkeit von Klausel(-mengen))}
Sei $F=\langle C_1,\ldots,C_n\rangle\mit n\geq1$ eine aussagenlogische Formel in Klauselform und sei $C=[L_1,\ldots,L_m]\mit m\geq0$ eine Klausel.

\subsubsection{Aufgabe 5.2 (a)}
Wenn eine in $F$ vorkommende Klausel unerfüllbar ist, dann ist $F$ unerfüllbar.

\begin{proof}
	Sei o.B.d.A. $C_1$ unerfüllbar (sonst Umnummerieren) und $I$ eine beliebige Interpretation. 
	Dann gilt $C^I=\bot$ und damit
	\begin{align*}
		F^I
		\overset{\text{Def}}&=
		\langle C_1,\ldots,C_n\rangle^I\\
		\overset{\text{Def}}&=
		\left(\bigwedge\limits_{i=1}^n C_i\right)^I\\
		\overset{\text{Def}}&=
		\stackrel{\ast}{\bigwedge\limits_{i=1}^n} (C_i)^I\\
		&=(C_1)^I\wedge \stackrel{n}{\bigwedge\limits_{i=2}^\ast} (C_i)^I\\
		&=\bot\wedge \stackrel{n}{\bigwedge\limits_{i=2}^\ast} (C_i)^I\\
		&=\bot
	\end{align*}
	Also ist $F$ unerfüllbar.
\end{proof}

\begin{proof}[Beweis des Tutors]
	\begin{align*}
		&\text{ eine in $F$ vorkommende Klausel ist unerfüllbar}\\
		&\Longleftrightarrow\text{ es gibt $i\in\lbrace1,\ldots,n\rbrace$ sodass $C_i$ ist unerfüllbar}\\
		&\Longleftrightarrow\text{ es gibt $i\in\lbrace1,\ldots,n\rbrace$ sodass für alle Interpretationen $I$ gilt: }I\not\models C_i\\
		&\implies\text{ für alle Interp. $I$ gibt es $i\in\lbrace1,\ldots,n\rbrace$ sodass gilt: }I\not\models C_i\\
		&\Longleftrightarrow F\text{ ist unerfüllbar}
	\end{align*}
\end{proof}

\subsubsection{Aufgabe 5.2 (b)}
$F$ kann unerfüllbar sein obwohl jede in $F$ vorkommende Klausel $C_i,i=1,\ldots,n$ erfüllbar ist.

\begin{proof}
	Betrachte
	\begin{align*}
		F:=\langle C_1,C_2\rangle=\langle[p],[\neg p]\rangle=p\wedge\neg p
	\end{align*}
	wobei $p\in\RR$ ein Atom ist. 
	Hier sind die (alle) Klauseln $C_1=p$ und $C_2=\neg p$  jeweils erfüllbar, aber $F$ ist nicht erfüllbar.
\end{proof}

\subsubsection{Aufgabe 5.2 (c)}
$C$ ist erfüllbar gdw. es ein Literal in $C$ gibt.

\begin{proof}
	\underline{Zeige ``$\implies$'':}\\
	Sei $C$ erfüllbar. 
	Dann ist $C\neq[]$ per Definition der leeren Klausel 
	(folgt aus Kontraposition: Es gibt kein Literal in $C\implies C=[]\implies C$ ist unerfüllbar). 	
	Folglich ist $m>0$, d. h. es existiert mindestens ein Literal $L_i$ in $C$.\nl
	\underline{Zeige ``$\Longleftarrow$'':}\\
	Angenommen es gibt ein Literal in $C$, d.h. $m>0$.
	Induktion über $m$:\\
	Induktionsanfang $m=1$: $C=[L_1]=L_1$ ist erfüllbar, da offenbar jedes Literal erfüllbar ist.\\
	Induktionsvoraussetzung: Sei $C_m=[L_1,\ldots,L_m]$ für beliebiges aber festes $m\in\N$ erfüllbar.
	Induktionsschritt: 
	\begin{align*}
		C_{m+1}
		&=[L_1,\ldots,L_m,L_{m+1}]\\
		&=[L_1,\ldots,L_m]\vee L_{m+1}\\
		&= C_{m}\vee L_{m+1}
	\end{align*}
	Da nach IV $C_m$ erfüllbar ist, gibt es eine Interpretation $\tilde{I}$, die $C_m$ erfüllt. 
	Da $L_{m+1}$ ein Literal ist, gibt es eine Interpretation $\hat{I}$, die $L_{m+1}$ erfüllt. 
	Somit gibt es auch eine Interpretation $I$, die 
	\begin{align*}
		(C_{m+1})^I=(C_{m}\vee L_{m+1})^I=(C_{m})^I\vee^\ast (L_{m+1})^I
	\end{align*}
	erfüllt. Folglich ist $C$ erfüllbar.\nl
	\underline{Zeige ``$\Longleftarrow$'' (Tutor-Version):}\\
	Sei nun $C=[L_1,\ldots,L_m]$ mit $m>0$. 
	(jedes $L_i$ steht dabei für eine aussagenlogische Variable oder die Negation einer Variablen.)
	Da $C$ eine verallgemeinerte Disjunktion (ODER-Verknüpfung) von Literalen ist, genügt es das Literal $L_m$ zu erfüllen. 
	Definiere also $I$ mit
	\begin{align*}
		p^I:=\left\lbrace\begin{array}{cl}
			\top,&\falls L_m=p\\		
			\bot,&\sonst
		\end{array}\right.\qquad\forall p\in\RR
	\end{align*}
	Dann gilt: $L^I=\top$. Damit folgt $C^I=\top$ und somit ist gezeigt, dass
\end{proof}

\subsection{Aufgabe 5.3 (Beweis von Lemma 3.29)}
Lemma 3.29 wird benötigt um zeigen, dass der Algorithmus, welcher Formeln in CNF umformt, korrekt ist.

\begin{proof}
	Sei $F$ eine verallgemeinerte Konjunktion von verallgemeinerten Disjunktionen
	%, d.h. $F$ ist eine verallgemeinerte Konjunktion von Klauseln.
	Die Umformungsregeln \eqref{KNF1}, \eqref{KNF3} und \eqref{KNF4} und ersetzen / modifizieren eine einzelne verallgemeinerte Disjunktion, liefern aber auch in der "Ausgabe" eine verallgemeinerte Disjunktion. 
	Die Regeln \eqref{KNF2} und \eqref{KNF5} ersetzen eine verallgemeinerte Disjunktion durch zwei verallgemeinerte Disjunktionen.\\
	Die semantische Äquivalenz von Regel \eqref{KNF1} folgt aus $\neg^\ast\circ\neg^\ast=\id$, von den Regeln \eqref{KNF3}, \eqref{KNF5} aus den De Morganschen  Regeln, sowie von den Regeln \eqref{KNF2} und \eqref{KNF4} aus 
	\begin{align*}
		(\varphi\wedge\psi)\equiv\big\langle[\varphi],[\psi]\big\rangle
		\qquad\text{bzw.}\qquad
		(\varphi\vee\psi)\equiv\big\langle[\varphi,\psi]\big\rangle
	\end{align*}
\end{proof}

\subsection{Aufgabe 5.4 (Terminierung der NNF Transformation}
Betrachte eine \textbf{Rangfunktion}
\begin{align*}
	r:\L(\RR)\to\N,\qquad r(F):=\left\lbrace\begin{array}{cl}
		1, &\falls F\in\RR\\
		2\cdot r(G)+1, &\falls F=\neg G\\
		r(G_1)+r(G_2)+2, &\falls F=(G_1\circ G_2)
	\end{array}\right.
\end{align*}

\subsubsection{Aufgabe 5.4 (a)}
Sei $F\in\L(\RR),\pi\in\P_F,F\lceil\pi\rceil=G$ und $H\in\L(\RR)$ mit $r(H)<r(G)$. 
Dann gilt:
\begin{align*}
	f\big(F\lceil\pi\mapsto H\rceil\big)<r(F)
\end{align*}

\begin{proof}
	Sei $F\in\L(\RR)$ beliebig.
	Beweis durch strukturelle Induktion über Position $\pi$:\\
	\ul{Induktionsanfang:} $\pi=\Lambda$
	\begin{align*}
		F&=F\lceil\pi\rceil=G\\
		F\lceil\pi\mapsto F\rceil]&=F\\
		r(F\lceil\pi\mapsto H\rceil)&=r(H)\stackrel{\text{Vor}}{<}r(G)=r(F)
	\end{align*}
	\ul{Induktionshypothese:} Sei $\pi$ eine Position, sodass für alle $F,G,H\in\L(\R)$ gilt:
	\begin{align*}
		\pi\in\P_F\text{ und }F\lceil\pi\rceil=G\text{ und } r(H)<r(G)\implies r\big(F\lceil\pi\mapsto H\rceil\big)<r(F)
	\end{align*}
	\ul{Induktionsbhauptung:} Die Aussage gilt dann auch für die Position $1\pi$ und $2\pi$.\nl
	\ul{Induktionsschritt:} Falls $1\pi\in\P_F$, dann gibt es zwei Möglichkeiten für die Struktur von $F$:
	\begin{enumerate}
		\item Falls $F=\neg F_1$:
			\begin{align*}
				r\big(F\lceil1\pi\mapsto H\rceil\big)
				&=r\big((\neg F_1)\lceil1\pi\mapsto H\rceil\big)\\
				&=r\big(\neg(F_1\lceil\pi\mapsto H\rceil)\big)\\
				&=2\cdot r\big(F_1\lceil\pi\mapsto H\rceil\big)+1\\
				\overset{\text{IH}}&{<}
				2\cdot r\big(F_1\big)+1\\
				&=r(\neg F_1)\\
				&=r(F)
			\end{align*}
		\item Falls $F=(F_1\circ F_2)$:
			\begin{align*}
				r\big(F\lceil1\pi\mapsto H\rceil\big)
				&=r\big((F_1\circ F_2)\lceil1\pi\mapsto H\rceil\big)\\
				&=r\big(F_1\lceil\pi\mapsto H\rceil\circ F_2)\big)\\
				&=r\big(F_1\lceil\pi\mapsto H\rceil\big)+r(F_2)+2\\
				\overset{\text{IH}}&{<}
				r\big(F_1\big)+r(F_2)+2\\
				&=r(\neg F_1\circ F_2)\\
				&=r(F)
			\end{align*}
	\end{enumerate}
	Analog zum zweiten Fall kann auch für $2\pi\in\P_F$ argumentiert werden.
\end{proof}

\subsubsection{Aufgabe 5.4 (b)}
Der NNF Transformationsalgorithmus terminiert.

\begin{proof}
	Mit jeder Regelanwendung wird der Rang der abgeleiteten Formel \underline{echt} kleiner. 
	Da die natürlichen Zahlen nach unten (bzgl. $<$) beschränkt sind, 
	kann ab einem gewissen Punkt keine Regel mehr angewendet werden, welche den Rang weiter verringern würde.
\end{proof}

\subsection{Aufgabe 5.5 (Dualisimus zwischen KF und DKF)}
Sei $F$ eine Formel in der nur die Junktoren $\vee,\wedge,\neg$ vorkommen.

\subsubsection{Aufgabe 5.5 (a)}
\begin{align*}
	\bar{\cdot}&:\L(\RR)\to\L(\RR) &F\mapsto\overline{F}:=\left\lbrace\begin{array}{cl}
		\neg A, &\falls A\in\RR\\
		\neg\overline{G}, &\falls F=\neg G\\
		(\overline{G_1}\circ\overline{G_2}), &\falls F=(G_1\circ G_2)
	\end{array}\right.\\
	\cdot^d&:\L(\RR)\to\L(\RR) &F\mapsto F^d:=\left\lbrace\begin{array}{cl}
		A, &\falls A\in\RR\\
		\neg G^d, &\falls F=\neg G\\
		\big((G_1)^d\vee(G_2)^d\big), &\falls F=(G_1\wedge G_2)\\
		\big((G_1)^d\wedge(G_2)^d\big), &\falls F=(G_1\vee G_2)
	\end{array}\right.\\
\end{align*}

\subsubsection{Aufgabe 5.5 (b)}
Es gilt $\neg F\equiv(\overline{F})^d$ für alle $F\in\L(\RR)$.

\begin{proof}
	Beweis durch strukturelle Induktion über $F\in\L(\RR)$:\\
	Induktionsanfang: Sei $F=p\in\RR$. Dann gilt
	\begin{align*}
		\neg F=\neg p=\neg p^d=(\neg p)^d=(\overline{p})^d=(\overline{F})^d
	\end{align*}
	Induktionsvoraussetzung: Seien $F_1,F_2\in\L(\RR)$ beliebige aber feste Formel mit den Eigenschaften
	\begin{align*}
		\neg F_1\equiv (\overline{F_1})^d
		\qquad\text{und}\qquad
		\neg F_2\equiv (\overline{F_2})^d
	\end{align*}
	Induktionsschritt:\\
	Sei $F=\neg F_1$. Dann gilt:
	\begin{align*}
		\neg F
		&=\neg\neg F_1\\
		\overset{\text{IV}}&{\equiv}
		\neg(\overline{F_1})^d\\
		&=(\neg\overline{F_1})^d\\
		&=(\overline{\neg F_1})^d\\
		&=(\overline{F})^d
	\end{align*}
	Sei $F=(F_1\wedge F_2)$. Dann gilt:
	\begin{align*}
		\neg F
		&=\neg(F_1\wedge F_2)\\
		\overset{\text{DM}}&{\equiv}
		(\neg F_1\vee\neg F_2)\\
		\overset{\text{IV}}&{\equiv}
		\big((\overline{F_1})^d\vee\overline{F_2})^d\big)\\
		&=\big((\overline{F_1}\wedge\overline{F_2})\big)^d\\
		&=(\overline{(F_1\wedge F_2)})^d\\
		&=(\overline{F})^d
	\end{align*}
	Sei $F=(F_1\vee F_2)$. Dann gilt:
	\begin{align*}
		\neg F
		&=\neg(F_1\vee F_2)\\
		\overset{\text{DM}}&{\equiv}
		(\neg F_1\wedge\neg F_2)\\
		\overset{\text{IV}}&{\equiv}
		\big((\overline{F_1})^d\wedge\overline{F_2})^d\big)\\
		&=\big((\overline{F_1}\vee\overline{F_2})\big)^d\\
		&=(\overline{(F_1\vee F_2)})^d\\
		&=(\overline{F})^d
	\end{align*}
\end{proof}

\subsubsection{Aufgabe 5.5 (c)}
Ersetze $\langle,\rangle$ durch $[,]$ und umgekehrt, negiere alle Atome alle aussagenlogischen Variablen und entferne Doppelnegation (zum Schluss).

\subsubsection{Aufgabe 5.5 (d)}
%TODO

\subsection{Sudoku $n^2\times n^2$}
Problem: Existiert eine Zuweisung, welche jeder Zelle einen Wert von 1 bis $n^2$ zuweist so, dass in jeder Zeile, in jeder  Spalte und in jedem Block jede Zahl aus $\lbrace1,2,\ldots,n^2\rbrace$ genau einmal vorkommt?

\begin{aufgabe}
	Finde eiene Formel (in CNF) für ein gegebenes Sudoku, sodass diese Formel genau dann erfüllbar ist, wenn das Soduku lösbar ist. 
	Aus dem (erfüllenden) Modell soll dann eine konkrete Lösung abgeleitet werden.\\
	Variablen: $R=\big\lbrace s_{x,y,z}\in\mathbb{B}^3\mid x,y,z\in\lbrace 1,\ldots,n\rbrace\big\rbrace$.
	Dabei soll $s_{x,y,z}$ auf wahr abgebildet wird g.d.w. in Zeile $x$, Spalte $y$ die Zahl $z$ steht.\\
	Formeln (für $n=3$):
	\begin{itemize}
		\item ``Jede Zelle bekommt (mindestens) einen Wert zugewiesen.'' (Überflüssig!)
		\begin{align*}
			\bigwedge\limits_{x=1}^{n^2}\bigwedge\limits_{y=1}^{n^2}\bigvee\limits_{z=1}^{n^2} s_{x,y,z}
		\end{align*}
		\item ``Jede Zelle bekommt höchstens einen Wert zugewiesen.''\\
		Meine Lösung (vielleicht richtig:)
		\begin{align*}
			\bigwedge\limits_{x=1}^{3^2}\bigwedge\limits_{y=1}^{3^2}\bigvee\limits_{z=1}^{3^2}\bigwedge\limits_{\hat{z}\in\lbrace1,2,\ldots,3^2\rbrace\setminus\lbrace z\rbrace} \neg s_{x,y,\hat{z}}
		\end{align*}
		Tutor-Lösung:
		\begin{align*}
			\bigwedge\limits_{x=1}^{n^2}
			\bigwedge\limits_{y=1}^{n^2}
			\bigwedge\limits_{z=1}^{n^2-1}
			\bigwedge\limits_{i=z+1}^{n^2}
			(\neg s_{x,y,z}\vee\neg s_{x,y,i})
		\end{align*}
		\item ``Jede Spalte enthält jede Zahl (mindestens) einmal.''
		\begin{align*}
			\bigwedge\limits_{y=1}^{n^2}\bigwedge\limits_{z=1}^{n^2}\bigvee\limits_{x=1}^{n^2} s_{x,y,z}
		\end{align*}
		\item ``Jede Zeile enthält jede Zahl (mindestens) einmal.''
		\begin{align*}
			\bigwedge\limits_{x=1}^{n^2}\bigwedge\limits_{z=1}^{n^2}\bigvee\limits_{y=1}^{n^2} s_{x,y,z}
		\end{align*}
		\item ``Jeder Block enthält jede Zahl (mindestens) einmal.''\\
		Die Menge aller Blöcke ist
		\begin{align*}
			B:=\left\lbrace 3\cdot(i,j)+\lbrace1,2,\ldots,n\rbrace^2\right\rbrace_{i,j=0}^{n-1}.
		\end{align*}
		Somit (meine Lösung, k. A: ob es stimmt):
		\begin{align*}
			\bigwedge\limits_{x\in B}\bigwedge\limits_{z=1}^{3^2}\bigvee\limits_{y=1}^{3^2} s_{x,y,z}
		\end{align*}
		Richtige Lösung anderer Gruppe:
		\begin{align*}
			\bigwedge\limits_{n_x=1}^n\bigwedge\limits_{n_y=1}^n\bigwedge\limits_{z=1}^{n^2}\bigvee\limits_{x=n_x\cdot n}^{(n_x\cdot(n+1))}\bigvee\limits_{y=n_y\cdot n}^{(n_y\cdot(n+1))-1} s_{x,y,z}
		\end{align*}
		Tutor-Lösung:
		\begin{align*}
			\bigwedge\limits_{z=1}^{n^2}\bigwedge\limits_{i=0}^{n-1}\bigwedge\limits_{j=1}^{n-1}\bigvee\limits_{x=1}^{n}\bigvee\limits_{y=1}^n s_{3\cdot i+x,3\cdot j+y,z}
		\end{align*}
	\end{itemize}
\end{aufgabe}

