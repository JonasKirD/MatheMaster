% !TEX root = THEOLOG.tex
% This work is licensed under the Creative Commons
% Attribution-NonCommercial-ShareAlike 4.0 International License. To view a copy
% of this license, visit http://creativecommons.org/licenses/by-nc-sa/4.0/ or
% send a letter to Creative Commons, PO Box 1866, Mountain View, CA 94042, USA.

\chapter{Wiederholung Aussagenlogik}

Seien $\K$ eine endliche Formelmenge und $F$ eine logische Aussage.
Wir wollen $\K\models F$ beweisen.
Wir nutzen dafür das Resolutionsverfahren.
Mit
\begin{align}\label{eq:chapter0Res}
	\neg\left(\bigwedge\limits_{K\in\K} K\to F\right)
	\equiv
	\neg\klammern{\neg\bigwedge\limits_{K\in\K} K\OR F}
	\overset{\text{De Morgan}}&{\equiv}
	\klammern{\bigwedge\limits_{K\in\K}K\wedge\neg F}
	\equiv K\cup\set{\neg F}
\end{align}
folgt
\begin{align*}
	\K\models F
	&\iff
	\klammern{\bigwedge\limits_{K\in\K} K\to F}\text{ ist allgemeingültig}\\
	&\iff
	\neg\klammern{\bigwedge\limits_{K\in\K} K\to F}\text{ ist unerfüllbar}\\
	\overset{\eqref{eq:chapter0Res}}&{\iff}
	K\cup\set{\neg F}\text{ ist unerfüllbar}.
\end{align*}
Nun können wir das Resolutionsverfahren anwenden.

\begin{erinnerung}
	Sei $C_1$ eine Klausel, in der das Atom $A$ vorkommt und $C_2$ eine Klausel, in der das negierte Atom $\neg\A$ vorkommt. 
	Sei $C$ das Resultat von:
	\begin{enumerate}
		\item Entferne alle Elemente der Form $A$ aus $C_1$.
		\item Entferne alle Elemente der Form $\neg A$ aus $C_2$.
		\item Verknüpfe die so erhaltenen Klauseln disjunktiv.
	\end{enumerate}
	Dies ist die \textbf{Resolutionsregel} und $C$ die \textbf{Resolvente} von $C_1$ und $C_2$ bzgl. $A$. 
	Kurzform zum Merken:
	\begin{align*}
		C_1=\big[ A,B\big],~
		C_2=\big[ \neg A,C\big]\implies C=[ B,C]
	\end{align*}
\end{erinnerung}

\section{Section}

