% This work is licensed under the Creative Commons
% Attribution-NonCommercial-ShareAlike 4.0 International License. To view a copy
% of this license, visit http://creativecommons.org/licenses/by-nc-sa/4.0/ or
% send a letter to Creative Commons, PO Box 1866, Mountain View, CA 94042, USA.

\chapter{Turingmaschinen und Grammatiken}

\section{Einführung}

Wir wollen Automatenmodelle für Typ-0- und Typ-1-Sprachen finden.\nl
\textbf{Churchsche These:}\\
Die (intuitiv) berechenbaren Funktionen sind genau die mit Turingmaschinen (und damit GOTO-,WHILE-,Pascal-Programmen,...) berechenbaren Funktionen.

\section{Turingmaschinen}
Eine Turingmaschine kann man sich als beidseitig unendliches Eingabeband mit Schreib-Lesekopf und endlicher Kontrolle vorstellen.
Zu jedem Zeitpunkt sind stets nur endlich viele Symbole auf dem Band verschieden von $\nb$.

\begin{definition}\label{def11.1}
	Eine \textbf{Turingmaschine} über dem Eingabealphabet $\Sigma$ hat die Form $\A=(Q,\Sigma,\Gamma,q_0,\Delta,F)$ wobei
	\begin{itemize}
		\item $Q$ endliche \textbf{Zustandsmenge} ist,
		\item $\Sigma$ das \textbf{Eingabealphabet} ist,
		\item $\Gamma$ das \textbf{Arbeitsalphabet} ist mit $\Sigma\cup\Gamma,\nb\in\Gamma\setminus\Sigma$
		\item $q_0\in Q$ der \textbf{Anfangszustand} ist,
		\item $F\subseteq Q$ die \textbf{Endzustandsmenge} ist und 
		\item $\Delta\subseteq Q\times\Gamma\times\Gamma\times\lbrace r,l,n\rbrace\times Q$ die \textbf{Übergangsrelation} ist.
	\end{itemize}
	Dabei bedeutet $\big(q,a,a',\begin{matrix}
		r\\
		l\\
		n
	\end{matrix},q'\big)$:
	Im Zustand $q$ mit $a$ auf dem gerade gelesenen Feld\\ (\textbf{Arbeitsfeld}) kann die Turingmaschine $\A$ das Symbol $a$ durch $a'$ ersetzen, in den Zustand $q'$ gehen und den Schreib-Lesekopf entweder um ein Feld nach rechts / links / nicht zu bewegen.\nl
	Die Maschine $\A$ heißt \textbf{deterministisch (DTM)}, falls es für jedes Tupel $(q,a)\in Q\times F$ \underline{höchstens ein} Tupel der Form $(q,a,\ldots,\ldots)\in\Delta$ gibt.\nl
	Einen Berechnungszustand (\textbf{Konfiguration}) einer Turingmaschine kann man beschreiben durch ein Wort $\alpha p\beta$ mit $\alpha,\beta\in\Gamma^+,q\in Q$:
	\begin{itemize}
		\item $q$ ist der momentane Zustand
		\item $\alpha$ ist die Beschriftung des Bandes links vom Arbeitsfeld
		\item $\beta$ ist die Beschriftung des Bandes beginnend beim Arbeitsfeld nach rechts
	\end{itemize}
	Die Übergangsrelation $\Delta$ ermöglicht die folgenden \textbf{Konfigurationsübergänge:}\\
	Seien $\alpha,\beta\in\Gamma^+,\beta'\in\Gamma^\ast,a,b,a'\in\Gamma,q,q'\in Q$.
	\begin{itemize}
		\item $\begin{aligned}
			\left.\begin{array}{lcr}
				\alpha qa\beta &\vdash_\A&\alpha a'q'\beta\\
				\alpha qa&\vdash_\A&\alpha a'q'\nb
			\end{array}\right\rbrace\falls(q,a,a',r,q)\in\Delta
		\end{aligned}$
		\item $\begin{aligned}
			\left.\begin{array}{lcr}
				\alpha bqa\beta' &\vdash_\A&\alpha q'ba'\beta'\\
				\alpha qa\beta'&\vdash_\A&\nb q'b a'\beta'
			\end{array}\right\rbrace\falls(q,a,a',l,q')\in\Delta
		\end{aligned}$
		\item $\begin{aligned}
			\alpha qa\beta'\vdash_\A \alpha q'a'\beta'\qquad\falls(q,a,a',n,q')\in\Delta
		\end{aligned}$
	\end{itemize}
	Weitere Bezeichnungen:
	\begin{itemize}
		\item Gilt $k\vdash_\A k'$ so heißt $k'$ \textbf{Folgekonfiguration} von $k$.
		\item Die Konfiguration $\alpha q\beta$ heißt \textbf{akzeptierend}, falls $q\in F$.
		\item Die Konfiguration $\alpha q\beta$ heißt \textbf{Stoppkonfiguration}, falls sie keine Folgekonfiguration hat.
		\item Die von $\A$ \textbf{akzeptierte Sprache} ist
		\begin{align*}
			L(\A):=\Big\lbrace w\in\Sigma^\ast:\nb q_0 w\nb\vdash_\A^\ast k,\text{ wobei $k$ akzeptierende Stoppkonfiguration ist}\Big\rbrace
		\end{align*}
	\end{itemize}
\end{definition}

\begin{definition}\label{def11.2}
	Die Sprache $L\subseteq\Sigma^\ast$ heißt \textbf{Turing-akzeptierbar} $:\Longleftrightarrow\exists$ NTM $\A:L=L(\A)$.
\end{definition}

\section{Zusammenfassende Tabelle}

\begin{tabular}{l|l|l|l}
	Sprachkl. & Automaten & Abschlusseigenschaften & entscheidbare Prob.\\ \hline
	$\L_3$ & DEA, NEA & $\cup,\cdot,\ast,\cap,C,\setminus$ & $\emptyset$, ÄQ, WP \\
	$\L_2$ & DPDA $\neq$ NPDA & $\cup,\cdot,\ast$ & WP, $\emptyset$\\
	$\L_1$ & LBA, DLBA$\overset{?}{=}$NLBA & $\cup,\cdot,\ast,\cap,C,\setminus$ & WP\\
	$\L_0$ & DTM=NTM & $\cup,\cdot,\ast,\cap$ & alle unentscheidbar\\
\end{tabular}

Abkürzungen:
\begin{itemize}
	\item WP = Wortproblem
	\item $\emptyset$ = Leerheitsproblem
	\item $C$ steht für Mengenkomplement
	\item LBA = linear beschränkter Automat; ist spezielle NTM.
\end{itemize}

\begin{itemize}
	\item Turingmaschine zum Wortproblem: Nicht entscheidbar\\
	Falls $w\in L$ geht TM in akzeptierenden Stoppzustand über\\
	Falls $w\not\in L$ bleibt TM stehen, akzeptiert aber nicht\\
	Falls $w\not\in L$ läuft die TM unendlich lange\\
	(Beides möglich, also unentscheidbar)
	\item Ob das Äquivalenzproblem für $\L_0$ entscheidbar ist, weiß man noch nicht.
\end{itemize}