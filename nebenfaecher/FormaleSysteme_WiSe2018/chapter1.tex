% This work is licensed under the Creative Commons
% Attribution-NonCommercial-ShareAlike 4.0 International License. To view a copy
% of this license, visit http://creativecommons.org/licenses/by-nc-sa/4.0/ or
% send a letter to Creative Commons, PO Box 1866, Mountain View, CA 94042, USA.

\chapter{Aussagenlogik}
\section{Syntax}

\begin{definition}\
	\begin{itemize}
		\item Ein \textbf{Alphabet} $\Sigma$ ist eine Menge.
		\item Ein \textbf{Zeichen} ist Element eines Alphabets.
		\item Das \textbf{leere Wort} schreiben wir als $\lambda$, o.B.d.A $\lambda\not\in\Sigma$
		\item Ein \textbf{Wort} über einem Alphabet ist $\lambda$ oder $aw$, falls $w$ ein Wort und $a\in\Sigma$ (also einfach alle möglichen Kombinationen).
		\item Die Menge aller Wörter über $\Sigma$ wird mit $\Sigma^\ast$ bezeichnet.
	\end{itemize}
\end{definition}

\begin{definition}
	Ein \textbf{Alphabet der Aussagenlogik} ist $(\RR,\J,\lbrace(,)\rbrace)$ wobei
	\begin{itemize}
		\item $\RR:=\lbrace p_1,p_2,\ldots\rbrace$ eine Menge von aussagenlogischen Variablen
		\item $\J:=\lbrace\neg,\wedge,\vee,\to,\leftrightarrow\rbrace$ die Menge von Junktoren
		\item und $\lbrace(,)\rbrace$ die Menge der beiden Sonderzeichen ist.
	\end{itemize}
	Die Elemente von $\RR$ heißen auch \textbf{Atome / atomare Formeln}.
\end{definition}

\begin{definition}\label{def3.5Formel}
	Eine \textbf{aussagenlogische Formel} ist wie folgt definiert:
	\begin{enumerate}
		\item Wenn $F\in\RR$, dann ist $F$ eine Formel.
		\item Wenn $F$ eine Formel ist, dann ist $\neg F$ ein Formel.
		\item Wenn $F,G$ Formeln sind, dann ist $(F\circ G)$ eine Formel mit $\circ\in\J$.
	\end{enumerate}
	Die Menge aller aussagenlogischen Formeln wird mit $\L(\RR)$ bezeichnet.
\end{definition}

\begin{beispiel}
	Das \textbf{Alphabet der Arithmetik} ist $(\Q,\lbrace+,-,\div,\times\rbrace,\lbrace(,)\rbrace)$.\\
	Ein \textbf{arithmetischer Ausdruck} ist:
	\begin{enumerate}
		\item $T\in\Q$ ist ein arithmetischer Ausdruck.
		\item Für $\circ\in\lbrace+,-,\div,\times\rbrace$ und $T_1,T_2$ arithmetische Ausdrücke ist $(T_1\circ T_2)$ ein arithmetischer Ausdruck.
	\end{enumerate}
\end{beispiel}

\begin{beispiel}
	Das \textbf{Alphabet der Natürlichen Zahlen} ist $(\lbrace 0\rbrace,\lbrace s\rbrace,\lbrace(,)\rbrace)$.
	Eine \textbf{natürliche} Zahl ist:
	\begin{enumerate}
		\item $0$ ist eine natürliche Zahl
		\item Wenn $N$ eine natürliche Zahl ist, ist $s(N)$ eine natürliche Zahl.
	\end{enumerate}
\end{beispiel}

Mit dem folgenden nützlichem Tool können wir Eigenschaften über aussagenlogische Formeln leicht beweisen:

\begin{satz}[Prinzip der strukturellen Induktion]\enter
	Jede Formel in $\L(\RR)$ besitzt Eigenschaft $E$, wenn:
	\begin{enumerate}
		\item \underline{Induktionsanfang:} $\forall A\in\RR:A$ besitzt Eigenschaft $E$.
		\item \underline{Induktionsschritte:}
		\begin{itemize}
			\item Wenn $F$ Eigenschaft $E$ besitzt, dann besitzt sie auch $\neg F$.
			\item Wenn $\circ\in\J$ und $F,G\in\L(\RR)$ die Eigenschaft $E$ besitzen, dann besitzt auch $(F\circ G)$ die Eigenschaft $E$.
		\end{itemize}
	\end{enumerate}
\end{satz}

\begin{proof}
	Sei $\E\subseteq\L(\RR)$ die Menge aller Formeln, die Eigenschaft $E$ erfüllen.
	Da $\E$ die drei Bedingungen in Definition \ref{def3.5Formel} erfüllt, gilt $\L(\RR)\subseteq\E$, da $\L(\RR)$ nach Definition die kleinste solche Menge ist.
\end{proof}

\begin{lemma}[Strukturelle Induktion über Zeichenreihen]\enter
	Sei $\Sigma$ ein Alphabet.
	Jede Zeichenreihe über $\Sigma$ besitzt eine Eigenschaft $E$, falls:
	\begin{enumerate}
		\item \underline{IA:} $\lambda$ besitzt Eigenschaft $E$.
		\item \underline{IS:} Wenn $w\in\Sigma\ast$ die Eigenschaft $E$ besitzt und $a\in\Sigma$, dann besitzt $aw$ die Eigenschaft $E$.
	\end{enumerate}
\end{lemma}

Funktionen auf $\L(\RR)$ haben stets die Gestalt
\begin{align*}
	f\colon\L(\RR)\to A,\qquad
	F\mapsto\left\lbrace\begin{array}{cl}
		a, &\falls F\in\RR\\
		f(G), &\falls F=\neg G\\
		f(G_1)+f(G_2),&\falls F=(G_1\circ G_2)\mit \circ\in\J
	\end{array}\right.
\end{align*}

\begin{definition}
	Sei $T$ ein arithmetischer Ausdruck.	
	Die \textbf{Menge aller (arithmetischen) Teilausdrücke} $\S_T$ ist:
	\begin{enumerate}
		\item $T\in\S_T$.
		\item Wenn $(T_1\circ T_2)\in\S_T\mit\circ\in\lbrace+,-,\div,\times\rbrace$, dann sind auch $T_1,T_2\in\S_T$.
	\end{enumerate}
\end{definition}

\begin{beispiel}
	Der Ausdruck $((3\times(9-6))+7)$ die Teilausdrücke
	\begin{align*}
		((3\times(9-6))+7),
		(3\times(9-6)),
		(9-6),
		9,6,7,3.
	\end{align*}
\end{beispiel}

\begin{definition}
	Die \textbf{Menge aller Teilformeln} von $F\in\L(\RR)$ ist $\S_F$:
	\begin{enumerate}
		\item $F\in\S_F$.
		\item Wenn $\neg G\in\S_G$, dann $G\in\S_F$.
		\item Wenn $(G_1\circ G_2)\in\S_f\mit\circ\in\J$, dann sind auch $G_1,G_2\in\S_F$.
	\end{enumerate}
	Die \textbf{Menge der in $F\in\L(\RR)$ vorkommenden Variablen} ist $\RR_F:=\S_F\cap\RR$.
\end{definition}

\begin{beispiel}
	Für $F=\neg((p_1\to p_2)\vee p_1)$ gilt:
	\begin{align*}
		\S_F&=\Big\lbrace
			\neg((p_1\to p_2)\vee p_1),
			((p_1\to p_2)\vee p_1),
			(p_1\to p_2), p_1,
			p_2
		\Big\rbrace\\
		\RR_F&=\lbrace p_1,p_2\rbrace
	\end{align*}
\end{beispiel}

\section{Semantik}
Was bedeuten eigentlich aussagenlogische Formeln? 
Sie können entweder wahr $\top$ oder falsch $\bot$ sein (zweiwertige Logik).
Wir nutzen \textit{Interpretationsfunktionen}, um Syntax auf Semantik abzubilden.
Wir betrachten zusätzlich Funktionen zu Junktoren um kennzeichnen diese mit einem $\ast$ als Superindex.

\begin{beispiel}
	Sei $\A$ die Menge der arithmetischen Ausdrücke.
	Dann betrachte die Interpretation
	\begin{align*}
		\cdot^I:=I:\A\to\Q,\qquad T\mapsto\left\lbrace\begin{array}{cl}
			T ,&\falls T\in\Q\\
			\big([T_1]^I\circ^\ast[T_2]^I\big), &\falls T=(T_1\circ T_2)
		\end{array}\right.
	\end{align*}
	Somit erhalten wir beispielsweise
	\begin{align*}
		[((3\times(9-6))+7)]^I=((3\times^\ast(9-^\ast6))+^\ast7)=16
	\end{align*}
\end{beispiel}

\begin{definition}\
	\begin{itemize}
		\item Eine \textbf{Struktur} besteht aus einer Menge und einer Menge darauf definierter Funktionen.
		\item Die Menge der \textbf{Wahrheitswerte} ist $\W:=\lbrace\top,\bot\rbrace$.
		\item Einstellige Funktionen vom Typ $\W\to\W$ sind genau:
		\begin{enumerate}
			\item Die \textbf{Negation} $\neg^\ast$ mit $\neg^\ast(\top):=\bot$ und $\neg^\ast(\bot):=\top$
			\item Die \textbf{Identität}.
			\item Die beiden Konstanten Funktionen, die auf $\bot$ bzw. $\top$ abbilden.
		\end{enumerate}
		\item Zweistellige Funktionen: Es gibt genau 16 Stück. Wichtig sind:
			Konjunktion, Disjunktion, Implikation und Äquivalenz
	\end{itemize}
	
	\begin{tabular}{cc||c|c|c|c}\\
		&& $\wedge^\ast$ & $\vee^\ast$ & $\to^\ast$ & $\leftrightarrow^\ast$\\ \hline
		\w & \w & \w & \w & \w & \w\\
		\w & \f & \f & \w & \f & \f\\
		\f & \w & \f & \w & \w & \f\\
		\f & \f & \f & \f & \w & \w
	\end{tabular}
\end{definition}

\begin{definition}\label{def3.9}
	Eine \textbf{(aussagenlogische) Interpretation} $I=(\W,\cdot^I)$ besteht aus einer Menge $\W$ von Wahrheitswerten und einer Abbildung $\cdot^I:=I:\L(\RR)\to\W$ die folgende Bedingung erfüllt:
	\begin{align*}
		[F]^I=\left\lbrace\begin{array}{cl}
			\neg^\ast[G]^I, &\falls F=\neg G\\
			\big([G_1]^I\circ^\ast[G_2]^I\big),&\falls F=(G_1\circ G_2)
		\end{array}\right.
	\end{align*}
	(Der Basisfall $F\in\RR$ ist also "interpretationssache".)
\end{definition}

\begin{proposition}\label{proposition3.10}
	Für jede Abbildung $g:\RR\to\W$ gibt es genau eine Interpretation $I=(\W,\cdot^I)$ so, dass $g(A)=[A]^I~\forall A\in\RR$.
\end{proposition}

\begin{proof}
	Setze
	\begin{align*}
		[F]^I:=\left\lbrace\begin{array}{cl}
			g(F),&\falls F\in\RR\\			
			\neg^\ast[G]^I, &\falls F=\neg G\\
			\big([G_1]^I\circ^\ast[G_2]^I\big),&\falls F=(G_1\circ G_2)
		\end{array}\right.
	\end{align*}
\end{proof}

\begin{proposition}\label{proposition3.11}
	%Sei $F\in\L(\RR)$ und 
	Seien $I=(\W,\cdot^I)$ und $I'=(\W,\cdot^{I'})$ Interpretationen.\\
	Wenn für alle $A\in\RR$ $[A]^I=[A]^{I'}$ gilt, dann gilt auch $I\equiv I'$.
\end{proposition}

Wir identifizieren eine Interpretation mit der Menge der Variablen, die auf \w~ abgebildet werden.

\begin{definition}\label{def3.12}
	Sei $F\in\L(\RR)$.
	\begin{itemize}
		\item $F$ heißt \textbf{erfüllbar} $:\Longleftrightarrow\exists I:F^I=\top$.
		\item $F$ heißt \textbf{allgemeingültig / Tautologie} $:\Longleftrightarrow\forall I:F^I=\top$.
		\item $F$ heißt \textbf{widerlegbar} $:\Longleftrightarrow\exists I:F^I=\bot$.
		\item $F$ heißt \textbf{unerfüllbar} $:\Longleftrightarrow\forall I:F^I=\bot$.
	\end{itemize}
\end{definition}

Um eine Formel $F$ für alle möglichen Interpretationen auszuwerten nutzt man \textbf{Wahrheitswertetabellen}, d.h. eine Tabelle mit 
\begin{itemize}
	\item Anzahl der Zeilen entspricht 2 hoch der Anzahl der Variablen, also $2^{|\R_F|}$ Zeilen
	\item Anzahl der Spalten ist Anzahl der Teilformeln von $F$, also $|\S_F|$.
	\item Für die ersten $n$ Spalten: \w\f\w\f\ldots; \w\w\f\f\ldots;\w\w\w\w\f\f\f\f\ldots
\end{itemize}

\begin{beispiel}\enter
	\begin{tabular}{c|c||c|c|c}
		$p$ & $q$ & $(p\to q)$ & $(p\to q)\to p$ & $(((p\to q)\to p)\to p)$\\ \hline
		$\top$ & $\top$ & $\top$ & $\top$ & $\top$ \\
		$\bot$ & $\top$ & $\top$ & $\bot$ & $\top$ \\
		$\top$ & $\bot$ & $\bot$ & $\top$ & $\top$\\
		$\bot$ & $\bot$ & $\top$ & $\bot$ & $\top$
	\end{tabular}
\end{beispiel}

\begin{bemerkung}
	Sei $F\in\L(\RR)$ und $T(F)$ eine dazugehörige Wahrheitswertetabelle.
	Dann gilt:
	\begin{itemize}
		\item $F$ ist erfüllbar $\gdw T(F)$ enthält ein $\top$ in der letzten Spalte.
		\item $F$ ist allgemeingültig $\gdw T(F)$ enthält kein $\bot$ in der letzten Spalte.
		\item $F$ ist widerlegbar $\gdw T(F)$ enthält $\bot$ in der letzten Spalte.
		\item $F$ ist unerfüllbar $\gdw T(F)$ enthält kein $\top$ in der letzten Spalte.
	\end{itemize}
\end{bemerkung}

\begin{definition}
	Eine Interpretation $I=(\W,\cdot^I)$ heißt \textbf{Modell} für eine Formel $F$, i.Z. $I\models F:\Longleftrightarrow F^I=\top$.\\
	Wenn $F$ allgemeingültig ist, schreiben wir $\models F$.
\end{definition}

\begin{satz}\label{satz3.14}
	$F$ allgemeingültig $\Longleftrightarrow\neg F$ unerfüllbar
\end{satz}

\begin{proof}
	\begin{align*}
		\models F &\Longleftrightarrow\big(\forall I:I\models F\big)\\
		&\Longleftrightarrow\big(\nexists I:I\models\neg F\big)\\
		&\Longleftrightarrow\neg F\text{ ist unerfüllbar}
	\end{align*}
\end{proof}

\begin{definition}\label{def3.15}
	Sei $\G$ eine Menge von Formeln.
	\begin{itemize}
		\item Eine Interpretation $I$ heißt \define{Modell} für $\G$, i.Z.
		\begin{align*}
			I\models \G:\Longleftrightarrow\forall F\in\G:I\models F
		\end{align*}
		\item $\G$ ist \textbf{erfüllbar}, wenn es ein Modell für $\G$ gibt.
		\item $\G$ ist \textbf{unerfüllbar}, wenn es kein Modell für $\G$ gibt.
		\item $\G$ ist \textbf{widerlegbar}, wenn es eine Interpretation gibt, die kein Modell für $\G$ ist
		\item $\G$ ist \textbf{allgemeingültig}, wenn alle Interpretationen auch Modelle für $\G$ sind.
	\end{itemize}
\end{definition}

\begin{definition}\label{def3.16}
	Eine Formel $F\in\L(\RR)$ ist eine \textbf{(aussagen-)logische Konsequenz} einer Menge von Formeln $\G$, i.Z.
	\begin{align*}
		\G\models F:\Longleftrightarrow\forall I:I\models\G\implies I\models F
	\end{align*}
\end{definition}

\begin{satz}\label{satz3.17}
	Seien $F,F_1,\ldots,F_n\in\L(\RR)$. Dann gilt:
	\begin{align*}
		\big\lbrace F_1,\ldots,F_n\big\rbrace\models F\Longleftrightarrow\models\Big(\big(\ldots(F_1\wedge F_2)\wedge\ldots\wedge F_n\big)\to F\Big)
	\end{align*}
\end{satz}

\section{Äquivalenz und Normalform}
\subsection{Semantische Äquivalenz}

\begin{definition}\label{def3.18}
	Zwei Formeln $F,G$ heißen \textbf{(semantisch) äquivalent}, i.Z.
	\begin{align*}
		F\equiv G:\Longleftrightarrow\forall I:I\models F\Longleftrightarrow I\models G
	\end{align*}
\end{definition}

\begin{bemerkung}
	$\equiv$ ist offenbar eine Äquivalenzrelation auf $\L(\RR)$.
\end{bemerkung}

\begin{satz}\label{satz3.19}
	Seien $F,G,H$ Formeln. Dann gilt:
	\begin{align*}
		(F\wedge F)&\equiv F\\
		(F\vee F)&\equiv F &&\text{Idempotenz}\\[1ex]
		(F\wedge G)&\equiv(G\wedge F)\\
		(F\vee G)&\equiv (G\vee F) &&\text{Kommutativität}\\[1ex]
		((F\wedge G)\wedge H)&\equiv(F\wedge(G\wedge H))\\
		((F\vee G)\vee H)&\equiv(F\vee(G\vee H)) &&\text{Assoziativität}\\[1ex]
		((F\wedge G)\vee F)&\equiv F\\
		((F\vee G)\wedge F)&\equiv F &&\text{Absorption}\\[1ex]
		(F\wedge(G\vee H))&\equiv((F\wedge G)\vee(F\wedge H))\\
		(F\vee(G\wedge H)&\equiv((F\vee G)\wedge(F\vee H) &&\text{Distributivität}\\[1ex]
		\neg\neg F&\equiv F &&\text{Doppelte Negation}\\[1ex]
		\neg(F\wedge G)&\equiv(\neg F\vee\neg G)\\
		\neg(F\vee G)&\equiv(\neg F\wedge\neg G)&&\text{de Morgan}\\[1ex]
		(F\vee G)&\equiv F,\text{ wenn $F$ allgemeingültig}\\
		(F\wedge G)&\equiv G\text{ wenn $F$ allgemeingültig} &&\text{Tautologie}\\[1ex]
		(F\vee G)&\equiv G,\text{ wenn $F$ unerfüllbar}\\
		(F\wedge G)&\equiv F,\text{ wenn $F$ unerfüllbar} &&\text{Unerfüllbarkeit}\\[1ex]
		(F\leftrightarrow G)&\equiv((F\wedge G\vee(\neg G\wedge\neg F)) &&\text{Äquivalenz}\\[1ex]
		(F\to G)&\equiv(\neg F\vee G) &&\text{Implikation}
	\end{align*}
\end{satz}

\begin{definition}\label{def3.20}
	Die Menge der \textbf{Positionen / Vorkommen} $\P_F$ in einer Formel $F\in\L(\RR)$ ist
	\begin{align*}
		&\pos\colon\L(\RR)\to\mathfrak{P}\big(\lbrace1,2\rbrace^\ast\big),\\
		&F\mapsto\lbrace\Lambda\rbrace\cup\left\lbrace\begin{array}{cl}
			\emptyset ,&\falls F\in\RR\\
			\lbrace 1\pi:\pi\in\pos(G)\rbrace, &\falls F=\neg G\\
			\lbrace 1\pi_1:\pi_1\in\pos(G_1)\rbrace\cup\lbrace2\pi_2:\pi_2\in\pos(G_2)\rbrace
			,&\falls F=(G_1\circ G_2)
		\end{array}\right.
	\end{align*}
\end{definition}

\begin{beispiel}
	Für $F=\neg(p\wedge(q\vee\neg p))$ ist 
	$\mathcal{P}_F=\lbrace\Lambda,1,11,12,121,122,1221\rbrace$.
	
	\begin{figure}[H]
		\begin{center}
			\begin{tikzpicture}[node distance=5em]
				\tikzstyle{every state}=[shape=circle,draw=black]
				\node[state] (1) {$\neg:\Lambda$};
				\node[state] (2) [below = of 1] {$\wedge:1$};
				\node[state] (3) [below left = of 2] {$p:11$};
				\node[state] (4) [below right = of 2] {$\vee:12$};
				\node[state] (5) [below left = of 4] {$q:121$};
				\node[state] (6) [below right = of 4] {$\neg:122$};
				\node[state] (7) [below of =6] {$p:1221$};
				\path 	(2) edge [, -] node {} (1)
    		  	   		(3) edge [, -] node {} (2)
	 	      			(4) edge [, -] node {} (2)
			  			(5) edge [, -] node {} (4)
	  					(6) edge [, -] node {} (4)
	  					(7) edge [, -] node {} (6)
				;	
			\end{tikzpicture}
			
			\caption{Beispiel für Positionen in Formel}
			\label{AbbPositionenInFormeln}
		\end{center}
	\end{figure}
\end{beispiel}

\begin{definition}\label{def3.21}
	Die \textbf{Teilformel von $F$ an der Position $\pi\in\P_F$}, i.Z. $F\lceil\pi\rceil$, ist wie folgt definiert:
	\begin{enumerate}
		\item $F\lceil\lambda\rceil:=F$.
		\item $F\lceil 1\pi\rceil:= G\lceil\pi\rceil$, wenn $F=\neg G$.
		\item $F\lceil i\pi\rceil:=G_i\lceil\pi\rceil$, wenn $F=(G_1\circ G_2)$ und $i\in\lbrace1,2\rbrace$.
	\end{enumerate}
\end{definition}

\begin{definition}\label{def3.22}
	Die \textbf{Ersetzung von $F$ an der Position $\pi\in\P_F$ durch $H$}, i.Z. $F\lceil\pi\mapsto H\rceil$ ist:
	\begin{enumerate}
		\item $F\lceil\lambda\mapsto H\rceil:=H$
		\item $F\lceil 1\pi\mapsto H\rceil:=\neg(G\lceil\pi\mapsto H\rceil)$, wenn $F=\neg G$.
		\item $F\lceil 1\pi\mapsto H\rceil:=\big(G_1\lceil\pi\mapsto H\rceil\circ G_2\big)$, wenn $F=(G_1\circ G_2)$.
		\item $F\lceil2\pi\mapsto H\rceil:=\big(G_1\circ G_2\lceil\pi\mapsto H\rceil\big)$, wenn $F=(G_1\circ G_2)$
	\end{enumerate}
\end{definition}

\begin{satz}\label{satz3.23}
	Seien $F,H,G\in\L(\RR),\pi\in\P_F,F\lceil\pi\rceil=G$ und $G\equiv H$.\\
	Dann gilt: $F\equiv F\lceil\pi\mapsto H\rceil$
\end{satz}

\begin{proof}
	Induktion über den Aufbau von $\pi$ (IA: $\pi=\lambda$, IS: $\pi=i\pi'$ mit $i\in\lbrace1,2\rbrace$).
\end{proof}

\begin{konvention}\
	\begin{itemize}
		\item Wenn $\pi\not\in\P_F$, dann sei $F\lceil\pi\mapsto H\rceil:=F$.
		\item Sei $F\lceil\pi\rceil=G$. Wenn aus dem Kontext ersichtlich ist, welches Vorkommen der Teilformel in $G$ ersetzt werden soll, dann schreiben wir $F\lceil G\mapsto H\rceil$ anstelle von $F\lceil\pi\mapsto H\rceil$.
		\item Satz \ref{satz3.23} erlaubt alle Vorkommen der Junktoren $\leftrightarrow$ und $\to$ zu ersetzen. Wir betrachten nun o.B.d.A. nur noch Formeln über $\RR\cup\lbrace\neg,\wedge,\vee,(,)\rbrace$.
	\end{itemize}
\end{konvention}

\subsection{Negationsnormalform}

\begin{definition}\label{def3.24}
	Eine Formel $F$ ist in \textbf{Negationsnormalform}, wenn alls in $F$ vorkommenden Negationszeichen $\neg$ unmittelbar vor aussagenlogischen Variablen stehen (also nicht vor Klammern).
\end{definition}

\begin{proposition}\label{prop3.25}
	Man kann jede Formel in eine semantisch äquivalente Formel in Negationsnormalform überführen:\\
	Solange $F$ nicht in Negationsnormalform ist, tue:
	\begin{itemize}
		\item Bestimme eine Teilformel der Form $\neg G$ von $F$ mit $G\not\in\RR$.
		\item Wenn $\neg G=\neg\neg H$, dann ersetze $\neg G$ durch $H$, i.Z.:
		\begin{align*}
			\frac{\neg\neg H}{H}
		\end{align*}
		\item Wenn $\neg G=\neg(G_1\wedge G_2)$, dann ersetze $\neg G$ durch $(\neg G_1\vee\neg G_2)$, i.Z.:
		\begin{align*}
			\frac{\neg(G_1\wedge G_2)}{(\neg G_1\vee\neg G_2)}
		\end{align*}
		\item Wenn $\neg G=\neg(G_1\vee G_2)$, dann ersetze $\neg G$ durch $(\neg G_1\wedge\neg G_2)$, i.Z.
		\begin{align*}
			\frac{\neg(G_1\vee G_2)}{(\neg G_1\wedge\neg G_2)}
		\end{align*}
	\end{itemize}
\end{proposition}

\subsection{Klauselform}

\begin{definition}\label{def3.26}
	Ein \textbf{Literal} ist eine aussagenlogische Variable (\textbf{positives Literal}) oder eine negierte aussagenlogische Variable (\textbf{negatives Literal}).
\end{definition}

\begin{notation}\
	\begin{itemize}
		\item Verallgemeinerte Disjunktion und verallgemeinerte Konjunktion:
		\begin{align*}
			\big[F_1,\ldots,F_n\big]:=\bigvee\limits_{i=1}^n F_i,\qquad
			\big\langle F_1,\ldots,F_n\big\rangle:=\bigwedge\limits_{i=1}^n F_i
			\qquad\forall F_j\in\L(\RR)
		\end{align*}
		\item Beachte $[F]=F=\langle F\rangle$ und $[~]^I=\bot,\langle\rangle^I=\top~\forall I$.
		Somit $(G\vee[~])\equiv G\equiv(G\wedge\langle\rangle)$.
	\end{itemize}
\end{notation}

\begin{definition}\label{def3.27}\
	\begin{itemize}
		\item Eine \textbf{Klausel} ist einer verallgemeinerte Disjunktion $[L_1,\ldots,L_n]$ mit $n\geq0,L_i$ Literal.
		\item Eine \textbf{duale Klausel} ist eine verallgemeinerte Konjunktion $\langle L_1,\ldots,L_n\rangle$ mit $n\geq0,_i$ Literal.
		\item Eine Formel ist in \textbf{konjunktiver Normalform / Klauselform} $:\gdw$ sie v.d.F. $\langle C_1,\ldots,C_m\rangle$ mit $m\geq0,C_i$ Klausel ist.
		\item Eine Formel ist in \textbf{disjunktiver Normalform / dualer Klauselform} $:\gdw$ sie v.d.F. $[C_1,\ldots,C_m]$ mit $m\geq0, C_i$ duale Klausel ist.
	\end{itemize}
\end{definition}

\begin{bemerkung}
	Sei $F$ eine Formel in Klauselform. Dann gilt:
	\begin{enumerate}
		\item $F$ unerfüllbar, wenn eine $F$ vorkommende Klausel unerfüllbar ist.
		\item Klausel unerfüllbar $\gdw$ sie leer ist $[~]$.
	\end{enumerate}
\end{bemerkung}

\begin{satz}\label{satz3.28}
	Jede Formel $F$ kann in eine semantisch äquivalente Formel $G$ in Klauselform (dualer Klauselform) überführt werden:\nl
	\texttt{ Setze $G:=\langle[F]\rangle$. 
	Solange $G$ nicht in konjunktiver Normalform ist, tue:
	\begin{itemize}
		\item Wähle ein Element $H$ aus $G$ aus, dass keine Klausel ist.
		\item Wähle ein Element $K$ aus $H$ aus, dass kein Literal ist.
		\item Wende eine der folgenden Regeln an, die anwendbar ist:
		\begin{align}
			&\frac{\neg\neg D}{D}\label{eqKNF1}\tag{KNF1} 
				&\langle[\neg\neg D,R]\rangle&\equiv\langle[D,R]\rangle\\
			&\frac{(D_1\wedge D_2)}{D_1\mid D_2}\label{eqKNF2}\tag{KNF2} 
				&\langle[(D_1\wedge D_2),R]\rangle&\equiv \langle[D_1,R],[D_2,R]\rangle\\
			&\frac{\neg(D_1\wedge D_2)}{\neg D_1,\neg D_2}\label{eqKNF3}\tag{KNF3}
				&\langle[\neg(D_1\wedge D_2),R]\rangle&\equiv\langle[\neg D_1,\neg D_2,R]\rangle\\
			&\frac{(D_1\vee D_2)}{D_1, D_2}\label{eqKNF4}\tag{KNF4}
				&\langle[D_1\vee D_2,R]\rangle&\equiv\langle[D_1,D_2,R]\rangle\\
			&\frac{\neg(D_1\vee D_2)}{\neg D_1\mid\neg D_2}\label{eqKNF5}\tag{KNF5}
				&\langle[\neg(D_1\vee D_2),R]\rangle&\equiv\langle[\neg D_1,R],[\neg D_2,R]\rangle
		\end{align}
	\end{itemize}		
	} Bei Anwendung einer Regel $\frac{E}{E_1\mid E_2}$ wird $H$ durch zwei Disjunktionen ersetzt: 
	Die erste erhält man aus $H$, indem man das Vorkommen von $E$ durch $E_1$ ersetzt.
	Die zweite erhält man aus $H$ indem das Vorkommen von $E$ durch $E_2$ ersetzt wird.
\end{satz}

\section{Beweisverfahren}

\begin{align*}
	\big\lbrace F_1,\ldots,F_n\big\rbrace\models F 
	&\Longleftrightarrow \Big(\big\langle F_1,\ldots,F_n\big\rangle\to F\Big)\text{ ist allgemeingültig}\\
	&\Longleftrightarrow\neg\Big(\big\langle F_1,\ldots,F_n\big\rangle\to F\Big)\text{ ist unerfüllbar}\\
	&\Longleftrightarrow\big\langle F_1,\ldots,F_n,\neg F\big\rangle\text{ ist unerfüllbar}
\end{align*}

\subsection{Resolution}
Resolution ist ein Verfahren zum Nachweis der Unerfüllbarkeit beliebiger Formeln (in Klauselform).
Eine Formel in Klauselform ist unerfüllbar, wenn sie eine leere Klausel enthält (oder sie in eine semantisch äquivalente Formel umgeformt werden kann, die die leere Klausel enthält).
Resolution ist also die Suche nach der leeren Klausel unter Verwendung der \textit{Resolutionsregel}.

\begin{definition}\label{def3.32}
	Sei $C_1$ eine Klausel, in der das Atom $A\in\RR$ vorkommt und $C_2$ eine Klausel, in der das negierte Atom $\neg\A$ vorkommt. Sei $C$ das Resultat von:
	\begin{enumerate}
		\item Entferne alle Elemente der Form $A$ aus $C_1$.
		\item Entferne alle Elemente der Form $\neg A$ aus $C_2$.
		\item Verknüpfe die so erhaltenen Klauseln disjunktiv.
	\end{enumerate}
	Dies ist die \textbf{Resolutionsregel} und $C$ die \textbf{Resolvente} von $C_1$ und $C_2$ bzgl. $A$. 
	Kurzform zum Merken:
	\begin{align*}
		C_1=\big[ A,B\big],~
		C_2=\big[ \neg A,C\big]\implies C=[ B,C]
	\end{align*}
\end{definition}

\begin{definition}\label{def3.33}
	Sei $F=\langle C_1,\ldots C_n\rangle$ eine Formel in Klauselform.
	\begin{enumerate}
		\item Die Folge $(C_1,\ldots,C_n)$ heißt \textbf{Resolutionsableitung} für $F$.
		\item Wenn $(C_1,\ldots,C_m)$ eine Resolutionsableitung für $F$ ist und $C_{m+1}$ durch Anwendung der Resolutionsregel auf zwei Elemente $C_i,C_j,i,j\leq m$ entstanden ist, dann ist $C_1,\ldots,C_m,C_{m+1})$ eine \textbf{Resolutionsableitung} für $F$.
		\item Eine Resolutionsableitung für $F$, die die leere Klausel $[~]$ enthält, heißt \textbf{Resolutionswiderlegung} für $F$.
	\end{enumerate}
\end{definition}

\begin{definition}\label{def3.34}
	Sei $F$ eine aussagenlogische Formel und $G\equiv\neg F$ in Klauselform.
	Ein \textbf{(aussagenlogischer) Resolutionsbeweis} für $F$ ist eine Resolutionswiderlegegung für $G$.
	$F$ heißt \textbf{Theorem} des Resolutionskalküls, wenn es einen Resolutionsbeweis für $F$ gibt, i.Z. $\vdash_r F$.\nl
	Forderungen:
	\begin{enumerate}
		\item Korrektheit: $\vdash_r F\implies\models F$
		\item Vollständigkeit: $\models F\implies\vdash_r F$.
	\end{enumerate}
\end{definition}

\section{Eigenschaften}

Idee: Um die Erfüllbarkeit einer unendlich großen Menge von Formeln zu betrachten, genügt es, endliche Teilmengen zu betrachten.

\begin{itemize}
	\item $\equiv$ ist eine Äquivalenzrelarion auf $\L(\RR)$.
	\item Sei $\L(\RR,n)\subseteq\L(\RR)$ die Menge aller Formeln mit höchstens $n$ Variablen.
	\item Dann finden wir $2^n$ verschiedene Interpretationen für $\L(\RR,n)$.
	\item Es gibt $2^{2^n}$ verschiedene durch $\equiv$ auf $\L(\RR,n)$ definierte Äquivalenzklassen.
\end{itemize}

\begin{satz}[Endlichkeitssatz]\label{satz345Endlichkeitssatz}\enter
	Eine Menge $\G$ von Formeln der Aussagenlogik ist erfüllbar $\gdw$ jede endliche Teilmenge $\H\subseteq\G$ erfüllbar ist.
\end{satz}

\begin{proof}
	Die Richtung "$\implies$" folgt direkt aus der Definition von Erfüllbarkeit von Mengen.\nl
	Sei jede endliche Teilmenge von $\G$ erfüllbar.
	Setze $\G_n:=\G\cap\L(\RR,n)$.\\
	Dann gibt es maximal $k\leq 2^{2^n}$ verschiedene durch $\equiv$ auf $\G_n$ definierte Äquivalenzklassen.
	Seien $G_1,\ldots,G_k$ Repräsentanten dieser Äquivalenzklassen.\\
	Somit gibt es für alle $G\in\G_n$ ein $i\in\lbrace 1,\ldots,k\rbrace:G\equiv G_i$.\\
	Folglich ist jedes Modell für $\big\lbrace G_1,\ldots,G_k\big\rbrace$ auch ein Modell für $\G_n$.\\
	Da $\big\lbrace G_1,\ldots,G_k\big\rbrace\subseteq\G$ endlich, besitzt sie ein Modell, nennen wir es $I_n$.\\
	Wegen $\G_1\subseteq\G_2\subseteq\ldots\subseteq\G_n$ ist $I_n$ auch ein Modell für alle $G_j$ mit $1\leq j\leq n$.\\
	Nun konstruieren wir eine Interpretation $I$ für $\G$ (Details unwichtig).
\end{proof}

\begin{korollar}\label{korollar3.46}
	Eine Menge von Formeln ist unerfüllbar $\gdw$ eine ihrer Teilmengen unerfüllbar ist.
\end{korollar}

\begin{lemma}[Resolutionslemma]\label{lemma3.47Resolutionslemma}\enter
	Sei $F=\big\langle C_1,\ldots,C_n\big\rangle$ Formel in Klauselform mit Klauseln $C_i$ 
	und seien $D_1\ldots,D_m$ die in einer Resolventenableitung von $F$ berechneten Resolventen.\\
	Dann gilt $F\equiv(F\wedge D_1)$ und $F\equiv\big\langle F,D_1,\ldots, D_m\big\rangle$
\end{lemma}

\begin{satz}\label{satz3.49}
	Das Resolutionsverfahren ist korrekt und vollständig, d.h.:
	\begin{align*}
		\forall F\in\L(\RR):\vdash_r F\Longleftrightarrow\models F
	\end{align*}
\end{satz}

