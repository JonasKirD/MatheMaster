% !TEX root = KAT.tex
% This work is licensed under the Creative Commons
% Attribution-NonCommercial-ShareAlike 4.0 International License. To view a copy
% of this license, visit http://creativecommons.org/licenses/by-nc-sa/4.0/ or
% send a letter to Creative Commons, PO Box 1866, Mountain View, CA 94042, USA.

\chapter{Kategorien: Definition und Beispiele}
\section{Zur Motivation}
Abstraktion in der Mathematik
\clearpage
\setlength{\voffset}{+45mm}
\pagestyle{empty} % disable pagenumbering
\begin{landscape}
\begin{tabular}{c||c|c|c|c}
	\makecell{konkrete\\ Objekte}
	&\makecell{Alle Teilmengen\\einer Menge $X$\\($\Potenzmenge{X}$ Potenzmenge)}
	&\makecell{Alle Unterobjekte\\eines math. Objektes ($\Sub X$)}
	&\makecell{Alle Automorphismen (Symmetrien)\\eines math. "Objektes" $X$\\($\Aut X$)}
	&\makecell{Alle Homomorphismen\\ zwischen math. Obj.\\ einer Familie $\X$}\\
	\hline\hline
	\makecell{konkrete\\ math. Struktur}
	&\makecell{Rechnen mit Teilmengen\\$\cap,\cup,\overline{\cdot},\emptyset,X$}
	&\makecell{Teilmengenbeziehung\\ "$\subseteq$"}
	&\makecell{Komposition von\\Permutationen\\
	\begin{tikzcd}[ampersand replacement=\&]
		X \arrow{r}{f} \arrow[bend right]{rr}{fg} \& X \arrow{r}{g} \& X
  	\end{tikzcd}
	}
	&\makecell{Komposition von\\ Homomorphismen\\
	\begin{tikzcd}[ampersand replacement=\&]      
		A \arrow{r}{f} \arrow{d}{fg} \& B \arrow{d}{gh} \arrow{ld}{g} \\
		C \arrow{r}{h}                  \& D                         
	\end{tikzcd}
	}\\
	\hline
	\makecell{Eigenschaften}
	&\makecell{Axiome für\\Mengenoperationen\\(z.B. $\cap$ ass., kommu.}
	&\makecell{reflexiv,\\transitiv,\\ antisymmetrisch}
	&\makecell{\ul{Ass.:} $(fg)h=f(gh)$\\\ul{Inv.:} $f^{-1}f=\id_X=ff^{-1}$\\\ul{Neu.:} $f\mal\id_X=f=\id_X f$}
	&\makecell{\ul{Ass.:} $(fg)h=f(gh)$\\\ul{Neu.:} $f\mal\id_B=\id_A\mal f=f$}\\
	\hline
	\makecell{abstrakte algebr.\\Strukturen}
	&\makecell{Boolesche Algebra\\$\langle B,\wedge,\vee,\overline{\cdot},0,1\rangle$}
	&\makecell{geordnete Menge\\$\langle P,\leq\rangle$}
	&\makecell{Gruppe\\$\langle G,\cdot,^{-1},e\rangle$}
	&\makecell{Kategorie\\(siehe \ref{section:1.2})}\\
	\hline
	\makecell{Rechtfertigung\\(Korrektheit,\\Soundness)}
	&\makecell{Jede Potenzmengenalgebra\\ist Boolsche Algebra.}
	&\makecell{$\langle\Sub X,\subseteq\rangle$ ist\\geordnete Menge}
	&\makecell{$\Aut(X)$ ist Gruppe}
	&\makecell{siehe \ref{section:1.3}}\\
	\hline
	\makecell{Darstellungssatz\\Vollständigkeit\\Completeness}
	&\makecell{Jede Boolsche\\ Algebra ist Unteralgebra\\einer Potenzmengenalgebra\\$\langle\Potenzmenge{X},\cap,\cup,\ldots\rangle$}
	&\makecell{ja}
	&\makecell{Satz von Cayley}
	&\makecell{siehe 3.6}\\ %TODO \ref{section:3.6}
\end{tabular}\\

Ergänzungen zu obiger Tabelle:
\begin{itemize}
	\item Satz von Cayley: Jede Gruppe ist isomorph zu Permutationsgruppe
\end{itemize}
\end{landscape}
\pagestyle{headings} %enable pagenumbering
\clearpage
\setlength{\voffset}{0mm}

\section{Definition}\label{section:1.2}
\begin{definition}
	Eine \define{Kategorie} $\Cat$ ist gegeben durch:\index{Kategorie}
	\begin{itemize}
		\item eine Klasse (vgl. \ref{section:0.3}
		) $\Ob(\Cat)$\\
		Die Elemente von $\Ob(\Cat)$ heißen \define{Objekte}.\index{Objekte}
		\item Für jedes Paar $(A,B)$ von Objekten $A,B\in\Ob(\C)$ gibt es eine Menge(!) $\Cat(A,B)$\\
		Die Elemente von $\Cat(A,B)$ heißen \define{Morphismen / Pfeile.} \index{Morphismus}\index{Pfeil|see{Morphismus}}
		Diese Mengen sind paarweise disjunkt, d.h.
		\begin{align*}
			\Cat(A,B)\cap\Cat(C,D)\neq\emptyset
			\implies A=C\AND B=D
		\end{align*}
	\end{itemize}
\end{definition}

\begin{notation}
	Schreibe $A\overset{a}{\longrightarrow}B$ oder $a\colon A\to B$ für $a\in\Cat(A,B)$.
\end{notation}

\begin{lemma}
	Für jedes Tripel $(A,B,C)$ von Objekten $A,B,C\in\Ob(\Cat)$ gibt es eine Abbildung, 
	\begin{align*}\index{Komposition}
		\mu_{A,B,C}^\Cat\colon\Cat(A,B)\times\Cat(B,C)\to\Cat(A,C)
	\end{align*}
	die sogenannte \define{Komposition} der Morphismen $a\in\Cat(A,B)$, $b\in\Cat(B,C)$ und wird mit $a;b$ oder $ab$ bezeichnet.
	Andere Bezeichnungen: $b\circ a$ oder $ba$.
	$$
	\begin{tikzcd}
                                   & B \arrow[rd, "b"] &   \\
A \arrow[ru, "a"] \arrow[rr, "ab"] &                   & C
	\end{tikzcd}
	$$
\end{lemma}

\begin{lemma}
	Für jedes Objekt $A\in\Ob(\Cat)$ gibt es einen Morphismus $1_A\in\Cat(A,A)$, genannt \define{identischer Morphismus}, falls die folgenden Axiome erfüllt sind:
	\begin{enumerate}\index{identischer Morphismus}
		\item[(ASS)] Die Morphismenkomposition ist assoziativ, d.h.
		\begin{align*}
			\forall a\in\Cat(A,B),\forall b\in\Cat(B,C),\forall c\in\Cat(C,D):(ab)c=a(bc)
		\end{align*}		 
		\item[(NEU)] Die identischen Morphismen sind neutrale Elemente bzgl. Komposition, d.h. 
		\begin{align*}
			\forall a\in\Cat(A,B):1_A a=a=a 1_B
		\end{align*}
	\end{enumerate}
	
		Schreibweise:
	$$ A\overset{1_A}{\longrightarrow}A\qquad
		 \begin{tikzcd}[cells={nodes={}}]
        \arrow[loop left, distance=3em, start anchor={[yshift=-1ex]west}, end anchor={[yshift=1ex]west}]{}{1_A} \arrow{r} \bullet 
        & \bullet%\arrow[loop right, distance=3em, start anchor={[yshift=1ex]east}, end anchor={[yshift=-1ex]east}]{}{\mathrm{id}_{B\times C}} 
    \end{tikzcd}\qquad
     \begin{tikzcd}[cells={nodes={}}]
        %\arrow[loop left, distance=3em, start anchor={[yshift=-1ex]west}, end anchor={[yshift=1ex]west}]{}{1_A}  
        \arrow{r}\bullet 
        & \bullet\arrow[loop right, distance=3em, start anchor={[yshift=1ex]east}, end anchor={[yshift=-1ex]east}]{}{1_B} 
    \end{tikzcd}	
	$$
\end{lemma}

\section{ToDo}\label{section:1.3}



