% This work is licensed under the Creative Commons
% Attribution-NonCommercial-ShareAlike 4.0 International License. To view a copy
% of this license, visit http://creativecommons.org/licenses/by-nc-sa/4.0/ or
% send a letter to Creative Commons, PO Box 1866, Mountain View, CA 94042, USA.

\chapter{Nichtlineare Halbgruppen} %4
\setcounter{section}{1}
Sei $(D,d)$ ein metrischer Raum.

\begin{definition}
	Eine \textbf{(stark stetige) Halbgruppe} auf $D$ ist eine Funktion\\ $S:[0,\infty]\to C(D,D)$ so, dass 
	\begin{enumerate}[label=(\roman*)]
		\item $S_0=\id$, d.h. $S_0(x)=x\qquad\forall x\in D$
		\item $\begin{aligned}
			S_{t+s}=S_t\circ S_s\qquad\forall s,t\geq0
		\end{aligned}$
		\item $\begin{aligned}
			[0,\infty)\to D,\qquad t\mapsto S_t(x)\text{ ist stetig}\qquad\forall x\in D
		\end{aligned}$
	\end{enumerate}
	(alternativer Name: \textbf{topologisches dynamisches System})\nl
	Manchmal betrachtet man auch \textbf{degenerierte (stark stetige) Halbgruppen}\\ $S\colon (0,\infty)\to C(D,D)$. 
	Definition wie oben, nur ersetze $[0,\infty]$ durch $(0,\infty)$ (auch in (iii)). Eigenschaft (i) fällt dann weg.
\end{definition}

\begin{beispiel}[Exponentialfunktion]\enter
	Sei $X$ ein Banachraum, $A\in\L(X)$ (d.h. beschränkter linearer Operator auf $X$) und 
	\begin{align*}
		S_t:=\sum\limits_{n=0}^\infty\frac{t^n\cdot A^n}{n!}=:\exp(t\cdot A)\qquad t\in[0,\infty]\text{ oder }t\in\C
	\end{align*}
	Die Reihe konvergiert absolut
	\begin{align*}
		\sum\limits_{n=0}^\infty \frac{t^n\cdot\Vert A^n\Vert}{n!}\leq\sum\limits_{n=0}^\infty\frac{t^n\cdot\Vert A\Vert^n}{	n!}=\exp\big(t\cdot\Vert A\Vert\big)<\infty
	\end{align*}
	und somit konvergiert sie auch in $\L(X)\subseteq C(X,X)$.\\
	Die Abbildung $S:\R_+\to\L(X)$ ist (als Potenzreihe mit Konvergenzradius $+\infty$) beliebig oft differenzierbar und 
	\begin{align*}
		\frac{\d}{\d t}S_t=AS_t=S_tA
	\end{align*}
	(Differenzierbarkeit bzgl. der Operatornorm in $\L(X)$!)\\
	Für alle $u_0\in X$ ist also der Orbit $u(t)=S_t u_0$ eine klassische Lösung des Cauchyproblems
	\begin{align*}
		\left\lbrace\begin{array}{r}
			\dot{u}-Au=0\\
			u(0)=u_0
		\end{array}\right.
	\end{align*}
\end{beispiel}

\begin{beispiel}[$C_0$-Halbgruppen / lineare Halbgruppen]\enter
	Eine stark-stetige Halbgruppe $S$ heißt auch \textbf{$C_0$-Halbgruppe} oder \textbf{lineare Halbgruppe}, 
	wenn $D=X$ ein Banachraum ist und alle $S_t$ stetige, lineare Operatoren sind.\\
	Für lineare Halbgruppen gelten folgende Eigenschaften:
\end{beispiel}

\begin{lemma}
	Ist $S$ eine lineare Halbgruppe auf einem Banachraum $X$, dann existieren Konstanten $M\geq1$ und $\omega\in\R$ so, dass 
	\begin{align*}
		\big\Vert S_t\big\Vert_{\L(X)}\leq M\cdot\exp(\omega\cdot t)\qquad\forall t\geq0
	\end{align*}
\end{lemma}

\begin{proof}
	Für alle $x\in X$ ist die Funktion $f\mapsto S_t x$ stetig auf $\R_+$ (Eigenschaft (iii)) und insbesondere auf dem Intervall $[0,1]$. 
	Damit ist die lineare Abbildung 
	\begin{align*}
		T:X\to C\big([0,1],X\big),\qquad x\mapsto\big(t\mapsto S_t x\big)
	\end{align*}
	wohldefiniert. 
%Prof: "Alle Abbildungen, die man hinschreiben kann, sind stetig." -> Er meint den Satz vom abgeschlossenen Graphen.
	Nun wollen wir den Satz vom abgeschlossenen Graphen anwenden. 
	$T$ ist außerdem abgeschlossen: 
	Es gelte $x_n\stackrel{n\to\infty}{\longrightarrow}x$ in $X$ und $T x_n\stackrel{n\to\infty}{\longrightarrow}f$ in $C\big([0,1],X\big)$. 
	Zu zeigen:\\
	Die Konvergenz $T x_n\stackrel{n\to\infty}{\longrightarrow}f$ (gleichmäßige Konvergenz) impliziert punktweise Konvergenz, d.h.
	\begin{align*}
		\big(\underbrace{T x_n}_{S_t x_n}\big)(t)\stackrel{n\to\infty}{\longrightarrow} f(t)\text{ in }X\qquad\forall t\in[0,1]
	\end{align*}
	Aus der Stetigkeit von $S_t$ folgt
	\begin{align*}
		f(t)=\limn S_t x_n=S_t x\qquad\forall t\in[0,1]
	\end{align*}
	d.h. $f=T x$. 
	Aus dem Satz vom abgeschlossenen Graphen folgt, dass $T$ stetig / beschränkt (äquivalent) ist.\\
	Sei $M:=\Vert T\Vert$. Dann ist 
	\begin{align*}
		\big\Vert S_t x\big\Vert_X&\leq\sup\limits_{s\in[0,1]}\big\Vert S_s x\big\Vert\leq M\cdot\Vert x\Vert&\forall& x\in X,\forall t\in[0,1]\\
		\implies\big\Vert S_t\big\Vert_{\L(X)}&\leq M &\forall& t\in [0,1]
	\end{align*}
	Sei $t\in\R_+$ beliebig. 
	Dann ist $t=n+\delta$ mit $n\in\N_0$ und $\delta\in[0,1[$ und somit
	\begin{align*}
		\big\Vert S_t\big\Vert
		&=\big\Vert S_{n+\delta}\big\Vert
		\overset{\text{(ii)}}{=}
		\big\Vert S_n S_\delta\big\Vert
		=\big\Vert S_1^n S_\delta\big\Vert
		\leq\big\Vert S_1\big\Vert^n\cdot\big\Vert S_\delta\big\Vert
		\leq M^{n+1}\\
		&=\underbrace{M^{n+\delta}}_{=\exp\big((n+\delta)\cdot\log(M)\big)}\cdot M^{1-\delta}
		\overset{\omega=\log(M)}\leq
		M^{1-\delta}\cdot\exp(\omega\cdot t)
		\overset{M\geq1}{\leq}
		M\cdot\exp(\omega\cdot t)
	\end{align*}
\end{proof}

\begin{definition}[Erzeuger]\enter
	Sei $S$ eine lineare Halbgruppe auf einem Banachraum $X$. 
	Dann definieren wir den \textbf{Erzeuger} von $S$
	\begin{align*}
		A&:=\left\lbrace(u,f)\in X\times X~\left|~f=\lim\limits_{t\to0^+}\frac{S_t u-u}{t}\right.\right\rbrace\mit\\
		\dom(A)&\hspace{3pt}=\left\lbrace u\in X~\left|~\lim\limits_{t\to0^+}\frac{S_t u-u}{t}\text{ existiert}\right.\right\rbrace
	\end{align*}
	Der Operator $A$ ist linear und eindeutig.
\end{definition}

\begin{lemma}
	Sei $S$ eine lineare Halbgruppe auf einem Banachraum $X$, mit Erzeuger $A$. 
	Dann gilt:
	\begin{enumerate}[label=(\alph*)]
		\item Für alle $u\in\dom(A)$ und alle $t\in\R_+$ ist $S_t u\in\dom(A)$, die Abbildung $t\mapsto S_t$ ist differenzierbar und 
		\begin{align*}
			\frac{\d}{\d t} S_t u=A S_t u=S_t A u
		\end{align*}
		\item Für alle $u\in X$ und alle $t\in\R_+$ ist
		\begin{align*}
			\int\limits_0^t S_s u\d s\in\dom(A)
		\end{align*}
		und
		\begin{align*}
			A\int\limits_0^t S_s u\d s=S_t u-u
		\end{align*}
		\item Der Erzeuger $A$ ist abgeschlossen und $\dom(A)$ ist dicht in $X$.
	\end{enumerate}
\end{lemma}

\begin{proof}
	\underline{Zeige (a):}\\
	sei $u\in\dom(A)$ und $t\in\R_+$. 
	Dann gilt für $h>0$:
	\begin{align*}
		\frac{S_h\big(S_t u\big)-S_t u}{h}
		\overset{\text{(iii)}}&=
		\frac{S_{h+t} u-S_t u}{h}
		\overset{\text{(iii)}}=
		\frac{S_t\big(S_h u\big)-S_t u}{h}
		=S_t\frac{S_h u-u}{h}\stackrel{h\to0^+}{\longrightarrow}S_t Au
	\end{align*}
	Wegen dieser Konvergenz ist $S_t u\in\dom(A)$ und 
	\begin{align*}
		A S_t u
		&=S_t Au
		=\frac{\d}{\d t}S_t u
	\end{align*}
	
	\underline{Zeige (b):}\\
	Sei $u\in X$, $t\in\R_+$. 
	Dann gilt für $h>0$:
	\begin{align*}
		\frac{S_h\left(\int\limits_0^t S_s u\d s\right)-\int\limits_0^t S_s \d s}{h}
		&=\frac{\int\limits_0^t S_{s+h}-\int\limits_0^t S_s u\d s}{h}\\
		\overset{\text{Subs}}&=
		\frac{1}{h}\cdot\left(\int\limits_h^{t+h} S_s u\d s-\int\limits_0^t S_s u\d s\right)\\
		&=\frac{1}{h}\cdot\left(\int\limits_t^{t+h} S_s u\d s-\int\limits_0^h S_s u\d s\right)
		\stackrel{h\to0^+}{\longrightarrow} S_t u-u 
	\end{align*}
	Also ist 
	\begin{align*}
		\int\limits_0^t S_s u\d s\in\dom(A)\qquad\text{und}\qquad
		A\int\limits_0^t S_s u\d s= S_t u-u
	\end{align*}

	\underline{Zeige (c):}\\
	Sei $u\in X$. 
	Dann ist für alle $t\in\R_+$ (siehe (b)):
	\begin{align*}
		\cdot\int\limits_0^t S_s u \d s\in\dom(A)\\
		\overset{\text{linear}}{\implies}
		\frac{1}{t}\cdot\int\limits_0^t \underbrace{S_s u}_{\stackrel{t\to 0^+}{\longrightarrow} u} \d s\in\dom(A)\\
	\end{align*}
	Also ist $u\in\dom(A)$ und somit $\dom(A)$ dicht in $X$. 
	Außerdem ist $A$ abgeschlossen, denn:\\
	Es gelte
	\begin{align*}
		\dom(A)\ni u_n\stackrel{n\to\infty}{\longrightarrow} u\text{ in }X
		\qquad\text{und}\qquad
		A u_n\stackrel{n\to\infty}{\longrightarrow} f\text{ in }X
	\end{align*}
	Dann gilt für alle $t\in\R_+$ und alle $n$, wegen (a)
	\begin{align*}
		S_t A u_n=\frac{\d}{\d t}S_t u_n
	\end{align*}
	bzw. (Integration)
	\begin{align*}
		\int\limits_0^t S_s A u_n\d s&=S_t u_n-u_n\\
		\downarrow n\to\infty&\qquad\downarrow n\to\infty\\
		\overset{n\to\infty}{\implies}
		\int\limits_0^t S_s f\d s&=S_t u-u\qquad|:t\\
		\implies
		\frac{S_t u-u}{t}&=\frac{1}{t}\cdot\int\limits_0^t S_s f\d s\stackrel{t\to0^+}{\longrightarrow} f
	\end{align*}
	Bei der Grenzwertbildung wird dominierte Konvergenz mit
	\begin{align*}
		\left\Vert S_s A u_n\right\Vert\leq M\cdot\exp(\omega\cdot s)\cdot\big\Vert A u_n\big\Vert
	\end{align*}
	und das erste Lemma verwendet.
	Damit ist $u\in\dom(A)$ und $Au=f$.
\end{proof}

\begin{lemma}
	Sei $S$ eine lineare Halbgruppe auf einem Banachraum $X$, mit Erzeuger $A$. 
	Seien $M\geq1$, $\omega\in\R$ so, dass 
	\begin{align*}
		\big\Vert S_t\big\Vert\leq M\cdot\exp(\omega\cdot t)\qquad\forall t\in\R_+
	\end{align*}
	Dann ist für alle $\lambda\in\R\mit\lambda>\omega$ der Operator $\lambda-A$ invertierbar und 
	\begin{align*}
		(\lambda-a)^{-1}f=\int\limits_0^\infty\exp(-\lambda\cdot t)\cdot S_t f\d t\qquad\forall f\in X
	\end{align*}
	Es gilt:
	\begin{align*}
		\left\Vert(\lambda-A)^{-1}\right\Vert_{\L(X)}\leq\frac{M}{\lambda-\omega}
	\end{align*}
\end{lemma}

\begin{proof}
	Wegen
	\begin{align*}
		\big\Vert\exp(-\lambda\cdot t)\cdot S_t f\big\Vert_X\leq\exp(-\lambda\cdot t)\cdot M\cdot\exp(\omega\cdot t)\cdot\Vert f\Vert
	\end{align*}
	ist $t\mapsto\exp(-\lambda\cdot t)\cdot S_t f$ für alle $f\in X$ und alle $\lambda>\omega$ integrierbar.\\
	Für alle $f\in X$ und alle $\lambda>\omega$ gilt:
	\begin{align*}
		(A-\lambda)\int\limits_0^T\exp(-\lambda\cdot t)\cdot S_t f\d t
		\overset{(\ast)}&{=}
		\exp(-\lambda\cdot t)\cdot S_t f-f\stackrel{T\to\infty}{\longrightarrow}-f
	\end{align*}
	Bei $(\ast)$ wird verwendet, dass $A-\lambda$ Erzeuger der linearen Halbgruppe $t\mapsto\exp(-\lambda\cdot t)\cdot S_t$ ist.\\
	Und weil $A$ abgeschlossen ist, ist
	\begin{align*}
		\int\limits_0^\infty\exp(-\lambda\cdot t)\cdot S_t f\d t&\in\dom(A)
		\qquad\text{und}\qquad
		(\lambda-A)\cdot\int\limits_0^\infty\exp(-\lambda\cdot t)\cdot S_t f\d t &=f
	\end{align*}
	Ähnlich gilt für $f\in\dom(A)$:
	\begin{align*}
		\int\limits_0^\infty\exp(-\lambda\cdot t)\cdot S_t(\lambda-A)f\d t=f
	\end{align*}
	Damit ist der Operator
	\begin{align*}
		R_\lambda:X\to X,\qquad f\mapsto\int\limits_0^\infty\exp(-\lambda\cdot t)\cdot S_t f\d t
	\end{align*}
	Links- und Rechtsinverse von $\lambda-A$. 
	Außerdem gilt:
	\begin{align*}
		\Big\Vert(\lambda-A)^{-1}\Big\Vert_{\L(X)}
		&=\Vert R_\lambda\Vert_{\L(X)}\\
		&=\sup\limits_{\Vert f\Vert\leq 1}\left\Vert\int\limits_0^\infty\exp(-\lambda\cdot t)\cdot S_t f\d t\right\Vert\\
		&\leq\sup\limits_{\Vert f\Vert\leq 1}\int\limits_0^\infty\exp(-\lambda\cdot t)\cdot M\cdot\exp(\omega\cdot t)\cdot\Vert f\Vert\d t\\
		&=\frac{M}{\lambda-w}
	\end{align*}
\end{proof}

Linearer Fall:
$$
\begin{tikzcd}[sep=tiny]
	& & S \arrow[drr, dash, bend left] \arrow[ddr, dash, bend right] \arrow[ddl, dash, bend left] \arrow[dll, dash, bend right]
	& & \\
	\parbox{50pt}{\scriptsize Definition\\(Ableitung\\in $0$)} 
	\arrow[dd, bend right]
	& & & 
	& \parbox{70pt}{\scriptsize Laplacetrans-\\formation}
	\arrow[dd, bend left]	\\
	&\parbox{50pt}{\scriptsize Lösung von \\$\dot{u}-Au=0$} 
	\arrow[uur, bend right]
	& 
	&\parbox{70pt}{\scriptsize Inversion der\\Laplacetrans-\\formation,\\Exponentialformel}
	\arrow[uul, bend left] 
	\\
	A \arrow[ur, dash, bend right] & & \arrow[ll, bend left] \parbox{70pt}{\scriptsize Invertieren von\\Operatoren} \arrow[rr, bend right]& & (\lambda - A)^{-1} \arrow[ul, dash, bend left]
\end{tikzcd}
$$

\begin{align*}
	f\colon\R_+\to X,\qquad\big\Vert f(t)\big\Vert\leq M\cdot\exp(\omega\cdot t)\\
	\hat{f}(x)=\int\limits_0^\infty\exp(-\lambda\cdot t)\cdot f(t)\d t
\end{align*}
Hierbei ist $\hat{f}$ die \textbf{Laplacetransformation} von $f$ ($\lambda\in\C,\Re(\lambda)>0$).\\
$\hat{f}$ ist analytisch / holomorph auf der Halbebene $\big\lbrace\Re(\lambda)>\omega\big\rbrace$. 
\begin{align*}
	\widehat{f'}(x)&=\lambda\cdot\hat{f}(x)-f(0)\\
	\int\limits_0^\infty\exp(-\lambda\cdot t)\cdot f'(t)\d t
	&=\big[\exp(-\lambda\cdot t)\big]_{t=0}^\infty+\lambda\cdot\int\limits_0^\infty\exp(-\lambda\cdot t)\cdot f(t)\d t
\end{align*}

Bezüglich der Wohlgestelltheit des Cauchyproblems mit \ul{linearem} $A$ gilt der folgende "Spezialfall" von Crandall-Ligget \ref{theoremCrandall-Liggett}:

\begin{theorem}[Lumer-Phillips]\enter
	Ein linearer Operator $A\subseteq X\times X$ auf einem Banachraum $X$ ist genau dann Erzeuger einer linearen Halbgruppe $S$ von Kontraktionen 
	(d.h. $\Vert S_t\Vert_{\L(X)}\leq 1$), wenn $-A$ $m$-akkretiv ist und $\dom(A)$ dicht in $X$ liegt.\\
	(Analog: quasi-Kontraktionen, d.h. $\Vert S_t\Vert\leq\exp(\omega\cdot t)$, wenn $-A$ $m$-akkretiv vom Typ $\omega$ und $\dom(A)$ dicht in $X$ ist.
	Wichtig ist die Konstante $M=1$!)
\end{theorem}

\begin{bemerkung}
	Zu der Konstante $M=1$ eine Bemerkung: Sei $S$ eine lineare Halbgruppe auf einem Banachraum $X$,
	\begin{align*}
		\Vert S_t\Vert_{\L(X)}\leq M\cdot\exp(\omega\cdot t)\qquad\forall t\in\R_+
	\end{align*}
	($M\geq1,\omega\in\R$). 
	Dann gibt es auf dem Banachraum $X$ eine äquivalente Norm $\normiii{\cdot}$, so dass
	\begin{align*}
		\normiii{S_t}_{\L(X)}\leq\exp(\omega\cdot t)\qquad\forall t\in\R_+
	\end{align*} 
\end{bemerkung}

\begin{proof}
	Setze
	\begin{align*}
		\normiii{x}:=\sup\limits_{t\geq0}\exp(-\omega\cdot t)\cdot\Vert S_t x\Vert\qquad\forall x\in X
	\end{align*}
	Das ist offenbar eine Norm auf $XR$ und es gilt 
	\begin{align*}
		\normiii{x}\leq M\cdot\Vert x\Vert
	\end{align*}
	(exponentielle Abschätzung für $S$) und 
	\begin{align*}
		\norm{x}=\norm{S_0 x}\leq\normiii{x},
	\end{align*}
	d.h. $\normiii{x}$ \underline{ist} äquivalente Norm auf $X$. 
	Außerdem gilt für alle $x\in X, s\in\R_+$:
	\begin{align*}
		\normiii{S_s x}
		&=\sup\limits_{t\geq0}\exp(-\omega\cdot t)\cdot\big\Vert S_t\cdot S_s\big\Vert\\
		&=\sup\limits_{t\geq0}\exp(-\omega\cdot t)\cdot\big\Vert S_{t+s}\big\Vert\\
		&=\sup\limits_{t\geq0}\exp\big(-\omega\cdot(t+s)\big)\cdot\big\Vert S_{t+s}\big\Vert\cdot\exp(\omega\cdot s)\\
		&=\exp(\omega\cdot s)\cdot\sup\limits_{t\geq s}\exp(-\omega\cdot t)\cdot\big\Vert S_{t}\big\Vert\\\
		&\leq\exp(\omega\cdot s)\cdot\normiii{x}\\
		\implies \normiii{S_s}_{\L(X)}&\leq\exp(\omega\cdot s)
	\end{align*}
\end{proof}

\begin{theorem}[Ohne Namen]\enter
	Ein linearer Operator $A$ auf einem Banachraum $X$ ist genau dann Erzeuger einer linearer Halbgruppe, 
	wenn $\dom(A)$ dicht in $X$ ist und wenn es eine äquivalente Norm $\normiii{\cdot}$ auf $X$ gibt bzgl. der $-A$ $m$-akkretiv vom Typ $\omega$ ist.
\end{theorem}

Mit diesem Theorem (+ obiger Bemerkung) sind alle Erzeuger von linearen Halbgruppen auf Banachräumen charakterisiert.

\begin{beispiel}[nichtlineare Halbgruppen]\enter
	Sei $A$ $m$-akkretiv vom Typ $\omega\in\R$ auf einem Banachraum $X$. Sei $D:=\overline{\dom(A)}$. 
	Aus dem Theorem von Crandall-Ligget \ref{theoremCrandall-Liggett} folgt, dass das Cauchyproblem \eqref{CPchapter3}
	\begin{align*}
		\left\lbrace\begin{array}{r}
			\dot{u}+Au\ni 0\\
			u(0)=u_0
		\end{array}\right.
	\end{align*}
	für alle $u_0\in D$ genau eine Integrallösung $u\in C\big(\R_+,X\big)$ besitzt. 
	Diese ist gleichzeitig Eulerlösung (d.h. Grenzwert von Lösungen des impliziten Eulerschemas). 
	Setze sodann:
	\begin{align*}
		S_t u_0:=u(t)\qquad\forall u_0\in D,\forall t\in\R_+
	\end{align*}
	wobei $u$ die eindeutige Lösung zum Anfangswert. $u_0$ ist. 
	Dann gilt:
	\begin{enumerate}[label=(\roman*)]
		\item $\begin{aligned}
			\forall u_0\in D:S_0 u_0=u_0,\text{ d.h. }S_0=\id_D
		\end{aligned}$
		\item $\begin{aligned}
			\forall u_0\in D,\forall t\geq 0:\R_s\to x,~
			s\mapsto S_{t+s} u_0=u(t+s)\text{ Integrallsg. zum Anfangsw. }u(t)
		\end{aligned}$\\
		Wegen Eindeutigkeit der Integrallösung folgt
		\begin{align*}
			S_{t+s}u_0=S_s\big( S_t u_0\big)\qquad\forall s\geq0
		\end{align*}
		d.h.
		\begin{align*}
			S_{t+s}=S_s\circ S_t
		\end{align*}
		\item Weil Integrallösungen stetige Funktionen $\R_+\to X$ sind, gilt für alle $u_0\in D$:
		\begin{align*}
			\R_+\to X,\qquad t\mapsto S_t u_0
		\end{align*}
		ist stetig. Außerdem (Abschätzung in Crandall-Ligget \ref{theoremCrandall-Liggett}):
		\begin{align*}
			\big\Vert S_t u_0-S_t\hat{u}_0\big\Vert\leq\exp(\omega\cdot t)\cdot\Vert u_0-\hat{u}_0\big\Vert\qquad\forall t\geq0,\forall u_0,\hat{u}_0\in D
		\end{align*}
		d.h. $S_t$ist Lipschitzstetig (und damit $S_t\in C(D,D)$) und 
		\begin{align*}
			\big\Vert S_t\big\Vert_{\text{Lip}}\leq\exp(\omega\cdot t)\qquad\forall t\geq0
		\end{align*}
		Damit ist $S$ eine (stark stetige, nichtlineare) Halbgruppe auf $D$.
	\end{enumerate}
\end{beispiel}

\begin{theorem}[Exponentialformel]\label{theoremExponentialformel}\enter
	Sei $A$ $m$-akkretiv vom Typ $\omega\in\R$ auf einem Banachraum $X$. 
	Dann gilt
	\begin{align*}
		S_t u_0=\limn\left(I+\frac{t}{n}\cdot A\right)^{-n}u_0\qquad\forall u_0\in D:=\overline{\dom(A)},\forall t\in\R_+
	\end{align*}
	wobei $S$ die von $A$ "erzeugte" Halbgruppe ist (wie oben) und
	\begin{align*}
		\big(I+h\cdot A\big)^{-n}=\left((I+h\cdot A)^{-1}\right)^n=J_h^n
	\end{align*}
\end{theorem}

\begin{proof}
	Ist $S$ die von $A$ erzeugte Halbgruppe auf $D$, dann ist $u(t)=S_t u_0$ Integrallösung von
	\begin{align*}
		\left\lbrace\begin{array}{r}
			\dot{u}+Au\ni 0\\
			u(0)=u_0
		\end{array}\right.
	\end{align*}
	bzw. auch Eulerlösung. 
	Sei $t\geq0$ und betrachte auf $[0,t]$ äquidistante Partitionen der Form:
	\begin{align*}
		0=t_0<\frac{t}{n}<\frac{2\cdot t}{n}<\ldots<t=\frac{n\cdot t}{n}
	\end{align*}
	Die Lösungen des zugehörigen impliziten Eulerschemas mit Anfangswert $u_0$ sind dann (im Zeitpunkt $t$) von der Form 
	\begin{align*}
		\left(I+\frac{t}{n}\cdot A\right)^{-n} u_0
	\end{align*}
	Wir wissen, dass die Näherungslösungen gleichmäßig auf $[0,t]$ (insbesondere im Punkt) konvergieren, wenn $n\to\infty$.\\
\end{proof}

Im nichtlinearen Fall:
$$
\begin{tikzcd}[sep=small]
	& & S \arrow[drr, dash, bend left] \arrow[ddr, dash, bend right] \arrow[ddl, dash, bend left] \arrow[dll, dash, bend right]
	& & \\
	\parbox{50pt}{\scriptsize noch nicht\\ bekannt} 
	\arrow[dd, bend right]
	& & & 
	& \parbox{50pt}{\scriptsize noch nicht\\ bekannt}
	\arrow[dd, bend left]	\\
	&\parbox{60pt}{\scriptsize Crandall-\\Ligget,\\Wohlgestelltheit\\ von $\dot{u}+Au\ni0$} 
	\arrow[uur, bend right]
	& 
	&\parbox{70pt}{\scriptsize Exponentialformel}
	\arrow[uul, bend left]
	\\
	A \arrow[ur, dash, bend right] & & \arrow[ll, bend left] \parbox{50pt}{\scriptsize Inversion von\\ nichtlinearen\\Operatoren} \arrow[rr, bend right]& 
	& (\lambda - A)^{-1} \arrow[ul, dash, bend left]
\end{tikzcd}
$$
