% This work is licensed under the Creative Commons
% Attribution-NonCommercial-ShareAlike 4.0 International License. To view a copy
% of this license, visit http://creativecommons.org/licenses/by-nc-sa/4.0/ or
% send a letter to Creative Commons, PO Box 1866, Mountain View, CA 94042, USA.

\chapter{Wohlgestelltheit des Cauchyproblems} %3
\setcounter{section}{1}
\begin{theorem}[Crandall-Liggett]\label{theoremCrandall-Liggett}\enter
	Sei $A\subseteq X\times X$ $m$-akkretiv vom Typ $\omega\in\R$, $X$ ein Banachraum.\\
	Dann besitzt das Cauchy-Problem
	\begin{align}\label{CPchapter3}\tag{CP}
		\text{(CP)}\left\lbrace\begin{array}{rl}
			\hat{u}+Au\ni 0&\text{ in }[0,T]\\
			u(0)=u_0&
		\end{array}\right.
	\end{align}
	für jedes $u_0\in\overline{\dom(A)}$ genau eine Integrallösung $u\in C\big([0,T],X\big)$ (oder $u\in C(\R_+,X)$ geht auch).\\
	Für je zwei Integrallösungen $u,\hat{u}$ von \eqref{CPchapter3} zu Anfangswerten $u_0$ bzw. $\hat{u}_0\in\overline{\dom(A)}$ gilt
	\begin{align*}
		\big\Vert u(t)-\hat{u}(t)\big\Vert
		\leq\exp(\omega\cdot t)\cdot\big\Vert u_0-\hat{u}_0\big\Vert\qquad\forall t\in[0,T]
	\end{align*}
	Setze dann
	\begin{align*}
		S(t)u_0:=u(t)\qquad\forall t\geq0,\forall u_0\in\overline{\dom(A)}
	\end{align*}
	wobei $u$ eindeutige Integrallösung von \eqref{CPchapter3} mit Anfangswert $u_0$ ist $(T\geq t)$:
	\begin{align*}
		S(t)u_0:\overline{\dom(A)}\to\overline{\dom(A)}
	\end{align*}
	ist Lipschitz-stetig mit Lipschitzkonstante $S(0)=\id$ und $S(t+a)=S(t)\circ S(s)$.
\end{theorem}

\begin{proof}
	Wir führen temporär den Begriff der Euler-Lösung ein:\\
	Eine Funktion $u\in C\big([0,T],X\big)$ heißt \textbf{Euler-Lösung} des Cauchyproblems 
	\begin{align*}
		\left\lbrace\begin{array}{rl}
			\dot{u}+Au\ni f\\
			u(0)=u_0
		\end{array}\right.\mit u_0\in X,f\in L^1(0,T,X)
	\end{align*}
	wenn es eine Folge 
	$(\Theta_n)_{n\in\N}=\left(\left(\pi_{\Theta_n},f_{\Theta_n},u_{0,\Theta_n}\right)\right)_{n\in\N}$ 
	von Diskretisierungen und eine Folge $(u_n)_{n\in\N}$ von Lösungen des impliziten Eulerschemas $(E_{\Theta_n})$ gibt mit den Eigenschaften:
	\begin{enumerate}
		\item $\pi_{\Theta_n}$ ist Partition von $[0,T]$
		\item $\limn\big|\pi_{\Theta_n}\big|=0$
		\item $\limn\big\Vert f_{\Theta_n}-f\big\Vert_{L^1}=0$
		\item $\limn\big\Vert u_{0,\Theta_n}-u_0\big\Vert=0$
		\item $\limn\big\Vert u_n-u\big\Vert_{C\big([0,T],X\big)}=0$
	\end{enumerate}
	Für Existenz einer Eulerlösung:\\
	Sei $\Theta_n=\big(\pi_{\Theta_n},0,u_{0,\Theta_n}\big)$ eine Folge von Diskretisierungen so, dass 
	\begin{enumerate}
		\item $\pi_{\Theta_n}$ ist Partition von $[0,T]$
		\item $\limn\big|\pi_{\Theta_n}\big|=0$ und $\big|\pi_{\Theta_n}\big|\cdot\omega<1$
		\item $\limn\big\Vert u_{0,\Theta_n}-u_0\big\Vert=0$
	\end{enumerate}
	wobei $u_0\in\overline{\dom(A)}$ gegeben ist.\\
	Weil $A$ $m$-akkretiv vom Typ $\omega$ ist, besitzt das implizite Eulerschema $(E_{\Theta_n})$ genau eine Lösung 
	$u_{\Theta_n}\in C\big([0,T],X\big)$. Erinnerung an das implizite Eulerschema:
	\begin{align*}
		\left\lbrace\begin{array}{c}
			u_{\Theta_n}(t_{i+1})+h_i\cdot A u_{\Theta_n}(t_{i+1})\ni u_{\Theta_n}(t_i)\\
			u(\underbrace{0}_{=t_0})=u_{0,\Theta_n}
		\end{array}\right.
	\end{align*}
	Dann gilt für alle $\hat{u}\in\dom(A)$:
	\begin{align*}
		&\limsup\limits_{n,m\to\infty}\left\Vert u_{\Theta_n}-u_{\Theta_m}\right\Vert_{C\big([0,T],X\big)}\\
		&=\limsup\limits_{n,m\to\infty}\sup\limits_{t\in[0,T]}\left\Vert u_{\Theta_n}(t)-u_{\Theta_m}(t)\right\Vert_X\\
		\overset{\ref{theoremEndeChpater2}}&\leq
		\limsup\limits_{n,m\to\infty}\sup\limits_{t\in[0,T]}
		\left(\big\Vert u_{0,\Theta_n}-\hat{u}\big\Vert+\big\Vert u_{0,\Theta_m}-\hat{u}\big\Vert+\sqrt{\big(|\pi_{\Theta_n}|+|\pi_{\Theta_m}|\big)}\cdot\Vert A\hat{u}\Vert\right)\\
		&= 2\cdot\big\Vert u_0-\hat{u}\big\Vert
	\end{align*}
	Nun bilden wir das Infimum über $\hat{u}$ und erhalten
	\begin{align*}
		\limsup\limits_{n,m\to\infty}\left\Vert u_{\Theta_n}-u_{\Theta_m}\right\Vert_{C\big([0,T],X\big)}=0
	\end{align*}
	Also ist $\big(u_{\Theta_n}\big)_{n\in\N}$ eine Cauchyfolge in $C\big([0,T],X\big)$ und wegen der Vollständigkeit konvergent. 
	Der Grenzwert $u:=\limn u_{\Theta_n}$ ist also eine Eulerlösung von \eqref{CPchapter3}. 
	Der Beweis zeigt, dass $u$ nicht von der Wahl der Folge $(\Theta_n)_{n\in\N}$ abhängt.\\
	Für den Rest des Theorems zeigt man, dass für zwei Eulerlösungen $u,\hat{u}$ 
	(bzw. eine Eulerlösung $u$ und eine Integrallösung $\hat{u}$) die Abschätzung 
	\begin{align*}
		&\big\Vert u(t)-\hat{u}(t)\big\Vert\\
		&\leq\big\Vert u(s)-\hat{u}(s)\big\Vert+\omega\cdot\int\limits_s^t\big\Vert u(\tau)-\hat{u}(\tau)\big\Vert\d\tau
		+\int\limits_s^t\big[u(\tau)-\hat{u}(\tau),f(\tau)-\hat{f}(\tau)\big]\d\tau
	\end{align*}
	für alle $0\leq s\leq t$ gilt. (Für uns ist die rechte Seite $f\equiv 0$.) Aus der Abschätzung folgt:
	\begin{enumerate}
		\item Die konstanten Lösungen sind zulässige Eulerlösungen, also:
		\begin{align*}
			\forall(\hat{u},\hat{f}):
			\Vert u(t)-\hat{u}\Vert\leq\Vert u(s)-\hat{u}\Vert
			+\omega\cdot\int\limits_s^t\Vert u(\tau)-\hat{u}\Vert\d\tau
			+\int\limits_s^t\big[u(\tau)-\hat{u},-\hat{f}\big]\d\tau
		\end{align*}
		für alle $0\leq s\leq t$, d.h. jede Eulerlösung \underline{ist} Integrallösung.
		\item (Mit dem Lemma von Gronwall folgt:) Für jede Eulerlösung $u$ und jede Integrallösung $\hat{u}$ (zu rechten Seiten $f=\hat{f}=0$)
		\begin{align*}
			\big\Vert u(t)-\hat{u}(t)\big\Vert
			\leq\exp(\omega\cdot t)\cdot\big\Vert u(0)-\hat{u}(0)\big\Vert\qquad\forall t\geq 0
		\end{align*}
		woraus Eindeutigkeit von Integral- (und Euler-)lösungen und die Abschätzung im Theorem folgen.
	\end{enumerate}
\end{proof}

\begin{theorem}[Regularität]\label{theoremChapter3Regularitaet}\enter
	Voraussetzungen wie in Crandall-Ligget. 
	Dann ist für alle $u_0\in\dom(A)$ die eindeutige Integrallösung $u$ von 
	\begin{align*}
		\left\lbrace\begin{array}{c}
			\dot{u}+Au\ni 0\\
			u(0)=u_0
		\end{array}\right.
	\end{align*}
	Lipschitzstetig mit Lipschitzkonstante
	\begin{align*}
		\Vert A u_0\Vert=\inf\big\lbrace\Vert f\Vert:f\in A u_0\big\rbrace
	\end{align*}
\end{theorem}

Wenn der Banachraum $X$ die \textit{Radon-Nikodym-Eigenschaft} besitzt, dann ist für jedes 
$u_0\in\dom(A)$ die Integrallösung $u\in W^{1,\infty}\big([0,T],X\big)$ eine starke Lösung und für 
$u_0\in\overline{\dom(A)}$ eine milde Lösung.\\
Hierzu die Beobachtung:\\
Jede Funktion $u\in W^{1,\infty}\big([0,T],X\big)$ ist \ul{Lipschitzstetig}, denn für $t\geq s$ gilt:
\begin{align*}
	\big\Vert u(t)-u(s)\big\Vert
	&=\left\Vert\int\limits_s^t u'(\tau)\d\tau\right\Vert\\
	\overset{\text{DU}}&\leq
	\int\limits_s^t\big\Vert u'(\tau)\big\Vert\d\tau\\
	&\leq\Vert u'\Vert_{L^\infty(0,T,X)}\cdot(t-s)
\end{align*}
Tatsächlich ist $W^{1,\infty}\big([0,T],X\big)$ ein abgeschlossener Unterraum des Banachraums\\ $\Lip\big([0,T],X\big)$. 
(die Normen
$\Vert u\Vert_{W^{1,\infty}}=\Vert u\Vert_{L^\infty}+\Vert u'\Vert_{L^\infty}$ und\\ 
$\Vert u\Vert_{\Lip}=\text{ Lipschitzkonstante von }u+\Vert u\Vert_{L^\infty}$ sind äquivalent.)\\
Im Fall von $X=\R$ oder $X=\C$ gilt der Satz von Radon-Nikodym bzw. der Satz von Rademacher:

\begin{satz}[Satz von Rademacher]\label{satzRademacher}
	\begin{align*}
		W^{1,\infty}\big([0,T]\big)=\Lip\big([0,T]\big)
	\end{align*}
\end{satz}

\begin{proof}
	Folgt aus dem Satz von Radon-Nikodym.
\end{proof}

Für beliebige Banachräume $X$ gilt die Gleichheit
\begin{align*}
	W^{1,\infty}\big([0,T],X\big)=\Lip\big([0,T],X\big)
\end{align*}
im Allgemeinen nicht.

\begin{beispiel}
	Sei $X=L^1(0,1)$ und betrachte
	\begin{align*}
		f:[0,1]\to X=L^1(0,1),\qquad f(t):=\indi_{[0,t]}
	\end{align*}
	Dann gilt für $t\geq s$:
	\begin{align*}
		\big\Vert f(t)-f(s)\big\Vert_{L^1(0,1)}
		=\left\Vert\indi_{[s,t]}\right\Vert_{L^1(0,1)}
		=|t-s|
	\end{align*}
	d.h. $f$ ist Lipschitzstetig mit Lipschitzkonstante 1.
	Somit gilt $f\in\Lip\big([0,T],X\big)$, aber $f\not\in W^{1,\infty}\big([0,T],X\big)$.
\end{beispiel}

\begin{definition}
	Ein Banachraum $X$ hat die \textbf{Radon-Nikodym-Eigenschaft}
	\begin{align*}
		:\Longleftrightarrow W^{1,\infty}\big([0,T],X\big)=\Lip\big([0,T],X\big)
	\end{align*}
\end{definition}

\begin{beispiel}\
	\begin{enumerate}
		\item reflexive Banachräume haben die Radon-Nikodym-Eigenschaft
		\item $L^1(0,1)$ hat die Eigenschaft nicht
		\item $l^1(\N)=l^1$ hat die Radon-Nikodym-Eigenschaft\\
		(siehe Buch von Diestel-Uhl: \textit{Vector measures})
	\end{enumerate}
\end{beispiel}

\begin{proof}[Beweis von Theorem \ref{theoremChapter3Regularitaet}]\enter
	Sei $u_0\in\dom(A)$ und sei $u$ (eindeutige) Integrallösung von $\dot{u}+Au\ni 0,~u(0)=u_0$. 
	Seien $\Theta_n:=\big(\pi_{\Theta_n},0,u_0\big)$ Diskretisierungen von \eqref{CPchapter3} mit
	\begin{align*}
		\lim\big|\pi_{\Theta_n}\big|=0\qquad\text{und}\qquad\big|\pi_{\Theta_n}\big|\cdot\omega<1
	\end{align*}
	und sei $u_{\Theta_n}$ die Lösung von $(E_{\Theta_n})$. 
	Dann gilt:
	\begin{align*}
		\big\Vert u_{\Theta_n}-u\big\Vert_{C\big([0,T],X\big)}\stackrel{n\to\infty}{\longrightarrow}0
	\end{align*}
	und insbesondere für alle $t,\hat{t}\in[0,T]$:
	\begin{align*}
		\big\Vert u(t)-u(\hat{t})\big\Vert_X
		&=\limn\left\Vert u_{\Theta_n}(t)-u_{\Theta_n}(\hat{t})\right\Vert_X\\
		\overset{\ref{theoremEndeChpater2},~\hat{u}\stackeq{!}0}&\leq
		\limn\sqrt{(t-\hat{t})^2+\big|\pi_{\Theta_m}\big|\cdot t+\big|\pi_{\Theta_n}\big|\cdot\hat{t}}\cdot\Vert A u_0\Vert\\
		&=\big| t-\hat{t}\big|\cdot\Vert Au_0\Vert
	\end{align*}
	Also ist $u$ Lipschitzstetig mit Lipschitzkonstante $\Vert A u_0\Vert$. 
	Falls $X$ die Radon-Nikodym-Eigenschaft hat, dann ist $u\in W^{1,\infty}\big([0,T],X\big)$. 
	Lemma ($u$ Integrallösung + $u\in W^{1,1}$) $\implies u$ ist starke Lösung. 
	Falls $X$ Radon-Nikodym-Eigenschaft hat und $i_0\in\overline{\dom(A)}$, 
	dann existiert eine Folge $(u_0^n)_{n\in\N}\subseteq\dom(A)$ mit $\limn\big\Vert u_0^n-u_0\big\Vert=0$.\\
	Die Lösungen $u_n$ zu den Anfangswerten $u_0^n$ sind starke Lösungen und konvergieren aufgrund der Abschätzung 
	\begin{align*}
		\big\Vert u_n(t)-u(t)\big\Vert\leq\exp(\omega\cdot t)\cdot\big\Vert u_0^n-u_0\big\Vert
	\end{align*}
	gleichmäßig auf $[0,T]$ gegen Lösung $u$ zum Anfangswert $u_0$. 
	Folglich ist $u$ eine milde Lösung.
\end{proof}

\begin{korollar}[Folgerungen zu Crandall-Ligget]\
	\begin{enumerate}[label=(\alph*)]
		\item Picard-Lindelöf: Sei $X=\R^n$ und sei $f:X\to X$ Lipschitz-stetig. 
		Dann besitzt die ODE
		\begin{align*}
			\left\lbrace\begin{array}{rr}
				\dot{u}+f(u)=0 & t\in[0,T]\\
				u(0)=u_0
			\end{array}\right.
		\end{align*}
		für alle $u_0\in X=\R^n$ genau eine Lösung $u\in W^{1,\infty}(0,T;\R^n)$. 
		Für je zwei Lösungen $u$ und $v$ zu Anfangswerten $u_0$ bzw. $v_0$ gilt
		\begin{align*}
			\big\Vert u(t)-v(t)\big\Vert\leq\exp(L\cdot t)\cdot\big\Vert u_0-v_0\big\Vert\qquad\forall t\in[0,T]
		\end{align*}
		wobei $L$ die Lipschitzkonstante von $f$ ist.\\
		(Der Operator $A=\big\lbrace(u,f(u)):u\in\R^n\big\rbrace=\graph(f)$ ist $m$-akkretiv vom Typ $L$, $\dom(A)=X$)\nl
		In den meisten Formulierungen von Picard-Lindelöf ist $f$ nur \ul{lokal} Lipschitz-stetig 
		(und hängt meist noch explizit von der Variable $t$ ab). Sei also $f\colon D\to\R^n$ ($D\subseteq \R^n$ offen) nur lokal Lipschitz-stetig, d.h. 
		\begin{align*}
			\forall u_0\in D:\exists r>0,\exists L\geq0:\forall u,\hat{u}\in B(u_0,r)\subseteq D:\big\Vert f(u)-f(\hat{u})\big\Vert\leq L\cdot\big\Vert u-\hat{u}\big\Vert
		\end{align*}
		Dann ist $f|_{B(u_0,r)}$ akkretiv vom Typ $L$. 
		Nach dem Lemma von Zorn besitzt diese Einschränkung eine maximal akkretive (vom Typ $L$) Fortsetzung.\\
		O.B.d.A. ist $\R^n$ mit der euklidischen Norm versehen. 
		Aus dem Theorem von Minty \ref{theoremMinty} folgt, dass jede maximal akkretive Fortsetzung von $f|_{B(u_0,r)}$ $m$-akkretiv (vom Typ $L$) ist.\\
		Sei $F$ eine solche $m$-akkretive Fortsetzung von $f|_{B(u_0,r)}$. 
		Dann besitzt 
		\begin{align*}
			\left\lbrace\begin{array}{r}
				\dot{u}+F(u)\ni 0\\
				u(0)=u_0
			\end{array}\right.
		\end{align*}
		eine globale Lösung $u\in C(\R_+,\R^n)$. Wegen Stetigkeit von $u$ gilt
		\begin{align*}
			u(t)\in B(u_0,r)\qquad\forall t\in[0,t_0]
		\end{align*}
		und 
		\begin{align*}
			F|_{B(u_0,r)}=f|_{B(u_0,r)}
		\end{align*}
		(Übung: Zeige, dass $F$ auf $B$ noch eine Funktion ist, und keine Relation).
		\item Sei $\Omega\subseteq\R^n$ offen und $p\in]1,\infty[$. Dann besitzt die \textbf{Diffusionsgleichung}
		\begin{align}\label{eqDiffusionsgleichung}\tag{Diffu}
			\left\lbrace\begin{array}{rl}
				\partial_t u(t,x)-\Delta_p u(t,x)=0 &\text{ in }(0,\infty)\times\Omega\\
				\partial_\nu u(t,x)=0 &\text{ in }(0,\infty)\times\partial\Omega\\
				u(0,x)=u_0(x) &\text{ in }\Omega
			\end{array}\right.
		\end{align}
		für alle $u_0\in L^2(\Omega)$ genau eine  "Lösung" $u\in C\big(\R_+, L^2(\Omega)\big)$.\\
		($u:(0,\infty)\times\Omega\to\R,~u(t,\cdot)\in L^2(\Omega),~u(t)(x):=u(t,x)$\\
		Zur Erinnerung: 
		$\Delta_p u=\div\left(|\nabla u|^{p-2}\cdot\nabla u\right)$ ist der $p$-Laplaceoperator und $\nu$ die äußere Normale an $\Omega$.)
		\begin{align*}
			-\Delta_p&=\partial\E\qquad\mit\\
			\E(u)&=\left\lbrace\begin{array}{cl}
				\frac{1}{p}\cdot\int\limits_\Omega |\nabla u|^p, &\falls\nabla u\in L^p(\Omega)\\
				+\infty ,&\sonst
			\end{array}\right.
		\end{align*}
		\item Ist $A\subseteq X\times X$ $m$-akkretiv vom Typ $\omega\in\R$ und $f\in X$, dann ist der \textbf{verschobene Operator}
		\begin{align*}
			A-f:=\Big\lbrace\big(\hat{u},\hat{f}-f\big):\big(\hat{u},\hat{f}\big)\in A\Big\rbrace
		\end{align*}
		wieder $m$-akkretiv vom Typ $\omega$. Aus Crandall-Ligget \ref{theoremCrandall-Liggett} folgt also, für $m$-akkretive $A$ vom Typ $\omega$, dass 			das Cauchyproblem
		\begin{align*}
			\left\lbrace\begin{array}{rl}
				\dot{u}+Au\ni f & t\in[0,T]\\
				u(0)=u_0
			\end{array}\right.
		\end{align*}
		für alle $u_0\in\overline{\dom(A)}$ und alle $f\in X$ genau eine Integrallösung $u\in C\big([0,T],X\big)$ besitzt. 
		Setzt man solche Integrallösungen zusammen, dann sieht man, dass das obige Cauchy-Problem für alle $u_0\in\overline{\dom(A)}$ und alle Treppenfunktionen $f\colon[0,T]\to X$\\
		(d.h. es existiert Partition $\pi:0=t_0<t_1<\ldots<t_N$ von $[0,T]$ so, dass $f$ auf allen $[t_i,t_{i+1})$ konstant ist) genau eine Integrallösung $u\in C\big([0,T],X\big)$ besitzt.\\
		Aus der Abschätzung
		\begin{align*}
			&\big\Vert u(t)-\hat{u}(t)\big\Vert\\
			&\leq
			\big\Vert u(s)-\hat{u}(s)\big\Vert
			+\omega\cdot\int\limits_s^t\big\Vert u(\tau)-\hat{u}(\tau)\big\Vert\d\tau
			+\int\limits_s^t\big[u(\tau)-\hat{u}(\tau),f(\tau)-\hat{f}(\tau)\big]\d\tau
		\end{align*}
		für je zwei Integrallösungen $u,\hat{u}$ zu Treppenfunktionen $f,\hat{f}$ einem Approximationsargument 
		(approximiere $f\in L^1(0,T,X)$ durch Treppenfunktionen $f_n$,\\$\Vert f_n-f\Vert_{L^1}\stackrel{n\to\infty}{\longrightarrow}0$) folgt, 
		dass das obige Cauchyproblem für alle $u_0\in\overline{\dom(A)}$ und alle $f\in L^1(0,T;X)$ genau eine Integrallösung besitzt.
	\end{enumerate}
\end{korollar}
