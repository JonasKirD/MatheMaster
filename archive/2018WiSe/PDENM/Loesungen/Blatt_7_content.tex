% This work is licensed under the Creative Commons
% Attribution-NonCommercial-ShareAlike 4.0 International License. To view a copy
% of this license, visit http://creativecommons.org/licenses/by-nc-sa/4.0/ or
% send a letter to Creative Commons, PO Box 1866, Mountain View, CA 94042, USA.
% vim: set noexpandtab:

\section{Aufgabenblatt 7}
\setcounter{section}{6}
\setcounter{subsection}{2}
\subsection{Aufgabe 6.3}
Betrachte
\begin{align*}
	-ε Δ u + b·∇u + cu = f \text{ in } u=0 \text{ auf } \Rand{Ω}
\end{align*}
mit $0 < ε \ll 1$ unter der Annahme
\begin{align*}
	c-\frac12 \div(b)\geq c_0 > 0 \text{ in } Ω
\end{align*}
Weiter untersuchen wir das Standard-Galerkin-FEM Verfahren auf einer regulären, affin äquivalenten Zerlegung von $Ω$. 
Es ist bekannt, dass die entstehende Bilinearform $a$ koerziv auf $H_0^1(Ω)$ bezüglich der Norm
\begin{align*}
	\norm{·} \coloneqq \left(ε \halfnorm{·}_{1, 2, Ω}^2+ \norm{·}_{0, 2, Ω}^2\right)^{\frac12}
\end{align*}
ist.
\subsubsection{Aufgabe 6.3 (a)}
Warum ist auf dem Standardweg (mit Hilfe des Céa-Lemmas \ref{theorem2.2CeasLemma}) in diesem Fall keine von $ε$ unabhängige Fehlerschätzung in der $\norm{·}_{ε}$-Norm möglich?

\begin{lösung}
	Da die Rechnung mit der Divergenz des Vektorfeldes $b$ zwei Mal auftaucht,
	erwähne ich sie hier.

\begin{lemma}[Divergenz-Lemma]\label{thm:aufg6.3-divergenz_lemma}\}\enter
	Für $b ∈ C^1(Ω)$ und $u, v ∈ H_0^1(Ω)$ gilt
	\begin{align*} 
		∫_{Ω} b · ∇u v &= - ∫_{Ω} u b · ∇v + \div b u v \\
			\text{und } ∫_Ω \left(b·∇u\right) u &= -\frac 12 ∫_{Ω} \div b \abs u^2.
		\end{align*}
	\end{lemma}

	\begin{proof}
		Die Produktregel für $b_i u$ ($i$ durchläuft die Koordinaten in $Ω$) ergibt
		$(b_i u)_{x_i} = (b_i)_{x_i} u + b_i u_{x_i}$, also
		$(b_i u)_{x_i} - (b_i)_{x_i} u = b_i u_{x_i}$.
		Mit partieller Integration sehen wir für $u, v ∈ H_0^1(Ω)$
		\begin{align*}
			∫_{Ω} \left(b · ∇u\right) v
			= ∫_{Ω} Σ_i b_i u_{x_i} v
			&= ∫_{Ω} Σ_i {(b_i u)}_{x_i} v - {(b_i)}_{x_i} u v
			= - ∫_{Ω} Σ_i b_i u v_{x_i} + {(b_i)}_{x_i} u v \\
			&= - ∫_{Ω} u \left(b · ∇v\right) + \div(b) u v. \\
			\intertext{Für den Fall $u = v$ vereinfacht sich dies zu}
			∫_{Ω} \left(b · ∇u\right) u
			&= - ∫_{Ω} \left(b · ∇u\right) u + \div b \abs u^2 \\
			⇒ ∫_Ω \left(b·∇u\right) u &= -\frac 12 ∫_{Ω} \div b \abs u^2. \qedhere
		\end{align*}
	\end{proof}

	Die Bedingung an $c$ und $b$ garantiert, dass es eine eindeutige Lösung der
	Gleichung gibt. Céas Lemma \ref{theorem2.2CeasLemma} gibt uns eine Fehlerschranke,
	die von der Stetigkeits- und Koerzivitätsschranke der Bilinearform abhängig ist.
	Die Bilinearform $a$ ist
	\begin{align*}
		a(u, v) = ∫_{Ω} ε ∇u · ∇v + b·∇u v + cuv \quad u, v ∈ H_0^1(Ω)
	\end{align*}
	Um die Stetigkeit von $a$ nachzuweisen, rechne wie üblich:
	\begin{align*}
		\abs {a(u, v)} & \leq \left(∫_{Ω} ε \abs{∇ u}^2 \right)^{\frac12}
		\left(∫_{Ω} ε \abs{∇ v}^2 \right)^{\frac12}
		+ \norm{b}_{∞} \left(∫_{Ω} \abs{∇u}^2\right)^{\frac12}
		\left(∫_{Ω} \abs{v}^2\right)^{\frac12} \\
		&\tab + \norm{c}_{∞} \left(∫_{Ω} \abs{u}^2\right)^{\frac12}
		\left(∫_{Ω} \abs{v}^2\right)^{\frac12} \\
		&= ε^{\frac12} \halfnorm{u}_{1, 2, Ω} ε^{\frac12} \halfnorm{v}_{1,2,Ω}
		+ \norm{b}_{∞} \halfnorm{u}_{1, 2, Ω} \norm{v}_{0, 2, Ω}
		+ \norm{c}_{∞} \norm{u}_{0, 2, Ω} \norm{v}_{0, 2, Ω} \\
		&\leq ε^{-\frac12}\tilde M(b, c) \left(ε \halfnorm{u}_{1, 2, Ω} + \norm{u}_{0, 2, Ω} \right)^{\frac12}
		\left(ε \halfnorm{v}_{1, 2, Ω} + \norm{v}_{0, 2, Ω}\right)^{\frac12} \\
		&= M(ε, b, c) \norm{u}_{ε} \norm{v}_{ε}.
	\end{align*}
	Die Stetigkeitskonstante $M$ ist also von $ε$ abhängig und steigt mit fallendem $ε$.
	Das wird auch nicht durch die Koerzivitätskonstante $α$ ausgeglichen.
	Berechne diese
	\begin{align*}
		a(u, u)
		&= ∫_{Ω} ε \abs{∇u}^2 + b · ∇u u + c u^2 \\
		\overset{\ref{thm:aufg6.3-divergenz_lemma}}&=
		ε \halfnorm u_{1, 2, Ω}^2 + ∫_{Ω} \left(c - \frac12 \div b\right) \abs u^2 \\
		\overset{\text{Vor.}}&{\geq}
		ε \halfnorm u_{1, 2, Ω}^2 + c_0 \norm{u}_{0, 2, Ω}^2
		\geq \min\{1, c_0\} \norm u_ε.
	\end{align*}
	Die Koerzivitätskonstante ist also $α = \min\{1, c_0\}$ und damit von $ε$ unabhängig.
	Damit ergibt Céas Lemma \ref{theorem2.2CeasLemma} keine $ε$-unabhängige Abschätzung, denn es besagt
	\begin{align*}
		\norm{u - u_h}_{ε} \leq \frac {ε^{-\frac12}M}{α} \inf_{v_h ∈ V} \norm{u - v_h}_{ε}
		\leq \frac {ε^{-\frac12}M}{α} C \halfnorm u_{2, 2, Ω}
	\end{align*}
	\begin{bemerkungnr} \label{bemerkung-aufg6.3Interpolationsfehler}
		Um einzuschätzen wie gut diese Abschätzung ist, sei hier eine Abschätzung für den Interpolationsfehler erwähnt, die in Lemma \ref{theorem4.16} gezeigt wurde:
		\begin{align*}
			% \norm{u - u_h}_{ε} & \leq \frac M{α} \inf_{v_h ∈ V} \norm{u - v_h}_{ε} \\
			\norm{u - I_hu}_{ε}
			& = \left(ε \halfnorm{u - I_h u}_{1, 2, Ω}^2 + \norm{u - I_hu}_{0, 2, Ω}^2\right)^{\frac12} \\
			& \leq \left(C ε h^2 \halfnorm u_{2, 2, Ω}^2 + C h^4 \halfnorm u_{2, 2, Ω}^2\right)^{\frac 12} \\
			& \leq C \left(ε^{\frac 12} h + h^2 \right) \halfnorm u_{2, 2, Ω} \\
			⇒ \norm{u - u_h}_{ε}
			& \leq \frac{\tilde M}{α} ε^{- \frac 12} C \left(ε^{\frac 12} h + h^2\right) \halfnorm u_{2, 2, Ω} \\
			&= C \left(h + \frac {h^2}{ε^{\frac 12}}\right) \halfnorm u_{2, 2, Ω}
		\end{align*}
	\end{bemerkungnr}
\end{lösung}

\subsubsection{Aufgabe 6.3 (b)}
Beweisen Sie stattdessen Term für Term für lineare Elemente eine Fehlerabschätzung der Form
\begin{align*}
	\norm {u-u_h}_{ε} \leq C \left(ε^{\frac12}h + h + h^2 \right)\halfnorm{u}_{2, 2, Ω}.
\end{align*}

\begin{proof} %[Simon Bechers erfolgreiche Variante]
	Die Koerzivität von $a$ und die Galerkin-Orthogonalität können wir nutzen, um $\norm{u-u_h}_{ε}$ abzuschätzen.
	Dabei sei $v_h ∈ V_h$ erstmal beliebig, sodass wir später einen passenden Kandidaten wählen können.
	\begin{align*}
		α \norm{u-u_h}_{ε}^2
		&\leq a(u-u_h, u-u_h) \\ %& \text{(Koerzivität)}\\
		&= a(u-u_h, u-v_h) \quad ∀ v_h ∈ V_h & \text{(Galerkin)} \\
		&= ε \scaProd{∇(u-u_h)}{∇(u-v_h)}_{0, Ω} & \eqqcolon \text{(I)} \\
		& \tab + \scaProd{b · ∇(u-u_h)}{u-v_h}_{0, Ω} & \eqqcolon \text{(II)} \\
		& \tab + \scaProd{c (u-u_h)}{u-v_h}_{0, Ω} & \eqqcolon \text{(III)}\\
		\text{(I)} %ε \scaProd{∇(u-u_h)}{∇(u-v_h)}_{0, Ω}
		\overset{\text{CS}}&{\leq}
		ε \halfnorm{u-u_h}_{1,2,Ω} \halfnorm{u-v_h}_{1,2,Ω} \\
		&\leq \norm{u-u_h}_{ε} ε^{\frac 12}\halfnorm{u-v_h}_{1,2,Ω} \\
		\text{(II)} %scaProd{b · ∇(u-u_h)}{u-v_h}_{0, Ω}
		\overset{\ref{thm:aufg6.3-divergenz_lemma}}&=
		\scaProd{-(u-u_h)}{b · ∇(u-v_h)}_{0, Ω}
		& \eqqcolon \text{(IV)} \\
		& \tab - \scaProd{\div(b)(u-u_h)}{u-v_h}_{0,Ω} & \eqqcolon \text{(V)}\\
		\text{(IV)} %\scaProd{-(u-u_h)}{b · ∇(u-v_h)}_{0, Ω}
		\overset{\text{CS}}&{\leq}
		\norm{u-u_h}_{0,2,Ω}\norm{b}_{0,∞,Ω}\halfnorm{u-v_h}_{1,2,Ω} \\
		&\leq \norm{u-u_h}_{ε}\norm{b}_{0,∞,Ω}\halfnorm{u-v_h}_{1,2,Ω} \\
		\text{(III)} + \text{(V)}
		\overset{\text{CS}}&{\leq}
		\norm{c - \div(b)}_{0, ∞, Ω} \norm{u - u_h}_{0, 2, Ω} \norm{u - v_h}_{0, 2, Ω} \\
		&\leq \norm{c - \div b}_{0, ∞, Ω} \norm{u - u_h}_{ε} \norm{u - v_h}_{0, 2, Ω}
		\intertext{Alles zusammen ergibt}
		α \norm{u - u_h}_{ε}^2
		&\leq \norm{u - u_h}_{ε}
		\bigg( \left(ε^{\frac12} + \norm{b}_{0, ∞, Ω}\right) \underbrace{\halfnorm{v - v_h}_{1, 2, Ω}}_{\leq C h \halfnorm u_{2, 2, Ω}} \\
		& \tab + \norm{c - \div(b)}_{0, ∞, Ω} \underbrace{\norm{u - v_h}_{0, 2, Ω}}_{\leq C h^2 \halfnorm u_{2, 2, Ω}} \bigg) \\
		&\leq C \norm{u - u_h}_{ε} (ε^{\frac12}h + h + h^2) \halfnorm u_{2, 2, Ω} \\
			⇒ \norm{u - u_h}_{ε}
			&\leq C \left(ε^{\frac12}h + h + h^2\right) \halfnorm u_{2, 2, Ω} \\
	\end{align*}
	Damit ist die geforderte Schranke gefunden. $C$ hängt von $α$, $b$ und $c$ ab, aber nicht von $ε$ und $h$.
	Leider ist diese Abschätzung wenig hilfreich, denn $\halfnorm u_{2, 2, Ω}$ steigt unbeschränkt für $ε → 0$.
\end{proof}
% \begin{proof}[Felix' schlechte Variante]
% 	Wir nutzen die Ideen aus Theorem \ref{theorem7.2} um den Fehler $\norm{u - u_h}_{ε}$ abzuschätzen. Leider kommt lediglich das Ergebnis
% 	\begin{align*}
% 		\norm{u - u_h}_{ε} \leq C (ε^{\frac 12} h + \frac {h^2}{ε^{\frac 12}} + h^2) \halfnorm u_{2, 2, Ω},
% 	\end{align*}
% 	wobei $C$ von $ε$ und $h$ unabhängig ist.
% 	Das ist für $h > ε^{\frac 12}$ schlechter als die geforderte Schranke.
% 	Im Theorem \ref{theorem7.2} kommt der Term $\frac {h^2}{ε^{\frac 12}}$ nicht vor, denn die dort genutzte $\norm{·}_{\text{SD}}$ beinhaltet einen Anteil in Richtung $b$, der in der $\norm{·}_{ε}$-Norm nicht auftaucht.
% 	Daher bezweifle ich (Felix), dass es besser geht.
%
% 	Wie im Beweis von Theorem \ref{theorem7.2} sei
% 	\begin{align*}
% 		I_h u &∈ V_h \text{ die Interpolierte von $u$} \\
% 		w_h \overset{\text{Def}}&= I_hu - u_h ∈ V_h, \\
% 		z_h \overset{\text{Def}}&= I_hu - u.
% 	\end{align*}
% 	Wir nutzen die Dreiecksungleichung
% 	\begin{align*}
% 		\norm{u - u_h}_{ε} \leq \norm{u - I_h u}_{ε} + \norm{I_h - u_h}_{ε} = \norm{w_h}_{ε} + \norm{z_h}_{ε}.
% 	\end{align*}
% 	Um den ersten Term abzuschätzen, nutze die Koerzivität von $a$ und die Galerkin-Orthogonalität:
% 	\begin{align*}
% 		\norm{w_h}_{ε}^2
% 		\leq α^{-1} a(w_h, w_h)
% 		&= α^{-1} (a(z_h, w_h) + \underbrace{a(u - u_h, w_h)}_{w_h ∈ V_h ⇒ = 0}) \\
% 		&\leq α^{-1} \abs{∫_{Ω} ε ∇z_h · ∇w_h} + \abs{∫_{Ω} (b · ∇z_h) w_h + c z_h w_h}.
% 	\end{align*}
% 	Nun schätze die einzelnen Terme mit Cauchy-Schwarz und Theorem \ref{theorem7.2} ab:
% 	\begin{align*}
% 		\abs{∫_{Ω} ε ∇z_h · ∇w_h}
% 		&\leq ε^{\frac 12} \halfnorm{z_h}_{1, 2, Ω} ε^{\frac 12} \halfnorm{w_h}_{1, 2, Ω} \\
% 		&\leq C ε^{\frac 12} h \halfnorm{u}_{2, 2, Ω} \norm{w_h}_{ε} \\
% 		\abs{∫_{Ω} (b · ∇z_h) w_h + c z_h w_h}
% 		\overset{\ref{thm:aufg6.3-divergenz_lemma}}&=
% 		\abs{∫_{Ω} -z_h b · ∇w_h - (\div b) z_h w_h + c z_h w_h} \\
% 		&\leq \norm b_{0, ∞, Ω} \norm{z_h}_{0, 2, Ω} \halfnorm{w_h}_{1, 2, Ω} \\
% 		&\tab + \norm{c - \div b}_{0, ∞, Ω} \norm{z_h}_{0, 2, Ω} \norm{w_h}_{0, 2, Ω} \\
% 		&\leq \norm{b}_{0, ∞, Ω} C h^2 \halfnorm{u}_{2, 2, Ω} ε^{- \frac 12} \norm{w_h}_{ε} \\
% 		&\tab + \norm{c - \div b}_{0, ∞, Ω} C h^2 \halfnorm u_{2, 2, Ω}\norm{w_h}_{ε} \\
% 		&\leq C h^2 \halfnorm u_{2, 2, Ω} ε^{-\frac 12} \norm{w_h}_{ε} \\
% 		⇒ \norm{w_h}_{ε}^2
% 		&\leq C(ε^{\frac 12} h + ε^{-\frac 12}h^2) \halfnorm u_{2, 2, Ω} \norm{w_h}_ε \\
% 		⇒ \norm{w_h}_{ε}
% 		&\leq C(ε^{\frac 12} h + ε^{-\frac 12}h^2) \halfnorm u_{2, 2, Ω}.
% 	\end{align*}
% 	Mit der Abschätzung für $\norm{z_h}_{ε}$ aus der Bemerkung \ref{bemerkung-aufg6.3Interpolationsfehler} ergibt sich
% 	\begin{align*}
% 		\norm{u - u_h}_{ε} \leq C (h + ε^{- \frac12}h^2 + ε^{\frac 12} h + ε^{-\frac 12}h^2) \halfnorm u_{2, 2, Ω}.
% 	\end{align*}
% 	Das ist das angekündigte Ergebnis. Die Konstante $C$ hängt unter anderem von $b$, $c$, der Formregularität der Triangulierung, aber nicht von $ε$ und $h$ ab.
% \end{proof}
\setcounter{section}{7}
\setcounter{subsection}{0}
\subsection{Aufgabe 7.1}
Sei $Ω \subseteq ℝ^d$ ein beschränktes Gebiet mit Lipschitz-Rand $\Rand{Ω}$.
Die primale gemischte Methode für die Poisson-Gleichung
\begin{align*}
	-\laplace u=-\div(\grad(u))=f \text{ in } Ω
\end{align*}
mit homogenen Dirichlet-Randbedingungen $u=0$ auf $\Rand Ω$ ist gegeben durch:\nl
Finde $(σ, u)∈ L^2(Ω)^d \times H_0^1(Ω)$ so, dass
\begin{align*}
	\begin{array}{rll}
		\scaProd{σ}{τ}_{0, Ω}- \scaProd{τ}{∇ u}_{0, Ω}&=0 & ∀ τ ∈ L^2(Ω)^d\\
		-\scaProd{σ}{∇ v}_{0, Ω}&=-\scaProd f v_{0, Ω} & ∀ v ∈ H_0^1(Ω)
	\end{array}
\end{align*}

\subsubsection{Aufgabe 7.1 (a)}
Leiten Sie die Formulierung im Detail her.
Schreiben Sie dazu die Poisson-Gleichung zunächst durch Einführung einer zweiten Variable $σ = ∇u$ in ein System partieller Differentialgleichungen erster Ordnung um.

\begin{lösung}
	Die schwache Formulierung für die Poisson-Gleichung mit Dirichlet-Rand\-be\-din\-gun\-gen ist
	\begin{align*}
		\scaProd {∇u}{∇v}_{0, Ω} = \scaProd fv_{0, Ω} && ∀ v ∈ H_0^1(Ω).
	\end{align*}
	Bezeichnet man nun $σ \overset{\text{Def}} = ∇u$ ergibt das
	\begin{align*}
		\scaProd {σ}{∇v}_{0, Ω} = \scaProd fv_{0, Ω} && ∀ v ∈ H_0^1(Ω)
	\end{align*}
	mit $σ ∈ L^2(Ω)$.
	Wir schreiben zusätzlich die Bedingung $σ - ∇u = 0$ in der schwachen Formulierung:
	\begin{align*}
		0 = \scaProd{σ - ∇u}{τ}_{0, Ω} = \scaProd{σ}{τ}_{0, Ω} - \scaProd{∇u}{τ} && ∀ τ ∈ L^2(Ω)^d
	\end{align*}
	Wie üblich sind diese beiden schwachen Formulierungen im Fall $u ∈ H^2(Ω) ∩ H^1_0(Ω)$ äquivalent zur starken Formulierung.
\end{lösung}

\subsubsection{Aufgabe 7.1 (b)}
Obiges gemischtes Problem liefert ein stabiles Sattelpunktsproblem.

\begin{proof}
	Was allgemein unter einem Sattelpunktsproblem zu verstehen ist und unter welchen Bedingungen so eines stabil ist, ist in der Bemerkung am Ende dieses Aufgabenblattes beschrieben.
	Die Übersetzung von den Variablen im allgemeinen Fall zu denen in unserem Problem ist:
	\begin{align*}
		X &= L^2(Ω)^d & M &= H_0^1(Ω) & a &= \scaProd{·}{·}_{0, Ω} & b &= \scaProd{·}{∇·}_{0, Ω} \\
		f &= 0 & g &= \scaProd f{·}_{0, Ω} & u &= σ & λ &= u \\
		v &= τ & μ &= v & V &= W
	\end{align*}
	Es ist also zuerst zu zeigen, dass $a$ koerziv auf $W$ ist, wobei
	\begin{align*}
		W \overset{\text{Def}}=
		\setDef{τ ∈ L^2(Ω)^d}{\scaProd{τ}{∇v} = 0\; ∀ v ∈ H_0^1(Ω)}.
	\end{align*}
	Da $a(w, w) = \scaProd ww = \norm{w}_{L^2(Ω)^d}^2$ ist $a$ koerziv mit Koerzivitätskonstante $1$, sogar auf ganz $L^2(Ω)^d$, nicht nur auf $W$.

	Nun betrachte die $\inf$-$\sup$-Bedingung. Sei $v ∈ H_0^1(Ω)$ mit $\norm v_{1, 2, Ω} = 1$ beliebig.
	Dann
	\begin{align*}
		\sup_{\substack{τ ∈ L^2(Ω)^d \\ \norm{τ}_{0, 2, Ω} = 1}} \scaProd{τ}{∇v}_{0, Ω}
		= \scaProd{\frac{∇v}{\norm{∇v}_{0, 2, Ω}}}{∇v}_{0, Ω}
		= \halfnorm{v}_{1, 2, Ω},
	\end{align*}
	da ein Skalarprodukt auf einer Kugel maximiert wird, wenn beide Argumente kollinear sind.
	Damit folgt mit der Friedrichsungleichung \ref{prop1.13FriedrichsUngleichung} für $H_0^1(Ω)$
	\begin{align*}
		\inf_{\substack{v ∈ H_0^1(Ω) \\ \norm v_{1, 2, Ω} = 1}}
		\sup_{\substack{τ ∈ L^2(Ω)^d \\ \norm{τ}_{0, 2, Ω} = 1}}
		\scaProd{τ}{∇v}_{0, Ω}
		= \inf_{\substack{v ∈ H_0^1(Ω) \\ \norm v_{1, 2, Ω} = 1}} \halfnorm v_{1, 2, Ω}
		\geq \inf_{\substack{v ∈ H_0^1(Ω) \\ \norm v_{1, 2, Ω} = 1}} \frac 1C \norm v_{1, 2, Ω}
		= \frac 1C > 0.
	\end{align*}
	Damit ist die $\inf$-$\sup$-Bedingung erfüllt und das Sattelpunktsproblem stabil.
\end{proof}

\subsection{Aufgabe 7.2}
Sei $Ω \subseteq ℝ^d$ ein beschränktes Gebiet mit Lipschitz-Rand $\Rand{Ω}$.
Die duale gemischte Methode für die Poisson-Gleichung 
\begin{align*}
	-\laplace u=-\div(\grad(u))=f \text{ in } Ω
\end{align*}
mit homogenen Dirichlet-Randbedingungen $u=0$ auf $\Rand Ω$ ist gegeben durch:\nl
Finde $(σ, u) ∈ H(\div, Ω) \times L^2(Ω)$: so, dass
\begin{align*}
	\begin{array}{rll}
		\scaProd{σ}{τ}_{0, Ω} + \scaProd{\div(τ)}{u}_{0, Ω} &= 0 & ∀ τ ∈ H(\div,Ω)\\
		\scaProd{\div(σ)}{v}_{0, Ω} &= -\scaProd{f}{v}_{0, Ω} & ∀ v ∈ L^2(Ω)
	\end{array}
\end{align*}
Der verwendete Raum
\begin{align*}
	H(\div,Ω) := %\coloneqq
	\setDef{τ ∈ L^2(Ω)^d}{\div(τ)∈ L^2(Ω)}
\end{align*}
kann auch als Vervollständigung von $C^{∞}(Ω)^d$ bzgl. der Norm 
\begin{align*}
	\norm v_{H(\div,Ω)}
	\overset{\text{Def}}=
	\left(\norm v_{0, 2, Ω}^2+ \norm {\div(v)}_{0, 2, Ω}^2\right)^{\frac12}
\end{align*}
definiert werden.

\subsubsection{Aufgabe 7.2 (a)}
Auf den ersten Blick schein $u$ nur in $L^2(Ω)$ zu liegen.
Zeigen Sie, dass tatsächlich aber $u∈ H_0^1(Ω)$ ist.
Was fällt bei den Randbedingungen auf?

\begin{proof}
	Wir zeigen, dass aus der ersten Bedingung
	\begin{align}\label{eqaufg7.2a-1}
		\scaProd{σ}{τ}_{0,Ω}+ \scaProd{\div(τ)}{u}_{0,Ω} &= 0 & ∀ τ ∈ H(\div,Ω)\\
	\end{align}
	folgt, dass $u$ schwach differenzierbar ist und $∇ u = σ$ in $Ω$ gilt.
	Dafür betrachte beliebige $φ ∈ C^{∞}_c(Ω)$ und $τ = φe_i ∈ H(\div, Ω)$, wobei $i ∈ \{1, \dots, d\}$ alle Komponenten durchläuft.
	Dafür gilt $\div τ = Σ_{j = 1}^d (τ_j)_{x_j} = φ_{x_j}$ und
	\begin{align*}
		∫_{Ω} σ_i φ
		= ∫_{Ω} Σ_{j = 1}^d σ_j (φe_i)_j
		\scaProd{σ}{τ}_{0, Ω}
		\overset{\eqref{eqaufg7.2a-1}}=
		\scaProd{\div τ}u_{0, Ω}
		= \scaProd{φ_{x_i}}u_{0, Ω}
		\quad ∀ φ ∈ C^{∞}_c(Ω)
	\end{align*}
	Also ist $u$ in die Richtung $x_i$ schwach differenzierbar mit der partiellen Ableitung $σ_i$. Da $i$ beliebig ist und $σ ∈ L^2(Ω)$ ist $u ∈ H^1(Ω)$ mit $∇u = σ$.

	Der Satz von Gauß erlaubt uns, Aussagen über Randintegrale zu finden.
	Randwerte von $u$ im $L^2(\Rand{Ω})$-Sinn sind wohldefiniert, da $u ∈ H^1(Ω)$.
	Für $τ ∈ H^1_0(Ω)^d \subseteq H(\div, Ω)$ gilt nun
	\begin{align*}
		∫_{Ω} (\div τ) u
		&= ∫_{\Rand{Ω}} (τu) · η \d S - ∫_{Ω} τ · ∇u
		= ∫_{\Rand{Ω}} (τu) · η \d S - ∫_{Ω} τ · σ \\
		\overset{\eqref{eqaufg7.2a-1}}{⇒}
		0 &= ∫_{\Rand{Ω}} (τu) · η \d S,
		= ∫_{\Rand{Ω}} (τ · η)u \d S,
	\end{align*}
	wobei $η$ der Einheitsnormalenvektor auf dem Rand ist.
	Da unter den betrachteten $τ ∈ H^1_0(Ω)^d$ alle möglichen Randwerte existieren, sind die Randwerte von $u$ gleich $0$ als $L^2(\Rand{Ω})$-Funktion, sprich $u ∈ H^1_0(Ω)$.
\end{proof}

\subsubsection{Aufgabe 7.2 (b)}
Obige gemischte Formulierung liefert ein stabiles Sattelpunktsproblem.

\begin{proof}
	Die Übersetzung von den Variablen im allgemeinen Fall zu denen in unserem Problem ist:
	\begin{align*}
		X &= H(\div, Ω) & M &= L^2(Ω) & a &= \scaProd{·}{·}_{0, Ω} & b &= \scaProd{\div(·)}{·}_{0, Ω} \\
		f &= 0 & g &= - \scaProd f{·}_{0, Ω} & u &= σ & λ &= u \\
		v &= τ & μ &= v & V &= W,
	\end{align*}
	wobei
	\begin{align*}
		W &=
		\setDef{τ ∈ H(\div, Ω)}{\scaProd{\div τ}{v}_{0, Ω} = 0\; ∀ v ∈ L^2(Ω)} \\
		&= \setDef{τ ∈ H(\div, Ω)}{\div τ = 0},
	\end{align*}
	also der Raum der Divergenz-freien $L^2(Ω)^d$-Funktionen.
	$a$ ist offensichtlich symmetrisch.
	Für $w ∈ W$ ist
	\begin{align*}
		a(w, w)
		&= \scaProd ww_{0, Ω}
		= \norm w_{0, 2, Ω}^2 + \underbrace{\scaProd{\div w}{\div w}_{0, Ω}}_{= 0, \text{ da } w ∈ W} \\
		&= \norm w_{0, 2, Ω}^2 + \norm{\div w}_{0, 2, Ω}^2
		= \norm w_{H(\div, Ω)}^2.
	\end{align*}
	Also ist $a$ auf $W$ mit Koerzivitätskonstante $1$ koerziv.

	Jetzt ist noch die $\inf$-$\sup$-Bedingung zu zeigen.
	Sei dafür $v ∈ L^2(Ω)$ mit $\norm v_{0, 2, Ω}$ beliebig.
	Wir wollen ein $τ ∈ H(\div, Ω)$ finden, sodass $\div τ$ möglichst nah dran ist an einem Vielfachen von $v$, damit $⟨\div τ, v⟩_{0, 2}$ maximal wird.
	Um Rechnungen zu vereinfachen, setzen wir $τ_i = 0$ für $i ∈ \{2, \dots, d\}$,
	denn dann ist $\div τ = (τ_1)_{x_1}$.
	Also sollte $τ_1$ eine Art Stammfunktion von $v$ sein.
	Damit punktweise Integration möglich ist, nehmen wir nicht $v$, sondern eine Approximation $φ ∈ C^{∞}_c(Ω)$ mit $\norm{v - φ}_{0, 2, Ω} \leq \frac12 \norm{v}_{0, 2, Ω}$.
	Dieser Schritt scheint viel zu verschenken und er tut es auch und eine bessere Annährung ergibt ein größeres $β$, aber wir sind gerade nur an einem beliebigen $β > 0$ interessiert.

	Damit $(τ_1)_{x_1} = φ$, setze
	\begin{align*}
		\tilde {τ_1}(x)
		&= ∫_{- ∞}^x φ(t, x_2, \dots x_d) \d t \\
		τ_1 &=
		\begin{cases}
			\tilde {τ_1}(x), &\text{für } x ∈ Ω \\
		0 &\text{sonst}
		\end{cases}
	\end{align*}
	$τ_1$ ist wohldefiniert, da $Ω$ beschränkt ist und daher tatsächlich nur über ein endliches Intervall integriert wird und $φ ∈ C^{∞}_c(Ω)$ darauf beschränkt ist.
	Nach dem Hauptsatz der Integral- und Differentialrechnung gilt tatsächlich $(τ_1)_{x_1} = φ$, zumindest im Inneren von $Ω$.

	Nun müssen wir zeigen, dass $τ_1 ∈ L^2(Ω)$, damit $τ ∈ H(\div, Ω)$.
	\begin{align*}
		\norm{τ_1}_{0, 2, Ω}^2
		% \leq \norm{\tilde τ_1}_{0, 2, ℝ^d}^2
		&= ∫_{Ω} \abs{∫_{- ∞}^x φ(t, x_2, \dots, x_d) \d t}^2 \d x
		\leq ∫_{Ω} \left( ∫_{-∞}^{∞} \abs{φ(t, x_2, \dots, x_d)} \d t \right)^2 \d x \\
		&= ∫_{Ω} \left( ∫_{-∞}^{∞} \abs{φ(t, x_2, \dots, x_d)} · 1 \d t \right)^2 \d x \\
		\overset{\text{Cauchy-Schwarz}}&{\leq}
		∫_{Ω}\left( \left( ∫_{-∞}^{∞} \abs{φ(t, x_2, \dots, x_d)}^2 \d t \right)^{\frac 12} \left(∫_{Ω} 1^2\right)^{\frac 12} \right)^2 \d x \\
		&= ∫_{Ω} ∫_{-∞}^{∞} \abs{φ(t, x_2, \dots, x_d)}^2 \d t \abs{Ω} \d x \\
		&\leq \abs{Ω} \abs{Ω_1} ∫_{Ω} \abs{φ(x_1, x_2, \dots, x_d)}^2 \d x \\
		&\leq \abs{Ω} \abs{Ω_1} \norm{φ}_{0, 2, Ω}^2
	\end{align*}
	Hierbei bezeichnet $\abs{Ω}$ das Maß von $Ω$ und $\abs{Ω_1}$ die maximale Ausdehnung von $Ω$ in die Richtung $x_1$.
	Der Integrand nach der ersten Abschätzung ist unabhängig von $x_1$ und daher kann $\abs{Ω_1}$ herausgezogen werden.
	Die Integrationsgrenzen $∞$ und $-∞$ sind tatsächlich auch beschränkt, da $Ω$ beschränkt ist und $φ$ Träger in $Ω$ hat.
	Damit gilt
	\begin{align*}
		\norm{τ}_{H(\div, Ω)}^2 &= \norm{τ}_{0, 2, Ω}^2 + \norm{\div τ}_{0, 2, Ω}^2 \leq \abs{Ω} \abs{Ω_1} \norm{φ}_{0, 2, Ω}^2 + \norm{φ}_{0, 2, Ω}^2 \\
		⇒ \norm{τ}_{H(\div, Ω)} &\leq \underbrace{\left(\abs{Ω} + \abs{Ω_1}\right)^{\frac 12}}_{:= C} \norm{φ}_{0, 2, Ω} % \coloneqq C
	\end{align*}
	Für das Supremum gilt demnach
	\begin{align*}
		\sup_{ρ ∈ H(\div, Ω)\setminus \{0\}} \frac {\scaProd{\div ρ}v_{0, Ω}}
		{\norm{ρ}_{H(\div, Ω)}}
		&\geq \frac {\scaProd{\div τ}v_{0, Ω}}
		{\norm{τ}_{H(\div, Ω)}} \\
		&\geq \frac {\scaProd{φ}v_{0, Ω}}
		{C \norm{φ}_{0, 2, Ω}} \\
		&\geq \frac{\scaProd vv_{0, Ω} - \scaProd {v - φ}v_{0, Ω}}
		{C \left(\norm{v}_{0, 2, Ω} + \norm{φ - v}_{0, 2, Ω}\right)} \\
		\overset{\text{Cauchy-Schwarz}}&{\geq}
		\frac{\norm v_{0, 2, Ω}^2 - \norm{v - φ}_{0, 2, Ω}\norm{v}_{0, 2, Ω}}
		{C \left(\norm{v}_{0, 2, Ω} + \frac 12\norm v_{0, 2, Ω}\right)} \\
		&\geq \frac{\norm v_{0, 2, Ω}^2 - \frac 12 \norm{v}_{0, 2, Ω}^2}
		{\frac 32 C \norm{v}_{0, 2, Ω}} \\
		&= \frac2{3·2C} \norm v_{0, 2, Ω}
	\end{align*}
	Damit lässt sich dann auch einfach das Infimum bestimmen:
	\begin{align*}
		\inf_{\substack{v ∈ L^2(Ω) \\ \norm v_{0, 2, Ω} = 1}}
		\sup_{ρ ∈ H(\div, Ω)\setminus \{0\}} \frac {\scaProd{\div ρ}v_{0, Ω}}
		{\norm{ρ}_{H(\div, Ω)}}
		\geq 
		\inf_{\substack{v ∈ L^2(Ω) \\ \norm v_{0, 2, Ω} = 1}}
		\frac 1{3C} \norm v_{0, 2, Ω} = \frac 1{3C} > 0.
	\end{align*}
	Damit handelt es sich um ein stabiles Sattelpunktsproblem.
\end{proof}
