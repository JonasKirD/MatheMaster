% !TEX root = PDENMW.tex
% This work is licensed under the Creative Commons
% Attribution-NonCommercial-ShareAlike 4.0 International License. To view a copy
% of this license, visit http://creativecommons.org/licenses/by-nc-sa/4.0/ or
% send a letter to Creative Commons, PO Box 1866, Mountain View, CA 94042, USA.

\chapter{Adaptive Finite Elements}
\section{Model Problem}
Let $\Omega\subset\R^d$ be a \textit{domain}, i.e. a bounded and open subset of $\R^d$. Our goal is to find a function $u\colon\Omega\to\R$ such that 
\begin{align*}
    -\laplace u & = f,\,\text{ in } \Omega, \\
    u & =u_0,\,\text{ on } \partial\Omega,
\end{align*}
where $\laplace$ is the \emph{Laplace operator}, $f$ and $u_0$ are given functions.
\section{Weak Formulation}
\begin{definition}
The \emph{Sobolev space} $W^{k,p}(\Omega)$ is the set of all $p$-integrable functions $v\colon\Omega\to\R$ with $k$ weak derivatives that are $p$-integrable. 
\begin{equation*}
    |v|_{W^{k,p}}:=\int\limits_\Omega|\nabla^kv|^p<\infty.
\end{equation*}
In particular, we write $W^{k,2}=:H^k$.
\end{definition}
Sobolev functions do not have point values. Nevertheless, the trace of $H^1$-functions on $\partial\Omega$ is well-defined due to the trace theorem. The image of the trace operator is denoted by $H^{\frac{1}{2}}(\partial\Omega)$. Let $\omega\in H^{\frac{1}{2}}(\partial\Omega)$. Define $H_\omega^1(\Omega)=\{ v\in H^1\colon v|_{\partial\Omega}=\omega\}$. Let $w\in H_0^1(\Omega)$ be a \textit{test function}. In order to obtain the \textit{weak formulation}, we multiply the partial differential equation in strong form with $w$ and integrate over $\Omega$ : 
\begin{equation*}
    -\laplace u=f\, \xrightarrowfix{\text{multiply}}-\laplace uw=fw\,\xrightarrowfix{\text{integrate}}\int\limits_\Omega-\laplace uw\,dx=\int\limits_\Omega fw\,dx.
\end{equation*}
Integration by parts: 
\begin{equation*}
    \int\limits_\Omega\nabla u\nabla w\,dx-\underbrace{\int\limits_{\partial\Omega}\partial_{n}uw \diff s}_{=0}=\int\limits_\Omega fw\diff x.
\end{equation*}
To go more abstract, we introduce the \textit{bilinear form} 
\begin{equation*}
    a(v,w):=\int\limits_\Omega\nabla v\nabla w\diff x,
\end{equation*}
and the \textit{linear form}
\begin{equation*}
    l(w):=\int\limits_\Omega fw\diff x.
\end{equation*}
Find $u\in H_{u_0}^1$ such that $a(u,w)=l(w)$ for all $w\in H_0^1$.
\section{Finite Elements}
Idea: approximate $H^1$ with a certain finite-dimensional subspace.
\begin{definition}
	A \textit{grid} is a finite-dimensional collection of closed simplices $\T=\{ T_1,\ldots, T_m\}$ such that $\bar{\Omega}=\bigcup_{i=1}^mT_i$ and for all $i,j$ either $T_i\cap T_j=\emptyset$ or $T_i\cap T_j$ is a common face. A grid with these properties is said to be \textit{conforming}.
\end{definition}
Finite elements are spaces of piecewise polynomial functions with respect to a grid. Most important: \textit{Lagrange spaces}: 
\begin{equation*}
	V_h^p:=\left\{ v_h\in C(\Omega)\mid\forall T\in\T\colon v_h|_T\text{ is polynomial of order}\leq p\right\}.
\end{equation*}
Note: $V_h^p\subset H^1$. Discrete problem/FE problem: find $u_h\in V_h^p\cap H^1_{u_0}$ such that 
\begin{equation*}
	\forall w_h\in V_h^p\cap H_0^1=\{ v_h\in V_h^p\mid v_h|_{\partial\Omega}=0\}\colon a(u_h,w_h)=l(w_h)
\end{equation*}
Our next aim are some statements about the existence and uniqueness of solutions.
Existence and uniqueness of solutions.
\section{Discretization Error}
We like to measure the error we commit with the discretization, i.e. estimate $\| u-u_h\|$. There are several possible norms we could employ. For example $\|\cdot\|_\infty$, $\|\cdot\|_{H^1}$ or $\|\cdot\|_{L^2}$. Either choice we make, the outcome as well as the proofs will be different. A relatively easy result is the following.
\begin{theorem}
There exists a constant $c$ such that $\| u-u_h\|_{H^1}\leq ch^p| u|_{H^{p+1}}$. $c$ depends on $\Omega$ and the grid.
\end{theorem}
Question: How big is $| u|_{H^{p+1}}$? Does $| u|_{H^{p+1}}$ exist? In other words: is $u\in H^{p+1}$? To find an answer to these questions is not easy and opens a whole new branch in mathematical analysis, called \textit{regularity theory}. A partial answer can be found in Braess %\cite{Braess07}.
\begin{theorem}%[Braess 2.7.2]
Let $f,u_0$ be sufficiently smooth. If $\Omega$ is convex, then $u\in H^2$. If $\Omega$ has a $C^2$ boundary, then $u\in H^s$ for $s\geq 2$.
\end{theorem}
\begin{example}
	
What can go wrong? 
\begin{equation*}
	\Omega=\{(x,y)\in\R^2\colon x^2+y^2<1,\,x<0\lor y>0\}.
\end{equation*} 
Boundary value problem:
\begin{align*}
	-\laplace u&=0\\
	u(r,\varphi)&=\sin(2/3\varphi)\quad \forall 0\leq\varphi\leq3/2\pi \\
	u&=0 \text{ elsewhere on } \partial\Omega.
\end{align*}
What does the solution looks like? Identify $\R^2$ with $\C$. The function $w(z)=z^{2/3}$ is holomorphic on $\Omega$. Cauchy-Riemann equations $\operatorname{Im}w(z)$ is harmonic and takes correct values on $\partial\Omega$. Therefore it is a solution to the BVP. Derivative: $w^\prime(z)=2/3z^{-1/3}$ tends to $\infty$ as $z\to 0$. Therefore the solution has a singularity at $0$ and $u\notin H^2$.
\end{example}
\begin{figure}[H]
\center

	\begin{tikzpicture}[scale=2]
	\def \xone{0};
	\def \yone{0};
	
	
	\coordinate (A) at (\xone,\yone);

	\draw (A) ++(-1.2,0) -- ++(2.4,0);
	\draw (A) ++(0,-1.2) -- ++(0,2.4);

	\draw (A) -- ++(0:1)
    	  (A) -- ++(-90:1);
	\draw (A) ++(0:1) arc (0:270:1);

	\node at (A) [above right]{0}
	
	\end{tikzpicture}

\caption{Omega}
\label{example_problem}
	
\end{figure}

Conditions for error estimates are not satisfied. You can see in experiments: the error decreases by $h^{\frac{1}{2}}$. \\
Is it possible to obtain the optimal order $\bigo(h)$ for $u \not \in H^2$ ?\\
\underline{Idea} problem specific grids:
\begin{itemize}
	\item use many small elements in areas where something \glqq happens\grqq, elsewhere use few big elemets
\end{itemize} 
This looks promising and we show with an easy example, that it could work properly.
\begin{example}
	Let $u: [0,1] \to \R $ be continuous. How well can we approximate $u$ by piecewise constant functions $u_{N}$?
	\begin{enumerate}[label = \Roman*)]
		\item Uniform grid:
			\begin{itemize}
				\item $0 = x_{0} < x_{1}< \dots < x_{N} = 1$ with uniform steplenght $h$
				\item Let $u \in W^{1,\infty}$ i.e. $\text{ess}\sup_{[0,1]} | u^\prime|< \infty$ 
				\item Define the approximating functions $u_{N}=u(x_{n-1})$ for all $x_{n-1}\leq x \leq x_{n}$ 
				\item pointwise error
					\begin{align*}
						|u(x) - u_{N}(x) | &= | u(x) - u(x_{n-1})|\\
										   &= \left| \int\limits_{x_{n-1}}^{x} u^\prime(t) \diff t \right|\\
										   &= h\cdot \text{ess}\sup_{[x_{n-1},x_{n}]}|u^\prime | \\
										   &= h\ \|u^\prime\|_{L^\infty(x_{n-1},x_{n})}
					\end{align*} 
					\begin{equation*}
						\implies\|u -u_{N}\|_{L^\infty} \leq h \ \|u^\prime\|_{L^\infty(0,1)}= h \ |u|_{W^{1,\infty}(0,1)}
				\end{equation*} 
			\end{itemize} 
			What happens if $u \not \in W^{1,\infty}$?\\
			The gridlenght $h$ is not defined for nonuniform grids. Instead use:
			\begin{equation*}
				\|u-u_{N}\|_{L^{\infty}(0,1)}  \leq \frac{1}{N} \|u^\prime\|_{L^\infty(0,1)}
			\end{equation*} 
			with $N := \frac{1}{h}$ the number of degrees of freedom. Furthermore is $N$ a measure for computational effort!
		\item Nonuniform grids:\\
			Let $u \not \in W^{1,1}(0,1)$ and $|u|_{W^{1,1}}>0$
			\begin{itemize}
				\item Define 
					\begin{equation*}
						\phi(x) := \int_{0}^x |u^\prime | \diff t.
					\end{equation*} 
					$\phi$ is monotonically increasing, $\phi(0)=0,\phi(1)=|u|_{W^{1,1}} > 0$.
				\item Define a grid $0 = x_{0} < x_{1}< \dots < x_{N} = 1$ such that 
					\begin{equation*}
						\phi(x_{n}) - \phi(x_{n-1}) = \frac{1}{N} |u^\prime|_{L^1(0,1)}= \frac{1}{N} |u|_{W^{1,1}}
					\end{equation*} 
				\item For every $x \in [x_{n-}x_{n})$ holds:
					\begin{align*}
						|u(x)-u_{N}(x)|&= |u(x)-u(x_{n-1})|\\
									   &= \left| \int_{x_{n-1}}^{x} u^\prime(t) \diff t\ \right|\\
									   &\leq \int_{x_{n-1}}^{x} |u^\prime(t)| \diff t\\\
									   &= \phi(x_{n}) - \phi(x_{n-1}) \\
									   &= \frac{1}{N} \|u^\prime\|_{L^{1}(0,1)}						   
					\end{align*} 
					\begin{equation*}
						\implies \|u - u_{N}\|_{L^{\infty}(0,1)} \leq \frac{1}{N} \|u^\prime\|_{L^{1}(0,1)}
					\end{equation*} 
			\end{itemize} 
	\end{enumerate} 
\end{example} 
