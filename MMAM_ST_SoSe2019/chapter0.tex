% !TEX root = NLANA.tex
% This work is licensed under the Creative Commons
% Attribution-NonCommercial-ShareAlike 4.0 International License. To view a copy
% of this license, visit http://creativecommons.org/licenses/by-nc-sa/4.0/ or
% send a letter to Creative Commons, PO Box 1866, Mountain View, CA 94042, USA.


\section*{Organisatorisches}%
\label{sec:Organisatorisches}

Bei Fragen während der Vorlesungs- und Prüfungsphase:

Email: schwartz@gsc.tu-darmstadt.de

Diese Information wird nur für die oben genannte Zeitspanne im Skript enthalten bleiben.

\chapter{Einführung}

\section{Beispiele}%
\label{sec:Beispiele}

\begin{definition}
	Ein \underline{strategisches Spiel in Normalform} besteht aus 
	\begin{itemize}
		\item eine Menge von \underline{Spielern}  $i=1,\ldots, N$
		\item zulässige \underline{Strategiemengen} $X_i, \quad i=1,\ldots, N$
		\item \underline{Auszahlungsfunktionen}  $f_i : \underbrace{X_1 \times \ldots \times X_N}_{:=X} \to \R$
	\end{itemize}
	\underline{Problem} von Spieler i:
	\begin{itemize}
		\item $\min_{x_i} f_i(x_1, \ldots, x_i, \ldots, x_N)$ mit $x_i \in X_i$
	\end{itemize}
\end{definition}

\begin{beispiel}
	(Schere, Stein, Papier)
	\begin{itemize}
		\item zwei Spieler : $i=1,2$
		\item Strategiemengen : $X_1 = X_2 = \left\{\text{Schere, Stein, Papier}\right\}$
		\item Auszahlungsfunktion : (Wer verliert, zahlt 1€ an den Gegner)

			Auszahlung für Spieler 1:
			\[
				f_1(x_1,x_2) = \begin{cases}
					\hfil -1 &\quad x_1 = \text{Schere}, x_2 =\text{Stein} \\
					\hfil +1 &\quad x_1 = \text{Schere}, x_2 = \text{Papier} \\
					\hfil 0 &\quad x_1 = \text{Schere}, x_2 = \text{Schere} \\
					\hfil \vdots &\quad \hfil \vdots
				\end{cases}
			\] 
			Auszahlung für Spieler 2:
			\[
				f_2(x_1,x_2) = \begin{cases}
					\hfil +1 &\quad x_1 = \text{Schere}, x_2 =\text{Stein} \\
					\hfil -1 &\quad x_1 = \text{Schere}, x_2 = \text{Papier} \\
					\hfil 0 &\quad x_1 = \text{Schere}, x_2 = \text{Schere} \\
					\hfil \vdots &\quad \hfil \vdots
				\end{cases}
			\]
			Die sogenannte Auszahlungsmatrix sieht dann folgendermaßen aus:
			\begin{center}
				\begin{tabular}{cc|c|c|c|}
					\\ \cline{3-5}
					&&\multicolumn{3}{c|}{{\color{red} Spieler 1}} \\ \cline{3-5}
					&&Schere &Stein & Papier \\ \hline
					\multicolumn{1}{ |c }{\multirow{3}{*}{\color{blue} Spieler 2}}&
					\multicolumn{1}{ |c| }{Schere} & \color{blue}0\color{black},\color{red} 0 & \color{blue}-1\color{black},\color{red} +1 & \color{blue}+1\color{black},\color{red} -1 \\ \cline{2-5}
					\multicolumn{1}{ |c }{}&
					\multicolumn{1}{ |c| }{Stein} & \color{blue}+1\color{black},\color{red} -1 & \color{blue}0\color{black},\color{red} 0 & \color{blue}-1\color{black},\color{red} +1 \\ \cline{2-5}
					\multicolumn{1}{ |c }{}&
					\multicolumn{1}{ |c|}{Papier} & \color{blue}-1\color{black},\color{red} +1 & \color{blue}+1\color{black},\color{red} -1 & \color{blue}0\color{black},\color{red} 0  \\ \hline
				\end{tabular}
			\end{center}
	\end{itemize}
\end{beispiel}

\begin{beispiel}
	(Gefangenendilemma):

	Angebot des Staatsanwalts:
	\begin{itemize}
		\item Verdächtiger gesteht, aber Partner schweigt:

			Geständiger muss als Kranzeuge 1 Jahr ins Gefängnis

			Schweigender Partner muss 10 Jahre ins Gefängnis
		\item Wenn beide gestehen:
			
			Beide müssen für 5 Jahre ins Gefängnis
		\item Wenn keiner gesteht:
			
			Beide müssen wegen anderer Vergehen für 2 Jahre ins Gefängnis
	\end{itemize}
	\[
	\implies
	\] 
	\begin{itemize}
		\item Spieler : $i=1, 2$
		\item Strategiemengen : $X_1 = X_2 = \{\text{schweigen, gestehen}\}$
	\end{itemize}
	\begin{center}
		\begin{tabular}{cc|c|c|}
			\\ \cline{3-4}
			&&\multicolumn{2}{c|}{{\color{red} Spieler 1}} \\ \cline{3-4}
			&&schweigen &gestehen \\ \hline
			\multicolumn{1}{ |c }{\multirow{2}{*}{\color{blue} Spieler 2}}&
			\multicolumn{1}{ |c| }{schweigen} & \color{blue}2\color{black},\color{red} 2 & \color{blue}10\color{black},\color{red} 1 \\ \cline{2-4}
			\multicolumn{1}{ |c }{}&
			\multicolumn{1}{ |c| }{gestehen} & \color{blue}1\color{black},\color{red} 10 & \color{blue}5\color{black},\color{red} 5 \\ \hline
		\end{tabular}
	\end{center}
\end{beispiel}

\begin{beispiel}
	(Koordinationsspiel, Kampf der Geschlechter)
	\begin{itemize}
		\item 2 Spieler : $i = 1, 2$
		\item Strategiemengen : $X_1 = X_2 = \{\text{A, B}\}$
			hierbei sind $A$ und $B$ z.B. die von den jeweiligen Spielern bevorzugten Aktivitäten (z.B. $A$ = in die Oper gehen vs. $B$ = zum Fußballspiel gehen)
		\item Auszahlungsmatrix:
			\begin{center}
				\begin{tabular}{c|c|c|}
					\cline{2-3}
					&A&B \\ \hline
					\multicolumn{1}{|c|}{A}& 2, 1& 0, 0 \\ \hline
					\multicolumn{1}{|c|}{B}& 0, 0& 1, 2 \\ \hline
				\end{tabular}
			\end{center}

			Auszahlungen werden hier im Gegensatz zu Normalfall maximiert!
	\end{itemize}
\end{beispiel}

\begin{beispiel}
	(Cournot-Oligopol)

	Firmen $F_i, \quad i=1,\ldots,N$ stellen das gleiche
	Gut her. Die Produktion von $x_i$ Einheiten kostet
	$c_i(x_i)$. Der Preis fällt mit dem Gesamtangebot, d.h.
	$p(\sum_{i=1}^{N} x_i)$.
	Das Problem der Firma ist, den Gewinn zu maximieren:
	\[
		\max_{x_i} p(\sum_{j=1}^N x_j)\cdot x_i - c_i(x_i) \qquad \mit x_i \geq 0 \quad\text{(+ weitere Restriktionen)}
	.\] 
\end{beispiel}

\section{Klassifikation von Spielen}%
\label{sec:Klassifikation von Spielen}

Wir werden nun Spiele nach unterschiedlichen Kriterien charakterisieren und einige Definitionen einführen.

\begin{definition} (Charakterisierung nach Auszahlungsfunktion)

	Ein Spiel $\Gamma=\{X_i, f_i\}_{i=1}^N$ heißt
	\begin{itemize}
		\item \underline{Nullsummenspiel}, wenn gilt
			\[
				\sum_{i=1}^N f_i(x_1, \ldots, x_N)=0 \qquad \forall (x_1, \ldots, x_N) \in X
			.\] 
		\item \underline{Konstansummenspiel}, wenn gilt
			\[
				\sum_{i=1}^N f_i(x_1, \ldots, x_N)=c \qquad \forall (x_1, \ldots, x_N) \in X
			.\] 
		\item \underline{Nichtnullsummenspiel}, wenn es kein Konstantsummenspiel ist.
	\end{itemize}
\end{definition}

\begin{definition} (Charakterisierung nach Größe der Strategiemengen) 

	Ein Spiel $\Gamma=\{X_i, f_i\}_{i=1}^N$ heißt
	\begin{itemize}
		\item \underline{endlich}, wenn alle Strategiemengen $X_i$ nur endlich viele Elemente haben.
		\item \underline{abzählbar}, wenn alle Strategiemengen $X_i$ abzählbar viele Elemente haben
		\item \underline{überabzählbar} oder \underline{kontinuierlich}, in allen andern Fällen
	\end{itemize}
\end{definition}

\begin{definition} (Charakterisierung nach Anzahl der Spieler)
	\begin{itemize}
		\item Ein endliches 2-Personen-Spiel heißt \underline{Bi-Matrixspiel} 
		\item Ein endliches 2-Personen-Nullsummenspiel heißt \underline{Matrixspiel} 
	\end{itemize}
\end{definition}

\begin{beispiel}
	(Endliches 2-Personen-Spiel)
	\begin{align*}
		X_1 = \{a_1, a_2, \ldots, a_m\} \text{ durchnummerieren } \rightarrow \quad X=\{1,\ldots,m\} \\
		X_2 = \{b_1, b_2, \ldots, b_n\} \text{ durchnummerieren } \rightarrow \quad X=\{1,\ldots,n\} 
	\end{align*}
Auszahlungsmatrizen für Spieler 1 und 2: \[A, B \in R^{m \times n} \mit a_{ij} = f_1(a_i, b_j), \quad b_{ij} = f_2(a_i, b_j)
.\]

Damit dieses Spiel ein Nullsummenspiel ist, müsste gelten \[
B = -A
.\] 
\end{beispiel}

Weitere zahlreiche Unterscheidungsmöglichkeiten von Spielen
\begin{itemize}
	\item kooperativ oder \underline{nichtkooperativ} 
	\item \underline{einmaliges} oder wiederholbares Spiel
	\item \underline{vollständige} versus unvollständige Information
	\item \underline{simultane} oder sequentielle Entscheidungen
\end{itemize}
