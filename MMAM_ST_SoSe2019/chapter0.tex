% !TEX root = NLANA.tex
% This work is licensed under the Creative Commons
% Attribution-NonCommercial-ShareAlike 4.0 International License. To view a copy
% of this license, visit http://creativecommons.org/licenses/by-nc-sa/4.0/ or
% send a letter to Creative Commons, PO Box 1866, Mountain View, CA 94042, USA.


\section*{Organisatorisches}%
\label{sec:Organisatorisches}

Bei Fragen während der Vorlesungs- und Prüfungsphase:

Email: schwartz@gsc.tu-darmstadt.de

\chapter{Einführung}

\section{Beispiele}%
\label{sec:Beispiele}

\begin{definition}
	Ein \underline{strategisches Spiel in Normalform} besteht aus 
	\begin{itemize}
		\item eine Menge von \underline{Spielern}  $i=1,\ldots, N$
		\item zulässige \underline{Strategiemengen} $X_i, \quad i=1,\ldots, N$
		\item \underline{Auszahlungsfunktionen}  $f_i : \underbrace{X_1 \times \ldots \times X_N}_{:=X} \to \R$
	\end{itemize}
	\underline{Problem} von Spieler i:
	\begin{itemize}
		\item $\min_{x_i} f_i(x_1, \ldots, x_i, \ldots, x_N)$ mit $x_i \in X_i$
	\end{itemize}
\end{definition}

\begin{beispiel}
	(Schere, Stein, Papier)
	\begin{itemize}
		\item zwei Spieler : $i=1,2$
		\item Strategiemengen : $X_1 = X_2 = \left\{\text{Schere, Stein, Papier}\right\}$
		\item Auszahlungsfunktion : (Wer verliert, zahlt 1€ an den Gegner)

			Auszahlung für Spieler 1:
			\[
				f_1(x_1,x_2) = \begin{cases}
					\hfil -1 &\quad x_1 = \text{Schere}, x_2 =\text{Stein} \\
					\hfil +1 &\quad x_1 = \text{Schere}, x_2 = \text{Papier} \\
					\hfil 0 &\quad x_1 = \text{Schere}, x_2 = \text{Schere} \\
					\hfil \vdots &\quad \hfil \vdots
				\end{cases}
			\] 
			Auszahlung für Spieler 2:
			\[
				f_2(x_1,x_2) = \begin{cases}
					\hfil +1 &\quad x_1 = \text{Schere}, x_2 =\text{Stein} \\
					\hfil -1 &\quad x_1 = \text{Schere}, x_2 = \text{Papier} \\
					\hfil 0 &\quad x_1 = \text{Schere}, x_2 = \text{Schere} \\
					\hfil \vdots &\quad \hfil \vdots
				\end{cases}
			\]
			Die sogenannte Auszahlungsmatrix sieht dann folgendermaßen aus:
			\begin{center}
				\begin{tabular}{c|c|c|c}
					&Schere &Stein & Papier \\ \hline
					Schere & 0, 0 & -1, +1 & +1, -1 \\ \hline
					Stein & +1, -1 & 0, 0 & -1, +1 \\ \hline
					Papier & -1, +1 & +1, -1 & 0, 0
				\end{tabular}
			\end{center}
			% Spieler 1 oben drüber, Spieler zwei links daneben, TODO Farben 1:blau 2:rot linke zahl ist 1
	\end{itemize}
\end{beispiel}

\begin{beispiel}
	Gefangenendilemma:

	Angebot des Staatsanwalts:
	\begin{itemize}
		\item Verdächtiger gesteht, aber Partner schweigt:
			Geständiger muss als Kranzeuge 1 Jahr ins Gefängnis
			Schweigender Partner muss 10 Jahre ins Gefängnis
		\item Wenn beide gestehen: Beide müssen für 5 Jahre ins Gefängnis
		\item Wenn keiner gesteht: Beide müssen wegen anderer Vergehen für 2 Jahre ins Gefängnis
	\end{itemize}
	$\implies$
	\begin{itemize}
		\item Spieler : $i=1, 2$
		\item Strategiemengen : $X_1 = X_2 = \{\text{schweigen, gestehen}\}$
	\end{itemize}
	\begin{tabular}{c|c|c}
		&schweigen&gestehen \\ \hline
		schweigen & 2, 2& 10, 1 \\ \hline
		gestehen & 1, 10& 5, 5
		%TODO gleiches mit Spielern wie oben
	\end{tabular}
\end{beispiel}

\begin{beispiel}
	Koordinationsspiel, Kampf der Geschlechter
	\begin{itemize}
		\item 2 Spieler : $i = 1, 2$
		\item Strategiemengen : $X_1 = X_2 = \{\text{A, B}\}$
			hierbei sind $A$ und $B$ z.B. die von den jeweiligen Spielern bevorzugten Aktivitäten (in die Oper gehen vs. zum Fußballspiel gehen)
		\item Auszahlungsmatrix :
			\begin{tabular}{c|c|c}
				&A&B \\ \hline
				A& 2, 1& 0, 0 \\ \hline
				B& 0, 0& 1, 2 
			\end{tabular}
			Auszahlungen werden hier im Gegensatz zu Normalfall maximiert!
	\end{itemize}
\end{beispiel}

\begin{beispiel}
	Cournat-Oligopol
	Firmen $F_i, \quad i=1,\ldots,N$ stellen das gleiche Gut her. Die Produktion von $x_i$ Einheiten kostet $c_i(x_i)$. Der Preis fällt mit dem Gesamtangebot, d.h. $p(\sum_{i=1}^{N} x_i$.
	Das Problem der Firma ist, den Gewinn zu maximieren:
	\[
		\max_{x_i} p(\sum_{j=1}^N x_j)x_i - c_i(x_i) \qquad \mit x_i \geq 0 \text{(+ weitere Restriktionen)}
	.\] 
\end{beispiel}

\section{Klassifikation von Spielen}%
\label{sec:Klassifikation von Spielen}

Eigenschaften der Auszahlungsfunktion:

\begin{definition}
	Ein Spiel $\Gamma=\{X_i, f_i\}_{i=1}^N$ heißt
	\begin{itemize}
		\item \underline{Nullsummenspiel}, wenn gilt
			\[
				\sum_{i=1}^N f_i(x_1, \ldots, x_N)=0 \qquad \forall (x_1, \ldots, x_N) \in X
			.\] 
		\item \underline{Konstansummenspiel}, wenn gilt
			\[
				\sum_{i=1}^N f_i(x_1, \ldots, x_N)=c \qquad \forall (x_1, \ldots, x_N) \in X
			.\] 
		\item \underline{Nichtnullsummenspiel}, wenn es kein Konstantsummenspiel ist.
	\end{itemize}
\end{definition}

Größe der Strategiemengen
\begin{definition}
	Ein Spiel $\Gamma=\{X_i, f_i\}_{i=1}^N$ heißt
	\begin{itemize}
		\item \underline{endlich}, wenn alle Strategiemengen $X_i$ nur endlich viele Elemente haben.
		\item \underline{abzählbar}, wenn alle Strategiemengen $X_i$ abzählbar viele Elemente haben
		\item \underline{überabzählbar} oder \underline{kontinuierlich}, in allen andern Fällen
	\end{itemize}
\end{definition}

Anzahl der Spieler
\begin{definition} Arten von Spielen
	\begin{itemize}
		\item Ein endliches 2-Personen-Spiel heißt \underline{Bi-Matrixspiel} 
		\item Ein endliches 2-Personen-Nullsummenspiel heißt \underline{Matrixspiel} 
	\end{itemize}
\end{definition}

\begin{beispiel}
	Endliches 2-Personen-Spiel
	\[
	\begin{align*}
		X_1 = \{a_1, a_2, \ldots, a_m\} \text{ durchnummerieren } &\rightarrow \quad $X=\{1,\ldots,m\}$ \\
		X_2 = \{b_1, b_2, \ldots, b_n\} \text{ durchnummerieren } &\rightarrow \quad $X=\{1,\ldots,n\}$ 
	\end{align*}
.\]
Auszahlungsmatrizen für Spieler 1 und 2: \[A, B \in R^{m \times n} \mit a_{ij} = f_1(a_i, b_j), \quad b_{ij} = f_2(a_i, b_j)
.\]

Bei einem Nullsummenspiel müsste gelten \[
B = -A
.\] 
\end{beispiel}

Weitere Unterscheidungsmöglichkeiten von Spielen
\begin{itemize}
	\item kooperativ oder \underline{nichtkooperativ} 
	\item \underline{einmaliges} oder wiederholbares Spiel
	\item \underline{vollständige} versus unvollständige Information
	\item \underline{simultane} oder sequentielle Entscheidungen
\end{itemize}





