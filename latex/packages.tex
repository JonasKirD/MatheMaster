% This work is licensed under the Creative Commons
% Attribution-NonCommercial-ShareAlike 4.0 International License. To view a copy
% of this license, visit http://creativecommons.org/licenses/by-nc-sa/4.0/ or
% send a letter to Creative Commons, PO Box 1866, Mountain View, CA 94042, USA.

% Folgende Packages sollten zuerst geladen werden:
\usepackage[utf8]{inputenc}	             % Paket für Umlaute; verwende utf8 Kodierung in TexWorks 

%%%%%%%%%%%%%%%%%%%%% PACKAGES (alphabetisch sortiert)
\usepackage{aligned-overset}             % besseres Einrücken, siehe: https://tex.stackexchange.com/questions/257529/overset-and-align-environment-how-to-get-correct-alignment
\usepackage{amsmath}                     % wichtig für align-Umgebung
\usepackage{amssymb}                     % wichtig für \mathbb{} usw.
\usepackage{amsthm}                      % damit kann man eigene Theorem-Umgebungen definieren, proof-Umgebungen, etc.
\usepackage{anyfontsize}                 % supports fallback scaling for non-existing font size
\usepackage{array}                       % Für Tabellen mit Zeilenumbrüchen in Zellen
\usepackage[english, ngerman]{babel}	 % Paket für Sprachselektion, in diesem Fall für deutsches Datum etc
\usepackage{bbm}                         % sorgt für Symbol für Indikatorfunktion
% cleverref siehe ganz unten
\usepackage{color}                       % bringt Farbe ins Spiel
\usepackage{csquotes}                    % Man kann normale Anführungsstriche "..." im Textmodus nutzen
	\MakeOuterQuote{"}
\usepackage[
	type={CC},
	modifier={by-nc-sa},
    version={4.0},
]{doclicense}                            % für CC Lizenz-Vermerk
\usepackage{enumitem}                    % für custom Liste mit default Buchstaben
\usepackage{epigraph}                    % für wohlformatierte Zitate
\usepackage{float}                       % figure-Umgebungen besser positionieren
\usepackage[T1]{fontenc}                 % ö,ü,ä werden richtig kodiert
\usepackage{graphicx}                    % für Grafiken
\usepackage{grffile}
\usepackage[backref]{hyperref}           % Inhaltsverzeichnis und ref-Befehle werden in der PDF-klickbar
% inputenc siehe ganz oben
\usepackage{lmodern}                     % provides a bigger set of font sizes
\usepackage{makecell}                    % für Zellen mit Zeilenumbrüchen in Zellen
\usepackage{makeidx}                     % erstellt Index / Stichwortverzeichnis
\usepackage{mathrsfs}                    % für fancy \mathscr-Symbole
\usepackage{mleftright}                  % für \DeclarePairedDelimiter
\usepackage{pdflscape}                   % damit kann man einzelne Seiten ins Querformat drehen
\usepackage{pgfplots}                    % damit kann man Funktionen plotten
	\pgfplotsset{compat=newest}
\usepackage{scrhack}                     % provides a hack for deprecated float environments used by some libs
\usepackage{setspace}                    % wichtig für Lesbarkeit. Schöne Zeilenabstände
\usepackage{tikz}
	\usepackage{tikz-qtree}
	\usetikzlibrary{arrows}
	\usetikzlibrary{automata}
	\usetikzlibrary{babel}
	\usetikzlibrary{calc}
	\usetikzlibrary{cd}
	\usetikzlibrary{fit}
	\usetikzlibrary{matrix}
	\usetikzlibrary{positioning}
	\usetikzlibrary{shapes.geometric}
\usepackage{xargs}                       % for multiple optional args in newcommand
\usepackage{xfrac}
\usepackage{xparse}                      % wichtig für \DeclarePairedDelimiter
\usepackage{xr}                          % man Referenzen aus anderen teX-files importieren und darauf verlinken

% Ich habe gelesen, dass man folgendes Package zuletzt einbinden soll:
\usepackage[english, ngerman, capitalise]{cleveref} % bessere Verweise








	








