% This work is licensed under the Creative Commons
% Attribution-NonCommercial-ShareAlike 4.0 International License. To view a copy
% of this license, visit http://creativecommons.org/licenses/by-nc-sa/4.0/ or
% send a letter to Creative Commons, PO Box 1866, Mountain View, CA 94042, USA.

% PACKAGES
\usepackage[english, ngerman]{babel}	% Paket für Sprachselektion, in diesem Fall für deutsches Datum etc
\usepackage[utf8]{inputenc}	% Paket für Umlaute; verwende utf8 Kodierung in TexWorks 
\usepackage[T1]{fontenc} % ö,ü,ä werden richtig kodiert
\usepackage{amsmath} % wichtig für align-Umgebung
\usepackage{amssymb} % wichtig für \mathbb{} usw.
\usepackage{amsthm} % damit kann man eigene Theorem-Umgebungen definieren, proof-Umgebungen, etc.
\usepackage{mathrsfs} % für \mathscr
\usepackage[backref]{hyperref} % Inhaltsverzeichnis und \ref-Befehle werden in der PDF-klickbar
\usepackage{graphicx}
\usepackage{grffile}
\usepackage{setspace} % wichtig für Lesbarkeit. Schöne Zeilenabstände
\usepackage{enumitem} % für custom Liste mit default Buchstaben
%\usepackage[normalem]{ulem} % für bessere Unterstreichung
%\usepackage{contour} % für bessere Unterstreichung
\usepackage{epigraph} % für das coole Zitat
\usepackage{float}            % figure-Umgebungen besser positionieren
\usepackage{xfrac}
\usepackage{xr} % man Referenzen aus anderen teX-files importieren und darauf verlinken
\usepackage{bbm} %sorgt für Symbol für Indikatorfunktion
\usepackage{color} % bringt Farbe ins Spiel
\usepackage{pdflscape} % damit kann man einzelne Seiten ins Querformat drehen
\usepackage{aligned-overset} % besseres Einrücken, siehe: https://tex.stackexchange.com/questions/257529/overset-and-align-environment-how-to-get-correct-alignment
\usepackage{pgfplots}
	\pgfplotsset{compat=newest}

\usepackage[
    type={CC},
    modifier={by-nc-sa},
    version={4.0},
]{doclicense} % für CC Lizenz-Vermerk

\usepackage{tikz}
	\usepackage{tikz-qtree}
	\usetikzlibrary{arrows}
	\usetikzlibrary{automata}
  \usetikzlibrary{babel}
	\usetikzlibrary{calc}
	\usetikzlibrary{cd}
	\usetikzlibrary{fit}
	\usetikzlibrary{matrix}
	\usetikzlibrary{positioning}
	\usetikzlibrary{shapes.geometric}
	
\usepackage{csquotes}
	\MakeOuterQuote{"}

\usepackage{xargs} % for multiple optional args in newcommand
\usepackage{lmodern} % provides a bigger set of font sizes
\usepackage{anyfontsize} % supports fallback scaling for non-existing font size
\usepackage{scrhack} % provides a hack for deprecated float environments used by some libs

% Ich habe gelesen, dass man folgendes Package zuletzt einbinden soll:
\usepackage[english, ngerman, capitalise]{cleveref} % bessere Verweise
