% This work is licensed under the Creative Commons
% Attribution-NonCommercial-ShareAlike 4.0 International License. To view a copy
% of this license, visit http://creativecommons.org/licenses/by-nc-sa/4.0/ or
% send a letter to Creative Commons, PO Box 1866, Mountain View, CA 94042, USA.

\newcommand{\TemplateUebung}[3]{
	% Argument no. 1 is the author
	% Argument no. 2 is the name of the lecture
	% Argument no. 3 is the worksheet number

	\documentclass[12pt,a4paper]{article} 
	
	\input{\directoryPrefix packages}
	\input{\directoryPrefix theoremenvironments}
	\input{\directoryPrefix commands}
	\input{\directoryPrefix commands_Willi}

	\author{#1}
	\parindent0cm %Ist wichtig, um führende Leerzeichen zu entfernen

	\usepackage{fancyhdr} %für Kopf und Fußzeile
		\pagestyle{fancy}	
	\usepackage[left=30mm,right=33mm,top=25mm,bottom=25mm]{geometry} % Randeinstellungen
	% Skript-Vorlage hat eine Breite von 418.25555pt
	% Die Randeinstellungen für left und right sind so gewählt, dass die Breite des templates genau mit der Skript-Vorlagen-Breite übereinstimmt.
	% Es gilt also: Overfull-box im Template gdw. overfull box im Skript.
	% Diese Eigenschaft ist wichtig, da der content in beiden Wrappern eingebunden wird.
	
	\setlength{\headheight}{15pt}% ...at least 14.9998 pt
	
	\rhead{#1}
	\chead{}
	\lhead{#2}
	\rfoot{Blatt #3}
	\cfoot{Version: \today}
	\lfoot{Seite \thepage}
}

% der folgende Command ist veraltet und sollte nicht mehr genutzt werden.
% Aber wem das Design besser gefällt, kann es gern verwenden.
\newcommand{\TemplateUebungAlt}[3]{
	% Argument no. 1 is the author
	% Argument no. 2 is the name of the lecture
	% Argument no. 3 is the worksheet number

	\documentclass[12pt,a4paper]{article} 
	
	% Throw warning to inform about usage of wrong template
	\PackageWarning{MyTemplates}{
		Hinweis: Du nutzt ein veraltetes Template. 
		Nutze \space \protect\templateUebung\space statt \protect\TemplateUebungAlt
	}
	
	\input{\directoryPrefix packages}
	\input{\directoryPrefix theoremenvironments}
	\input{\directoryPrefix commands}
	\input{\directoryPrefix commands_Willi}

	\author{#1}
	\parindent0cm %Ist wichtig, um führende Leerzeichen zu entfernen

	\usepackage{scrpage2}
	\pagestyle{scrheadings}
	\clearscrheadfoot

	\ihead{#1}
	\chead{}
	\ohead{#2}
	\ifoot{Blatt #3}
	\cfoot{Version: \today}
	\ofoot{Seite \pagemark}
}