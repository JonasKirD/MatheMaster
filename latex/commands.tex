% This work is licensed under the Creative Commons
% Attribution-NonCommercial-ShareAlike 4.0 International License. To view a copy
% of this license, visit http://creativecommons.org/licenses/by-nc-sa/4.0/ or
% send a letter to Creative Commons, PO Box 1866, Mountain View, CA 94042, USA.

% Wichtige Mengen
\newcommand{\R}{\mathbb{R}}              % reelle Zahlen
\newcommand{\Rn}{\R^n}                   % der R^n
\newcommand{\N}{\mathbb{N}}              % natürliche Zahlen
\newcommand{\Q}{\mathbb{Q}}              % rationale Zahlen
\newcommand{\Z}{\mathbb{Z}}              % ganze Zahlen
\newcommand{\C}{\mathbb{C}}              % komplexe Zahlen
\renewcommand{\L}{\mathcal{L}}           % L_p-Räume
\newcommand{\ii}{\mathrm{i}\mkern1mu}    % imaginäre Einheit

% Schlüsselwörter
\renewcommand{\mit}{\text{ mit }}
\newcommand{\falls}{\text{falls }}
\newcommand{\und}{\text{ und }}

% Commands für Stochastik / Statistik
\newcommand{\A}{\mathcal{A}}           % Sigma-Algebra
\renewcommand{\P}{\mathbb{P}}          % Wahrscheinlichkeitsmaß
\newcommand{\E}{\mathbb{E}}            % Erwartungswert
\newcommand{\B}{\mathcal{B}}           % Borel-Sigma-Algebra
\newcommand{\Var}{\mathbb{V}\text{ar}} % Varianz
\newcommand{\Cov}{\mathbb{C}\text{ov}} % Kovarianz
\newcommand{\indi}{\mathbbm{1}}        % Indikatorfunktion

% Verteilungen
\DeclareMathOperator{\Bin}{Bin}        % Binomialverteilung
\newcommand{\Nor}{\mathcal{N}}         % Normalverteilung
\DeclareMathOperator{\Poi}{Poi}        % Poissonverteilung
\DeclareMathOperator{\Exp}{Exp}        % Exponentialverteilung
\DeclareMathOperator{\Cauchy}{Cauchy}  % Cauchyverteilung

% Wichtige Operatoren (alphabetisch)
\DeclareMathOperator*{\argmax}{arg\,max} % argmax
\DeclareMathOperator*{\argmin}{arg\,min} % argmin
\newcommand{\bigcupdot}{
	\mathop{\vphantom{\bigcup}\mathpalette\setbigcupdot\cdot}\displaylimits
}                                        % Disjunkte Vereinigung
	\newcommand{\setbigcupdot}[2]{\ooalign{\hfil$#1\bigcup$\hfil\cr\hfil$#2$\hfil\cr\cr}}
\newcommand*{\bigtimes}{
	\mathop{\raisebox{-.5ex}{\hbox{\huge{$\times$}}}}
}                                        % Kartesisches Produkt / Kreuzprodukt 
\renewcommand{\d}{\mathrm{d}}            % Differential d ohne Space
\newcommand{\ds}{~\mathrm{d}}            % Differential d mit Space für Integrale
\newcommand{\Def}{\text{Def}}            % wird genutzt für z.B. $\overset{\Def}{=}$
\DeclareMathOperator{\diag}{diag}        % Diagonalmatrix
\DeclareMathOperator*{\esssup}{ess\,sup} % essenzielles Supremum
\DeclareMathOperator{\grad}{grad}        % Gradient, alternativ zu \nabla
\DeclareMathOperator{\graph}{graph}      % Graph einer Funktion
\DeclareMathOperator{\Hom}{Hom}          % Raum linearer Funktionen
\newcommand{\ideal}{\trianglelefteq}     % Ideal
\DeclareMathOperator{\im}{im}            % Bild
\newcommand{\laplace}{\triangle}         % Laplace Operator
\DeclareMathOperator{\meas}{meas}        % Das Maß einer Menge, meist Lebesguemaß gemeint
\newcommand{\properideal}{
	\mathrel{\ooalign{$\lneqq$\cr\raise.51ex\hbox{$\lhd$}\cr}}
}                                        % echtes Ideal
\DeclareMathOperator{\SO}{SO}            % Orthogonale Gruppe
\DeclareMathOperator{\Span}{span}        % Span / Lineare Hülle
\DeclareMathOperator{\supp}{supp}        % Träger
\DeclareMathOperator{\tr}{tr}            % Spur
\newcommand{\Vertiii}[1]{
	{\left\vert\kern-0.25ex\left\vert\kern-0.25ex\left\vert #1\right\vert\kern-0.25ex\right\vert\kern-0.25ex\right\vert}
}                                        % dreifach gestrichene Norm

% WHITESPACE COMMANDS
\newcommand{\nl}{\\[\baselineskip]}      % Zeilenumbruch mit freier Zeile darunter OHNE underfull-hbox-warning
\newcommand{\enter}{$ $\newline}         % Zeilenumbruch ohne Einschränkungen
\newcommand\tab[1][1cm]{\hspace*{#1}}    % praktischer Tabulator

% Folgende Commands sollten nicht mehr genutzt werden und sind nur noch zwecks "Abwärtskompatibilität" vorhanden
\newcommand{\stackeq}[1]{\stackrel{#1}{=}} 
\newcommand{\ul}[1]{\underline{#1}}      

% TEXT ÜBER UND UNTER ZEICHEN
\newcommand{\stackrelnew}[3]{\underset{#1}{\overset{#2}{#3}}}
