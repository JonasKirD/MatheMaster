% This work is licensed under the Creative Commons
% Attribution-NonCommercial-ShareAlike 4.0 International License. To view a copy
% of this license, visit http://creativecommons.org/licenses/by-nc-sa/4.0/ or
% send a letter to Creative Commons, PO Box 1866, Mountain View, CA 94042, USA.

% STANDARD SHORTCUTS
%%%%%%%%%%%%%%%%%%%%%%%%%%%%%%%%%%%%%%%%%%%%
\newcommand{\R}{\mathbb{R}}				 % reelle Zahlen
\newcommand{\Rn}{\R^n}					 % der R^n
\newcommand{\N}{\mathbb{N}}				 % natürliche Zahlen
\newcommand{\Q}{\mathbb{Q}}				 % rationale Zahlen
\newcommand{\Z}{\mathbb{Z}}				 % ganze Zahlen
\newcommand{\C}{\mathbb{C}}			   % komplexe Zahlen
\renewcommand{\mit}{\text{ mit }}   % mit
\newcommand{\falls}{\text{falls }} % falls
\renewcommand{\d}{\text{ d}}        % Differential d
\DeclareMathOperator{\tr}{tr} % spur
\DeclareMathOperator{\diag}{diag}
\DeclareMathOperator{\Hom}{Hom}
\DeclareMathOperator{\Span}{span}
\DeclareMathOperator{\im}{im}
\DeclareMathOperator{\SO}{SO}
\newcommand{\ideal}{\trianglelefteq} %Ideal
\newcommand{\properideal}{\mathrel{\ooalign{$\lneqq$\cr\raise.51ex\hbox{$\lhd$}\cr}}} %echtes Ideal
\DeclareMathOperator{\supp}{supp}                 % Träger
\newcommandx{\bracket}[2][1=\cdot, 2=\cdot]{[#1,#2]}

% ETWAS SPEZIELLERE ZEICHEN
%%%%%%%%%%%%%%%%%%%%%%%%%%%%%%%%%%%%%%%%%%%%
% disjunkte Vereinigung
\newcommand{\bigcupdot}{
	\mathop{\vphantom{\bigcup}\mathpalette\setbigcupdot\cdot}\displaylimits
}
\newcommand{\setbigcupdot}[2]{\ooalign{\hfil$#1\bigcup$\hfil\cr\hfil$#2$\hfil\cr\cr}}
% großes Kreuz
\newcommand*{\bigtimes}{\mathop{\raisebox{-.5ex}{\hbox{\huge{$\times$}}}}} 
% dreifach gestrichene Norm
\newcommand{\Vertiii}[1]{{\left\vert\kern-0.25ex\left\vert\kern-0.25ex\left\vert #1 
    \right\vert\kern-0.25ex\right\vert\kern-0.25ex\right\vert}}
% korrektes argmin und argmax
\DeclareMathOperator*{\argmax}{arg\,max}
\DeclareMathOperator*{\argmin}{arg\,min}

% WHITESPACE COMMANDS
%%%%%%%%%%%%%%%%%%%%%%%%%%%%%%%%%%%%%%%%%%%%
% Zeilenumbruch mit freier Zeile darunter OHNE underfull-hbox-warning
\newcommand{\nl}{\\[\baselineskip]}
% nicht restriktiver newline command
\newcommand{\enter}{$ $\newline} 
% praktischer Tabulator
\newcommand\tab[1][1cm]{\hspace*{#1}}

% TEXT ÜBER ZEICHEN
\newcommand{\stackeq}[1]{\stackrel{#1}{=}} 

% TEXT ÜBER UND UNTER ZEICHEN
\newcommand{\stackrelnew}[3]{\underset{#1}{\overset{#2}{#3}}}

% UNDERLINE (wird nicht mehr genutzt)
% besseres underline 
%\renewcommand{\ULdepth}{1pt}
%\contourlength{0.5pt}
%\newcommand{\ul}[1]{
%	\uline{\phantom{#1}}\llap{\contour{white}{#1}}
%}
\newcommand{\ul}[1]{\underline{#1}} %Umleitung des Commands, da man sich gegen obigen entschieden hat. Dieser erzeugt zu viel Whitespace vor und nach dem Unterstrichenem.

