% This work is licensed under the Creative Commons
% Attribution-NonCommercial-ShareAlike 4.0 International License. To view a copy
% of this license, visit http://creativecommons.org/licenses/by-nc-sa/4.0/ or
% send a letter to Creative Commons, PO Box 1866, Mountain View, CA 94042, USA.

% Wichtige Mengen
\newcommand{\R}{\mathbb{R}}              % reelle Zahlen
\newcommand{\Rn}{\R^n}                   % der R^n
\newcommand{\N}{\mathbb{N}}              % natürliche Zahlen
\newcommand{\Q}{\mathbb{Q}}              % rationale Zahlen
\newcommand{\Z}{\mathbb{Z}}              % ganze Zahlen
\newcommand{\C}{\mathbb{C}}              % komplexe Zahlen
\newcommand{\K}{\mathbb{K}}              % Körper
\renewcommand{\L}{\mathcal{L}}           % L_p-Räume
\newcommand{\ii}{\mathrm{i}\mkern1mu}    % imaginäre Einheit
\renewcommand{\O}{\mathcal{O}}           % Landau-O

% Schlüsselwörter
\renewcommand{\mit}{\text{ mit }}       % sprachliches mit für align-Umgebungen
\newcommand{\falls}{\text{falls }}      % sprachliches falls für Fallunterscheidungen
\newcommand{\sonst}{\text{sonst }}      % sprachliches sonst für Fallunterscheidungen
\newcommand{\und}{\text{ und }}         % sprachliches und für align-Umgebungen
\newcommand{\Def}{\text{Def}}           % Voraussetzung, z.B. $\overset{\Def}{=}$
\newcommand{\Vor}{\text{Vor}}           % Voraussetzung, z.B. $\overset{\Vor}{=}$

% Commands für Stochastik / Statistik
\newcommand{\A}{\mathcal{A}}           % Sigma-Algebra
\renewcommand{\P}{\mathbb{P}}          % Wahrscheinlichkeitsmaß
\newcommand{\E}{\mathbb{E}}            % Erwartungswert
\newcommand{\B}{\mathcal{B}}           % Borel-Sigma-Algebra
\newcommand{\Var}{\mathbb{V}\text{ar}} % Varianz
\newcommand{\Cov}{\mathbb{C}\text{ov}} % Kovarianz
\newcommand{\indi}{\mathbbm{1}}        % Indikatorfunktion

% Verteilungen
\DeclareMathOperator{\Bin}{Bin}        % Binomialverteilung
\newcommand{\Nor}{\mathcal{N}}         % Normalverteilung
\DeclareMathOperator{\Poi}{Poi}        % Poissonverteilung
\DeclareMathOperator{\Exp}{Exp}        % Exponentialverteilung
\DeclareMathOperator{\Cauchy}{Cauchy}  % Cauchyverteilung

% Wichtige Operatoren (alphabetisch)
\newcommand{\AND}{\wedge}                % logisches Und
\DeclareMathOperator*{\argmax}{arg\,max} % argmax
\DeclareMathOperator*{\argmin}{arg\,min} % argmin
\newcommand{\bigcupdot}{
	\mathop{\vphantom{\bigcup}\mathpalette\setbigcupdot\cdot}\displaylimits
}                                        % Disjunkte Vereinigung
	\newcommand{\setbigcupdot}[2]{\ooalign{\hfil$#1\bigcup$\hfil\cr\hfil$#2$\hfil\cr\cr}}
\newcommand*{\bigtimes}{
	\mathop{\raisebox{-.5ex}{\hbox{\huge{$\times$}}}}
}                                        % Kartesisches Produkt / Kreuzprodukt 
\renewcommand{\d}{\mathrm{d}}            % Differential d ohne Space
\newcommand{\ds}{~\mathrm{d}}            % Differential d mit Space für Integrale
\DeclareMathOperator{\diag}{diag}        % Diagonalmatrix
\DeclareMathOperator{\diam}{diam}        % Diameter, Durchmesser einer Menge
\let\divison\div                         % save it before overwriting:
	\DeclareMathOperator{\BigDiv}{div}   % Divergenz
	\renewcommand{\div}{\BigDiv}         % Divergenz
\DeclareMathOperator{\dom}{dom}          % Domain, Definitionsbereich
\DeclareMathOperator*{\esssup}{ess\,sup} % essenzielles Supremum
\DeclareMathOperator{\grad}{grad}        % Gradient, alternativ zu \nabla
\DeclareMathOperator{\graph}{graph}      % Graph einer Funktion
\DeclareMathOperator{\Hom}{Hom}          % Raum linearer Funktionen
\DeclareMathOperator{\id}{id}            % identische Abbildung
\newcommand{\ideal}{\trianglelefteq}     % Ideal
\DeclareMathOperator{\im}{im}            % Bild
\DeclareMathOperator{\inner}{int}        % Das innere einer Menge
\newcommand{\laplace}{\triangle}         % Laplace Operator
\newcommand{\limn}{\lim\limits_{n\to\infty}} % Standard-Limes
\newcommand{\mal}{\cdot}                 % hier kann man \cdot ein- und ausschalten
\DeclareMathOperator{\meas}{meas}        % Das Maß einer Menge, meist Lebesguemaß gemeint
\newcommand{\OR}{\vee}                   % logisches Oder
\newcommand{\properideal}{
	\mathrel{\ooalign{$\lneqq$\cr\raise.51ex\hbox{$\lhd$}\cr}}
}                                        % echtes Ideal
\newcommand{\rand}{\partial}             % Rand einer Menge
\newcommand{\Potenzmenge}[1]{\mathfrak{P}\left(#1\right)} % Potenzmenge
\DeclareMathOperator{\rg}{rg}            % Bild einer Funktion, Rang einer Matrix
\DeclareMathOperator{\SO}{SO}            % Orthogonale Gruppe
\DeclareMathOperator{\Span}{span}        % Span / Lineare Hülle
	\newcommand{\spann}{\Span}           % Span / Lineare Hülle
\DeclareMathOperator{\supp}{supp}        % Träger
\DeclareMathOperator{\tr}{tr}            % Spur

% Normen, Betrag, Skalarprodukt
% Quelle: https://tex.stackexchange.com/questions/94410/easily-change-behavior-of-declarepaireddelimiter
% Beispiel: \set{...}; \set*{...}=\set[]{...}; \set[\big]{...}, \set[\Big]{...}, ...
\NewDocumentCommand\xDeclarePairedDelimiter{mmm} % unär
{
	\NewDocumentCommand#1{som}{
   		\IfNoValueTF{##2}
	    {\IfBooleanTF{##1}{#2##3#3}{\mleft#2##3\mright#3}}
    	{\mathopen{##2#2}##3\mathclose{##2#3}}
	}
}

\NewDocumentCommand\xDeclarePairedDelimiterBin{mmm} % binär
{
	\NewDocumentCommand#1{somm}{
   		\IfNoValueTF{##2}
	    {\IfBooleanTF{##1}{#2##3,##4#3}{\mleft#2##3,##4\mright#3}}
    	{\mathopen{##2#2}##3,##4\mathclose{##2#3}}
	}
}
 
\xDeclarePairedDelimiter{\set}{\lbrace}{\rbrace}
\xDeclarePairedDelimiter{\norm}{\Vert}{\Vert}
\xDeclarePairedDelimiter{\abs}{\vert}{\vert}
\xDeclarePairedDelimiterBin{\scaProd}{\langle}{\rangle}
\xDeclarePairedDelimiter{\klammern}{(}{)}

\NewDocumentCommand\normiii{som}{
	\IfNoValueTF{#2} %falls optionales Argument leer (d.h. kein "\Big")
	{\IfBooleanTF{#1}
		{\vert\kern-0.25ex \vert\kern-0.25ex \vert #3 \vert\kern-0.25ex \vert\kern-0.25ex\vert} %normale Darstellung
		{\mleft\vert\kern-0.25ex \mleft\vert\kern-0.25ex \mleft\vert #3 \mright\vert\kern-0.25ex \mright\vert\kern-0.25ex \mright\vert}
	}
    {\mathopen{#2\vert\kern-0.25ex #2\vert\kern-0.25ex #2\vert} #3 \mathclose{#2\vert\kern-0.25ex #2\vert\kern-0.25ex #2\vert}}
}

% WHITESPACE COMMANDS
\newcommand{\nl}{\\[\baselineskip]}      % Zeilenumbruch mit freier Zeile darunter OHNE underfull-hbox-warning
\newcommand{\enter}{$ $\newline}         % Zeilenumbruch ohne Einschränkungen
\newcommand\tab[1][1cm]{\hspace*{#1}}    % praktischer Tabulator

% overset und Co.
\newcommand{\stackrelnew}[3]{\underset{#1}{\overset{#2}{#3}}}
\newcommand{\stackeq}[1]{\overset{#1}{=}}  

% Hervorhebungen
\newcommand{\ul}[1]{\underline{#1}}    
\newcommand{\define}[1]{\textbf{#1}}     % Definitionen sind standardmäßig fett
\newcommand{\undefine}[1]{\textit{#1}}   % Markiert das Nutzen von "externen" Resultaten oder Begriffen, die nicht in der Vorlesung eingeführt wurden.
\newcommand{\betone}[1]{\underline{#1}}  % Das zu Betonende ist standardmäßig unterstrichen

% Warnings and errors
\newcommand{\throwWarning}[1]{\PackageWarning{42}{#1}} % Erzeugt LaTeX-Warnung mit Message = #1 