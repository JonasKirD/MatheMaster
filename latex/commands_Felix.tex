% This work is licensed under the Creative Commons
% Attribution-NonCommercial-ShareAlike 4.0 International License. To view a copy
% of this license, visit http://creativecommons.org/licenses/by-nc-sa/4.0/ or
% send a letter to Creative Commons, PO Box 1866, Mountain View, CA 94042, USA.

% hier einige Befehle, die ich im Sinne der selbst-erklärheit der Codes sinnvoll finde:

\newcommand{\abs}[1]{\left\lvert#1\right\rvert}     % absolute value
\newcommand{\measure}[1]{\abs{#1}}    % measure (size) of a set
\newcommand{\halfnorm}[1]{\left\lvert#1\right\rvert}     % Halbnorm
\newcommand{\norm}[1]{\left\lVert#1\right\rVert}    % norm
\newcommand{\scaProd}[2]{\left\langle#1, #2\right\rangle} % scalar product, could also use \rangle, \langle
\newcommand{\dualpair}[2]{\left\langle #1, #2 \right\rangle} % dual pairing, could also be #1 \left(#2\right)
\newcommand{\Rand}[1]{\partial #1}  % Rand einer Menge. Wenn im Code \partial steht, verwirrt mich das, denn ich erwarte eine differenzierte Funktion 
\newcommand{\setDef}[2]{\left\{#1 \,:\, #2\right\}}            % a typical set definition: #1 is the thing in the set #2 the condition on the variables {#1 : #2}
% alternative to ':' is \middle\vert

\usepackage{silence}                  % Package that lets ignore warnings and errors for unicodechar redefining
\WarningFilter{newunicodechar}{Redefining Unicode character}

\usepackage{newunicodechar}               % makes this remapping possible
% \RequirePackage{eurosym}                  % provides euro symbol

% Ebene 2 german „“
\newunicodechar{•}{\item}

% Ebene 3
\newunicodechar{≡}{\ensuremath{\equiv}}
\newunicodechar{≈}{\ensuremath{\approx}}
\newunicodechar{…}{\dots}
\newunicodechar{⊇}{\ensuremath{\supseteq}}

% Ebene 4 
\newunicodechar{✓}{\checkmark}
\newunicodechar{·}{\ensuremath{\cdot}}
\newunicodechar{ℱ}{\ensuremath{\mathcal F}}
\newunicodechar{⁚}{\ensuremath{\colon}}
\newunicodechar{±}{\ensuremath{\pm}}
\newunicodechar{∓}{\ensuremath{\mp}}
\newunicodechar{⊃}{\ensuremath{\supset}}

% Ebene 5
\newunicodechar{α}{\ensuremath{\alpha}}
\newunicodechar{β}{\ensuremath{\beta}}
\newunicodechar{γ}{\ensuremath{\gamma}}
\newunicodechar{δ}{\ensuremath{\delta}}
\newunicodechar{ε}{\ensuremath{\varepsilon}}
\newunicodechar{ζ}{\ensuremath{\zeta}}
\newunicodechar{η}{\ensuremath{\eta}}
\newunicodechar{θ}{\ensuremath{\theta}}
\newunicodechar{ι}{\ensuremath{\iota}}
\newunicodechar{κ}{\ensuremath{\kappa}}
\newunicodechar{λ}{\ensuremath{\lambda}}
\newunicodechar{μ}{\ensuremath{\mu}}
\newunicodechar{ν}{\ensuremath{\nu}}
\newunicodechar{ξ}{\ensuremath{\xi}}
\newunicodechar{π}{\ensuremath{\pi}}
\newunicodechar{ρ}{\ensuremath{\rho}}
\newunicodechar{σ}{\ensuremath{\sigma}}
\newunicodechar{τ}{\ensuremath{\tau}}
\newunicodechar{υ}{\ensuremath{\upsilon}}
\newunicodechar{φ}{\ensuremath{\varphi}}
\newunicodechar{ϕ}{\ensuremath{\phi}}
\newunicodechar{χ}{\ensuremath{\chi}}
\newunicodechar{ψ}{\ensuremath{\psi}}
\newunicodechar{ω}{\ensuremath{\omega}}
\newunicodechar{ϑ}{\ensuremath{\vartheta}}
\newunicodechar{ς}{\ensuremath{\varsigma}}
\newunicodechar{ϱ}{\ensuremath{\varrho}}
\newunicodechar{≤}{\ensuremath{\leq}}
\newunicodechar{≥}{\ensuremath{\geq}}
\newunicodechar{⊆}{\ensuremath{\subseteq}}


% Ebene 5 top row:
\newunicodechar{⟨}{\ensuremath{\left\langle}}
\newunicodechar{⟩}{\ensuremath{\right\rangle}}

% Ebene 6 NEO top row:
\newunicodechar{¬}{\ensuremath{\neg}}
\newunicodechar{∧}{\ensuremath{\wedge}}
\newunicodechar{∨}{\ensuremath{\vee}}
\newunicodechar{⊥}{\ensuremath{\perp}}
\newunicodechar{∡}{\ensuremath{\measuredangle}}
\newunicodechar{∥}{\ensuremath{\parallel}}
\newunicodechar{⟶}{\ensuremath{\to}}            % ! Differs from NEO !
\newunicodechar{∞}{\ensuremath{\infty}}
\newunicodechar{∅}{\ensuremath{\emptyset}}

\newunicodechar{Ξ}{\ensuremath{\Xi}}
\newunicodechar{√}{\ensuremath{\sqrt}}
\newunicodechar{Λ}{\ensuremath{\Lambda}}
\newunicodechar{ℂ}{\ensuremath{\mathbb{C}}}
\newunicodechar{Ω}{\ensuremath{\Omega}}
\newunicodechar{×}{\ensuremath{\times}}
\newunicodechar{Ψ}{\ensuremath{\Psi}}
\newunicodechar{Γ}{\ensuremath{\Gamma}}
\newunicodechar{Φ}{\ensuremath{\Phi}}
\newunicodechar{ℚ}{\ensuremath{\mathbb{Q}}}
\newunicodechar{∘}{\ensuremath{\circ}}
\newunicodechar{⊂}{\ensuremath{\subset}}
\newunicodechar{∫}{\ensuremath{\int}}
\newunicodechar{∀}{\ensuremath{\forall}}
\newunicodechar{∃}{\ensuremath{\exists}}
\newunicodechar{∈}{\ensuremath{\in}}
\newunicodechar{Σ}{\ensuremath{\sum}} % not capital sigma which is smaller!
\newunicodechar{ℕ}{\ensuremath{\mathbb{N}}}
\newunicodechar{ℝ}{\ensuremath{\mathbb{R}}}
\newunicodechar{∂}{\ensuremath{\partial}}
\newunicodechar{Δ}{\ensuremath{\Delta}}
\newunicodechar{∇}{\ensuremath{\nabla}}
\newunicodechar{∪}{\ensuremath{\cup}}
\newunicodechar{∩}{\ensuremath{\cap}}
\newunicodechar{Π}{\ensuremath{\prod}} % not capital pi which is smaller!
\newunicodechar{ℤ}{\ensuremath{\mathbb{Z}}}
\newunicodechar{⇒}{\ensuremath{\implies}}
\newunicodechar{⇐}{\ensuremath{\impliedby}}
\newunicodechar{⇔}{\ensuremath{\iff}}
\newunicodechar{↦}{\ensuremath{\mapsto}}
\newunicodechar{Θ}{\ensuremath{\Theta}}

