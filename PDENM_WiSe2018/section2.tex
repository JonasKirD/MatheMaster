% This work is licensed under the Creative Commons
% Attribution-NonCommercial-ShareAlike 4.0 International License. To view a copy
% of this license, visit http://creativecommons.org/licenses/by-nc-sa/4.0/ or
% send a letter to Creative Commons, PO Box 1866, Mountain View, CA 94042, USA.
% vim: set noexpandtab:

\section{Abstrakte Variations-Analysis}
Die \textbf{Poissongleichung} ist
\begin{align}
	\left\{
	\begin{array}{r l l}\label{PoissonEquation}\tag{Poi}
		-\Delta u &\equiv f &\text{ auf }\Omega \\
		u &\equiv 0 &\text{ auf }\partial\Omega=:\Gamma.
	\end{array}\right.
\end{align}
Durch Multiplikation mit $v$ und Integration erhalten wir
\begin{align*}
	&\stackrel{ v|_{\partial\Omega}\equiv0}{\implies}
	\int\limits_\Omega f\cdot v\d x
	\overset{\eqref{PoissonEquation}}=
	\int\limits_\Omega-\laplace u\cdot v\d x
	\overset{\eqref{eqGreen}}=
	\int\limits_\Omega\nabla u\cdot\nabla v\d x-\underbrace{\int\limits_{\partial\Omega}\frac{\partial u}{\partial n}v\d x}_{=0}
\end{align*}
Definiere:
\begin{align*}
	\text{Bilinearform } a(u,v)&:=\int\limits_\Omega\nabla u\cdot\nabla v\d x\\
	\text{Linearform } l(v)&:=\int\limits_\Omega f\cdot v\d x
\end{align*}
$a$ und $l$ sind wohldefiniert für $f\in L^2(\Omega),~u,v\in H^1(\Omega)$.
Um $v|_{\partial\Omega}=0$ und $u|_{\partial\Omega}=0$ sicherzustellen, muss $u,v\in H^1_0(\Omega)$ gelten.\nl
\textbf{Variationsformulierung der Poissongleichung:}\\
Finde $u\in H_0^1(\Omega)$ so, dass
\begin{align*}
	a(u,v)=l(v)\qquad\forall v\in H^1_0(\Omega).
\end{align*}
Dieses Problem gilt es zu lösen.
Allgemeiner formuliert erhält man das folgende Problem:\\
Sei $V$ ein Hilbertraum. Finde $u\in V$ so, dass
\begin{align*}
	a(u,v)=l(v)\qquad\forall v\in V.
\end{align*}

Eigenschaften der oben definieren Bilinearform $a$:
\begin{itemize}
	\item $a$ ist stetig auf $V$, d. h.
	\begin{align*}
		\exists M>0:\forall u,v\in V:\big|a(u,v)\big|\leq M\cdot\Vert u\Vert_V\cdot\Vert v\Vert_V
	\end{align*}
	denn:
	\begin{align*}
		\big| a(u,v)\big|=\left|\int\limits_\Omega\nabla u\cdot\nabla v\d x\right|
		\overset{\eqref{eqCS}}\leq
		|u|_{1,2,\Omega}\cdot|v|_{1,2,\Omega} % stand so drin, aber macht wenig Sinn
		\overset{\Def}\leq
		\underbrace{1}_{=:M}\cdot\Vert u\Vert_{1,2,\Omega}\cdot\Vert v\Vert_{1,2,\Omega}
	\end{align*}
	\item $a$ ist \textbf{koerziv}, d. h.
	\begin{align*}
		\exists\alpha>0:\forall v\in V:a(v,v)\geq\alpha\cdot\Vert v\Vert^2_V
	\end{align*}
	denn:
	\begin{align*}
		a(v,v)
		\overset{\Def}=
		\int\limits_\Omega\nabla v\cdot\nabla v\d x
		\overset{\Def}=
		\int\limits_\Omega\sum\limits_{i=1}^d\left(\frac{\partial v}{\partial x_i}\right)^2\d x
		\overset{\Def}=
		|v|^2_{1,2,\Omega}
		\stackrel{\text{Fried}}{\geq}
		\frac{1}{c_F^2}\cdot\Vert v\Vert^2_{1,2,\Omega}
	\end{align*}
\end{itemize}

Eigenschaften der oben definierten Linearform $l$:
\begin{itemize}
	\item $l$ ist stetig auf $V$, d. h.
	\begin{align*}
		\exists M^\ast>0:\forall v\in V: \big|l(v)\big|\leq M^\ast\cdot\Vert v\Vert_V
	\end{align*}
	denn:
	\begin{align*}
		\big| l(v)\big|
		\overset{\Def}=
		\left|\int\limits_\Omega f\cdot v\d x\right|
		\overset{\eqref{eqCSW02}}{\leq}
		\Vert f\Vert_{0,2,\Omega}\cdot\Vert v\Vert_{0,2,\Omega}
		\stackrel{\Def}{\leq}
		\underbrace{\Vert f\Vert_{0,2,\Omega}}_{=:M^\ast}\cdot\Vert v\Vert_{1,2,\Omega}
	\end{align*}
\end{itemize}

\begin{theorem}[Lax-Milgram]\label{theorem2.1LaxMilgram}\enter
	Sei $V$ ein Hilbertraum, $a$ eine stetige, koerzive Bilinearform auf $V$ und $b$ eine stetige Linearform auf $V$.\\
	Dann hat das Variationsproblem
	\begin{align*}
		\text{Bestimme $u\in V$ so, dass }a(u,v)=b(v)\qquad\forall v\in V.
	\end{align*}
	eine eindeutige Lösung.
\end{theorem}

\begin{proof}
	Wir verwenden in diesem Beweis folgendes Resultat der Funktionalanalysis:

	\begin{thm}[Darstellungssatz von Riesz-Fréchet]\enter
		Sei $(H,\langle\cdot,\cdot\rangle)$ ein Hilbertraum. Dann existiert zu jedem stetigen Funktional $A\in H^\ast$ genau ein $h\in H$ so, dass
		\begin{align*}
			A(v)=\langle v,h\rangle\qquad\forall v\in H
			\qquad\und\qquad\Vert A\Vert_{H^\ast}=\Vert h\Vert_H
		\end{align*}
	\end{thm}
	
	\underline{Schritt 1:}\\
	Transformation in einen dualen Operator über dem Dualraum $V^\ast$ von $V$.
	Betrachte also für festes $u\in V$ den Operator
	\begin{align*}
		Au:V\to \R,\qquad v\mapsto Au(v):=a(u,v)
	\end{align*}
	Offenbar ist $Au$ eine stetige Linearform auf $V$, also $ Au\in V^\ast$. Dann gilt:
	\begin{align*}
		\langle \underbrace{Au}_{\hat{=}h},v\rangle\overset{\text{Riesz}}{=}(Au)(v)\overset{\Def}&=a(u,v)&\forall& u,v\in V \\
		b\overset{\text{Vor}}{\in} V^\ast\implies \langle b,v\rangle\overset{\text{Riesz}}&{=}b(v) &\forall& v\in V
	\end{align*}
	Damit folgt:
	\begin{align*}
		\text{Variationsproblem}&\Longleftrightarrow\langle Au,v\rangle=\langle b,v\rangle\qquad\forall v\in V\\
		&\Longleftrightarrow Au=b\text{ in }V^\ast
	\end{align*}

	\underline{Schritt 2:}\\
	Die Abbildung $Au\colon V\to\R$ kann auch als Abbildung $A:V\to V^\ast$ ($u\mapsto Au$) aufgefasst werden.
	Eigenschaften von $A$:
	\begin{enumerate}[label=(\alph*)]
		\item $A$ ist linear auf $V$
		\begin{align*}
			\big\langle A(\alpha u+\beta v),w\big\rangle
			&=a(\alpha u+\beta v,w)\\
			&=\alpha a(u,w)+\beta a(v,w)\\
			&=\alpha\langle Au,w\rangle+\beta\langle Av,w\rangle\qquad\forall u,v,w,\in V
		\end{align*}
		\item $A$ ist stetig auf $V$:
		\begin{align*}
			\Vert A v\Vert_{V}
			&=\Vert A u\Vert_{V^\ast}\\
			&=\sup\limits_{v\neq0}\frac{\big|\langle Au,v\rangle\big|}{\Vert v\Vert_V}\\
			&=\sup\limits_{v\neq0}\frac{\big|a(u,v)\big|}{\Vert v\Vert_V}\\
			&\leq
			\sup\limits_{v\neq0}\frac{M\cdot\Vert u\Vert_V\cdot\Vert v\Vert_V}{\Vert v\Vert_V}\\
			&=M\cdot\Vert u\Vert_V
		\end{align*}
	\end{enumerate}

	\underline{Schritt 3:} Setze
	\begin{align*}
		u^{n+1}:=P(u^n):=u^n+\tau\cdot(b-A u^n)\qquad\forall\tau\in\R_{\neq0}
	\end{align*}
	$u^0$ ist eine beliebiger Startwert.
	\begin{align*}
		P:V\to V,\qquad u\mapsto u+\tau\cdot(b-Au)
	\end{align*}
	Jeder Fixpunkt von $P$ ist eine Lösung von $Au=b$ und umgekehrt.\\
	Es bleibt also zu zeigen, das $P$ einen eindeutigen Fixpunkt hat.
	Dafür nutzen wir den \textit{Banach'schen Fixpunktsatz} \ref{BanachscherFixpunktsatz}.\nl
	Wähle also $N=V$.
	$P$ ist eine Kontraktion, wenn $\tau$ richtig gewählt wird.
	\begin{align*}
		&\quad\Vert P(u)-P)v)\Vert^2_V\\
		&=\big\langle P(u)-P(v),P(u)-P(v)\big\rangle_V\\
		&=\big\langle u-v-\tau\cdot A(u-v),u-v-\tau\cdot A(u-v)\big\rangle_V\\
		&=\langle u-v,u-v\rangle_V-2\cdot\tau\cdot\underbrace{\big\langle A(u-v),u-v\big\rangle_V}_{\stackrel{\text{Def.}}{=}a(u-v,u-v)\stackrel{\text{koerz}}{\geq}\alpha\cdot\Vert u-v\Vert^2_V}+\tau^2\cdot\underbrace{\big\langle A(u-v),A(u-v)\big\rangle_V}_{=\Vert A(u-v)\Vert^2_V\leq M^2\cdot\Vert u-v\Vert^2_V}\\
		\overset{\tau>0}&{\leq}
		\Vert u-v\Vert^2_V\cdot\big(1-2\cdot\tau\cdot\alpha+\tau^2\cdot M^2\big)
	\end{align*}
	Wähle $\tau:=\frac{\alpha}{M^2}>0$.
	Dann gilt:
	\begin{align*}
		1-2\cdot\tau\cdot\alpha+\tau^2\cdot M^2=1-\frac{\alpha^2}{M^2}>0
	\end{align*}
	$P$ ist eine Kontraktion, also hat $P$ nach dem Banachschen Fixpunktsatz einen eindeutigen Fixpunkt und damit hat auch $Au=b$ eine eindeutige Lösung.
\end{proof}

Schwierigkeit:
\begin{align*}
	a(u,v)=b(v)\qquad\forall v\in V
\end{align*}
ist ein unendlichdimensionales Problem.\\
\underline{Idee:} Approximation in einem endlich-dimensionalen Unterraum\\
Sei $V_h\subseteq V$ ein endlich-dimensionaler Unterraum von $V$.
Betrachte das Problem
\begin{align*}
	\text{Finde }u_n\in V_h \text{ so, dass }\qquad \\
	a(u_n,v_h)=l(v_h)\qquad\forall v_h\in V_h
\end{align*}
Da $a$ und $l$ auf $V_h$ die Voraussetzungen des Satzes von Lax-Milgram auf $V_h$ erfüllen, gibt es ein eindeutiges $u_n\in V_h$, das das Problem löst.\nl
Sei $\lbrace\varphi_i\rbrace_{i=1}^n$ eine Basis von $V_h$.
\begin{align*}
	\Big(a(u_n,v_h)=l(v_h)\qquad\forall v_h\in V_h\Big)\overset{\text{Basis}}\Longleftrightarrow
	\Big(a(u_n,\varphi_i)=l(\varphi_i)\qquad\forall i\in\lbrace1,\ldots,n\rbrace\Big)
\end{align*}
und
\begin{align*}
	u_n\overset{\text{Basis}}&=\sum\limits_{j=1}^n u_j\cdot\varphi_j &\exists&u_1,\ldots,u_n\in\R\\
	\implies\sum\limits_{j=1}^n a(\varphi_j,\varphi_i)\cdot u_j
	&=a\left(\sum\limits_{j=1}^n u_j\cdot\varphi_j,\varphi_i\right)	
	\overset{\text{Lin}}=l(\varphi_i)&\forall& i\in\lbrace1,\ldots,n\rbrace
\end{align*}
Daraus ergibt sich ein lineares Gleichungssystem:
\begin{align*}
	\left.\begin{array}{ll}
		A:=(a_{i,j})_{i,j=1,\ldots,n} &a_{i,j}:=a\big(\varphi_j,\varphi_i\big)\\
		b:=(b_i)_{i=1,\ldots,n} &b_i:=l(\varphi_i)\\
		u:=(u_j)_{j=1,\ldots,n}
	\end{array}\right\rbrace A\cdot u=b
\end{align*}

\begin{theorem}[Céa's Lemma]\enter\label{theorem2.2CeasLemma}
	Sei $V$ ein Hilbertraum, $V_h\subseteq V$ ein abgeschlossener Unterraum, $a:V\times V\to\R$ eine $M$-stetige (beschränkte), $\alpha$-koerzive Bilinearform auf $V$ und $l:V\to\R$ eine stetige Linearform.
	Dann gilt:
	\begin{align*}
		\exists C=\frac{M}{\alpha}>0:\Vert u-u_h\Vert_V\leq C\cdot\inf\limits_{v_h\in V_h}\Vert u-v_h\Vert_V
	\end{align*}
\end{theorem}

\begin{proof}
	Wir betrachten das stetige und diskrete Problem:
	\begin{align*}
		a(u,v)&=l(v) &\forall& v\in V\\
		a(u_h,v_h)&= l(v_h) &\forall& v_h\in V_h\subseteq V
	\end{align*}
	Wir erhalten die sogenannte \textbf{Galerkin Orthogonalität}, indem wir die Differenz dieser beiden Gleichungen bilden 
	(mit $v=v_h$ in der ersten Gleichung, da $V_h\subseteq V$):
	\begin{align}\label{eqGalerkinOrthogonalitaet}\tag{GO}
		a(u-u_h,v_h)=0\qquad v_h\in V_h
	\end{align}
	Also:
	\begin{align*}
		\alpha\cdot\Vert u-u_h\Vert^2_V
		\overset{\text{koerziv}}&\leq
		a(u-u_h,u-u_h)\\
		&=a(u-u_h,u-v_v+v_h- u_h)\\
		\overset{\text{bilin}}&=
		a(u-u_h,u-v_h)+\underbrace{a(u-u_h,\underbrace{v_h-u_h}_{\in V_h})}_{\overset{\eqref{eqGalerkinOrthogonalitaet}}=0}\\
		\overset{\text{stetig}}&{\leq}
		M\cdot\Vert u-u_h\Vert_V\cdot\Vert u-v_h\Vert_V\\
		\implies
		\Vert u-u_h\Vert_V&\leq\frac{M}{\alpha}\cdot\Vert u-v_h\Vert_V\qquad\forall v_h\in V_h\\
		\implies
		\Vert u-u_h\Vert_V&\leq \frac{M}{\alpha}\cdot\inf\limits_{v_h\in V_h}\Vert u-v_h\Vert_V
	\end{align*}
\end{proof}

\begin{theorem}[Dualitätsargument, Aubin-Nitsche]\enter\label{theorem2.3AubinNitscheDualitaetsargument}
	Sei $V$ ein Hilbertraum, $V_h\subseteq V$ ein abgeschlossener Unterraum, $a$ eine stetige, koerzive Bilinearform und $l$ eine stetige Linearform.
	Sei weiterhin $H$ ein Hilbertraum mit Skalarprodukt $\langle\cdot,\cdot\rangle_H$ und Norm $\Vert\cdot\Vert_H$ so, dass $V\hookrightarrow H$ und $V$ liegt dicht in $H$ bzgl. $\Vert\cdot\Vert_H$.
	Sei zusätzlich $u_\varphi\in V$ für jedes feste $\varphi\in H$ die eindeutige Lösung von
	\begin{align}\label{Dtheorem2.3}\tag{D}
		a(v,u_\varphi)=\langle\varphi,v\rangle_H\qquad\forall v\in V
	\end{align}
	Dann gilt:
	\begin{align*}
		\Vert u-u_h\Vert_H\leq M\cdot\Vert u-u_h\Vert_V\cdot\sup\limits_{\begin{subarray}{c}
			\varphi\in H\\
			\Vert\varphi\Vert_H=1
		\end{subarray}}\inf\limits_{v_h\in V_h}\Vert u_\varphi-v_h\Vert_V
	\end{align*}
\end{theorem}

\begin{proof}
	$v\mapsto\langle\varphi,v\rangle_H$ ist eine stetige Linearform auf $V$.
	Aus Lax-Milgram \ref{theorem2.1LaxMilgram} folgt, dass es eine eindeutige Lösung $u_\varphi\in V$ gibt.

	\begin{align*}
		\langle u-u_h,\varphi\rangle_H
		&=\langle\varphi,u-u_h\rangle_H\\
		&=a(u-u_h,u_\varphi)\\
		&=a(u-u_h,u_\varphi-v_h)\mit v_h\in V\text{ beliebig}\\
		&\leq M\cdot\Vert u-u_h\Vert_V\cdot\Vert u_\varphi-v_h\Vert_V\\
		\Vert u-u_h\Vert_H
		&=\sup\limits_{\begin{subarray}{c}
			\varphi\in H\\
			\Vert\varphi\Vert=1
		\end{subarray}}\langle u-u_h,\varphi\rangle_H\\
		&\leq M\cdot\Vert u-u_h\Vert_V\cdot\sup\limits_{\begin{subarray}{c}
			\varphi\in H\\
			\Vert\varphi\Vert=1
		\end{subarray}}\inf\limits_{v_h\in V_h}\Vert u_\varphi-v_h\Vert_V
	\end{align*}
\end{proof}

\begin{theorem}\label{theorem2.4}
	Seien $(X,\Vert\cdot\Vert_X)$, $(Y,\Vert\cdot\Vert_Y)$ Banachräume mit $X\stackrel{C}{\hookrightarrow} Y$ und $a_0,a_1$ stetige Bilinearformen auf $X$. 	Gelte weiterhin:
	\begin{itemize}
		\item $a_0$ ist symmetrisch und koerziv auf $X$
		\item $\begin{aligned}
			\exists A>0:\forall u,v\in X:\big|a_1(u,v)\big|\leq A\cdot\Vert u\Vert_X\cdot\Vert v\Vert_Y
		\end{aligned}$
	\end{itemize}
	Erfülle
	\begin{align*}
		a(u,v):=a_0(u,v)+a_1(u,v)\qquad\forall u,v\in X
	\end{align*}
	die Ungleichung
	\begin{align*}
		a(v,v)>0\qquad\forall v\in X\setminus\lbrace0\rbrace.
	\end{align*}
	Dann sind die beiden Probleme
	\begin{align*}
		\text{Finde }u\in X\text{ so, dass }
		a(u,v)=l(v)\qquad\forall v\in X\\
		\text{Finde }\tilde{u}\in X\text{ so, dass }
		a(v,\tilde{u})=l(v)\qquad\forall v\in X
	\end{align*}
	jeweils eindeutig lösbar für alle stetigen Linearformen $l$.
\end{theorem}

\begin{theorem}\label{theorem2.5}
	Gleiche Voraussetzungen wie Theorem \ref{theorem2.4}.\\
	Sei zusätzlich $X_h\subseteq X$ ein endlich-dimensionaler Unterraum von $X$.\\
	Dann ist das Problem
	\begin{align*}
		\text{Finde }u_h\in X_h\text{ so, dass }
		a(u_h,v_h)=l(v_h)\qquad\forall v_h\in X_h
	\end{align*}
	eindeutig lösbar für jede feste stetige Linearform $l$.
	Zusätzlich gilt
	\begin{align*}
		\Vert u-u_h\Vert_X\leq\frac{M}{\alpha}\cdot\inf\limits_{v_h\in X_h}\Vert u-u_h\Vert_X
	\end{align*}
	unter der Voraussetzung
	\begin{align*}
		a(v,v)&\geq\beta\cdot\Vert v\Vert^2_X &\forall& v\in X\\
		\big|a(u,v)\big|&\leq M\cdot\Vert u\Vert_X\cdot\Vert v\Vert_Y &\forall& u,v\in X
	\end{align*}
\end{theorem}