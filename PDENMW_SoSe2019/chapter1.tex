% !TEX root = PDENMW.tex
% This work is licensed under the Creative Commons
% Attribution-NonCommercial-ShareAlike 4.0 International License. To view a copy
% of this license, visit http://creativecommons.org/licenses/by-nc-sa/4.0/ or
% send a letter to Creative Commons, PO Box 1866, Mountain View, CA 94042, USA.

\chapter{Geometry of the bisection method}
In the previous example the function $u$ was known. Thus the grid can be constructed specifically for $u$. For finite element problems the function $u $ is not known! There you use an initial grid and modify it repeatedly based on error estimations.

Possible modifications include
\begin{itemize}
  \item moving vertices (r-refinement) (We do not do this here.)
	\item split elements (h-refinement)
\end{itemize}
There are various ideas to split elements, but most of theory exists for bisection refinement.
\section{Bisection refinement}
\begin{figure}[H]
	\center
	\begin{tikzpicture}[scale=1]
	\def \xone{0};
	\def \yone{0};
	\def \h{3};

	% 2d triangle
	\coordinate (A) at (\xone ,\yone);
	\coordinate (B) at ($ (A) + (\h,0) $);
	\coordinate (C) at ($ (A) + (\h/3,2*\h/3) $);
	\coordinate (M) at ($ (A) + (\h/2,0) $);


	\draw (A) -- (B) -- (C) --cycle;
	\draw[red!50] (M) -- (C);
	\filldraw (M) circle (1pt);
		 
		  
	% first 3d triangle
	\coordinate (A1) at (\xone,\yone -\h);
	\coordinate (B1) at ($ (A1) + (\h,0) $);
	\coordinate (C1) at ($ (A1) + (\h/3,2*\h/3) $);
	\coordinate (D1) at ($ (A1) + (\h,1.5*\h/3) $);
	\coordinate (M1) at ($ (A1) + (\h/2,0) $);

		
	\filldraw[red!30] (M1) --(D1) -- (C1) -- cycle;
	\draw (A1) -- (B1) -- (C1) --cycle;
	\draw (B1) -- (C1) -- (D1) --cycle;
	\draw[dotted] (A1) -- (D1);
	\filldraw (M1) circle (1pt);
	
	
	\end{tikzpicture}
	
	\caption{bisection refinement in 2D/3D}
	\label{ch_m_bisection_refine}
	
\end{figure}

There are two challenges:
\begin{itemize}
	\item grid conformity
	\begin{figure}[H]
	\center
	\begin{tikzpicture}[scale=1]
	\def \xone{0};
	\def \yone{0};
	\def \h{3};
	
	% first rectangle
	\coordinate (A) at (\xone,\yone);
	\coordinate (B) at ($ (A) + (\h,0) $);
	\coordinate (C) at ($ (A) + (\h,0.7*\h) $);
	\coordinate (D) at ($ (A) + (0,0.7*\h) $);
	\coordinate (M) at ($ (A) + (\h/2,0.35*\h) $);


	\fill[red!20] (A) --(B) --(D) -- cycle;		
	\draw (A) -- (B) -- (C) --(D) --cycle;
	\draw (B) --(D);
	\filldraw (A) circle (1pt)
			  (B) circle (1pt)
			  (C) circle (1pt)
			  (D) circle (1pt)
			  (M) circle (1pt);
	
	\draw[-to] (11/10*\h,\h/3) -- ++(\h/2,0);
	
	
	% second rectangle
	\coordinate (A1) at (\xone + 1.75*\h,\yone);
	\coordinate (B1) at ($ (A1) + (\h,0) $);
	\coordinate (C1) at ($ (A1) + (\h,0.7*\h) $);
	\coordinate (D1) at ($ (A1) + (0,0.7*\h) $);
	\coordinate (M1) at ($ (A1) + (\h/2,0.35*\h) $);
	
	
	\fill[green!20] (A1) --(B1) --(D1) -- cycle;	
	\fill[red!20] (C1) --(B1) --(D1) -- cycle;	
	\draw (A1) -- (B1) -- (C1) --(D1) --cycle;
	\draw (B1) --(D1);
	\draw (A1) --(M1);
	\filldraw (A1) circle (1pt)
			  (B1) circle (1pt)
			  (C1) circle (1pt)
			  (D1) circle (1pt)
			  (M1) circle (1pt);
	 
	\end{tikzpicture}
	
	\caption{hanging nodes}
	\label{ch_m_hanging_nodes}
	
\end{figure}
\item elements must remain \enquote{well shaped}
	\begin{figure}[H]
	\center
	\begin{tikzpicture}[scale=1]
	\def \xone{0};
	\def \yone{0};
	\def \h{2};

	% 2d triangle
	\coordinate (A) at (\xone ,\yone);
	\coordinate (B) at ($ (A) + (\h,0) $);
	\coordinate (C) at ($ (A) + (\h,\h) $);
	\coordinate (M) at ($ (A) + (\h,\h/2) $);

	\coordinate (M11) at ($ (A) + (\h,3*\h/4) $);
	\coordinate (M12) at ($ (A) + (\h,\h/4) $);

	\coordinate (M21) at ($ (A) + (\h,\h/8) $);
	\coordinate (M22) at ($ (A) + (\h,3*\h/8) $);
	\coordinate (M23) at ($ (A) + (\h,5*\h/8) $);
	\coordinate (M24) at ($ (A) + (\h,7*\h/8) $);


	\draw (A) -- (B) -- (C) --cycle;
	\draw (A) -- (M);

	\draw (A) -- (M11);
	\draw (A) -- (M12);

	\draw (A) -- (M21);
	\draw (A) -- (M22);
	\draw (A) -- (M23);
	\draw (A) -- (M24);

	

	\end{tikzpicture}
	
	\caption{$\delta$ can get very small}
	\label{ch_1_shape_shape_regularity}
	
\end{figure}

\end{itemize}
In particular: the constant $c$ in
\begin{equation*}
  \norm{u - u_{N}}_{H^{1}} \leq c\ h\ \halfnorm{u}_{H^{2}}
\end{equation*}
depends on
\begin{equation*}
  \delta = \max_{T \in \T} \delta_{T}
\end{equation*}
with
\begin{equation*}
	\delta_{T}= \frac{\text{radius of circumcircle of }T}{\text{radius of incircle of }T}.
\end{equation*}
$\delta_{T}$ is called \enquote{shape coefficient} or \enquote{aspect ratio}.

Aim: refinement strategies, that keep $\delta_{T}$ bounded.
\section{Partition of a single simplex}
From the 70s there are many papers considering different strategies to keep $\delta_T$ bounded. They are tough reading. For example
\begin{description}
  \item[Longest edge bisection] This works well in 2D, but not in higher dimensions.
  \item[Oldest edge bisection]
  \item[Type bisection] See the following:
\end{description}
\begin{definition}
	We describe a simplex $T$ as a ordered set of its vertices and a type $t\in \N$.
	\begin{equation*}
    T = \set{z_{0}, \dots , z_{d}}_{t}, \qquad t\in \set{0,\dots ,d-1}
	\end{equation*}
	The refinement rule is solely based on this information.
	\begin{enumerate}[label = \arabic*)]
		\item The refinement edge is $\overline{z_{0}z_{d}}$.
		\item The new midpoint $\bar{z}= \frac{z_{0} + z_{d}}{2}$ becomes a new index.
		\item connect $\bar{z}$ to all vertices of $T$ except $z_{0},z_{d}$ \\
			$\implies$ two new simplices $T_{1},T_{2}$ are created with common face
			 \begin{equation*}
         S = \set{\overline{z},z_{1},\dots ,z_{d-1} }
     \end{equation*}
		\item vertex ordering and type for the children $T_{1},T_{2}$ are defined as:
			\begin{align*}
        T_{1}&= \set*{ z_{0},\overline{z},\underbrace{z_{1},\dots ,z_{t}}_{\rightarrow},\underbrace{z_{t+1},\dots ,z_{d-1}}_{{\rightarrow}}}_{(t+1)\mod d}\\
				T_{2}&= \set*{ z_{d},\overline{z},\underbrace{z_{1},\dots ,z_{t}}_{\rightarrow},\underbrace{z_{d-1},\dots ,z_{t+1}}_{{\leftarrow}}}_{(t+1)\mod d}\\
			\end{align*}
		$\rightarrow$ means increasing, $\leftarrow$ means decreasing order.

    Note that $\set*{\underbrace{z_{d},\dots ,z_{0}}_{\rightarrow}} \coloneqq \emptyset$
	\end{enumerate}
\end{definition}
\begin{example}\
	\begin{itemize}
		\item $d=2$:
			\begin{align*}
      T &= \set{z_{0},z_{1},z_{2}}_{0} & \overline{z}&= \frac{1}{2}\left(z_{0} + z_{2} \right)\\
      T_{1} &= \set{ z_{0},\overline{z},z_{1}}_{1}  & T_{2} &= \set{z_{0},\overline{z},z_{1}}_{1}
			\end{align*}
			$T_1$ and $T_2$ are independent of $t$!
			\begin{figure}[H]
	\center
	\begin{tikzpicture}[scale=1]
	\def \xone{0};
	\def \yone{0};
	\def \h{3};

	% 2d triangle
	\coordinate (A) at (\xone ,\yone);
	\coordinate (B) at ($ (A) + (\h,0) $);
	\coordinate (C) at ($ (A) + (\h/3,2*\h/3) $);
	\coordinate (M) at ($ (A) + (\h/2,0) $);


	\filldraw[red!30] (A) -- (M) -- (C) --cycle;
	\filldraw[blue!30] (B) -- (M) -- (C) --cycle;
	\draw (A) -- (B) -- (C) --cycle;
	\draw (M) -- (C);
	\filldraw (M) circle (1pt);
	
	\node at (A) [below] {2};
	\node at (B) [below] {0};
	\node at (C) [above] {1};

	\node[red!40] at (A) [left] {0};
	\node at (B) [right] {0};
	\node[red!30] at (C) [above left] {2};
	\node[blue!30] at (C) [above right] {2};
	\node[red!30] at (M) [below left] {1};
	\node[blue!30] at (M) [below right] {1};
	\end{tikzpicture}
	
	\caption{2D example for the numbering rule}
	\label{ch_1_example_numbering_alg}
	
\end{figure}

		\item $d= 3$: \quad $T = \set{ z_{0},z_{1},z_{2},z_{3} }_{t}$
      \begin{align*}
        t&=0: & T_{1}&= \set{ z_{0},\overline{z}, z_{1},z_{2}}_{1} & T_{2}&= \set{z_{3},\overline{z}, z_{2},z_{1}}_{1}\\
        t&=1: & T_{1}&= \set{z_{0},\overline{z}, z_{1},z_{2}}_{2} & T_{2}&= \set{z_{3},\overline{z}, z_{1},z_{2} }_{2}\\
        t&=2: & T_{1}&= \set{ z_{0},\overline{z}, z_{1},z_{2} }_{0} & T_{2}&= \set{ z_{3},\overline{z}, z_{1},z_{2} }_{0}
			\end{align*}
	\end{itemize}
\end{example}
If you renumber the vertices of $T$ the shape of $T$ does not get modified, but the shapes of the children get modified. You can get up to $\frac{1}{2}\left(d+1 \right)!$ different shapes for the children.\\
Repeated refinement leads to an infinite binary tree $\mathbb{F}(T_{0})$.
\begin{definition}
	The \textit{generation} $g(T)$ of $T \in \mathbb{F}(T_{0})$ is its number of ancestors.
\end{definition}
\underline{Question:} Is the set of shape coefficients bounded?\nl
Let us first look at a particularly regular case:
\begin{definition}\
	\begin{enumerate}[label = \Roman*)]
		\item The Kuhn simplex has the corners
			\begin{align*}
				k_{0}&=0 \in \R^d\\
				k_{1}&=e_{1}, k_{2}=e_1 + e_{2}\\
				k_{i}&= \sum_{j=1}^{i} e_{j}
			\end{align*}
			\begin{figure}[H]
	\center
	\begin{tikzpicture}[scale=1]
	\def \xone{0};
	\def \yone{0};
	\def \h{3};

	% 2D triangle
	\coordinate (A) at (\xone ,\yone);
	\coordinate (B) at ($ (A) + (\h,0) $);
	\coordinate (C) at ($ (A) + (\h,\h) $);

	\draw (A) -- (B) -- (C) --cycle;
	
	\node at (A) [below] {$(0,0)$};
	\node at (B) [below] {$(1,0)$};
	\node at (C) [above] {$(1,1)$};

	% 3D triangle
	\coordinate (A) at (\xone + 1.8*\h ,\yone - 0.4*\h);
	\coordinate (B) at ($ (A) + (\h,0) $);
	\coordinate (C) at ($ (A) + (1.7*\h ,0.7*\h) $);
	\coordinate (D) at ($ (A) + (1.7*\h ,1.7*\h) $);

	\draw (A) -- (B) -- (C) -- (D) --cycle;
	\draw (B) -- (D);
	\draw[dotted] (A) -- (C);
	
	\node at (A) [below] {$(0,0,0)$};
	\node at (B) [below] {$(1,0,0)$};
	\node at (C) [right] {$(1,1,0)$};
	\node at (D) [above] {$(1,1,1)$};


	\end{tikzpicture}
	
	\caption{2D/3D Kuhn-simplex}
	\label{ch_1_kuhn_simplex}
	
\end{figure}

		\item Let $\pi$ be a permutation of $\set{1,\dots ,d \right}$. The Kuhn-simplex $T_{\pi}$ has vertices
			\begin{align*}
				k_{0}&=0 \\
				k_{i}&= \sum_{j=1}^{d} e_{\pi(j)}
			\end{align*}		
      Fun fact: all Kuhn simplices together fill the unit cube.
	\end{enumerate}
\end{definition}
\begin{theorem}\
	\begin{itemize}
		\item The $2^g$ descendants in the $g$-th generation of a Kuhn-simplex are all pairwise congruent.
		\item There are only $d$ different simplex shapes in the entire infinite tree $\mathbb{F}(T_{\pi})$
		\item All descendants of generation $d$ are similar to $T_{\pi}$ with scaling $\frac{1}{2}$.
			\begin{figure}[H]
	\center
	\begin{tikzpicture}[scale=1]
	\def \xone{0};
	\def \yone{0};
	\def \h{3};

	% 2D triangle
	\coordinate (A) at (\xone ,\yone);
	\coordinate (B) at ($ (A) + (\h,0) $);
	\coordinate (C) at ($ (A) + (\h,\h) $);
	\coordinate (M1) at ($ (A) + (0.5*\h,0.5*\h) $);
	\coordinate (M2) at ($ (A) + (0.5*\h,0) $);
	\coordinate (M3) at ($ (A) + (\h,0.5*\h) $);

	\draw (A) -- (B) -- (C) --cycle;
	\draw[red!40,thick] (M1) -- (B);
	\draw[green!40,thick] (M1) -- (M2);
	\draw[green!40,thick] (M1) -- (M3);
	
	\node at (A) [below] {$(0,0)$};
	\node at (B) [below] {$(1,0)$};
	\node at (C) [above] {$(1,1)$};

	% 3D triangle
	\coordinate (A) at (\xone + 1.8*\h ,\yone - 0.4*\h);
	\coordinate (B) at ($ (A) + (\h,0) $);
	\coordinate (C) at ($ (A) + (1.7*\h ,0.7*\h) $);
	\coordinate (D) at ($ (A) + (1.7*\h ,1.7*\h) $);
	\coordinate (M) at ($ (A) + (0.85*\h ,0.85*\h) $);

	\filldraw[green!30] (B) -- (C) --(M);
	\draw (A) -- (B) -- (C) -- (D) --cycle;
	\draw (B) -- (D);
	\draw[dotted] (A) -- (C);
	
	\node at (A) [below] {$(0,0,0)$};
	\node at (B) [below] {$(1,0,0)$};
	\node at (C) [right] {$(1,1,0)$};
	\node at (D) [above] {$(1,1,1)$};


	\end{tikzpicture}
	
	\caption{2D/3D refinement of a Kuhn-simplex }
	\label{ch_1_kuhn_generation_scaling}
	
\end{figure}

	\end{itemize}
\end{theorem}
\begin{proof}[Proof sketch]
	% (proof sketch)\\
	We show: all descendants in the $g$-th generation of a Kuhn-simplex of type $0$ ($0 \leq g < d$) are congruent to
	\begin{equation*}
    T_{g}= \set{0, \frac{k_{d-g+1}}{2},\dots ,\frac{k_{d}}{2},k_{1},\dots ,k_{d-g} }_{g}
	\end{equation*}
	\underline{Induction over $g$}:
	\begin{align*}
		g=0&: & T_{0}= \set{0,k_{1},\dots ,k_{d} }_{0} \text{is the Kuhn simplex}\\
		g\mapsto g+1&: &\text{bisect } T_{g}\quad ( \text{bisection edge } \overline{0k_{d-g}})\\
		 && C_{0}= \set{ 0, \frac{k_{d-g}}{2},\frac{k_{d-g+1}}{2},\dots , \frac{k_{d}}{2},k_{1},\dots ,k_{d-g-1}}_{g+1}= T_{g+1}\\
     && C_{1}= \set{ k_{d-g}, \frac{k_{d-g}}{2},\frac{k_{d-g+1}}{2},\dots , \frac{k_{d}}{2},k_{d-g-1},\dots ,k_{1} }_{g+1} % not obviously equal to T_{g+1}
	\end{align*}
	Construct a series of reflection that maps $C_{1}$ onto $T_{g}$ :
	\begin{enumerate}
		\item Reflect at the plane perpendicular to $k_{d-g}$ in the point $\frac{k_{d-g}}{2}$
			\begin{itemize}
				\item $k_{d-g} \mapsto 0$
				\item $k_{l} \mapsto k_{d-g}-k_{l}$ ($l < d-g$)
				\item $k_{l} \mapsto k_{l}$ ($l \geq d-g$)
			\end{itemize}
    \item Swap $e_{1}\leftrightarrow e_{d-g-1}$, $e_{2} \leftrightarrow e_{d-g-2},\dots $ (Reflection at diagonals)
			\begin{itemize}
				\item $0 \mapsto 0$
				\item $k_{d-g}-k_{l} \mapsto k_{d-g-l}$ $(l< d-g)$
				\item $\frac{k_{l}}{2}\mapsto \frac{k_{l}}{2}$ $(l \geq d-g)$
			\end{itemize}
	\end{enumerate}
	Result:
	\begin{equation*}
		C'_{1}= \set{ 0, \frac{k_{d-g}}{2},\dots ,\frac{k_{d}}{2},k_{1},\dots ,k_{d-g-1} }_{g+1} = T_{g+1}
	\end{equation*}
	$C_{0}$ and $C_{1}$ are congruent.
	\begin{equation*}
		T_{d}= \set{ 0, \frac{k_{1}}{2},\dots ,\frac{k_{d}}{2} }_{0} \quad \implies \text{ scaled Khun simplex} \qedhere
	\end{equation*}
\end{proof}
\begin{theorem}
	Let $T$ be an arbitrary $d$-simplex. Then there is a constant $C(T_{0})$ such that
	\begin{equation*}
		\delta_{T} \leq C(T_{0})
	\end{equation*}
	for all $T \in \mathbb{F}(T_{0})$.
\end{theorem}
\begin{proof_}\
	\begin{itemize}
		\item Let $T_{0}$ be of type 0.
		\item Let $\hat{T}_{0}$ be the Kuhn simplex of type 0.
		\item There is an affine map $F$ with $F(\hat{T}_{0})=T_{0}$ \\
			$\implies$ For each $T \in \mathbb{F}(T_{0})$ there is a unique $\hat{T} \in \mathbb{F}(\hat{T}_{0})$ such that
			\begin{equation*}
				T = F(\hat{T})
			\end{equation*}
			%TODO ask if the refinement is correct, because of the step from d to b and c
			\begin{figure}[H]
	\center
	\begin{tikzpicture}[scale=1]
	\def \xone{0};
	\def \yone{0};
	\def \h{5};
	
	% blank triangle a
	\coordinate (A) at (\xone ,\yone);
	\coordinate (B) at ($ (A) + (\h,0) $);
	\coordinate (C) at ($ (A) + (\h/3,\h/3) $);
	\coordinate (S) at ($ (A) + (2*\h/5,\h/8) $);
	\draw (A) -- (B) -- (C) --cycle;
	\node at (S) {$a$};	

	% after first bisection
	\coordinate (A) at ($(A) + (0, -\h/2 )$);
	\coordinate (B) at ($ (A) + (\h,0) $);
	\coordinate (C) at ($ (A) + (\h/3,\h/3) $);
	\coordinate (M) at ($ (A) + (\h/2,0) $);
	\coordinate (S11) at ($ (A) + (\h/4,\h/10) $);
	\coordinate (S12) at ($ (A) + (3*\h/5,\h/10) $);

	\draw (A) -- (B) -- (C) --cycle;
	\draw (M) -- (C);
	\node at (S11) {$b$};	
	\node at (S12) {$c$};	

	% after second bisection
	\coordinate (A) at ($(A) + (0, -\h/2 )$);
	\coordinate (B) at ($ (A) + (\h,0) $);
	\coordinate (C) at ($ (A) + (\h/3,\h/3) $);
	\coordinate (M) at ($ (A) + (\h/2,0) $);
	\coordinate (M1) at ($ (A) + (\h/6,\h/6) $);
	\coordinate (M2) at ($ (A) + (2*\h/3,\h/6) $);
	\coordinate (S21) at ($ (A) + (\h/5,\h/12) $);
	\coordinate (S22) at ($ (A) + (3*\h/10,\h/6) $);
	\coordinate (S23) at ($ (A) + (5*\h/10,\h/6) $);
	\coordinate (S24) at ($ (A) + (7*\h/10,\h/12) $);


	\draw (A) -- (B) -- (C) --cycle;
	\draw (M) -- (C);
	\draw (M) -- (M1);
	\draw (M) -- (M2);

	\node at (S21) {$a$};	
	\node at (S22) {$d$};	
	\node at (S23) {$d$};	
	\node at (S24) {$a$};	

	% something is probably off here 
	% after third bisection
	\coordinate (A) at ($(A) + (0, -\h/2 )$);
	\coordinate (B) at ($ (A) + (\h,0) $);
	\coordinate (C) at ($ (A) + (\h/3,\h/3) $);
	\coordinate (M) at ($ (A) + (\h/2,0) $);
	\coordinate (M1) at ($ (A) + (\h/6,\h/6) $);
	\coordinate (M2) at ($ (A) + (2*\h/3,\h/6) $);
	\coordinate (M3) at ($ (A) + (\h/4,0) $);
	\coordinate (M4) at ($ (A) + (5*\h/12,\h/6) $);
	\coordinate (M5) at ($ (A) + (3*\h/4,0) $);
	%\coordinate (S21) at ($ (A) + (\h/5,\h/12) $);
	%\coordinate (S22) at ($ (A) + (3*\h/10,\h/6) $);
	%\coordinate (S23) at ($ (A) + (5*\h/10,\h/6) $);
	%\coordinate (S24) at ($ (A) + (7*\h/10,\h/12) $);

	% draw lines 
	\draw (A) -- (B) -- (C) --cycle;
	\draw (M) -- (C);
	\draw (M) -- (M1);
	\draw (M) -- (M2);
	\draw (M1) -- (M3);
	\draw (M1) -- (M4);
	\draw (M2) -- (M4);
	\draw (M2) -- (M5);

	%\node at (S21) {$a$};	
	%\node at (S22) {$d$};	
	%\node at (S23) {$d$};	
	%\node at (S24) {$a$};	

\end{tikzpicture}
	
	\caption{example for iterated refinement}
	\label{ch_1_example_refined_triangle}
	
\end{figure}

	\end{itemize}
	This does not mean that the recurrent refinement will produce no more than $d$ similarity classes, but the total number remains finite.
	 \begin{equation*}
		 \implies \delta_{T} \text{ remains bounded}
	\end{equation*}
	If $T_{0}$ has type $>0$ : Refine until reaching descendants of type $0$.
\end{proof_}



