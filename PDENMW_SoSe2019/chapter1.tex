% !TEX root = PDENMW.tex
% This work is licensed under the Creative Commons
% Attribution-NonCommercial-ShareAlike 4.0 International License. To view a copy
% of this license, visit http://creativecommons.org/licenses/by-nc-sa/4.0/ or
% send a letter to Creative Commons, PO Box 1866, Mountain View, CA 94042, USA.

\chapter{Geometry of the bisection method}
In the previous example the function $u$ was known. Thus the grid can be constructed specifically for $u$. For finite element problems the function $u $ is not known! There you use an initial grid and modify it repeatedly.\\
possible modifications:
\begin{itemize}
	\item moving vertices (r-refinement)
	\item split elements (h-refinement)
\end{itemize} 
There are various ideas to split elements, but most of theory exists for bisection refinement.
\section{Bisection refinement}
%TODO add pictures here

There are two challenges:
\begin{itemize}
	\item grid conformity
	\item elements must remain \glqq well shaped\grqq 
\end{itemize} 
In particular: the constant $c$ in 
\begin{equation*}
	\|u - u_{N}\|_{H^{1}} \leq c\ h\ |u|_{H^{2}}
\end{equation*} 
depends on 
\begin{equation*}
	\delta = \max \delta_{T} 
\end{equation*} 
with
\begin{equation*}
	\delta_{T}= \frac{\text{radius of circum-circle of }T}{\text{radius of incircle of }T}.
\end{equation*} 
$\delta_{T}$ is called \glqq shape coefficient\grqq\ or \glqq aspect ratio\grqq.\\
\par 
Aim: refinement strategies, that keep $\delta_{T}$ bounded.

\section{Partition of a single simplex}
\begin{definition}
	We describe a simplex $T$ as a ordered set of its vertices and a type $t\in \N$.
	\begin{equation*}
		T = \left\{z_{0}, \dots , z_{d}\right\}_{t}, \qquad t\in \left\{ 0,\dots ,d-1 \right\}
		
	\end{equation*}
	The refinement rule is solely based on this information.
	\begin{enumerate}[label = \arabic*)]
		\item The refinement edge is $\overline{z_{0}z_{d}}$.
		\item The new midpoint $\bar{z}= \frac{\overline{z_{0}z_{d}}}{2}$ becomes a new index.
		\item connect $\bar{z}$ to all vertices of $T$ except $z_{0},z_{d}$ \\
			$\implies$ two new simplices $T_{1},T_{2}$ with common face 
			 \begin{equation*}
				S = \left\{\overline{z},z_{1},\dots ,z_{d-1} \right\}
			\end{equation*}
		\item vertex oredring and type for the children $T_{1},T_{2}$:
			\begin{align*}
				T_{1}&= \left\{ z_{0},\overline{z},\underbrace{z_{1},\dots ,z_{t}}_{\rightarrow},\underbrace{z_{t+1},\dots ,z_{d-1}}_{{\rightarrow}} \right\}_{(t+1)\mod d}\\
				T_{2}&= \left\{ z_{d},\overline{z},\underbrace{z_{1},\dots ,z_{t}}_{\rightarrow},\underbrace{z_{d-1},\dots ,z_{t+1}}_{{\leftarrow}} \right\}_{(t+1)\mod d}\\
			\end{align*}
		$\rightarrow, \leftarrow$ means increasing or decreasing order.	\\
		$\underbrace{\left\{z_{d},\dots ,z_{0}\right\} }_{\rightarrow} =\emptyset$		
	\end{enumerate}
\end{definition}
\begin{example}
	%TODO formatieren, sodass der neue Anstrich vorne beginnt
	\begin{itemize}
		\item $d=2$:\\
			\begin{align*}
				T &= \left\{ z_{0},z_{1},z_{2} \right\}_{0} & \overline{z}&= \frac{1}{2}\left(z_{0} + z_{2} \right)\\
				T_{1} &= \left\{ z_{0},\overline{z},z_{1} \right\}_{1}  & T_{2} &= \left\{ z_{0},\overline{z},z_{1} \right\}_{1}
			\end{align*}
			independent of $t$
			%TODO add picture	
		\item $d= 3$: \quad $T = \left\{ z_{0},z_{1},z_{2},z_{3} \right\}_{t}$
			\begin{align*}
				t&=0: & T_{1}&= \left\{ z_{0},\overline{z}, z_{1},z_{2} \right\}_{1} & T_{2}&= \left\{ z_{3},\overline{z}, z_{2},z_{1} \right\}_{1}\\
				t&=1: & T_{1}&= \left\{ z_{0},\overline{z}, z_{1},z_{2} \right\}_{2} & T_{2}&= \left\{ z_{3},\overline{z}, z_{1},z_{2} \right\}_{2}\\
				t&=2: & T_{1}&= \left\{ z_{0},\overline{z}, z_{1},z_{2} \right\}_{0} & T_{2}&= \left\{ z_{3},\overline{z}, z_{1},z_{2} \right\}_{0}
			\end{align*}
	\end{itemize}
\end{example}
If you renumber the vertices of $T$ the shape of $T$ does not get modified, but the shapes of the children get modified. You can get up to $\frac{1}{2}\left(d+1 \right)!$ different shapes for the children.
