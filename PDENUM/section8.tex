% This work is licensed under the Creative Commons
% Attribution-NonCommercial-ShareAlike 4.0 International License. To view a copy
% of this license, visit http://creativecommons.org/licenses/by-nc-sa/4.0/ or
% send a letter to Creative Commons, PO Box 1866, Mountain View, CA 94042, USA.
% vim: set noexpandtab:


\section{Gemischte Methoden} %8
\subsection{Startpunkt}
Idee: Minimierung auf einem Unterraum\\
Gegeben: Funktional $\Phi\colon V\to\R,u\mapsto\frac{1}{2}a(u,u)-\underbrace{(f,u)}_{=f(u)}$\\
Wir wollen dieses Funktional minimieren ($\Phi\to\min$) auf $W\subseteq V$ wobei
\begin{align*}
	W:=\Big\lbrace v\in V:b(q,v)=0~\forall q\in Q\Big\rbrace
\end{align*}
$a\colon V\times V\to\R$: ist eine koerzive, stetige Bilinearform\\
$f(\cdot)=(f,\cdot)$: ist stetige Lineaform $f(\cdot)\colon V\to\R$ \\
$b\colon Q\times V\to\R$: stetige Bilinearform\\
$Q$: passender Funktionenraum\nl
$b$ hat verschwindendes Integral-Mittel:
\begin{align*}
	b(q,v)=\int\limits_\Omega q\cdot v\d x,\qquad
	Q=\spann\lbrace\indi\rbrace\hat{=}\R
\end{align*}
Folglich ist $W$ der Raum aller Funktionen aus $V$, bei welchem das Integral-Mittel verschwindet.\nl
Bezeichne $u^\ast$ eine Lösung des Minimierungsproblems mit Bedingung
\begin{align*}
	\frac{\d}{\d t}\Phi\big(u^\ast+t\cdot w\big)\Big|_{t=0}=0\qquad\forall w\in W
\end{align*}
Also:
\begin{align*}
	\Phi\big(u^\ast+t\cdot w\big)
	&=\frac{1}{2}a\big(u^\ast+t\cdot w,u^\ast+t\cdot w\big)-f\big(u^\ast+t\cdot w\big)\\
	0&=a\big(u^\ast,w\big)-f(w)\qquad\forall w\in W
\end{align*}
Das ist ein Variationsproblem in $W$.\\
Problem: Es ist oftmals schwer $W$ durch konforme Finite-Elemente-Räume zu approximieren.

\begin{beispiel}\label{beispiel8.1} %nonumber in Lecture, but i'll give one
	\begin{align*}
		V:&=\big(H_0^1(\Omega)\big)^d,d\in\lbrace1,2\rbrace,\qquad Q=L^2(\Omega)\\
		a(v,w)&=\int\limits_\Omega\nabla v:\nabla w\d x\\
		b(q,v)&=\int\limits_\Omega q\div(v)\d x\\
		f(v)&=\int\limits_\Omega f\cdot v\d x\mit\\
		\Big( A:B:&=\sum\limits_{i,j=1}^d A_{i,j}\cdot B_{i,j}\qquad\forall A,B\in\R^{d\times d} \Big)\\
		W&=\big\lbrace v\in V:b(q,v)=0~\forall q\in Q\big\rbrace\\
		&=\left\lbrace v\in\big(H_0^1(\Omega)\big)^d:\int\limits_\Omega q\div(v)\d x=0~\forall q\in L^2(\Omega)\right\rbrace\\
		&=\left\lbrace v\in\big(H_0^1(\Omega)\big)^d:\div(v)=0\right\rbrace
	\end{align*}
	$W$ ist also der Raum der Divergenzfreien Funktionen
\end{beispiel}

Kurzer Exkurs zu Minimierung mit Nebenbedingungen:
\begin{align*}
	f(x,y)&=x\cdot y\to\min\quad\text{unter}\quad x^2+y^2=1\\
	F(x,y,\lambda):&=f(x,y)+\lambda\cdot\big(x^2+y^2-1\big)
\end{align*}
$F$ ist die sogenannte Lagrange-Funktion und $\lambda$ der Lagrange-Multiplikator.\\
Optimalitätsbedingungen:
\begin{align*}
	\frac{\partial F}{\partial x}=0\qquad
	\frac{\partial F}{\partial y}=0\qquad
	\frac{\partial F}{\partial\lambda}=0
\end{align*}
Nun versuchen wir dies auf unser Problem anzuwenden:
Wir betrachten also die Funktion (mit Lagrange-Mulitplikator $p$):
%Ich weiß, dass es deine Regel bricht, aber an dieser Stelle sieht es besser auf, wenn man die align Umgebungen direkt hintereinander macht.
\begin{align*}
	F(u,p):&=\Phi(u)-b(p,u)\\
	&=\frac{1}{2}a(u,u)-f(u)-b(p,u)
\end{align*}
\begin{align*}
	\frac{\partial}{\partial t}F\big(u+tv,p+sq\big)\Big|_{s=t=0}
	&=0
	\qquad\forall v\in V,\forall q\in Q\\
	\frac{\partial}{\partial s}F\big(u+tv,p+sq\big)\Big|_{s=t=0}
	&=0
	\qquad\forall v\in V,\forall q\in Q
\end{align*}
\begin{align*}
	F\big(u+tv,p+sq\big)
	&=\frac{1}{2}a\big(u+tv,u+tv\big)-f\big(u+tv\big)-b\big(p+sq,u+tv\big)
\end{align*}
\begin{align*}
	&\implies \left\lbrace\begin{array}{rl}
		a(u,v)-f(v)-b(p,v)=0 &\forall v\in V\\
		-b(q,u) =0 &\forall q\in Q
	\end{array}\right.
\end{align*}
 \begin{align*}
	a(u,w)&=f(w) \qquad\forall w\in W
\end{align*}

\begin{beispiel}[Weiter Beispiel \ref{beispiel8.1}]
	\begin{align*}
		\int\limits_\Omega\nabla u:\nabla v\d x-\int\limits_\Omega p\div(v)\d x
		&=\int\limits_\Omega f\cdot v\d x
		&&\forall v\in V\\
		\int\limits_\Omega q\div(u)\d x
		&=0
		&&\forall q\in Q=L^2(\Omega)\\
		\overset{q=\div(u)\in L^2(\Omega)}{\implies}
		\div(u)&=0
		&& \text{ in } \Omega
	\end{align*}
	Für $u\in\big(H^2(\Omega)\big)^d$ und $p\in H^1(\Omega)$ gilt:
	\begin{align*}
		\int\limits_\Omega\nabla u:\nabla v\d x
		\overset{\text{part Int}}&=
		-\int\limits_\Omega \Delta u\cdot v\d x\\
		\int\limits_\Omega p\div(v)\d x
		\overset{\text{part Int}}&=
		-\int\limits_\Omega v\cdot\nabla p\d x\\
		\implies
		\int\limits_\Omega\big(-\Delta u+\nabla p\big)\cdot v\d x
		&=\int\limits_\Omega f\cdot v\d x\qquad\forall v\in V\\
	\end{align*}
	\begin{align*}
		\stackrelnew{\text{der Variationsanalysis}}{\text{Fundamentallemma}}{\implies}
		\left\lbrace\begin{array}{rl}
			-\Delta u+\nabla p &=f\text{ in }\Omega\\
			\div(u)&=0\text{ in }\Omega
		\end{array}\right.
	\end{align*}
	Das entstandene Problem heißt \textbf{Stokes-Problem}.
\end{beispiel}

\subsection{Lösbarkeit von gemischten Problemen}
Seien $V,Q$ Hilberträume, $a\colon V\times V\to\R$ bilinear, stetig und koerziv, $b\colon Q\times V\to\R$ bilinear und stetig.\nl
Finde $(u,p)\in V\times Q$ so, dass
\begin{align*}
	a(u,v)-b(p,v)&=f(v) &\forall v\in V\\
	b(q,u) &=0 &\forall q\in Q
\end{align*}
Setze
\begin{align*}
	A\big((u,p),(v,q)\big) &:=a(u,v)-b(p,v)+b(q,v)\\
	B\big((v,q)\big) &:= f(v)
\end{align*}
Damit erhalten wir das äquivalente Problem:
Finde $(u,p)\in V\times Q$ so, dass
\begin{align*}
	A\big((u,p),(v,q)\big)=F\big(v,q)\big) \qquad\forall(v,q)\in V\times Q
\end{align*}
Setze $X:=V\times Q$ mit der kanonischen Norm
\begin{align*}
	\big\Vert(v,q)\big\Vert_X:=\sqrt{\Vert v\Vert_V^2+\Vert q\Vert_Q^2}
\end{align*}
$A$ ist stetig und bilinear auf $X$ und $F$ ist stetig auf $X$.\\
Ist $A$ koerziv auf $X$?
\begin{align*}
	A\big((v,q),(v,q)\big)
	&=a(v,v)-b(q,v)+b(q,v)
	=a(v,v)\geq\alpha\cdot\Vert v\Vert^2_V\\
	\implies
	A\big((0,q),(0,q)\big)&=0\qquad\forall q\in Q
\end{align*}
Somit ist $A$ \underline{nicht} koerziv.
Folglich kann Lax-Milgram nicht angewandt werden.

\begin{align}\label{eq8.2Problem1MixedForm}\tag{1}
		\left\lbrace
		\begin{array}{rlrr}
			\text{Finde }(u,p)\in V\times Q\text{ so, dass }&\\
			a(u,v)-b(p,v)&=f(v) &\forall v\in V\\
			b(q,u)&=0 &\forall q\in Q
		\end{array}\right.\\
		\label{eq8.2Problem2MixedForm}\tag{2}
		\text{Finde $u\in W$ so, dass }
		a(u,v)= f(v)\qquad\forall v\in W
\end{align}

Wenn $\big(u^\ast,p^\ast\big)$ eine Lösung von \eqref{eq8.2Problem1MixedForm} ist, dann löst $u^\ast$ \eqref{eq8.2Problem2MixedForm}.\\
Aus $\big(u^\ast,p^\ast\big)$ löst\eqref{eq8.2Problem1MixedForm} folgt:
\begin{align*}
	b\big(q,u^\ast\big)=0\qquad\forall q\in Q
	\implies u^\ast\in W
	\implies u^\ast\text{ löst }\eqref{eq8.2Problem2MixedForm}
\end{align*}

Sei $v\in W\subseteq V$ beliebig.
\begin{align*}
	a\big(u^\ast,v\big)-\underbrace{b(p^\ast,v)}_{\overset{v\in W}{=}0}=f(v)\\
	a(u^\ast,v)=f(v)\qquad\forall v\in W
	\implies u^\ast\text{ löst }\eqref{eq8.2Problem2MixedForm}
\end{align*}

\underline{Problem (1):}
Finde $(u,p)\in V\times Q:$
\begin{align*}
	a(u,v)-b(p,v)&=f(v) &\forall&v\in V\\
	b(q,u)&=0 &\forall& q\in Q
\end{align*}

\underline{Problem (2):}
Finde $u\in W$
\begin{align*}
	a(u,v)&=f(v) \qquad\forall v\in W
\end{align*}
Mit
\begin{align*}
	W&=\big\lbrace v\in V:b(q,v)=0~\forall q\in Q\big\rbrace
\end{align*}

Angenommen $u\in W$ löst Problem (2).
Also erfüllt $u$ die zweite Gleichung in Problem (1).
Um eine Funktion $p\in Q$ in (1) zu finden, wir brauchen
\begin{align*}
	b(p,v)=a(u,v)-f(v)\qquad\forall v\in V
\end{align*}
Für $w\in W:$
\begin{align*}
	b(p,w)=a(u,w)-f(w)\overset{(2)}{=}0
\end{align*}
mit $V=W+W^\perp$ und $W^\perp:=\big\lbrace v\in V:(v,w)=0~\forall w\in W\big\rbrace$
\begin{align*}
v\in V\implies\exists! w\in W,w^\perp\in W^\perp:v=w+w^\perp
\end{align*}
Es reichte auch zu zeigen:
\begin{align}\label{eqProblem3}\tag{3}
	b(p,v)=a(u,v)-f(v)\qquad\forall v\in W^\perp
\end{align}
Sei $B:V\to Q^\ast$ eine Funktion wobei $Q^\ast$ die Menge aller linearen stetigen Funktionale auf $Q$ ist.
\begin{align*}
	\langle Bv,q\rangle&:=b(q,v)\qquad\forall q\in Q,\forall v\in V
\end{align*}
$B^\ast:Q\to V^\ast$:
\begin{align*}
	\langle B^\ast q,v\rangle&:=b(q,v)\qquad\forall q\in Q,\forall v\in V
\end{align*}
$B$ und $B^\ast$ sind offenbar stetig und linear.
Außerdem gilt:
\begin{align*}
	\eqref{eqProblem3}&\Longleftrightarrow\big\langle B^\ast p,v\big\rangle
	=\underbrace{a(u,v)-f(v)}_{=:\langle g,v\rangle,g\in V^\ast}\qquad\forall v\in V\\
	&\Longleftrightarrow B^\ast p=g\text{ in }V^\ast
\end{align*}

$B^\ast$ ist injektiv ist äquivalent zu
\begin{align*}
	\overset{B^\ast\text{ linear}}&{\Longleftrightarrow}
	\exists\beta>0:\big\Vert B^\ast p\big\Vert_{V^\ast}\geq\beta\cdot\Vert p\Vert_Q &\forall& p\in Q\\
	\overset{\text{Def }\Vert\cdot\Vert}&\Longleftrightarrow \exists\beta>0:
	\sup\limits_{v\in V\setminus\lbrace0\rbrace}\frac{b(p,v)}{\Vert v\Vert_V}\geq\beta\Vert p\Vert_Q &\forall&p\in Q\\
	&\Longleftrightarrow\exists\beta>0:
	\inf\limits_{q\in Q\setminus\lbrace0\rbrace}\sup\limits_{v\in V\setminus\lbrace0\rbrace}\frac{b(q,v)}{\Vert v\Vert_V\cdot\Vert q\Vert_Q}\geq\beta
\end{align*}
Die letzte Ungleichung heißt \textbf{$\inf$-$\sup$-Bedingung / LBB-Bedingung}
\setcounter{satz}{-1}
\begin{satz}\label{theorem8.0infsupAequivalenz}%nonumber
	Für eine stetige Bilinearform $b$ auf $Q\times V$ mit zugehörigen Operatoren
	$B\colon V\to Q^\ast$ und $B^\ast\colon Q\to V^\ast$ sind die folgenden Bedingungen äquivalent:
	\begin{enumerate}[label=(\roman*)]
		\item $\begin{aligned}
			\exists\beta>0:\inf\limits_{q\in Q\setminus\lbrace0\rbrace}\sup\limits_{v\in V\setminus\lbrace 0\rbrace}\frac{b(q,v)}{\Vert v\Vert_V\cdot\Vert q\Vert_Q}\geq\beta
		\end{aligned}$
		\item $\begin{aligned}
			B\colon W^\perp\to Q^\ast
		\end{aligned}$ ist ein Isomorphismus mit
		\begin{align*}
			\Vert B v\Vert_{Q^\ast}\geq\beta\cdot\Vert v\Vert_V\qquad\forall v\in W^\perp
		\end{align*}
		\item $\begin{aligned}
			B^\ast\colon Q\to\big\lbrace g\in V^\ast:\langle g,v\rangle=0~\forall v\in W\big\rbrace
		\end{aligned}$ ist ein Isomorphismus mit
		\begin{align*}
			\big\Vert B^\ast q\big\Vert_{V^\ast}\geq\beta\cdot\Vert q\Vert_Q\qquad\forall q\in Q
		\end{align*}
	\end{enumerate}
\end{satz}

\setcounter{satz}{0} %Die Nummerierung war in der VL so.
\begin{theorem}\label{theorem8.1}
	Seien $Q,V$ Hilberträume.
	Sei $a\colon V\times V\to\R$ stetig, $V$-elliptisch (= koerziv), bilinear.
	Sei $b\colon V\to\R$ stetig, bilinear mit
	\begin{align*}
		\exists\beta>0:\inf\limits_{q\in Q}\sup\limits_{v\in V}\frac{b(q)}{\Vert v\Vert_V\cdot\Vert q\Vert_Q}\geq\beta
	\end{align*}
	Dann gibt es für jedes $f\in V^\ast$ eine eindeutige Lösung des Problems
	\begin{align*}
		\text{Finde }u\in W:a(u,v)=f(v)\qquad\forall v\in W
	\end{align*}
	Zusätzlich existiert ein eindeutiges $p\in Q$ mit
	\begin{align*}
		b(p,v)=a(u,v)-f(v)\qquad\forall v\in W^\perp
	\end{align*}
	Das Paar $(u,p)\in V\times Q$ löst:\\
	\tab Finde $(u,p)\in V\times Q$ so, dass
	\begin{align*}
		a(v,v)-b(p,v)&=f(v) &\forall& v\in V\\
		b(q,u)&=0 &\forall& q\in Q
	\end{align*}
\end{theorem}

\begin{bemerkung}
	Man kann die Voraussetzungen abschwächen.
	Es reicht aus, dass $a$ elliptisch (koerziv) auf $W$ ist.
\end{bemerkung}

Betrachte erneut unser Beispiel, das Stokes Problem:
\begin{align*}
	\left\lbrace\begin{array}{rl}
		-\Delta u+\nabla p=f&\text{ in }\Omega\\
		\div(u)=0&\text{ in }\Omega\\
		u=0&\text{ auf }\partial\Omega
	\end{array}\right.
\end{align*}

Wähle $V=\big(H_0^1(\Omega)\big)^d$ und $Q=L_0^2(\Omega):=\left\lbrace q\in L^2(\Omega):\int\limits_\Omega q(x)\d x=0\right\rbrace$.
Unsere Setting ist
\begin{align*}
	a(v,w&)=\int\limits_\Omega\nabla v:\nabla w\d x,\qquad
	b(q,v)=\int\limits_\Omega q\div(v)\d x\\
	\Vert v\Vert_V&=\Vert v\Vert_{1,2,\Omega}=\left(\sum\limits_{i=1}^d\Vert v_i\Vert^2_{1,2,\Omega}\right)^{\frac{1}{2}}\\
	\Vert q\Vert_Q&=\inf\limits_{c\in\R}\Vert q+c\Vert_{0,2,\Omega}=\Vert q\Vert_{0,2,\Omega}\text{ für }q\in L^2_0(\Omega)
\end{align*}

Die $\inf$-$\sup$-Bedingung ist erfüllt für $(V,Q)$.

\subsection{Konforme Diskretisierungen} %8.3
Wähle Finite-Elemente-Räume $V_h\subseteq V,Q_h\subseteq Q$.
Diskretes Problem:\\
Finde $(u_h,p_h)\in V_h\times Q_h$ so, dass
\begin{align*}
	a(u_h,v_h)-b(p_h,v_h)&=f(v_h) &\forall& v_h\in V_h\\
	b(q_n,u_n)&=0 &\forall& q_h\in Q_h
\end{align*}

Seien $\lbrace\varphi_i\rbrace$, $\lbrace\psi_i\rbrace$ Basen von $V_h$ bzw. $Q_h$.
\begin{align*}
	A&:=(a_{i,j}), &a_{i,j}&:=a(\varphi_j,\varphi_i)\\
	B&:=(b_{i,j}), &b_{i,j}&:=b(\psi_i,\varphi_j)\\
	F&:=(f_i), &f_i&:=f(\varphi_i)\\
	AU-B^T p&~=F &U&:=(u_i),~P(p_i)\\
	BU&~=0 &u_h&:=\sum\limits_i u_i\varphi_i,~p_h:=\sum\limits_i p_i\psi_i\\
	\begin{pmatrix}
		A & -B^T\\
		B & 0
	\end{pmatrix}\cdot\begin{pmatrix}
		U\\
		p
	\end{pmatrix}
	=\begin{pmatrix}
		F\\
		0
	\end{pmatrix}
\end{align*}

\subsection*{Lösbarkeit von diskreten Problemen}
Stetigkeit und Elliptizität vererben sich für stetige Probleme.
Eine \textbf{diskrete Version der $\inf$-$\sup$-Bedingung} muss angenommen / geprüft werden.
\begin{align*}
	\exists\beta>0:\inf\limits_{q_h\in Q_h}\sup\limits_{v_h\in V_h}\frac{b(q_h,v_h)}{\Vert v_h\Vert_V\cdot\Vert q_h\Vert_V}\geq\beta
\end{align*}
Wir nehmen an, dass $\beta$ ist unabhängig von Netzparameter $h$ ist.
Folglich ist das diskrete Problem eindeutig lösbar.\\
Paare von $(V_h,Q_h)$ erfüllen diese diskrete $\inf$-$\sup$-Bedingung:
\begin{align*}
	V_h:S_{h,0}^{r,0},\qquad Q:S_h^{r-1,0},\qquad r\geq2,\qquad P_2/P_1\\
	r\geq 2,\qquad Q_h=S_h^{r-1,1},\qquad V_h=P_r+\text{ Bereicherung}\supseteq S_{h,0}^{r,0}
\end{align*}
Die Variante in der ersten Zeile wird \textbf{Taylor-Hood-Element} genannt.

\subsubsection*{Fehlerschätzung}
$a\colon V\times V\to\R$ sei bilinear, stetig mit Konstante $M$ und koerziv (V-elliptisch) mit Konstante $\alpha$, also
\begin{align*}
	\big|a(v,w)\big|&\leq M\Vert v\Vert_V\Vert w\Vert_V\\
	a(v,v)&\geq\alpha\Vert v\Vert_V^2
\end{align*}
$b\colon Q\times V\to\R$ sei bilinear, stetig mit Konstante $C_b$:
\begin{align*}
	\big|b(q,v)\big|&\leq C_b\Vert q\Vert_Q\Vert v\Vert_V
\end{align*}
Die $\inf$-$\sup$-Bedingung für $(V,Q)$
\begin{align*}
	\exists\beta>0:\inf\limits_{q\in Q}\sup\limits_{v\in V}\frac{b(q,v)}{\Vert q\Vert_Q\Vert v\Vert_V}\geq\beta
\end{align*}
ist erfüllt. Ebenso die uniforme $\inf$-$\sup$-Bedingung für $(V_h,Q_h)$:
\begin{align*}
	\exists\tilde{\beta}>0:\inf\limits_{q_h\in Q_h}\sup\limits_{v_h\in V_h}\frac{b(q_h,v_h)}{\Vert q_h\Vert_Q\Vert v_h\Vert_V}\geq\tilde{\beta}
\end{align*}
wobei $\tilde{\beta}$ unabhängig von $h$ ist.

\begin{align*}
	W&=\Big\lbrace v\in V:b(q,v)=0~\forall q\in Q\Big\rbrace\text{ divergenzfreie Funktionen}\\
	W^\perp&=\Big\lbrace v\in V:(v,w)=0~\forall w\in W\Big\rbrace\\
	W_h&=\Big\lbrace v_h\in V_h:b(q_h,v_h)=0~\forall q_h\in Q_h\Big\rbrace\text{ diskrete divergenzfreie Funktionen}\\
	W_h^\perp&=\Big\lbrace v_h\in V_h:(v_h,w_h)=0~\forall w_h\in W_h\Big\rbrace
\end{align*}

Stetiges Problem: (P)

\begin{minipage}[t]{0.49\textwidth}
	Finde $(u,p)\in V\times Q$:
	\begin{align*}
		a(u,v)-b(p,v)&=f(v) &\forall&v\in V\\
		b(q,v)&=0 &\forall& q\in Q\\
	\end{align*}
\end{minipage} %zwischen Minipages darf kein leere Zeile stehen, da es sonst nicht funktioniert
\begin{minipage}[t]{0.49\textwidth}
	Finde $u\in W$:
	\begin{align*}
		a(u,v)=f(v)\qquad\forall v\in W
	\end{align*}
\end{minipage}

Diskretes Problem: ($P_h$)\\
\begin{minipage}[t]{0.49\textwidth}
	Finde $(u_h,p_h)\in V_h\times Q_h$:
	\begin{align*}
		a(u_h,v_h)-b(p_h,v_h)&=f(v_h) &\forall&v_h\in V_h\\
		b(q_h,v_h)&=0 &\forall& q_h\in Q_h
	\end{align*}
\end{minipage} %zwischen Minipages darf kein leere Zeile stehen, da es sonst nicht funktioniert
\begin{minipage}[t]{0.49\textwidth}
	Finde $u_h\in W_h$:
	\begin{align*}
		a(u_h,v_h)=f(v_h)\qquad\forall v_h\in W_h
	\end{align*}
\end{minipage}

\begin{bemerkung}
	Da im Allgemeinen $W_h\not\subseteq W$ gilt, werden Techniken wie Céas Lemma \ref{theorem2.2CeasLemma} nicht funktionieren.\\
	Selbst wenn $W_h\subseteq W$ gilt, bleibt die Frage der Approximationseigenschaften von $W_h$.
\end{bemerkung}

\begin{theorem} %nonumber
	Unter obigen Voraussetzungen gilt
	\begin{align*}
		\inf\limits_{w_h\in W_h}\big\Vert u-w_h\big\Vert_V\leq c\cdot\inf\limits_{v_h\in V_h}\big\Vert u-v_h\big\Vert_V
	\end{align*}
	d.h. der Raum $W_h$ hat dieselben Approximationseigenschaften wie $V_h$.
\end{theorem}

\begin{proof}
	Sei $v_h\in V_h$ beliebig.
	Wir konstruieren eine Funktion $r_h\in V_h$ so, dass $w_h:=v_h+r_h\in W_h$.
	Wir wollen, dass $w_h\in W_h$.
	Es gilt:
	\begin{align*}
		w_h\in W_h
		&\Longleftrightarrow \Big(b(q_h,w_h)=0\qquad&\forall& q_h\in Q_h\Big)\\
		&\Longleftrightarrow \Big(b(q_h,v_h)+b(q_h,r_h)=b(q_h,u) &\forall&q_h\in Q_h\Big)\\
		&\Longleftrightarrow\Big( b(q_h,r_h)=b(q_h,u-v_h) &\forall&q_h\in Q_h\Big)
	\end{align*}
	Aus der diskreten $\inf$-$\sup$-Bedingung folgt, dass es eine eindeutige Lösung $r_h\in W_h^\perp$ von
	\begin{align*}
		b(q_h,r_h)=b(q_h,u-v_h)\qquad\forall q_h\in Q_h
	\end{align*}
	gibt. Es folgt:
	\begin{align*}
		\Vert r_h\Vert&\leq\frac{1}{\tilde{\beta}}\Vert B_h r_h\Vert_{Q_h^\ast}\\
		&=\frac{1}{\tilde{p}}\sup\limits_{q_h\in Q_h}\frac{b(q_h,r_h)}{\Vert q_h\Vert_Q}\\
		&=\frac{1}{\tilde{\beta}}\sup\limits_{q_h\in Q_h}\frac{b(q_h,u-v_h)}{\Vert q_h\Vert_Q}\\
		&\leq\frac{c_b}{\tilde{\beta}}\Vert u-v_h\Vert_V
	\end{align*}

	\begin{align*}
		\Vert u-w_h\Vert_V
		&\leq\Vert u-v_h\Vert_V+\Vert\underbrace{v_h-w_h}_{=r_h}\Vert_V
		\leq\left(1-\frac{c_b}{\tilde{\beta}}\right)\Vert u-v_h\Vert_V\\
		\inf\limits_{z_h\in W_h}\Vert u-z_h\Vert_V
		&\leq\Vert u-w_h\Vert_V
		\leq\left(1+\frac{c_b}{\tilde{\beta}}\right)\Vert u-v_h\Vert_V\qquad\forall v_h\in V_h\\
		\implies
		\inf\limits_{z_h\in W_h}\Vert u-z_h\Vert_V
		&\leq\inf\limits_{v_h\in V_h}\Vert u-v_h\Vert_V\left(1+\frac{c_b}{\tilde{\beta}}\right)
	\end{align*}
\end{proof}

\begin{theorem} %nonumber
	Unter obigen Voraussetzungen gilt
	\begin{align*}
		\Vert u-u_h\Vert_V&\leq\left(1+\frac{M}{\alpha}\right)\inf\limits_{v_h\in W_h}\Vert u-v_h\Vert_V+\frac{c_b}{\alpha}\inf\limits_{q_h\in Q_h}\Vert p-q_h\Vert_Q
	\end{align*}
\end{theorem}

\begin{bemerkung}
	Die Abschätzung von $u$ hängt im Allgemeinen vom Raum $Q_h$ ab.
\end{bemerkung}

\begin{proof}
	$V_h\subseteq V$
	Für die Probleme (P) bzw. ($P_h$) gilt:
	\begin{align*}
		&(P) &a&(u,v_h)-b(p,v_h)-b(p,v_h)&=&f(v_h) &\forall&v_h\in V_h\subseteq V\\
		&(P_h) &a&(u_h,v_h)-b(p_h,v_h)-b(p,v_h)&=&f(v_h) &\forall&v_h\in V_h\\
		&\implies
		&a&(u-u_h,v_h)&=&b(p-p_h,v_h) &\forall&v_h\in V
	\end{align*}
	Dies wird auch \textbf{Galerkin Orthogonalität} genannt.
	Sei $w_h\in W_h$ beliebig und setze $v_h:=w_h-u_h\in W_h$.
	Dann gilt:
	\begin{align*}
		\alpha\Vert\underbrace{w_h-u_h}_{=v_h}\Vert_V^2
		&\leq a(w_h-u_h,v_h)\\
		&=a(w_h-u,v_h)+a(u-u_h,v_h)\\
		&=a(w_h-u,v_h)+b(p-p_h,v_h)-\underbrace{b(q_h-p_h,v_h)}_{=0~\forall q_h\in Q_h}\\
		&=a(w_h-u,v_h)+b(p-q_h,v_h)\\
		\overset{\text{stetig}}&\leq
		M\Vert u-w_h\Vert_V\Vert v_h\Vert_V+c_b\Vert p-q_h\Vert_Q\Vert v_h\Vert_V\\
		\implies
		\Vert w_h-u_h\Vert_V
		&\leq \frac{M}{\alpha}\Vert u-w_h\Vert_V+\frac{c_b}{\alpha}\Vert p-q_h\Vert_Q &\forall& q_h\in Q_h,\forall w_h\in W_h\\
		\implies
		\Vert u-u_h\Vert_V
		&\leq\Vert u-w_h\Vert_V+\Vert w_h-u_h\Vert_V\\
		&\leq\left(1+\frac{M}{\alpha}\right)\Vert u-w_h\Vert_V+\frac{c_b}{\alpha}\Vert p-q_h\Vert_Q &\forall& q_h\in Q_h,\forall w_h\in W_h
	\end{align*}
	Bilden des Infimums $\inf\limits_{q_h\in W_h}$ liefert die Behauptung.
\end{proof}

\begin{theorem}
	Gelten obige Voraussetzungen. Dann gilt:
	\begin{align*}
		\Vert p-p_h\Vert_Q&\leq\left(1+\frac{c_b}{\tilde{\beta}}\right)\inf\limits_{q_h\in Q_h}\Vert p-q_h\Vert_Q+\frac{M}{\tilde{\beta}}\Vert u-u_h\Vert_V
	\end{align*}
\end{theorem}

\begin{proof}
	Sei $q_h\in Q$ beliebig. Dann gilt aus der $\inf$-$\sup$-Bedingung:
	\begin{align*}
		\Vert p_h-q_h\Vert_Q
		&\leq\frac{1}{\tilde{\beta}}\sup\limits_{v_h\in V_h}\frac{b(p_h-q_h,v_h)}{\Vert v_h\Vert_V}
	\end{align*}
	Nun setzen wir
	\begin{align*}
		b(p_h-q_h,v_h)
		&=b(p_h-p,v_h)+b(p-q_h,v_h)\\
		&=a(u_h-u,v_h)+b(p-q_h,v_h)
	\end{align*}
	in obige Ungleichung ein und erhalten
	\begin{align*}
		\Vert p_h-q_h\Vert_Q
		&\leq\frac{1}{\tilde{\beta}}\sup\limits_{v_h\in V_h}\frac{b(p_h-q_h,v_h}{\Vert v_h\Vert_V}\\
		&=\frac{1}{\tilde{\beta}}\sup\limits_{v_h\in V_h}\frac{a(u_h-u,v_h)+b(p-q_h,v_h)}{\Vert v_h\Vert_V}\\
		&=\frac{1}{\tilde{\beta}}\sup\limits_{v_h\in V_h}\frac{b(p-q_h,v_h)-a(u-u_h,v_h)}{\Vert v_h\Vert_V}\\
		&\leq\frac{1}{\tilde{\beta}}\sup\limits_{v_h\in V_h}\frac{c_b\Vert p-q_h\Vert_Q\Vert v_h\Vert_V+M\Vert u-u_h\Vert_V\Vert v_h\Vert_V}{\Vert v_h\Vert_V}\\
		&=\frac{1}{\tilde{\beta}}c_b\Vert p-q_h\Vert_Q+M\Vert u-u_h\Vert_V\\
		&=\frac{1}{\tilde{\beta}}\Big(c_b\Vert p-q_h\Vert-Q+M\Vert u-v_h\Vert_V\Big)\qquad\forall q_h\in Q_h
	\end{align*}

	\begin{align*}
		\Vert p-p_h\Vert_Q
		&\leq\Vert p-q_h\Vert_Q+\Vert q_h-p_h\Vert_Q\\
		&\leq \left(1+\frac{c_b}{\tilde{\beta}}\right)\Vert p-q_h\Vert_Q+\frac{M}{\tilde{\beta}}\Vert u-u_h\Vert_V\qquad\forall q_h\in Q_h
	\end{align*}
	Bilden des Infimums liefert die Behauptung.
\end{proof}

\begin{bemerkung}
	\begin{align*}
		\Vert p-p_h\Vert_Q
		&\leq c\inf\limits_{v_h\in V_h}\Vert u-v_h\Vert_V+c\inf\limits_{q_h\in Q_h}\Vert p-q_h\Vert_Q
	\end{align*}
\end{bemerkung}

\textbf{Stokes}:
\begin{align*}
	V&=\big(H_0^1(\Omega)\big)^d, &\Vert&\cdot\Vert_V=\Vert\cdot\Vert_{1,2,\Omega}\\
	Q&=L_0^2(\Omega), &\Vert&\cdot\Vert_Q=\Vert\cdot\Vert_{0,2,\Omega}
\end{align*}

Wähle Elemente der Ordnung $r$ für $V_h$ und Elemente der Ordnung $r-1$ für $Q_h$ würde zu
\begin{align*}
	\Vert p-p_h\Vert_{0,2}&=\O(h^r)\\
	\Vert u-u_h\Vert_{1,2}&=\O(h^r)
\end{align*}
führen.

\begin{beispiel}[Beispiele für Stokes-Probleme]\
	\begin{enumerate}
		\item Mini-Element
		%TODO Add Sketch for $u,p,p_1$
		\item (klassisches) Taylor-Hood Element
		%TODO Add Sketch with $P_2,P_r,P_1,P_{r-1},r\geq2$
		\item Bernardi-Rangel Element
		%TODO Add Sketch
	\end{enumerate}
\end{beispiel}

Die $\inf$-$\sup$-Bedingung spielt eine große in der Theorie, ist aber in der Praxis oft schwer zu zeigen.

\subsection{Nachweis der \texorpdfstring{$\inf$-$\sup$}{inf-sup}-Bedingung (diskrete Version)}
Man nimmt stets an, dass die stetige Version der $\inf$-$\sup$-Bedingung gilt.

\begin{lemma}[Fortin, 1977]\label{lemmaFortin}\enter
	Angenommen die stetige $\inf$-$\sup$-Bedingung
	\begin{align*}
		\inf_{q\in Q\setminus\{0\}}\sup_{v\in V\setminus\{0\}}\frac{b(v,q)}{\Vert v \Vert_V\Vert q\Vert_Q}\geq\beta>0
	\end{align*}
	gilt. Dann gilt die diskrete $\inf$-$\sup$-Bedingung
	\begin{align*}
		\inf_{q_h\in Q_h\setminus\{0\}}\sup_{v_h\in V_h\setminus\{0\}}\frac{b(q_h,v_h)}{\Vert v_h \Vert_V\Vert q_h\Vert_Q}\geq\beta^\ast>0
	\end{align*}
	gleichmäßig im Gitterparameter $h$ genau dann, wenn
	es eine Familie von linearen Interpolationsoperatoren $\{I_h\colon V\to V_h\}_h$ gibt, so dass
	\begin{align}\label{eqLemmaFortinBed1}
		b(q_h,v)=b(q_h,I_h v)\qquad\forall q_h\in Q_h,\forall v\in V
	\end{align}
	und
	\begin{align}\label{eqLemmaFortinBed2}
		\Vert I_h v\Vert_V\leq C\Vert v\Vert_V\qquad\forall v\in V
	\end{align}
	gilt, wobei $C$ unabhängig von $h$ ist.
\end{lemma}

\begin{proof}\
	\begin{enumerate}[label=(\roman*)]
		\item Angenommen es gibt die Familie der $I_h$ mit den beiden obigen Eigenschaften \eqref{eqLemmaFortinBed1} und \eqref{eqLemmaFortinBed2}.
			Sei $q_h\in Q_h\setminus\{0\}$ beliebig. Dann
			% Ich (Felix) mag ja die Variante mit \norm{q_h}_Q = 1 lieber....
		\begin{align*}
			\sup\limits_{v_h\in V_h\setminus\{0\}}\frac{b(q_h,v_h)}{\Vert v_h \Vert_V\Vert q_h\Vert_Q}
			\overset{I_h V\subseteq V_h}&\geq
			\sup\limits_{v\in V\setminus\{0\}}\frac{b(q_h,I_h v)}{\Vert I_h v\Vert_V\Vert q_h\Vert_Q}\\
			\overset{\eqref{eqLemmaFortinBed1}}&=
			\sup\limits_{v\in V\setminus\{0\}}\frac{b(q_h,v)}{\Vert I_h v\Vert_V\Vert q_h\Vert_Q}\\
			\overset{\eqref{eqLemmaFortinBed2}}&\geq
			\frac{1}{C}\sup\limits_{v\in V\setminus\{0\}}\frac{b(q_h,v)}{\Vert v\Vert_V\Vert q_h\Vert_Q}\\
			&\geq\frac{\beta}{C}>0
		\end{align*}
		Das $\beta$ kommt aus der stetigen $\inf$-$\sup$-Bedingung.
		Wir haben also die diskrete $\inf$-$\sup$-Bedingung (u.a.) aus der stetigen inf-sup-Bedingung gefolgert.
		\item Angenommen die diskrete $\inf$-$\sup$-Bedingung ist erfüllt.
			Definiere wie in der Herleitung der $\inf$-$\sup$-Bedingung.
		\footnote{$Q_h^\ast$ ist der Dualraum von $Q_h$}
		\begin{align*}
			B_h\colon V_h\to Q_h^\ast,\qquad
		\dualpair{B_h v_h}{q_h}\coloneqq b(q_h,v_h)\qquad\forall q_h\in Q_h,\forall v_h\in V_h
		\end{align*}
		Also ist $B_h\colon W_h^\perp\to Q_h^\ast$ ein Isomorphismus (Äquivalenz gemäß Theorem \ref{theorem8.0infsupAequivalenz}.) % vor Theorem \ref{theorem8.1}). % TODO: Die Referenz im pdf schickt einen zum Beispiel 8.1, das in der Vorlesung keine Nummer hatte und das eigentlich referenzierte Theorem hat keine Nummer.
		\begin{align*}
			\Vert v_h\Vert_V
			&\leq\frac{1}{\beta^\ast}\Vert B_h v_h\Vert_{Q_h^\ast}
			\overset{\text{Def Norm}}=
			\frac{1}{\beta^\ast}\sup_{q_h\in Q_h\setminus\{0\}}\frac{b(q_h,v_h)}{\Vert q_h\Vert_Q}  \quad \forall v_h \in W_h^{\perp}
		\end{align*}
		Für ein gegebenes $v\in V$ definieren wir $I_h v\in W_h^\perp\subseteq V_h$ als Lösung von
		\begin{align*}
			B_h(I_h v)=g_h\text{ in }Q_h^\ast
		\end{align*}
		wobei
		\begin{align*}
			\dualpair{g_h}{q_h} \coloneqq b(q_h,v)\qquad\forall q_h\in Q_h
		\end{align*}
		Also gilt
		\begin{align*}
			b(q_h,I_h v)
			\overset{\text{Def}}=
			\dualpair{B_h(I_h v)}{q_h}
			=\dualpair{g_h}{q_h}
			=b(q_h,v)\qquad\forall q_h\in Q_h
		\end{align*}
		Somit haben wie die erste Bedingung \eqref{eqLemmaFortinBed1} gezeigt.
		Nun zur zweiten Bedingung \eqref{eqLemmaFortinBed2}:
		\begin{align*}
			\Vert I_h v\Vert_V
			&\leq \frac{1}{\beta^\ast}\sup_{q_h\in Q_h\setminus\{0\}}\frac{b(q_h,I_h v)}{\Vert q_h\Vert_Q}\\
			&=\frac{1}{\beta^\ast}\sup_{q_h\in Q_h\setminus\{0\}}\frac{b(q_h,v)}{\Vert q_h\Vert_Q}\\
			&\leq\frac{1}{\beta^\ast}\sup_{q_h\in Q_h\setminus\{0\}}\frac{c_b\Vert q_h\Vert_{Q}\Vert v\Vert_V}{\Vert q_h\Vert_Q}\\
			&=\frac{1}{\beta^\ast}\sup_{q_h\in Q_h\setminus\{0\}}c_b\Vert v\Vert_V\\
			&=\frac{c_b}{\beta^\ast}\Vert v\Vert_V
		\end{align*}
	\end{enumerate}
\end{proof}

\begin{bemerkung}
	Für nichtkonforme Räume $V_h$ ändert sich folgendes:
	\begin{align*}
		\Vert v_h\Vert_V&\leadsto\Vert v_h\Vert_h\\
		\Vert I_h v\Vert_V&\leadsto\Vert I_h v\Vert_h\\
		b(q,v)&=\sum\limits_{K\in\T}\int\limits_K q\div(v)\d x \text{ für das Stokes-Problem}
	\end{align*}
\end{bemerkung}

Sei $V_h = P_1^{nc}$ das Crouzeix-Raviart-Element und $Q_h$ stückweise konstant, also $P_0$. % nc für non-conforming
Das Crouzeix-Raviart-Element ist stückweise linear und nur in den Mittelpunkten der Dreiecksseiten stetig.
Wir wollen Lemma \ref{lemmaFortin} anwenden.
Wie wählen wir $I_h\colon V\to V_h$? Oft ist $V = H^1_0(\Omega)^d$ und auf diesem ergibt klassische Interpolation mit Punktwerten keinen Sinn. %\nl % einfach neuer Absatz ist sinnvoll

Der natürliche Weg ist folgender:
\begin{align*}
	\frac{1}{\measure E}\int\limits_E I_h v\d\gamma=\frac{1}{\measure E}\int\limits_E v\d\gamma=(I_h v)(m_E).
\end{align*}
Also bleibt das Integralmittel erhalten.
$m_E$ ist hier der Mittelpunkt der Kange $E$.
Dann ist $I_h$ ein \textbf{Fortin-Operator}, d.h. er er erfüllt die beiden Bedingungen \eqref{eqLemmaFortinBed1},\eqref{eqLemmaFortinBed2} in Lemma \ref{lemmaFortin}:
\begin{align*}
	b(q_h,I_h v)
	&=\sum\limits_{K}\int\limits_K \underbrace{q_K}_{\text{konstant auf }K}\div(I_h v)\d x\\
	&=\sum\limits_{K} q_K\int\limits_K \div(I_h v)\d x\\
	\overset{\text{Gauß}}&=
	\sum\limits_{K} q_K\int\limits_{\partial K} (I_h v)\cdot n_K\d \gamma\\
	\overset{n_E\text{ konst. auf }E}&=\sum\limits_K q_K\sum\limits_{E\subseteq\partial K}\int\limits_E(I_h v)\cdot n_E\d\gamma\\
	\overset{\text{Def }I_h}&=
	\sum\limits_K q_K\sum\limits_{E\subseteq\partial K}\int\limits_E v\cdot n_E\d\gamma\\
	\overset{\text{selbe Argumente rückwärts}}&=
	b(q_h,v)
\end{align*}
Um die in $h$ gleichmäßige Stetigkeit von $I_h$ nachzuweisen, nutze die $\halfnorm{·}_{1, K}$-Halbnorm, denn sie ist äquivalent zur vollen Norm und leichter zu handhaben.
\begin{align*}
	|I_h v|^2_{1,K}
	&=\int\limits_K\nabla I_h v:\nabla I_h v\d x\\
	&=-\int\limits_K I_h v\cdot\underbrace{\laplace I_h v}_{\overset{\text{stückweise linear}}{=}\vec{0}}\d x+\int\limits_{\partial K}I_h v\cdot\underbrace{\big(\nabla I_h v\cdot n_K\big)}_{\text{konstant auf jeder Kante $E\subseteq\partial K$}}\d\gamma\\
	&=-\int\limits_K v\cdot\underbrace{\laplace I_h v}_{=\vec{0}}\d x+\sum\limits_{E\subseteq\partial K}\int\limits_{E} v\cdot\big(\nabla I_h v\cdot n_K\big)\d\gamma\\
	\overset{\text{selbe Schritte rückwärts}}&=
	\int\limits_K\nabla v\cdot\nabla I_h v\d x\\
	\overset{\text{CS}}&\leq
	|v|_{1,K}|I_h v|_{1,K}\\
	\implies\Vert I_h v\Vert_h&\leq\vert v\vert_1
\end{align*}

\subsection{Ausblick}
Hier folgt eine Sammlung von Themen, die in der Vorlesung nicht behandelt wurden und von weiterführendem Interesse sind.

Wie die linearen Gleichungssysteme, die aus den diskreten Problemen herauskommen, gelöst werden, haben wir nicht betrachtet.
Dafür gibt es zum Beispiel multi-mesh-Methoden.

Wie die Integrale, gerade die der Datenfunktionen berechnet wurden, haben wir nicht betrachtet.
Häufig ist dies nicht exakt möglich, sodass die diskreten Probleme numerische Integration benötigt.
Häufig ist es auch ausreichend, nur Integrale der Form $∫uv$ oder $∫∇u·∇v$ exakt zu berechnen.
Wie exakt die Quadraturformeln sein müssen, hängt davon ab, wie gut dann tatsächlich die restliche Lösungsverfahren ist.

Wir haben immer angenommen, das das Gebiet ein Polygon ist, sodass es mit Dreiecken aufgeteilt werden kann.
Eine runde Form kann mit immer mehr Dreiecken immer besser angenähert werden, häufig sind aber auch Elemente mit gekrümmten Kanten nötig und möglich.
Es muss auch geklärt werden, wie überhaupt der Fehler gemessen wird und wie Randwerte auf den Rand des diskreten Gebietes projiziert werden.

Wir haben keine Gleichungen höherer Ordnung betrachtet.

Wir haben keine zeitabhängigen Systeme betrachtet. Bei diesen werden die betrachteten Wege mit ODE-Lösern kombiniert.