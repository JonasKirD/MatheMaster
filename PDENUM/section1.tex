% This work is licensed under the Creative Commons
% Attribution-NonCommercial-ShareAlike 4.0 International License. To view a copy
% of this license, visit http://creativecommons.org/licenses/by-nc-sa/4.0/ or
% send a letter to Creative Commons, PO Box 1866, Mountain View, CA 94042, USA.
% vim: set noexpandtab:

\section{Sobolev-Räume}
Bezeichnungen für dieses Kapitel:

\begin{itemize}
	\item $d\geq1$ sei die Raumdimension
	\item $\Omega\subseteq\R^d$ sei offen und beschränkt
	\item $p\in[1,\infty)$ reelle Zahl
	\item $q\in(1,\infty]\mit\frac{1}{p}+\frac{1}{q}=1$ \textbf{konjungierter / dualer Exponent}
	\item $\alpha=(\alpha_1,\ldots,\alpha_d)\in\N_0^d$ Multiindex mit
	\begin{align*}
		|\alpha|:=\alpha_1+\ldots+\alpha_d\\
		D^\alpha\varphi:=\frac{\partial^{|\alpha|}\varphi}{\partial x_1^{\alpha_1}\hdots\partial x_d^{\alpha_d}}
	\end{align*}
	\item $L^p(\Omega):=\left\lbrace f:\Omega\to\R:f\text{ messbar und }\int\limits_\Omega |f(x)|^p\d\mathcal{L}(x)<\infty\right\rbrace$ Lebesgue-Räume
	\item Für einen Funktionenraum $\mathcal{F}$ bezeichnet $\mathcal{F}_0$ den Teilraum der Funktionen aus $\mathcal{F}$ mit kompaktem Träger.
\end{itemize}

\begin{bemerkung}\
	\begin{enumerate}
		\item Da $\Omega$ beschränkt ist, gilt $L^p(\Omega)\subseteq L^1(\Omega)$ und die kanonische Injektion ist stetig.
		\item Es gilt die Gauß-Formel:
		\begin{align}\label{GaussFormel}
			\int\limits_\Omega\varphi\cdot D^\alpha\psi\d x=(-1)^{|\alpha|}\cdot\int\limits_\Omega\psi\cdot D^\alpha\varphi\d x\qquad\forall \varphi,\psi\in C_0^\infty(\Omega)
		\end{align}
	\end{enumerate}
\end{bemerkung}

\begin{definition}[schwache Ableitung]
	Seien $\varphi,\psi\in L^1(\Omega)$ und sei $\alpha\in\N_0^\alpha$ ein Multiindex.
	Dann heißt $\psi$ die \textbf{$\alpha$-te schwache Ableitung} von $\varphi:\Longleftrightarrow$
	\begin{align*}
		\forall v\in C_0^\infty(\Omega):\int\limits_\Omega\varphi\cdot D^\alpha v\d x=(-1)^{|\alpha|}\cdot\int\limits_\Omega\psi\cdot v\d x
	\end{align*}
	Kurzschreibweise: $\psi=D^\alpha\varphi$
\end{definition}

\begin{bemerkung}\ 
	\begin{enumerate}
		\item Die $\alpha$-te schwache Ableitung ist eindeutig bestimmt im Sinne des $L^1$ (also bis auf Lebesgue-Nullmengen).
		\item Ist $\varphi\in C^{|\alpha|}(\Omega)$, dann existiert die schwache $\alpha$-te 	Ableitung, die mit der klassischen Ableitung übereinstimmt.
	\end{enumerate}
\end{bemerkung}

\begin{beisp}
	$d=1$, $\Omega=(-1,1)$, $\varphi(x):=|x|$\\
	Behauptung: $\varphi'(x)=\left\lbrace\begin{array}{cl}
		-1, & \falls -1<x<0\\
		1, & \falls 0\leq x<1
	\end{array}\right.$\\
	Die schwache Ableitung existiert also und der Wert an der Stelle 0 ist nicht relevant.
	\begin{proof}
		Sei $v \in C_0^\infty(\Omega)$ beliebig. Dann
		\begin{align*}
			\int\limits_{-1}^1 \varphi\cdot v' \d x
			&=\int\limits_{-1}^0 \varphi\cdot v' \d x + \int\limits_{0}^1 \varphi v' \d x \\
			\overset{\text{part}}&=
			\big[\varphi\cdot v\big]_{-1}^0 - \int\limits_{-1}^0 (-1) v \d x + \big[\varphi\cdot v\big]_{0}^1 - \int\limits_{0}^1 (1) v \d x \\
			\overset{v = 0 \text{ auf Rand}}&=
			-\int\limits_{-1}^0 (-1) v \d x - \int_{0}^1 (1) v \d x \\
			\overset{\text{Def}}&=
			-\int\limits_{-1}^1 \varphi'(x)v\d x
		\end{align*}
	\end{proof}
\end{beisp}

\begin{definition}[Sobolev-Räume]
	Für $k\in\N_0$, $p\in[1,\infty)$ definieren wir den \textbf{Sobolev-Raum} wie folgt;
	\begin{align}
		W^{k,p}(\Omega):=\Big\lbrace\varphi\in L^p(\Omega):D^\alpha\varphi\text{ (schwache Abl.) existiert und }D^\alpha\varphi\in L^p(\Omega)~\forall|\alpha|\leq k\Big\rbrace
	\end{align}
	Als Norm vereinbaren wir
	\begin{align*}
		\Vert\varphi\Vert_{k,p,\Omega}:=\left(\sum\limits_{|\alpha|\leq k}\left\Vert D^\alpha\varphi\right\Vert^p_{L^p}\right)^{\frac{1}{p}}
		=\left(\sum\limits_{|\alpha|\leq k}\int\limits_\Omega\left| D^\alpha\varphi(x)\right|^p\d x\right)^{\frac{1}{p}}
	\end{align*}
	Durch
	\begin{align*}
		|\varphi|_{k,p,\Omega}:=\left(\sum\limits_{|\alpha|= k}\left\Vert D^\alpha\varphi\right\Vert^p_{L^p}\right)^{\frac{1}{p}}
	\end{align*}
	wird eine Halbnorm definiert.\\
	Für $p=2$ schreiben wir $H^k(\Omega):=W^{k,2}(\Omega)$.
\end{definition}

\begin{satz}[Eigenschaften der Sobolev-Räume]
	Es gilt:
	\begin{enumerate}
		\item $\left(W^{k,p}(\Omega),\Vert\cdot\Vert_{k,p,\Omega}\right)$ ist ein Banachraum.
		\item $C^\infty(\overline{\Omega})$ liegt dicht in $W^{k,p}(\Omega)$.
		\item $H^k(\Omega)$ ist ein Hilbertraum mit dem Skalarprodukt
		\begin{align*}
			\langle \varphi,\psi\rangle_k:=\sum\limits_{|\alpha|\leq k}\int\limits_\Omega D^\alpha\varphi\cdot D^\alpha\psi\d x.
		\end{align*}
	\end{enumerate}
\end{satz}

\begin{satz}[Glätte von stückweise glatten Funktionen]\label{satz1.4}\enter
	Seien $\Omega_1,\Omega_2\subseteq\Omega$ zwei nichtleere, offene, beschränkte und disjunkte Teilmengen mit stückweise glattem Rand.
	Gelte
	$\overline{\Omega}=\overline{\Omega_1}\cup\overline{\Omega_2}$.
	Weiterhin sei $\varphi\in L^p(\Omega)$ so, dass
	\begin{align*}
		\exists k\geq 1:\forall i\in\lbrace1,2\rbrace:\varphi|_{\Omega_i}\in C^k(\overline{\Omega_i}).
	\end{align*}
	Dann gilt:
	\begin{align*}
		\varphi\in W^{k,p}(\Omega)\Longleftrightarrow\varphi\in C^{k-1}(\Omega)
	\end{align*}
\end{satz}

\begin{proof}
	Siehe Übung.
\end{proof}

\begin{definition}
	Die Vervollständigung des $C_0^\infty(\Omega)$ bzgl. der Norm $\Vert\cdot\Vert_{k,p,\Omega}$ wird mit $W_0^{k,p}(\Omega)$ bezeichnet.
	Außerdem setze $H_0^k(\Omega):=W_0^{k,2}(\Omega)$.
\end{definition}

\begin{definition}[Lipschitz-Rand]\enter
	$\Omega$ hat einen \textbf{Lipschitz-Rand} $:\gdw\exists N\in\N,\exists U_1,\ldots,U_N\subseteq\R^d$ offen, sodass
	\begin{enumerate}
		\item $\begin{aligned}
			\partial\Omega\subseteq\bigcup\limits^N_{i=1} U_i
		\end{aligned}$
		\item $\begin{aligned}
			\forall i\in\lbrace1,\ldots,N\rbrace:\partial\Omega\cap U_i
		\end{aligned}$
		ist darstellbar als Graph einer Lipschitz-stetigen Funktion
	\end{enumerate}
	Das Gebiet $\Omega$ wird dann \textbf{Lipschitz-Gebiet} genannt.
\end{definition}

\begin{bemerkung}
	Für Lipschitz-Gebiete existiert fast überall auf $\partial\Omega$ der äußere Normalenvektor.
\end{bemerkung}

\begin{satz}[Spursatz]\label{satz1.7Spursatz}\enter
	Seien $\Omega$ eine Lipschitz-Gebiet, $k\in\N$, $l\in\lbrace 0,\ldots,k-1\rbrace$.\\
	Dann gibt es eine stetige lineare Abbildung
	\begin{align*}
		\gamma_l:W^{k,p}(\Omega)\rightarrow L^p(\partial\Omega)
	\end{align*}
	mit der Eigenschaft
	\begin{align*}
		\gamma_l(\varphi)=\frac{\partial^l}{\partial n^l}\varphi|_{\partial\Omega}\qquad\forall\varphi\in C^k(\overline{\Omega}).
	\end{align*}
\end{satz}

\begin{bemerkung}\
	\begin{itemize}
		\item $\gamma_0$ bildet die Funktion $\varphi$, die auf $\Omega$ lebt, auf ihre Randdaten ab.
		\item  $\gamma_1$ gibt die Normalenableitung zurück.
		\item Das Bild von $W^{k,p}(\Omega)$ unter $\gamma_l$ ist ein abgeschlossener Unterraum von $L^p(\partial\Omega)$, genauer:
		\begin{align*}
			\gamma_l\left(W^{k,p}(\Omega)\right)=W^{k-l-\frac{1}{p},p}(\partial\Omega)
		\end{align*}
	\end{itemize}
\end{bemerkung}

\begin{satz}
	Sei $\Omega$ ein Lipschitz-Gebiet.
	Dann gilt:
	\begin{align*}
		W_0^{k,p}(\Omega)=\left\lbrace\varphi\in W^{k,p}(\Omega):
		\forall l\in\lbrace0,\ldots,k-1\rbrace:\gamma_l(\varphi)=0\right\rbrace
	\end{align*}
\end{satz}

\begin{definition}
	Seien $(X,\Vert\cdot\Vert_X)$ und $(Y,\Vert\cdot\Vert_Y)$ zwei normierte Vektorräume.
	\begin{enumerate}
		\item Eine lineare Abbildung $A:X\to Y$ heißt \textbf{kompakt} $:\gdw$ das Bild der abgeschlossenen Einheitskugel in $X$ im Raum $Y$ relativkompakt ist.
		\item $X$ heißt \textbf{stetig eingebettet} in $Y$, in Zeichen $X\hookrightarrow Y:\gdw X\subseteq Y$ und die kanonische Injektion $I:X\to Y,~\varphi\mapsto\varphi$ stetig ist.
		\item $X$ heißt \textbf{kompakt eingebettet} in $Y$, in Zeichen $X\stackrel{C}{\hookrightarrow} Y:\gdw X\subseteq Y$ und die kanonische Injektion $I:X\to Y,~\varphi\mapsto\varphi$ kompakt ist.
	\end{enumerate}
\end{definition}

\begin{bemerkung}\
	\begin{enumerate}
		\item $\begin{aligned}
			X\hookrightarrow Y\Longrightarrow\exists c>0:\forall\varphi\in X:\Vert\varphi\Vert_Y\leq c\cdot\Vert\varphi\Vert_X
		\end{aligned}$
		\item $\begin{aligned}
			X\stackrel{C}{\hookrightarrow} Y\Longrightarrow X\hookrightarrow Y
		\end{aligned}$
		\item Ist $X\stackrel{C}{\hookrightarrow} Y$ und $(\varphi_n)_{n\in\N}\subseteq X$ eine beschränkte Folge in $X$, so besitzt $(\varphi_n)_{n\in\N}\subseteq Y$ eine konvergente Teilfolge.
	\end{enumerate}
\end{bemerkung}

\begin{satz}[Einbettung]\label{satz1.10}\
	\begin{enumerate}
		\item Sei $p<d$. Dann gilt:
		\begin{align*}
			W^{k,p}(\Omega)\hookrightarrow W^{k-1,q}(\Omega)\qquad\forall q\in\left[1,\frac{p\cdot d}{d-p}\right]\\
			W^{k,p}(\Omega)\stackrel{C}{\hookrightarrow} W^{k-1,q}(\Omega)\qquad\forall q\in\left[1,\frac{p\cdot d}{d-p}\right)
		\end{align*}
		\item Sei $p=d$. Dann gilt:
		\begin{align*}
			W^{k,p}(\Omega)\hookrightarrow W^{k-1,q}(\Omega)\qquad\forall q\in\left[1,\infty\right)
		\end{align*}
		\item Sei $k>\frac{d}{p}$. Dann gilt:
		\begin{align*}
			W^{k,p}(\Omega)\stackrel{C}{\hookrightarrow} C^l(\overline{\Omega})\qquad\forall 0\leq l\leq k-\frac{d}{p}
		\end{align*}
	\end{enumerate}
\end{satz}

\begin{bemerkung}\
	\begin{itemize}
		\item Es gilt stets
		\begin{align*}
			W^{k,p}(\Omega)\stackrel{C}{\hookrightarrow} W^{k-1,p}(\Omega)
		\end{align*}
		\item Falls $p=2$, $d\in\lbrace2,3\rbrace$:
		\begin{align*}
			H^2(\Omega)\hookrightarrow C(\overline{\Omega})
		\end{align*}
		Für $H^1(\Omega)$-Funktionen sind Punkte weiter nicht sinnvoll.
		\item Falls $p=2,~d=2$:
		\begin{align*}
			H^1(\Omega)\stackrel{C}{\hookrightarrow} L^q(\Omega)\qquad\forall q\in[1,\infty)
		\end{align*}
		\item Falls $p=2,~d=3$:
		\begin{align*}
			&H^1(\Omega)\stackrel{C}{\hookrightarrow} L^q(\Omega)\qquad\forall q\in[1,6)\\
			&H^1(\Omega)\stackrel{C}{\hookrightarrow} L^6(\Omega)
		\end{align*}
	\end{itemize}
\end{bemerkung}

\begin{beisp}
	Für $d=2$ gibt es $H^1(\Omega)$-Funktionen, die nicht stetig sind:
	\begin{itemize}
		\item Sei $\Omega=$ Einheitskreis
		\item $u(x,y):=\ln\left(\ln\left(\frac{4}{\sqrt{x^2+y^2}}\right)\right)$
	\end{itemize}
	$u$ hat einen Pol im Ursprung, d.h. $u\not\in C(\overline{\Omega})$, aber:\\
	\ul{Behauptung:}
	\[\|u\|^2_{1,2,\Omega}\leq 8\pi + \frac{2\pi}{\ln(4)}< \infty\]
	\[\implies u \in H^1(\Omega)\]
	\begin{proof}
		\[\|u\|^2_{1,2,\Omega} = \underbrace{\int\limits_\Omega |u|^2 \d x \d y}_{:=\text{ I}} +
		\underbrace{\int\limits_\Omega |u_x|^2 + |u_y|^2 \d x \d y}_{:=\text{ II}}\]
		Es gilt für alle $z\geq 4$:
		\[\ln(\ln(z))\leq \sqrt{z}\]
		Mit der Koflächenformel erhalten wir:
		\begin{align*}
			\text{I} &= \int\limits_\Omega |u|^2 \d x \d y \\
			&\leq \int\limits_\Omega \left(\frac{4}{\sqrt{x^2 + y^2}}\right)^2 \d x \d y \\
			\overset{\text{Koflä.}}&= \int\limits_0^{2\pi}\int\limits_0^1\left(\frac{4}{r}\right)^2 r \d r \d \varphi \\
			&= 8 \pi
		\end{align*}
		Als nächstes betrachten wir II. Dafür schauen wir uns zunächst die partiellen Ableitungen von $u$ an:
		\begin{align*}
			u_x &= \frac{1}{\ln\left(\frac{4}{r}\right)}\frac{r}{4}\left(-\frac{4}{r^2}\right)\frac{x}{r} = - \frac{x}{r^2\ln\left(\frac{4}{r}\right)} \\
			u_y &= - \frac{y}{r^2\ln\left(\frac{4}{r}\right)} \\
			\implies u_x^2 + u_y^2 &= \frac{x^2+y^2}{\left(r^2\ln\left(\frac{4}{r}\right)\right)^2} \\
			&= \frac{r^2}{\left(r^2\ln\left(\frac{4}{r}\right)\right)^2} \\
			&= \frac{1}{r^2\left(\ln\left(\frac{4}{r}\right)\right)^2} \\
		\end{align*}
		Also schlussendlich:
		\begin{align*}
			\text{II} \overset{\text{vgl. I}}&= \int\limits_0^{2\pi}\int\limits_0^1\frac{r}{r^2\left(\ln\left(\frac{4}{r}\right)\right)^2} \d r \d \varphi \\
			 &= 2\pi \int\limits_0^1 \frac{1}{r^2\left(\ln\left(\frac{4}{r}\right)\right)^2} \d r \\
			 &=\left[2\pi\left(\frac{1}{\ln\left(\frac{4}{r}\right)}\right)\right]_{r=0}^1 \\
			 &= \frac{2\pi}{\ln(4)}
		\end{align*}
	\end{proof}
\end{beisp}

\begin{proposition}\label{prop1.11}
	Sei $\Omega\subseteq\R^d$ ein beschränktes Lipschitz-Gebiet und sei $F$ lineares, stetiges Funktional auf $H^1(\Omega)$ und $B$ eine beschränkte, symmetrische Bilinearform auf $H^1(\Omega)$ mit
	\begin{align*}
		B(u,u)\geq0\qquad\forall u\in H^1(\Omega).
	\end{align*}
	Sei $g$ eine Funktion mit $g(x)=1~\forall x\in\overline{\Omega}$ so, dass
	\begin{align*}
		|F(g)|+\sqrt{B(g,g)}>0.
	\end{align*}
	Dann gilt:\\
	Die Normen $\Vert\cdot\Vert_{1,2,\Omega}$ und
	\begin{align*}
		\Vertiii{f}:=|f|_{1,2,\Omega}+|F(f)|+\sqrt{B(f,f)}
	\end{align*}
	sind äquivalent auf $H^1(\Omega)$, d. h. es existieren $c_1,c_2>0$ so, dass
	\begin{align*}
		c_1\cdot\Vert f\Vert_{1,2,\Omega}\leq\Vertiii{f}
		\leq
		c_2\cdot\Vert f\Vert_{1,2,\Omega}
		\qquad\forall f\in H^1(\Omega)
	\end{align*}
\end{proposition}

\begin{proof}
	\underline{Zeige $\Vertiii{\cdot}$ ist eine Norm:}
	Dies wird in der Übung gezeigt.\nl
	\underline{Zeige $\Vertiii{f}\leq c_2\cdot\Vert f\Vert_{1,2,\Omega}$:}
	\begin{align*}
		\Vertiii{f} &=
		|f|_{1,2,\Omega}+|F(f)|+\sqrt{B(f,f)}\\
		&\leq
		\Vert f\Vert_{1,2,\Omega} +c_F\cdot\Vert f\Vert_{1,2,\Omega}+\sqrt{c_B\cdot\Vert f\Vert^2_{1,2,\Omega}}\\
		&\leq
		(1+c_F+\sqrt{c_B})\cdot\Vert f\Vert_{1,2,\Omega}
	\end{align*}

	%(iii)
	\underline{Zeige $\Vert f\Vert_{0,2,\Omega}\leq c\cdot\Vertiii{f}$ indirekt:}
	Angenommen es gibt eine Folge $(f_n)_{n\in\N}\subseteq H^1(\Omega)$ mit
	\begin{align}\label{proof1.9}\tag{$\ast$}
		\Vert f_n\Vert_{0,2,\Omega}=1\text{ und }\Vertiii{f_n}<\frac{1}{n}~\forall n\in\N
	\end{align}
	Dann gilt für alle $n\in\N$:
	\begin{align*}
		|f_n|_{1,2,\Omega}<\frac{1}{n}
		\implies
		\Vert f_n\Vert^2_{1,2,\Omega}\leq 2\
	\end{align*}
	Deshalb ist $(f_n)_{n\in\N}$ eine beschränkte Folge in $H^1(\Omega)$.
	Wegen $H^1(\Omega)\stackrel{C}{\hookrightarrow} L^2(\Omega)$ gibt eine Teilfolge $(f_m)_{m\in\N}$, die in $L^2(H)$ konvergiert.
	\begin{align*}
		&\stackrel{\eqref{proof1.9}}{\implies}
		(f_m)_{m\in\N}\text{ ist eine Cauchy-Folge in }H^1(\Omega)\\
		&\implies
		f_m\stackrel{m\to\infty}{\longrightarrow} f\in H^1(\Omega)\\
		&\implies
		\Vert f\Vert_{0,2,\Omega}=1\text{ und } |f|_{1,2,\Omega}=0\\
		&\implies
		f\text{ ist konstant, also } f\equiv \lambda\cdot g\mit\lambda\in\R
	\end{align*}
	Mit $\Vertiii{f}=0$ erhalten wir
	\begin{align*}
		\Vertiii{f} = 0
		&=|f|_{1,2,\Omega}+|F(f)|+\sqrt{B(f,f)}\\
		&=0+|\lambda|\cdot|F(g)|+|\lambda|\cdot\sqrt{B(g,g)}\\
		&=|\lambda|\cdot\underbrace{\Big(|F(g)|+\sqrt{B(g,g)}\Big)}_{>0}\\
		&\implies\lambda=0\implies f\equiv 0
	\end{align*}
	Das ist ein Widerspruch zur Annahme $\Vert f\Vert_{0,2,\Omega}=1$.
	Also folgt
	\begin{align*}
		\Vert f\Vert_{0,2,\Omega}\leq c\cdot\Vertiii{f}\qquad\forall f\in H^1(\Omega).
	\end{align*}

	\underline{Zeige $\|f\|_{1,2,\Omega}\leq c_1 \Vertiii{f}$}

	\begin{align*}
		\Vert f\Vert_{1,2,\Omega}
		&\leq
		|f|_{1,2,\Omega}+\Vert f\Vert_{0,2,\Omega}\\
		&\leq
		\Vertiii{f}+c\cdot\Vertiii{f}\\
		&=c_1\cdot\Vertiii{f}\qquad\forall f\in H^1(\Omega)
	\end{align*}
\end{proof}

\begin{korollar}[Poincaré-Ungleichung]\label{korollarPoincareUngleichung}\enter %1.12
	Aus Proposition \ref{prop1.11} folgt mit
	\begin{align*}
		B(u,v)=0\quad\text{ und }\quad F(u)=\int\limits_\Omega u(x)\d x
		\qquad\forall u,v\in H^1(\Omega)
	\end{align*}
	die Äquivalenz der Normen
	\begin{align*}
		\Vert\cdot\Vert_{1,2,\Omega}
		\qquad\text{und}\qquad
		|\cdot|_{1,2,\Omega}+|F(\cdot)|.
	\end{align*}
	Setze
	\begin{align*}
		V:=\left\lbrace\varphi\in H^1(\Omega):\int\limits_\Omega u(x)\d x=0\right\rbrace.
	\end{align*}
	Dann sind sogar $\Vert\cdot\Vert_{1,2,\Omega}$ und $|\cdot|_{1,2,\Omega}$ äquivalent auf $V$.
\end{korollar}

\begin{proof}
	Es ist zu zeigen, dass $F$ ein lineares stetiges Funktional auf $H^1(\Omega)$  ist. Das ist einfach.
\end{proof}

\begin{proposition}[Friedrichs Ungleichung]\label{prop1.13FriedrichsUngleichung}\enter
	$|\cdot|_{k,p,\Omega}$ ist eine Norm auf $W_0^{k,p}(\Omega)$, welche äquivalent zu der Norm $\Vert\cdot\Vert_{k,p,\Omega}$ ist.\nl
	Bemerkung: Für $k=1,~p=2$ ist das eine direkte Folgerung auf Proposition \ref{prop1.11} mit
	\begin{align*}
		B(u,v)\equiv0\text{ und } F(u)=\int\limits_{\partial\Omega} u\d \gamma.
	\end{align*}
\end{proposition}