% This work is licensed under the Creative Commons
% Attribution-NonCommercial-ShareAlike 4.0 International License. To view a copy
% of this license, visit http://creativecommons.org/licenses/by-nc-sa/4.0/ or
% send a letter to Creative Commons, PO Box 1866, Mountain View, CA 94042, USA.
% vim: set noexpandtab:

\section{Schwache Lösungen} %3
Hier: skalare, lineare, elliptische partielle Differentialgleichungen zweiter Ordnung, also
\begin{align}\label{eqPDE}
	-\sum\limits_{i,j=1}^d\frac{\partial}{\partial x_i}\left(A_{i,j}(x)\cdot\frac{\partial u}{\partial x_j}\right)+\sum\limits_{i=1}^d a_i(x)\cdot\frac{\partial u}{\partial x_i}+\alpha(x)\cdot u=f(x)\text{ in }\Omega
\end{align}
Vereinfachende Annahmen:
\begin{align*}
	\begin{array}{ll}
		f\in L^2(\Omega)&\alpha \in C(\Omega)\\
		a_i\in C^1(\Omega)~\forall i\in\lbrace1,\ldots,d\rbrace, & A_{i,j}\in C^1(\Omega)~\forall i,j\in\lbrace1,\ldots,d\rbrace\\
		A_{i,j}(x)=A_{j,i}(x)~\forall x\in\Omega, i,j\in\lbrace1,\ldots,d\rbrace, & a=\big(a_1,\ldots, a_d\big)^T\\
		A=\big(A_{i,j}\big)^d_{i,j=1},& 0<\lambda_0:=\inf\limits_{x\in\Omega}\inf\limits_{z\in\R^d\setminus\lbrace0\rbrace}\frac{z^T\cdot A(x)\cdot z}{z^T\cdot z}
	\end{array}
\end{align*}

\subsection*{Einige Spezialfälle}
\begin{itemize}
	\item Poisson-Gleichung: $\alpha\equiv 0,~a\equiv0,~A=I$
	\item Membran-Gleichung: $\alpha\equiv0,~a\equiv 0$
	\item Reaktionsdiffusionsgleichung: $a\equiv 0$
	\item Konvektions-Diffusions-Gleichung: ist der allgemeine Fall
\end{itemize}

\subsection*{Randbedingungen}
\begin{itemize}
	\item Die \textbf{homogene Dirichlet-Randbedingung} ist
	\begin{align*}
		u\equiv0\qquad\text{auf}\qquad\Gamma:=\partial\Omega
	\end{align*}
	\item Die \textbf{Neumann-Randbedingung} ist
	\begin{align*}
		n^T\cdot A\cdot\nabla u\equiv q\qquad\text{auf}\qquad\Gamma,\qquad\forall g\in L^2(\Gamma)
	\end{align*}
	wobei $n$ der äußere Normalenvektor ist.
	\item Mischform: Dirichlet-Randbegingung auf $\Gamma_D$ und Neumann-Randbedingung auf $\Gamma_N$ für
	$\Gamma=\Gamma_D\stackrel{\cdot}{\cup}\Gamma_N,~\Gamma_D\cap\Gamma_N=\emptyset$
	\item Die \textbf{Robin-Randbedingung} ist
	\begin{align*}
		\gamma\cdot u+n^T\cdot A\cdot\nabla u\equiv h\qquad\text{auf}\qquad\Gamma
	\end{align*}
\end{itemize}

Nun bringen wir die allgemeine Form der Gleichung \eqref{eqPDE} in eine lesbare Form.
Sei dazu $v\in C_0^\infty(\Omega)$.
Wir multiplizieren auf beiden Seiten mit v und integrieren anschließend über ganz $\Omega$.
Dann gilt:
\begin{align*}
	&\int\limits_\Omega f\cdot v\d x\\
	\overset{\eqref{eqPDE}}&=
	\int\limits_\Omega-\sum\limits_{i,j=1}^d\frac{\partial}{\partial x_i}\left(A_{i,j}(x)\cdot\frac{\partial u}{\partial x_j}\right)\cdot v+\sum\limits_{i=1}^d a_i(x)\cdot\frac{\partial u}{\partial x_i}\cdot v+\alpha(x)\cdot u\cdot v\d x\\
	&=
	\int\limits_\Omega-\sum\limits_{i,j=1}^d\frac{\partial}{\partial x_i}\left(A_{i,j}(x)\cdot\frac{\partial u}{\partial x_j}\right)\cdot v\d x
	+\int\limits_\Omega\sum\limits_{i=1}^d a_i(x)\cdot\frac{\partial u}{\partial x_i}\cdot v+\alpha(x)\cdot u\cdot v\d x\\
	\overset{\text{partGauß}}&=
	\int\limits_\Omega\sum\limits_{i,j=1}^d A_{i,j}(x)\cdot\frac{\partial u}{\partial x_j}\cdot\frac{\partial v}{\partial x_i}\d x+\int\limits_\Omega\sum\limits_{i=1}^d a_i(x)\cdot\frac{\partial u}{\partial x_i}\cdot v+\alpha(x)\cdot u\cdot v\d x\\
	&\qquad
	-\underbrace{\int\limits_\Gamma\underbrace{\sum\limits_{i,j=1}^d n_i\cdot A_{i,j}(x)\cdot\frac{\partial u}{\partial x_j}}_{=n^T\cdot A\cdot\nabla u}\cdot v\d\gamma}_{=0}\\
	&=
	\int\limits_\Omega\nabla u^T\cdot A\cdot\nabla v+a\cdot\nabla u\cdot v+\alpha\cdot u\cdot v\d x
\end{align*}

Dies ist die \textbf{schwache Formulierung} von \eqref{eqPDE}.

\begin{definition}[Schwache Lösung]\ %3.1
	\begin{enumerate}[label=(\roman*)]
		\item Eine Funktion $u\in H_0^1(\Omega)$ heißt \textbf{schwache Lösung einer Konvektions-\\Diffusions-Gleichung mit homogener Dirichlet-Randbedingung}
		\begin{align*}
			:\Longleftrightarrow
			\int\limits_\Omega\nabla u^T\cdot A\cdot\nabla v+a\cdot\nabla u\cdot v+\alpha\cdot u\cdot v\d x
			=\int\limits_\Omega f\cdot v\d x
			\qquad\forall v\in H_0^1(\Omega)
		\end{align*}
		\item Eine Funktion
		\begin{align*}
			u\in H_D^1(\Omega):=\big\lbrace\varphi\in H^1(\Omega):\varphi|_{\Gamma_D}\equiv 0\big\rbrace
		\end{align*}
		heißt \textbf{schwache Lösung einer Konvektions-Diffusions-Gleichung  mit gemischten Randbedingungen}
		\begin{align*}
			:\Longleftrightarrow
			\int\limits_\Omega \nabla u^T\cdot A\cdot\nabla v+a\cdot\nabla u\cdot v+\alpha\cdot u\cdot v\d x
			=\int\limits_\Omega f\cdot v\d x+\int\limits_{\Gamma_N}g\cdot v\d\gamma\\
			\qquad\forall v\in H^1_D(\Omega)
		\end{align*}
		\item Eine Funktion $u\in H^1(\Omega)$ heißt \textbf{schwache Lösung einer Konvektions-\\Diffusions-Gleichung mit Neumann-Randbedingung}
		\begin{align*}
			:\Longleftrightarrow
			\int\limits_\Omega\nabla u^T\cdot A\cdot\nabla v+a\cdot\nabla u\cdot v+\alpha\cdot u\cdot v\d x
			=\int\limits_\Omega f(x)\cdot v\d x+\int\limits_{\Gamma} g\cdot v\d\gamma\\
			\qquad\forall v\in H^1(\Omega)
		\end{align*}
	\end{enumerate}
\end{definition}

\begin{bemerkung}\ %NoNumber
	\begin{itemize}
		\item Jede "klassische" Lösung ist auch eine schwache Lösung.
		\item Jede schwache Lösung, welche glatt genug ist ($C^2$) ist auch eine "klassische" Lösung.
		\item schwächere Annahmen an die Problemdaten:
		\begin{align*}
			&\alpha\in L^\infty(\Omega), &&\alpha\geq0\\
			&a_i\in L^\infty(\Omega), &&\div(a)=\sum\limits_{i=1}^d\frac{\partial a_i}{\partial x_i}\in L^\infty(\Omega)\\
			&A_{i,j}\in L^\infty(\Omega)
		\end{align*}
		\item Unter der inhomogenen Dirichletbedingung
		\begin{align*}
			u|_\Gamma\equiv u_D
		\end{align*}
		gehört die schwache Lösung zu
		\begin{align*}
			u_D+H_0^1(\Omega)=\big\lbrace v\in H^1(\Omega):v|_\Gamma\equiv u_D\big\rbrace
		\end{align*}
		Der Testraum ist immer $H^1_0(\Omega)$.
	\end{itemize}
\end{bemerkung}

\begin{theorem}[Existenz und Eindeutigkeit von schwachen Lösungen]\
	\begin{enumerate}[label=(\roman*)]
		\item Sei
		\begin{align*}
			\alpha-\frac{1}{2}\cdot\div(a)\geq0.
		\end{align*}
		Dann hat die Konvektions-Diffusions-Gleichung mit homogenen\\ Dirichlet-Randbedingungen eine eindeutige schwache Lösung.
		\item Sei
		\begin{align*}
			\alpha-\frac{1}{2}\cdot\div(a)\geq0\qquad\text{ und }\qquad a\cdot n\geq0\text{ auf }\Gamma_N.
		\end{align*}
		Dann hat die Konvektions-Diffusions-Gleichung mit gemischten Randbedingungen eine eindeutige Lösung.
		\item Sei
		\begin{align*}
			\alpha-\frac{1}{2}\cdot\div(a)\geq0,\qquad \alpha\geq\alpha_0>0\qquad\text{ und }\qquad a\cdot n\geq 0\text{ auf }\Gamma.
		\end{align*}
		Dann hat die Konvektions-Diffusions-Gleichung mit Neumann-Randbedingung eine eindeutige Lösung.
		\item Sei
		\begin{align*}
			\alpha=0,\qquad-\frac{1}{2}\cdot\div(a)=0\qquad\text{und}\qquad a\cdot n\geq 0\text{ auf }\Gamma\\
		\end{align*}
		und gelte
		\begin{align*}
		\int\limits_\Omega f(x)\d x+\int\limits_\Gamma g\d\gamma=0.
	\end{align*}
		Dann hat die Konvektions-Diffusions-Gleichung mit Neumann-Randbedingung eine eindeutige schwache Lösung $u$, welche
		\begin{align*}
			\int\limits_\Omega u(x)\d x=0
		\end{align*}
		erfüllt.
	\end{enumerate}
\end{theorem}

\begin{proof}
	\underline{Zeige (i):}\\
	$V=H_0^1(\Omega)$, $\Vert\cdot\Vert_V=\Vert\cdot\Vert_{1,2,\Omega}\cong\vert\cdot\vert_{1,2,\Omega}$.
	Die Linearform
	\begin{align*}
		l\colon V\to\R,\qquad l(v):=\int\limits_\Omega f\cdot v\d x\qquad\forall v\in V
	\end{align*}		
	 ist stetig, denn es gilt:
	\begin{align*}
		|l(v)|
		\overset{\Def}=
		\left|\int\limits_\Omega f\cdot v\d x\right|
		\overset{\eqref{eqCSW02}}&\leq
		\Vert f\Vert_{0,2}\cdot\Vert v\Vert_{0,2}
		\overset{\Def}\leq
		\underbrace{\Vert f\Vert_{0,2}}_{\overset{\overset{\text{Vor}}{\in}\L^2}{<}\infty,\text{ konst.}}\cdot\Vert v\Vert_{1,2}
	\end{align*}
	Betrachte die Bilinearform
	\begin{align*}
		a(v,w):=\int\limits_\Omega\nabla v^T\cdot A\cdot\nabla w+\big(a\cdot\nabla v+\alpha\cdot v\big)\cdot w\d x\qquad\forall v,w\in V.
	\end{align*}
	Sie ist stetig, denn:
	\begin{align*}
		\big|a(v,w)\big|
		\overset{\Def}&=
		\left|\int\limits_\Omega\nabla v^T\cdot A\cdot\nabla w+\big(a\cdot\nabla v+\alpha\cdot v\big)\cdot w\d x\right|\\
		&\leq
		\Vert A\Vert_{L^\infty}\cdot\Vert\nabla v\Vert_{0,2}\cdot\Vert\nabla w\Vert_{0,2}\\
		&\quad+\Vert a\Vert_{L^\infty}\cdot\Vert\nabla v\Vert_{0,2}\cdot\Vert w\Vert_{0,2}\\
		&\quad+\Vert\alpha\Vert_{L^\infty}\cdot\Vert v\Vert_{0,2}\cdot\Vert w\Vert_{0,2}\\
		&\leq
		\left(\Vert A\Vert_{L^\infty}+\Vert a\Vert_{L^\infty}+\Vert\alpha\Vert_{L^\infty}\right)\cdot\Vert v\Vert_{1,2}\cdot\Vert w\Vert_{1,2}
	\end{align*}
	Koerzivität:
	\begin{align*}
		a(v,v)&\geq\underbrace{\beta}_{>0}\cdot\Vert v\Vert^2_{1,2}\qquad\forall v\in V=H_0^1(\Omega)\\
		a(v,v)&=
		\int\limits_\Omega\nabla v^T\cdot A\cdot\nabla v+\big(a\cdot\nabla v+\alpha\cdot v\big)\cdot v\d x\\
		&=\int\limits_\Omega\nabla v^T\cdot A\cdot\nabla v+\big(a\cdot\nabla v\big) v+\alpha\cdot v^2\d x\\
		&=\int\limits_\Omega\nabla v^T\cdot A\cdot\nabla v+\frac{a}{2}\nabla \left(v^2\right)+\alpha\cdot v^2\d x\\
		&=\int\limits_\Omega\underbrace{\nabla v^T\cdot A\cdot\nabla v}_{\geq\lambda_0\cdot|\nabla v(x)|^2\geq0}+\underbrace{\left(\alpha-\frac{1}{2}\cdot			\div(a)\right)}_{\stackrel{\text{Vor}}{\geq}0}\cdot v^2\d x+\underbrace{\int\limits_{\Gamma}\frac{1}{2}\cdot(a\cdot n)\cdot v^2\d\gamma}_{=0}\\
		&\geq
		\lambda_0\cdot |v|^2_{1,2}\\
		\overset{\text{Fried}}&{\geq}
		\frac{\lambda_0}{C_f^2}\cdot\Vert v\Vert^2_{1,2}
	\end{align*}

	\underline{Zeige (ii):}\\
	Homogene Dirichlet-Randbedingungen auf $\Gamma_D$ und Neumann-Randbedingungen auf $\Gamma_N$.
	\begin{align*}
		\left\{\begin{array}{rl}
			\alpha-\frac{1}{2}\cdot\div(a)\geq0 &\text{ in }\Omega\\
			a\cdot n\geq0&\text{ auf }\Gamma_N
		\end{array}\right.
	\end{align*}
	Also gilt
	\begin{align*}
		l(v)&=\int\limits_\Omega f\cdot v\d x+\int\limits_{\Gamma_N}g\cdot v\d\gamma\\
		\big|l(v)\big|&\leq\underbrace{\Vert f\Vert_{0,2}}_{\leq\Vert v\Vert_{1,2}}\cdot\Vert v\Vert_{0,2}+\Vert g\Vert_{0,2,\Gamma}\cdot\underbrace{\Vert v\Vert_{0,2,\Gamma_N}}_{\leq\Vert v\Vert_{1,2,\Omega}}
	\end{align*}
	Erinnerung:
	\begin{align*}
		H^1_D(\Omega)=\big\lbrace v\in H^1(\Omega):v|_{\Gamma_D}\equiv 0\big\rbrace
	\end{align*}
	Die Abschätzung für $a$ ist analog zu (a), nur ist das letzte Integral nicht Null, sondern:
	\begin{align*}
		\int\limits_\Gamma\frac{1}{2}\cdot(a\cdot n)\cdot v^2\d\gamma
		&=\underbrace{\int\limits_{\Gamma_D}\frac{1}{2}\cdot(a\cdot n)\cdot v^2\d\gamma}_{=0}+\underbrace{\int\limits_{\Gamma_N}\frac{1}{2}\cdot\underbrace{(a\cdot n)}_{\geq0}\cdot v^2\d\gamma}_{\geq0}
	\end{align*}

	\underline{Zeige (iii):}
	\begin{align*}
		\left\{\begin{array}{rl}
			\alpha-\frac{1}{2}\cdot\div(a)\geq0&\text{ in }\Omega\\
			\alpha\geq\alpha_0>0&\text{ in }\Omega\\
			a\cdot n\geq0&\text{ auf }\Gamma
		\end{array}\right.
	\end{align*}
	Wende Theorem \ref{theorem2.4} an:
	\begin{align*}
		X&=H^1(\Omega),\qquad Y=L^2(\Omega)\\
		l(v)&=\int\limits_\Omega f\cdot v\d x+\int\limits_\Gamma g\cdot v\d\gamma\\
		a_0(v,w)&=\int\limits_\Omega\nabla v^T\cdot A\cdot\nabla w+\alpha\cdot v\cdot w\d x\\
		a_1(v,w)&=\int\limits_\Omega(a\cdot\nabla v)\cdot v\d x
	\end{align*}
	Stetigkeit von $l$ und $a_0$ sowie Symmetrie sind klar. Bleibt nur Koerzivität zu zeigen:
	\begin{align*}
		a(v,v)&=\int\limits_\Omega\nabla v^T \cdot A \cdot \nabla v+\underbrace{\alpha}_{\geq\alpha_0>0}\cdot v^2\d x\\
		&\geq\lambda_0\cdot|v|^2_{1,2,\Omega}+\alpha_0\cdot\Vert v\Vert^2_{0,2,\Omega}\\
		&\geq\underbrace{\min\lbrace\lambda_0,\alpha_0\rbrace}_{>0}\cdot\Vert v\Vert^2_{1,2,\Omega}
	\end{align*}
	Stetigkeit von $a_1$:
	\begin{align*}
		\big| a_1(v,w)\big|&\leq\Vert a\Vert_{L^\infty}\cdot\underbrace{|\nabla v|_{0,2,\Omega}}_{\leq\Vert v\Vert_{1,2,\Omega}}\cdot\underbrace{\Vert w		\Vert_{0,2,\Omega}}_{\leq\Vert w\Vert_{1,2,\Omega}}
	\end{align*}
	Es bleibt nur noch zu zeigen, dass
	\begin{align*}
		a(v,v)>0\qquad\forall v\in H^1(\Omega)\setminus\lbrace0\rbrace
	\end{align*}
	Also
	\begin{align*}
		a(v,v)
		&=\int\limits_{\Omega} \nabla v^T\cdot A\cdot \nabla v+\underbrace{(a\cdot\nabla v)\cdot v}_{=\frac{a}{2}\cdot\nabla(v^2)}+\alpha\cdot v^2\d x\\
		&=\int\limits_{\Omega}\nabla v^T\cdot A\cdot\nabla v+\underbrace{\left(\alpha-\frac{1}{2}\cdot\div(a\right)}_{\geq0}\cdot v^2\d x+\int\limits_\Gamma\frac{1}{2}\cdot\underbrace{(a\cdot n)}_{\geq0}\cdot v^2\d\gamma\\
		&\geq\lambda_0\cdot |v|^2_{1,2,\Omega}\\
		&\geq 0\qquad\forall v\in H^1(\Omega)
	\end{align*}
	Aber wir müssen noch den Fall "$=0$" ausschließen. 
	Also ist zu zeigen, dass
	\begin{align*}
		a(v,v)=0\implies v\equiv 0
	\end{align*}
	gilt.
	\begin{align*}
		0&=a(v,v)\geq\underbrace{\lambda_0}_{>0}\cdot|v|^2_{1,2,\Omega}\geq0 \\
		\implies& |v|_{1,2,\Omega}=0 \\
		\implies& v\equiv c\text{ konstant auf }\Omega\\
		0&=a(c,c)=\int\limits_\Omega\underbrace{\nabla c^T}_{=0}\cdot A\cdot\underbrace{\nabla c}_{=0}+(a\cdot\underbrace{\nabla c}_{=0})\cdot c+\alpha\cdot c^2\d x\\
		&=\int\limits_\Omega\alpha\cdot c^2\d x\\
		&\geq\underbrace{\alpha_0}_{>0}\cdot\Vert c\Vert^2_{0,2,\Omega}\\
		&\implies\Vert c\Vert_{0,2,\Omega}=0\\
		&\implies c=0\\
		&\implies v\equiv 0
	\end{align*}

	\underline{Zeige (iv):}
	\begin{align*}
		&\left\{\begin{array}{rl}
			a\cdot n=0&\text{ auf }\Gamma\\
			\alpha=0,~\div(a)=0&\text{ in  }\Omega\\
		\end{array}\right. \\
		&V=\left\lbrace v\in H^1(\Omega):\int\limits_\Omega v(x)\d x=0\right\rbrace\mit|\cdot|_{1,2}\sim\Vert\cdot\Vert_{1,2}\text{ (siehe Poincaré-Ungleichung)}
	\end{align*}
	Also sind die Bilinearform
	\begin{align*}
		a(v,w)=\int\limits_\Omega\nabla v^T\cdot A\cdot\nabla w+(a\cdot\nabla v)\d x
	\end{align*}
	und die Linearform
	\begin{align*}
		l(v)=\int\limits_\Omega f(x)\cdot v(x)\d x+\int\limits_\Gamma g\cdot v\d\gamma
	\end{align*}
	stetig. Zur Koerzivität:
	\begin{align*}
		a(v,v)&=\int\limits_\Omega\nabla v^T\cdot A\cdot\nabla v+(a\cdot\nabla v)\cdot v\d x\\
		&=\int\limits_\Omega\nabla v^T\cdot A\cdot \nabla v-\Big(\frac{1}{2}\cdot\underbrace{\div(a)}_{=0}\Big)\cdot v^2\d x+\int\limits_\Gamma\frac{1}{2}\cdot\underbrace{(a\cdot n)}_{=0}\cdot v^2\d\gamma\\
		&\geq\lambda_0\cdot|v|_{1,2,\Omega}^2\\
		&\geq\frac{\lambda_0}{C^2_p}\cdot\Vert v\Vert^2_{1,2,\Omega}
	\end{align*}
	Aus Lax Milgram \ref{theorem2.1LaxMilgram} folgt, dass das Variationsproblem eine eindeutig Lösung \ul{in $V$} hat.
	Aber: Die schwache Lösung benötigt $X=H^1(\Omega)$.
	Wir wissen, dass das Problem
	\begin{align*}
		\text{Finde $u\in V$ so, dass }a(u,v)=l(v)\qquad\forall v\in V
	\end{align*}
	eine eindeutige Lösung besitzt.
	Es bleibt zu zeigen, dass $a(u,\indi)=l(\indi)$ für eine konstante Funktion $\indi$ gilt.
	\begin{align*}
		a(u,\indi)=l(\indi)
		&\Longleftrightarrow
		\int\limits_\Omega\underbrace{\nabla u^T\cdot A\cdot\nabla\indi}_{=0}+(a\cdot\nabla u)\cdot\indi\d x
		=
		\int\limits_\Omega f(x)\d x+\int\limits_\Gamma g\d\gamma\stackeq{\text{Vor}}0\\
		&\Longleftrightarrow
		a\cdot \nabla u=0\\
		\overset{\text{part. Int.}}&{\Longleftrightarrow}
		0=-\int\limits_\Omega\underbrace{(\div(a))}_{\equiv0}\cdot u\d x+\int\limits_\Gamma\underbrace{(a\cdot n)}_{\equiv 0}\cdot u\d\gamma
	\end{align*}
	Also ist die schwache Lösung in $V$ die schwache Lösung in $H^1$, die wir gesucht haben.
\end{proof}

\begin{theorem}[Regularität]\label{theorem3.3}\enter
	Sei $\Gamma$ eine $C^1$-Mannigfaltigkeit oder $\Omega$ konvex und $f\in L^2(\Omega)$.
	Wenn wir Neumann-Randbedingungen betrachten und wir annehmen, dass es eine Funktion
	\begin{align*}
		u_g\in H^2(\Omega)\mit g=\gamma_{1}(u_g)\overset{\Def}{=}\frac{\partial u_g}{\partial n}
	\end{align*}
	gibt. Dann gilt:
	\begin{align*}
		\Vert u\Vert_{2,2,\Omega}\leq C\cdot\Big(\Vert f\Vert_{0,2,\Omega}+\Vert u\Vert_{0,2}\Big)
	\end{align*}
	wobei $C$ nur von $\Omega, A,a,\alpha$ abhängt (also nicht von $f$).
\end{theorem}