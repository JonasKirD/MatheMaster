% This work is licensed under the Creative Commons
% Attribution-NonCommercial-ShareAlike 4.0 International License. To view a copy
% of this license, visit http://creativecommons.org/licenses/by-nc-sa/4.0/ or
% send a letter to Creative Commons, PO Box 1866, Mountain View, CA 94042, USA.
% vim: set noexpandtab:

\setcounter{chapter}{0}
\renewcommand{\thechapter}{\Alph{chapter}}
\chapter{Anhang}
\setcounter{equation}{1}
\section{partielle Integration mit Satz von Gauß}
Im Beispiel aus der Vorlesung erhalten wir folgende Gleichheit:
\begin{align}
	\int_{\Omega} u_xv_x + u_yv_y \d x\d y = -\int_{\Omega} (u_{xx} + u_{yy})v\d x\d y + \int_{\partial\Omega} u_x v n_x + u_y v n_y \d\gamma
\end{align}
bzw. die \textbf{erste Greensche Formel}, in Kurzschreibweise:
\begin{align}\label{eqGreen}\tag{Green}
	\int\limits_\Omega\nabla u\cdot\nabla v\d x
	=\int\limits_{\partial\Omega}v\cdot\nabla u\cdot n\d S
\end{align}
Dies wird hier noch einmal im Detail erklärt.\nl
Zunächst gehen wir zu einer kürzeren und allgemeineren Notation über. 
Wir schreiben die Summen der partiellen Ableitungen als Skalarprodukt von Gradienten auf, z.B. :
\[u_xv_x + u_yv_y = \nabla u \cdot \nabla v\]
Außerdem bedienen wir uns einer Indentität, die man durch Nachrechnen für passende Funktionen nachweisen kann:
\begin{align}
	\div(fg) = f \cdot \nabla g + g \div(f)
\end{align}
Wir nutzen diese Gleichung spezifisch mit $f=\nabla u$ und $g=v$.
Nun können wir (A.2) beweisen:
\begin{proof}
	\begin{align*}
		\int_{\Omega} u_xv_x + u_yv_y \d x\d y 
		\overset{\text{Notation}}&=
		\int_{\Omega} \nabla u \cdot \nabla v \d x\d y \\
		\overset{\text{A.3}}&=
		\int_{\Omega}\div(v\nabla u) \d x\d y - \int_{\Omega} v \div(\nabla u) \d x\d y \\
		\overset{\text{Gauß}}&=
		\int_{\partial\Omega}v \nabla u \cdot n \d\gamma - \int_{\Omega} v \div(\nabla u) \d x\d y \\
		\overset{\div(\nabla) = \laplace}&=
		\int_{\partial\Omega}v \nabla u \cdot n \d\gamma - \int_{\Omega} v \laplace u \d x\d y \\
		\overset{\text{Def}}&=
		\int_{\partial\Omega} u_x v n_x + u_y v n_y \d\gamma - \int_{\Omega} (u_{xx} + u_{yy})v\d x\d y 
	\end{align*}
\end{proof}

\section{Koflächenformel}
\begin{satz}
	Sei $f:B(0,R)\rightarrow \R$ integrierbar. 
	Dann ist $f$ für fast alle $r>0$ über der Sphäre $S_r := \partial B(0,r)$ integrierbar und es gilt:
	\begin{align*}
		\int_{B(0,R)}f(x) \d\lambda(x) &= \int_0^R \left(\int_{S_r} f(x) \d\gamma(x)\right) \d r \\
																	 &= \int_0^R \left(\int_{S_1} f(rx) \d\gamma(x)\right)r^{n-1} \d r
	\end{align*}
\end{satz}

\begin{proof}
	Siehe Forster \textit{Analysis 2}.
\end{proof}

\section{Cauchy-Schwarz-Ungleichung auf Sobolevräumen}
Es gilt
\begin{align}\label{eqCS}\tag{CS}
	\left|\int\limits_\Omega\nabla u\cdot\nabla v\d x\right|
	&\leq\vert u\vert_{1,2,\Omega}\cdot\vert v\vert_{1,2,\Omega}
	&\forall& u,v\in W_0^{1,2}(\Omega)\\
	\left|\int\limits_\Omega u\cdot v\d x\right|&\leq\Vert u\Vert_{0,2,\Omega}\cdot\Vert v\Vert_{0,2,\Omega}
	&\forall& u,v\in W^{0,2}(\Omega)\label{eqCSW02}\tag{CS}
\end{align}

\begin{proof}
	\underline{Zur ersten Ungleichung:}
	Seien $u,v\colon \Omega\subseteq\R^d\to\R$ aus $W_0^{1,2}(\Omega)$ beliebig.
	Bezeichne $E:=\lbrace e_1,\ldots,e_d\rbrace$ die Standardbasis des $\R^d$.
	Wir wissen aus der Friedrichs-Ungleichung \ref{prop1.13FriedrichsUngleichung}, dass $\big(W_0^{1,2}(\Omega),\vert\cdot\vert_{1,2,\Omega}\big)$ ein normierter Raum ist.
	Man sieht leicht, dass $\big(W_0^{1,2}(\Omega),\langle\cdot,\cdot\rangle_{1,2,\Omega,H}\big)$ ein Hilbertraum ist mit Skalarprodukt
	\begin{align*}
		\langle u,v\rangle_{1,2,\Omega,H}
		:=\sum\limits_{|\alpha|=1}\int\limits_\Omega D^\alpha u\cdot D^\alpha v\d x
	\end{align*}
	Also haben wir eine Cauchy-Schwarz-Ungleichung für die Semi-Norm zur Verfügung und es gilt:
	\begin{align*}
		\left|\int\limits_\Omega\nabla u\cdot\nabla v\d x\right|
		\overset{\Def~\nabla}&{=}		
		\left|\int\limits_\Omega\left(\sum\limits_{i=1}^d\frac{\partial u}{\partial_i}e_i\right)\cdot\left(\sum\limits_{j=1}^d\frac{\partial v}{\partial_j}e_j\right)\d x\right|\\
		&=\Bigg|\int\limits_\Omega\sum\limits_{i,j=1}^d\frac{\partial u}{\partial_i}\frac{\partial v}{\partial_j}\underbrace{e_i\cdot e_j}_{=\delta_{i,j}}\d x\Bigg|\\
		&=\left|\int\limits_\Omega\sum\limits_{i=1}^d\frac{\partial u}{\partial_i}\frac{\partial v}{\partial_i}\d x\right|\\
		\overset{\text{Lin}}&=
		\left|\sum\limits_{i=1}^d\int\limits_\Omega\frac{\partial u}{\partial_i}\frac{\partial v}{\partial_i}\d x\right|\\
		%\overset{\text{DU}}&\leq
		%\sum\limits_{i=1}^d\left|\int\limits_\Omega\frac{\partial u}{\partial_i}\frac{\partial v}{\partial_i}\d x\right|\\
		&=
		\left|\sum\limits_{|\alpha|=1}\int\limits_\Omega D^\alpha u D^\alpha v\d x\right|\qquad(|\alpha|=1\gdw\alpha\in E)\\
		%&\leq
		%\left|\sum\limits_{|\alpha|\leq1}\int\limits_\Omega D^\alpha u D^\alpha v\d x\right|\\
		\overset{\Def}&=
		\big|\langle u,v\rangle_{1,2,\Omega,H}\big|\\
		\overset{\text{CF}}&\leq
		\vert u\vert_{1,2,\Omega}\cdot\vert v\vert_{1,2,\Omega}
	\end{align*}
	Bei CF wird die gewöhnliche Cauchy-Schwartz-Ungleichung benutzt:
	\begin{align}\label{eqCSAllgemein}\tag{CS}
		\big|\langle u,v\rangle\big|_H\leq\Vert u\Vert_H\cdot\Vert v\Vert_H\qquad\forall u,v\in H
	\end{align}
	\underline{Zur zweiten Ungleichung:}
	\begin{align*}
		\left|\int\limits_\Omega u\cdot v\d x\right|
		&=\left|\sum\limits_{|\alpha|\leq 0}\int\limits_\Omega D^\alpha u\cdot D^\alpha v\d x\right|\\
		\overset{\Def}&=
		\Big|\langle u,v\rangle_{0,2,\Omega}\Big|\\
		\overset{\text{CS}}&\leq
		\Vert u\Vert_{0,2,\Omega}\cdot\Vert v\Vert_{0,2,\Omega}
	\end{align*}
\end{proof}

\section{Fixpunktsätze}
\begin{satz}[Banach'scher Fixpunktsatz]\label{BanachscherFixpunktsatz}\enter
	Sei $W$ ein Banachraum, $Q:N\to N$ mit $N\subseteq W$ sei eine Kontraktion, d. h.
	\begin{align*}
		\exists\gamma<1:\forall u,v\in N:\Vert Q(u)-Q(v)\Vert_V\leq\gamma\cdot\Vert u-v\Vert_V.
	\end{align*}
	Dann hat $Q$ genau einen Fixpunkt.
\end{satz}
