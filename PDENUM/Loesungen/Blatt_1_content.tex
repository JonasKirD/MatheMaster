% This work is licensed under the Creative Commons
% Attribution-NonCommercial-ShareAlike 4.0 International License. To view a copy
% of this license, visit http://creativecommons.org/licenses/by-nc-sa/4.0/ or
% send a letter to Creative Commons, PO Box 1866, Mountain View, CA 94042, USA.

\section{Aufgabenblatt 1}
\subsection{Aufgabe 1.1}
Die Funktion
\begin{align*}
	H:(-1,1)\to\R,\qquad H(x):=\left\lbrace\begin{array}{cl}
		0, & \falls x<0\\
		1, & \falls x\geq0
	\end{array}\right.
\end{align*}
besitzt keine schwache Ableitung $f\in L^1((-1,1))$.

\begin{proof}[Beweis. (Kurz)]
	Folgt aus Aufgabe 1.3 da, $H$ nicht stetig ist.
\end{proof}

\begin{proof}[Beweis. (Ausführlich)]
	Offenbar ist die Funktion
	\begin{align*}
		\tilde{H}:\equiv H|_{(-1,0)\cup(0,1)}.
	\end{align*}
	beliebig oft stetig differenzierbar mit $\tilde{H}'\equiv0$. 
	Somit ist die schwache Ableitung von $\tilde{H}$ bereits die Nullfunktion und es gilt nach Definition ($\alpha=(0)$) für alle $\tilde{v}\in C_0^\infty((-1,0)\cup(0,1))$:
	\begin{align*}
		\int\limits_{(-1,0)\cup(0,1)}\tilde{H}(x)\cdot \tilde{v}'(x)\d x=-1\cdot\int\limits_{(-1,0)\cup(0,1)} 0\cdot \tilde{v}\d x=0
	\end{align*}

	Nehmen wir nun an, dass $H$ eine schwache Ableitung $f\in L^1((-1,1))$ besitzt. 
	Dann gilt per Definition ($\alpha:=(0)$) für alle $v\in C_0^\infty((-1,1))$:
	\begin{align*}
		-1\cdot\int\limits_{(-1,1)} f(x)\cdot v(x)\d x
		\overset{\text{Def}}&=
		\int\limits_{(-1,1)} H(x)\cdot v'(x)\d x\\
		\overset{\text{Def }H}&=
		\int\limits_{[0,1)} v'(x)\d x\\
		\overset{\text{HS}}&=
		\lim\limits_{k\uparrow 1} v(k)-v(0)\\
		\overset{\text{komp. T}}&=
		-v(0)\\
		&\implies v(0)=\int\limits_{(-1,1)}f(x)\cdot v(x)\d x
	\end{align*}

	Wählen wir jetzt $v_n\in C^\infty_c((-1,1))$ mit den folgenden Eigenschaften:
	\begin{itemize}
		\item $v_n(x)\in[0,1]\quad \forall x\in (-1,1)$
		\item $v_n(0)=1$
		\item $v_n(x)\stackrel{n\rightarrow\infty}{\longrightarrow}0\quad\forall x\neq 0$
	\end{itemize}
	Dies ist eine Folge von $C^\infty_c((-1,1))$-Funktionen, die sich auf den Nullpunkt zusammenziehen
	und beschränkt ist.
	Dann erhalten wir
	\begin{align*}
		1 &=\lim\limits_{n\rightarrow\infty} v_n(0) \\
		&=\lim\limits_{n\rightarrow\infty}-\int\limits_{(-1,1)} f(x)\cdot v_n(x)\d x \\
		&=-\int\limits_{(-1,1)} f(x)\cdot\lim\limits_{n\rightarrow\infty} v_n(x)\d x \\
		&=-\int\limits_{\{0\}} f(x)\cdot\lim\limits_{n\rightarrow\infty} v_n(x)\d x 
		-\int\limits_{(-1,1)\setminus\{0\}} f(x)\cdot\lim\limits_{n\rightarrow\infty} v_n(x)\d x \\
		&= 0 -\int\limits_{(-1,1)\setminus\{0\}} f(x)\cdot 0\d x \\
		&= 0
	\end{align*}
	Demnach kann $f(x)$ nicht existieren.
\end{proof}

\subsection{Aufgabe 1.2}
Sei $\Omega:=(0,2)$ und 
\begin{align*}
	v(x)&:=\left\lbrace\begin{array}{cl}
		x, & \falls x\in[0,1]\\
		2-x, &\falls x\in(1,2]
	\end{array}\right.\\
	v_n(x)&:=\left\lbrace\begin{array}{cl}
		-n\cdot\frac{(x-1)^2}{2}+1-\frac{1}{2\cdot n}, & \falls x\in\left[1-\frac{1}{n},1+\frac{1}{n}\right]\\
		v(x),&\sonst
	\end{array}\right.\\
\end{align*}
Dann gilt $\limn v_n=v$ in $H^2((0,2))$.

\begin{proof}
	Die Differenz
	\begin{align*}
		v(x)-v_n(x)=\left\lbrace\begin{array}{cl}
			x+n\cdot\frac{(x-1)^2}{2}-1+\frac{1}{2\cdot n}, & \falls x\in\left[1-\frac{1}{n},1\right]\\
			1-x+n\cdot\frac{(x-1)^2}{2}+\frac{1}{2\cdot n}, & \falls x\in\left(1,1+\frac{1}{n}\right]\\
			0,&\sonst
		\end{array}\right.\stackrel{\text{}}{\geq}0\\
	\end{align*}
	hat die schwache Ableitung
	\begin{align*}
		\frac{\d}{\d x}(v(x)-v_n(x))=\left\lbrace\begin{array}{cl}
			1+n\cdot(x-1), & \falls x\in\left[1-\frac{1}{n},1\right]\\
			-1+n\cdot(x-1), & \falls x\in\left(1,1+\frac{1}{n}\right]\\
			0,&\sonst
		\end{array}\right.
	\end{align*}

	\begin{align*}
		&\Vert v-v_n\Vert_{1,2,\Omega}
		\overset{\text{Def}}=
		\Bigg(\underbrace{\int\limits_0^2|v -v_n|^2\d x}_{=:I}+\underbrace{\int\limits_0^2\left|\frac{\d}{\d x} \big(v(x)-v_n(x)\big)\right|^2\d x}_{=:II}\Bigg)^{\frac{1}{2}}\\
		I&=\int\limits_{1-\frac{1}{n}}^1 x+n\cdot\frac{(x-1)^2}{2}-1+\frac{1}{2\cdot n}\d x+
		\int\limits_1^{1+\frac{1}{n}}1-x+n\cdot\frac{(x-1)^2}{2}+\frac{1}{2\cdot n}\d x\\
		&=\left[\frac{x^2}{2}+\frac{n}{2}\cdot\frac{(x-1)^3}{3}-x+\frac{x}{2\cdot n}\right]_{x=1-\frac{1}{n}}^1
		+\left[x-\frac{x^2}{2}+\frac{n}{2}\cdot\frac{(x-1)^3}{3}+\frac{x}{2\cdot n}\right]_{x=1}^{1+\frac{1}{n}}\\
		&=\frac{1}{2}-1+\frac{1}{2\cdot n}-\frac{(1-\frac{1}{n})^2}{2}-\frac{n}{2}\cdot\frac{1}{-3\cdot n^3}+1-\frac{1}{n}-\frac{1-\frac{1}{n}}{2\cdot n}\\
		&~~~+1+\frac{1}{n}-\frac{(1+\frac{1}{n})^2}{n}+\frac{n}{2}\cdot\frac{1}{3\cdot n^3}+\frac{1+\frac{1}{n}}{2\cdot n}-1+\frac{1}{2}-\frac{1}{2\cdot n}\\
		&=\frac{1}{2}-\frac{(1-\frac{1}{n})^2}{2}-\frac{n}{2}\cdot\frac{1}{-3\cdot n^3}-\frac{1-\frac{1}{n}}{2\cdot n}
		-\frac{(1+\frac{1}{n})^2}{2}+\frac{n}{2}\cdot\frac{1}{3\cdot n^3}+\frac{1+\frac{1}{n}}{2\cdot n}+\frac{1}{2}\\
		&=1-\frac{n^2+1}{n^2}+\frac{1}{3\cdot n^2}+\frac{1}{n^2}\\
		&=\frac{1}{3\cdot n^2}
	\end{align*}
	
	\begin{align*}
		II&=\int\limits_{1-\frac{1}{n}}^1 |\underbrace{1+n\cdot(x-1)}_{\geq0}|\d x+
		\int\limits_1^{1+\frac{1}{n}}|\underbrace{-1+n\cdot(x-1)}_{\leq0}|\d x\\
		&=\int\limits_{1-\frac{1}{n}}^1 1+n\cdot(x-1)\d x+
		\int\limits_1^{1+\frac{1}{n}}1-n\cdot(x-1)\d x\\
		&=\left[x+n\cdot\left(\frac{x^2}{2}-x\right)\right]_{x=1-\frac{1}{n}}^1+\left[x-n\cdot\left(\frac{x^2}{2}-x\right)\right]_{x=1}^{1+\frac{1}{n}}\\
		&=1+n\cdot\left(\frac{1}{2}-1\right)-1+\frac{1}{n}-n\cdot\left(\frac{\left(1-\frac{1}{n}\right)^2}{2}-1+\frac{1}{n}\right)\\
		&~~~+1+\frac{1}{n}-n\cdot\left(\frac{\left(1+\frac{1}{n}\right)^2}{2}-1-\frac{1}{n}\right)-1+n\cdot\left(\frac{1}{2}-1\right)\\
		%&=-n+\frac{2}{n}-n\cdot\left(\frac{\left(1-\frac{1}{n}\right)^2}{2}-1+\frac{1}{n}\right)
		%-n\cdot\left(\frac{\left(1+\frac{1}{n}\right)^2}{2}-\frac{1}{n}\right)\\
		\overset{\text{CAS}}&=
		\frac{1}{n}\\
		&\implies\Vert v-v_n\Vert_{1,2,\Omega}=\sqrt{\frac{1}{3\cdot n^2}+\frac{1}{n}}
		\stackrel{n\to\infty}{\longrightarrow}0
	\end{align*}
\end{proof}

\section*{Aufgabe 1.3}
Seien $\Omega_1,\Omega_2\subseteq\Omega$ zwei nichtleere, offene, beschränkte und disjunkte Teilmengen von $\Omega$ mit stückweise glattem Rand und $\overline{\Omega}=\overline{\Omega_1}\cup\overline{\Omega_2}$. 
Weiter sei $\varphi\in L^p(\Omega)$ so, dass $\varphi|_{\Omega_i}\in C^1(\Omega_i),~i\in\lbrace1,2\rbrace$ gilt. 
Dann gilt:
\begin{align*}
	\varphi\in W^{1,p}(\Omega)\Longleftrightarrow\varphi\in C(\Omega)
\end{align*}

\begin{proof}
	\begin{align*}
		\Psi^{(i)}|_{\Omega_i}:=\partial\left(\varphi|_{\Omega_i}\right)
	\end{align*}
	Sei $\vec{n}$ der Normalenvektor an $\Omega_1$ und setze
	\begin{align*}
		\Gamma&:=\partial\Omega_1\cap\partial\Omega_2\\
		[\varphi](\underbrace{\gamma}_{\in\Gamma}):=\lim\limits_{t\searrow 0}\varphi(\underbrace{x+t\cdot n}_{\in\Omega_2})-\lim\limits_{t\searrow0}\varphi(\underbrace{x-t\cdot n}_{\in\Omega_1})
	\end{align*}
	Sei $v\in C_0^\infty(\Omega)$ Testfunktion.
	\begin{align*}
		-\int\limits_\Omega\varphi\frac{\partial v}{\partial x_i}\d x
		&=-\int\limits_{\Omega_1}\varphi\cdot\frac{\partial v}{\partial x_i}\d x
		-\int\limits_{\Omega_2}\varphi\cdot\frac{\partial v}{\partial x_i}\d x\\
		&=-\int\limits_{\partial\Omega_1}\varphi v\cdot n_i\d S+\int\limits_{\Omega_1}\frac{\partial\varphi}{\partial x_i}v\d x
		-\int\limits_{\Gamma}\varphi v\cdot(-n_i)\d S+\int\limits_{\Omega_2}\frac{\partial\varphi}{\partial x_i}v\d x\\
		&=\int\limits_\Omega\frac{\partial\varphi}{\partial x_i}\d x-\int\limits_\Gamma[\varphi] v n_i\d S\qquad\forall v\in C_0^\infty(\Omega)
	\end{align*}

	\underline{Zeige ``$\Rightarrow$'':}\\
	Wähle $n:=\begin{pmatrix}
		1\\ 
		0\\
		0
	\end{pmatrix}$. 
	Dann gilt:
	\begin{align*}
		\int\limits_\Gamma[\varphi] v n_i\d S=0\implies[\varphi]\equiv0
	\end{align*}

	\underline{Zeige ``$\Leftarrow$'':}
	\begin{align*}
		&\implies[\varphi]\equiv0\\
		&\implies -\int\limits_\Omega\varphi\cdot\frac{\partial v}{\partial x_i}
		=\int\limits \frac{\partial\varphi}{\partial x}v\d x\\
		&\implies\text{ Definition der schwachen Ableitung}
	\end{align*}
\end{proof}

\subsection{Aufgabe 1.4}
Sei $u_s:[0,1]\to\R\mit u_s(x):=x^s\cdot(1-x)^s,~s>0$. 
Dann gilt:
\begin{enumerate}[label=(\alph*)]
	\item $\forall s>0:u_s\in C^0([0,1])$ (also stetig)
	\item $u_s\in H_0^1([0,1])\Longleftrightarrow s>\frac{1}{2}$
\end{enumerate}

\begin{proof}
	\underline{Zeige (a):}\\
	Klar wegen $s>0$.\nl
	\underline{Zeige (b):}\\
	$L^2$-Norm:
	\begin{align*}
		\int\limits_0^1 (u_s(x))^2\d x=\int\limits x^{2\cdot s}\cdot\underbrace{(1-x)}_{\leq1}^{2\cdot s}\d x
		\leq\int\limits_0^1 x^{2\cdot s}\d x
		=\frac{1}{2\cdot s+1}\cdot x^{2\cdot s+1}|_{x=0}^1=\frac{1}{2\cdot s+1}<\infty
	\end{align*}
	$H^1$-Halbnorm:
	\begin{align*}
		u_s'(x)&=s\cdot x^{s-1}\cdot(1-x)^s--s\cdot x^s\cdot(1-x)^{s-1}\\
		&=s\cdot x^{s-1}\cdot(1-x)^{s-1}\cdot(a-2\cdot x)\\
		\implies
		\int\limits_0^1(u_s'(x))^2\d x
		&=\int\limits_0^1 s^2\cdot x^{2\cdot s-2}\cdot(1-x)^{2\cdot s-2}\cdot (1-2\cdot x)^2\d x\\
		\overset{t=x-\frac{1}{2}}&=
		\int\limits_{-\frac{1}{2}}^{\frac{1}{2}} s^2\cdot\left(\frac{1}{2}+t\right)^{2\cdot s-2}\cdot\left(\frac{1}{2}-t\right)^{2\cdot s-2}\cdot(-2\cdot t)^2\d t\\
		\overset{\text{Sym}}&=
		2\cdot\int\limits_{0}^{\frac{1}{2}}s^2\cdot\left(\frac{1}{2}+t\right)^{2\cdot -2}\cdot\left(\frac{1}{2}-t\right)^{2\cdot s-2}\cdot 4\cdot t^2\d t\\
		&=8\cdot s^2\cdot\int\limits_0^{\frac{1}{2}}\left(\frac{1}{4}-t^2\right)^{2\cdot s-2}\cdot \underbrace{t^2}_{\leq t}\d t
	\end{align*}
	Für $s>\frac{1}{2}$ gilt
	\begin{align*}
		\int\limits_0^{\frac{1}{2}}\big( u_s'(x)\big)^2\d x
		&\leq
		8\cdot s^2\cdot\int\limits_0^{\frac{1}{2}}\left(\frac{1}{4}-t^2\right)^{2\cdot s-2}\cdot t\d t\\
		\overset{v=\frac{1}{4}-t^2}&=
		4\cdot s^2\cdot\int\limits_0^{\frac{1}{4}} v^{2\cdot s-2}\d v\\
		&=\frac{s^2}{2\cdot s-1}\cdot\frac{1}{4^{2\cdot s-2}}<\infty
	\end{align*}
	Für $s=\frac{1}{2}$ gilt:
	\begin{align*}
		\int\limits_0^1\big(u_s'(x)\big)^2\d x=8\cdot s^2\cdot\int\limits_0^{\frac{1}{2}}\frac{t^2}{\frac{1}{4}-t^2}\d t=\infty
	\end{align*}

	%TODO Bitte die folgenden beiden Zeilen im Beweis oben ergänzen um die Lesbarkeit zu erhöhen.\\
	%\underline{Zeige ``$\Rightarrow$'':}\\
	%\underline{Zeige ``$\Leftarrow$'':}\\
\end{proof}

\subsection{Aufgabe 1.5}
Sei $v\in H^1((0,1))$ gegeben. 
Geben Sie je einen expliziten Spuroperator für $v(0)$ und $v(1)$ an, der mit Hilfe von Integralen über $v$ und $v'$ geschrieben wird.

\begin{proof}
	%TODO
\end{proof}

\subsection{Aufgabe 1.6 (Friedrichs Ungleichung)}
Sei $\Omega\subseteq Q\subseteq\R^d\mit Q:=I_1\times\ldots\times I_d$ mit minimaler Seitenlänge $l_{\min}$, d.h.
\begin{align*}
	l_{\min}=\min\limits_{i\in\lbrace1,\ldots,d\rbrace}\int\limits_{I_i} 1\d t.
\end{align*}
Dann gilt:
\begin{align*}
	\forall v\in W_0^{1,p}(\Omega):\Vert v\Vert_{0,p,\Omega}\leq l_{\min}\cdot|v|_{1,p,\Omega}
\end{align*}

\begin{proof}
	Sei $v\in C_0^\infty(\Omega)$. 
	Wir wissen bereits, dass $C_0\infty(\Omega)$ dicht in $W_0^{1,p}(\Omega)$ liegt.\\
	Sei O.B.d.A $l_{\min}=\int\limits_{I_1}1\d t\mit I_1=(\tilde{x}_1^l,\tilde{x}_1^r)$ (sonst Umordnen). 
	Dann gilt:
	\begin{align*}
		v(x)&=\underbrace{v\left(\tilde{x}_1^l, x_2,\ldots,x_d\right)}_{=0}+
		\int\limits_{\tilde{x}_1^l}^{x_1}\frac{\partial v}{\partial x_1}(y,x_2,\ldots,x_d)\d y
	\end{align*}
	
	\begin{align*}
		|v(x)|^p&\leq\left(\int\limits_{\tilde{x}_1^l}^{x_1}\left|\frac{\partial v}{\partial x_1}\left(y,x_2,\ldots,x_d\right)\right|\d y\right)^p\\
		&\leq
		\left(\int\limits_{I_1}\left|\frac{\partial v}{\partial x_1}\left(y,x_2,\ldots,x_d\right)\right|\d y\right)^p\\
		\overset{\text{Hölder}}&{\leq}
		\left(\int\limits_{I_1} 1^q\d x\right)^{\frac{p}{q}}\cdot\left(\int\limits_{I_1}\left|\frac{\partial v}{\partial x_1}\left(y,x_2,\ldots,x_d\right)\right|\d y\right)^p
	\end{align*}

	\begin{align*}
		\Vert v\Vert_{0,p,\Omega}^p
		\overset{v=0\text{ auf }Q\setminus\Omega}&=
		\Vert v\Vert^p_{0,p,Q}\\
		&=\int\limits_Q\left|v(x)\right|^p\d x\\
		&=\int\limits_{I_1}\hdots\int\limits_{I_d} |v(x)|^p\d x_d,\ldots,x_1\\
		&\leq
		\int\limits_{I_1}\underbrace{\int\limits_{I_2}\hdots\int\limits_{I_d}l_1^{p-1}\int\limits_{I_1}\left|\frac{\partial v}{\partial x_1}(y,x_2,\ldots,x_d)\right|^p\d y\d x_d\ldots\d x_2}_{=l_1^{p-1}\cdot\left\Vert\frac{\partial v}{\partial x_1}\right\Vert^p_{0,p,Q}}\d x_1\\
		&\leq \int\limits_{I_1} l_1^{p-1}\cdot\left\Vert\frac{\partial v}{\partial x_1}\right\Vert^p_{0,p,\Omega}\d x_1\\
		&=l_1^P\cdot\left\Vert\frac{\partial v}{\partial x_1}\right\Vert^P_{0,p,\Omega}
	\end{align*}
\end{proof}
