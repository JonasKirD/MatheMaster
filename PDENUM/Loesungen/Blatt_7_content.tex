% This work is licensed under the Creative Commons
% Attribution-NonCommercial-ShareAlike 4.0 International License. To view a copy
% of this license, visit http://creativecommons.org/licenses/by-nc-sa/4.0/ or
% send a letter to Creative Commons, PO Box 1866, Mountain View, CA 94042, USA.

\section{Aufgabenblatt 7}
\subsection*{Aufgabe 6.3}
Betrachte
\begin{align*}
	-\varepsilon\Delta u+b\cdot\nabla u+cu=f\text{ in }u=0\text{ auf }\partial\Omega
\end{align*}
mit $0<\varepsilon<<1$ unter der Annahme
\begin{align*}
	c-\frac{1}{2}\div(b)\geq c_0>0\text{ in }\Omega
\end{align*}
Weiter untersuchen wir das Standard-Galerkin-FEM Verfahren auf einer regulären, affin äquivalenten Zerlegung von $\Omega$. 
Es ist bekannt, dass die entstehende Bilinearform $a$ koerziv auf $H_0^(\Omega)$ bezüglich der Norm
\begin{align*}
	\Vert\cdot\Vert:=\big(\varepsilon|\cdot|_1^2+\Vert\cdot\Vert_0^2\big)^{\frac{1}{2}}
\end{align*}
ist.
\subsubsection*{Aufgabe 6.3 (a)}
Warum ist auf dem Standardweg (mit Hilfe des Céa-Lemmas) in diesem Fall keine von $\varepsilon$ unabhängige Fehlerschätzung in der $\Vert\cdot\Vert_\varepsilon$-Norm möglich?

\begin{lösung}
	%TODO
\end{lösung}

\subsubsection*{Aufgabe 6.3 (b)}
Beweisen Sie stattdessen Term für Term für lineare Elemente eine Fehlerabschätzung der Form
\begin{align*}
	\Vert u-u_h\Vert_\varepsilon\leq C\left(\varepsilon^{\frac{1}{2}}h+h+h^2\right)|u|_2.
\end{align*}

\begin{proof}
	Wir nutzen die Galerkin-Orthogonalität.
	%TODO
\end{proof}

\subsection{Aufgabe 7.1}
Sei $\Omega\subseteq\R^d$ ein beschränktes Gebiet mit Lipschitz-Rand $\partial\Omega$.
Die primale gemischte Methode für die Poisson-Gleichung
\begin{align*}
	-\Delta u=-\div(\grad((u))=f\text{ in }\Omega
\end{align*}
mit homogenen Dirichlet-Randbedingungen $u=0$ auf $\partial\Omega$ ist gegeben durch:\nl
Finde $(\sigma,u)\in L^2(\Omega)^d\times H_0^1(\Omega)$ so, dass
\begin{align*}
	\begin{array}{rll}
		(\sigma,\tau)_{0,\Omega}-(\tau,\nabla u)_{0,\Omega}&=0 &\forall\tau\in L^2(\Omega)^d\\
		-(\sigma,\nabla v)_{0,\Omega}&=-(f,v)_{0,\Omega} &\forall v\in H_0^1(\Omega)
	\end{array}
\end{align*}

\subsubsection{Aufgabe 7.1 (a)}
Leiten Sie die Formulierung im Detail her.
Schreiben Sie dazu die Poisson-Gleichung zunächst durch Einführung einer zweiten Variable $\sigma=\nabla u$ in ein System partieller Differentialgleichungen erster Ordnung um.

\begin{lösung}
	%TODO
\end{lösung}

\subsubsection{Aufgabe 7.1 (b)}
Obiges gemischtes Problem liefert ein stabiles Sattelpunktproblem.

\begin{proof}
	%TODO
\end{proof}

\subsection{Aufgabe 7.2}
Sei $\Omega\subseteq\R^d$ ein beschränktes Gebiet mit Lipschitz-Rand $\partial\Omega$.
Die duale gemischte Methode für die Poisson-Gleichung 
\begin{align*}
	-\Delta u=-\div(\grad((u))=f\text{ in }\Omega
\end{align*}
mit homogenen Dirichlet-Randbedingungen $u=0$ auf $\partial\Omega$ ist gegeben durch:\nl
Finde $(\sigma,u)\in H(\div,\Omega)\times L^2(\Omega)$: so, dass
\begin{align*}
	\begin{array}{rll}
		(\sigma,\tau)_{0,\Omega}+(\div(\tau),u)_{0,\Omega}&=0&\forall\tau\in H(\div,\Omega)\\
		(\div(\sigma),v)_{0,\Omega} &=-(f,v)_{0,\Omega} &\forall v\in L^2(\Omega)
	\end{array}
\end{align*}
Der verwendete Raum
\begin{align*}
	H(\div,\Omega):=\left\lbrace\tau\in L^2(\Omega)^d:\div(\tau)\in L^2(\Omega)\right\rbrace
\end{align*}
kann auch als Vervollständigung von $C^\infty(\Omega)^d$ bzgl. der Norm 
\begin{align*}
	\Vert v\Vert_{H(\div,\Omega)}:=\left(\Vert v\Vert_{0,\Omega}^2+\Vert\div(v)\Vert_{0,\Omega}^2\right)^{\frac{1}{2}}
\end{align*}
definiert werden.

\subsubsection{Aufgabe 7.2 (a)}
Auf den ersten Blick schein $u$ nur in $L^2(\Omega)$ zu liegen.
Zeigen Sie, dass tatsächlich aber $u\in H_0^1(\Omega)$ ist.
Was fällt bei den Randbedingungen auf?

\begin{proof}
	%TODO
\end{proof}

\subsubsection{Aufgabe 7.2 (b)}
Obige gemischte Formulierung liefert ein stabiles Sattelpunktsproblem.

\begin{proof}
	%TODO
\end{proof}
