% This work is licensed under the Creative Commons
% Attribution-NonCommercial-ShareAlike 4.0 International License. To view a copy
% of this license, visit http://creativecommons.org/licenses/by-nc-sa/4.0/ or
% send a letter to Creative Commons, PO Box 1866, Mountain View, CA 94042, USA.

\section{Aufgabenblatt 3}
\subsection{Aufgabe 3.1}
Gegeben sei die Differentialgleichung 
\begin{align*}
	\frac{\d^4 u(x)}{\d x^4}=f(x)\qquad\forall x\in(0,1)
\end{align*}
mit den Randbedingungen
\begin{enumerate}[label=(\alph*)]
	\item $\begin{aligned}
		u(0)=u'(0)=u(1)=u'(1)=0
	\end{aligned}$
	\item $\begin{aligned}
		u(0)=u''(0)=u(1)=u''(1)=0
	\end{aligned}$
	\item $\begin{aligned}
		u(0)=u''(0)=u'(1)=u'''(1)=0
	\end{aligned}$
\end{enumerate}
Geben Sie jeweils eine schwache Formulierung an.

\begin{lösung}
	\underline{Zu (a):}\\
	Multiplikation mit Testfunktion $v$ und Integration über $\Omega$ liefert
	\begin{align*}
		\int\limits_0^1 \frac{\d^4 u(x)}{\d x^4}\cdot v\d x=\int\limits_0^1 f(x)\cdot v(x)\d x
	\end{align*}
	mit $v\in C_c^\infty\big((0,1)\big)$. Partielle Integration liefert
	\begin{align*}
		\left[\frac{\d^3 u(x)}{\d x^3}\cdot v\right]_{x=0}^1
		-\int\limits_0^1 \frac{\d^3 u(x)}{\d x^3}\cdot\frac{\d v}{\d x}\d x=\int\limits_0^1 f\cdot v\d x\\
		\left[\frac{\d^3 u(x)}{\d x^3}\cdot v\right]_{x=0}^1 -
		\left[\frac{\d^2 u(x)}{\d x^2}\cdot \frac{\d v(x)}{\d x}\right]_{x=0}^1
		+\int\limits_0^1 \frac{\d^2 u(x)}{\d x^2}\cdot\frac{\d^2 v}{\d x^2}\d x=\int\limits_0^1 f\cdot v\d x\\
	\end{align*}
	Für $v$ gelten dieselben Randbedingungen, wie für $u$, da beide aus dem selben Raum $V$ kommen müssen, den es noch zu bestimmen gilt. 
	Dementsprechend, sind die beiden Randintegrale Null und wir erhalten
	\begin{align*}
		\underbrace{\left[\frac{\d^3 u(x)}{\d x^3}\cdot v\right]_{x=0}^1}_{=0} -
		\underbrace{\left[\frac{\d^2 u(x)}{\d x^2}\cdot \frac{\d v(x)}{\d x}\right]_{x=0}^1}_{=0}
		+\int\limits_0^1 \frac{\d^2 u(x)}{\d x^2}\cdot\frac{\d^2 v}{\d x^2}\d x=\int\limits_0^1 f\cdot v\d x\\
	\end{align*}
	Wir hören an dieser Stelle auf partiell zu integrieren, da die beiden Funktionen unter dem Integral minimale Anforderungen an den Raum $V$ stellen.
	Nach weiterer Betrachtung ergibt sich $H^2_0((0,1))$ als geeignetes $V$ und wir erhalten die Problemstellung\nl
	Finde $u\in H^2_0\big((0,1)\big)$ so, dass
	\begin{align*}
		\int\limits_0^1 \frac{\d^2 u(x)}{\d x^2}\cdot\frac{\d^2 v(x)}{\d x^2}\d x=\int\limits_0^1 f\cdot v\d x\qquad\forall v\in H_0^2\big((0,1)\big)
	\end{align*}

	\underline{Zu (b):}
	\begin{align*}
		&\int\limits_0^1 \frac{\d^4 u(x)}{\d x^4}\d x\\
		\overset{\text{p.I.}}&=
		\left[\frac{\d^3 u(x)}{\d x^3}\right]_{x=0}^1-\int\limits_0^1 \frac{\d^3 u(x)}{\d x^3}\cdot\frac{\d v(x)}{\d x}\d x\\
		\overset{\text{p.I.}}&=
		\left[\frac{\d^3 u(x)}{\d x^3}\cdot v(x)\right]_{x=0}^1- 
		\left[\frac{\d^2 u(x)}{\d x^2}\cdot\frac{\d v(x)}{\d x}\right]_{x=0}^1
		+\int\limits_0^1 \frac{\d^2 u(x)}{\d x^2}\cdot\frac{\d^2 v(x)}{\d x^2}\d x\\
	\end{align*}
	Finde $u\in H^2(0,1)\cap H^1_0(\Omega)$ so, dass
	\begin{align*}
		\int\limits_0^1 \frac{\d^2 u(x)}{\d x^2}\cdot\frac{\d^2 v(x)}{\d x^2}\d x=\int\limits_0^1 f(x)\cdot v(x)\d x\qquad\forall v\in H^1((0,1))\cap H^1_0(\Omega)
	\end{align*}
	Angenommen $u\in C^4(\Omega)$. Dann gilt
	\begin{align*}
		&\int\limits_0^1 \frac{\d^4 u(x)}{\d x^4}\cdot v\d x-\left[\frac{\d^3 u(x)}{\d x^3}\cdot v(x)\right]_{x=0}^1+\left[\frac{\d^2 u(x)}{\d x^2}\cdot\frac{\d v(x)}{\d x}\right]_{x=0}^1=\int\limits_0^1 f(x)\cdot v(x)\d x
		\\&\qquad\forall v\in H^2((0,1))\cap H_0^1((0,1))\\
		&\int\limits_0^1 \frac{\d^4 u(x)}{\d x^4}\cdot v\d x
		=\int\limits_0^1 f(x)\cdot v(x)\d x\qquad\forall v\in C_c^\infty(0,1)
	\end{align*}
	Mit dem \textit{Fundamentallemma} folgt:
	\begin{align*}
		%\overset{\text{Fundamentallemma}}&{\implies}
		&\frac{\d^4 u(x)}{\d x^4}\equiv f\text{ in }(0,1)\\
		&\Rightarrow
		\left(\frac{\d^2 u(x)}{\d x^2}\cdot\frac{\d v(x)}{\d x}\right)(1)-\left(\frac{\d^2 u(x)}{\d x^2}\cdot\frac{\d v(x)}{\d x}\right)(0)=0~\forall v\in H^2((0,1))\cap H^1_0((0,1))
	\end{align*}

	\underline{Zu (c):}\\
	Finde
	\begin{align*}
		u\in V:=\big\lbrace w\in H^2(\Omega):w(0)=w'(1)=0\big\rbrace
	\end{align*}
	so, dass
	\begin{align*}
		\int\limits_0^1 f(x)\cdot v(x)\d x=\int\limits_0^1\frac{\d^2 u(x)}{\d x^2}\cdot\frac{\d^2 v(x)}{\d x^2}\d x\qquad\forall v\in V.
	\end{align*}
\end{lösung}

\subsection{Aufgabe 3.2}
Sei $\Omega=(0,1)^2$. Wie sieht eine schwache Formulierung des Problems
\begin{align*}
	-5\cdot\frac{\partial^2 u(x,y)}{\partial x^2}+6\cdot\frac{\partial^2 u(x,y)}{\partial x\partial y}-8\cdot\frac{\partial^2 u(x)}{\partial y^2}+4\cdot\frac{\partial u(x,y)}{\partial y}+u(x,y)&=f(x,y) \text{ in }\Omega\\
	u&\equiv 0\text{ auf }\partial\Omega
\end{align*}
aus?

\begin{lösung}
	Sei $A=(a_{i,j})_{i,j\in\lbrace1,2\rbrace}\in\R^{2\times 2}$. 
	Dann gilt:
	\begin{align*}
		A\cdot\nabla u&=\begin{pmatrix}
			a_{1,1} & a_{1,2}\\
			a_{2,1} & a_{2,2}
		\end{pmatrix}\cdot\begin{pmatrix}
			\frac{\partial u}{\partial x}\\
			\frac{\partial u}{\partial y}
		\end{pmatrix}=\begin{pmatrix}
			a_{1,1}\cdot\frac{\partial u}{\partial x}+a_{1,2}\cdot\frac{\partial u}{\partial y}\\
			a_{2,1}\cdot\frac{\partial u}{\partial x}+a_{2,2}\cdot\frac{\partial u}{\partial y}
		\end{pmatrix}\\
		\implies
		\div(A\cdot\nabla u)&=\frac{\partial}{\partial x}\left(a_{1,1}\cdot\frac{\partial u}{\partial x}+a_{1,2}\cdot\frac{\partial u}{\partial y}\right)+\frac{\partial}{\partial y}\left(a_{2,1}\cdot\frac{\partial u}{\partial x}+a_{2,2}\cdot\frac{\partial u}{\partial y}\right)\\
		&=a_{1,1}\cdot\frac{\partial^2 u}{\partial x^2}+\big(a_{1,2}+a_{2,1}\big)\cdot\frac{\partial^2 u}{\partial x\partial y}+a_{2,2}\cdot\frac{\partial^2 u}{\partial y^2}
	\end{align*}
	Setze also 
	\begin{align*}
		A:=\begin{pmatrix}
			5 & -3\\
			-3 & 8
		\end{pmatrix}.
	\end{align*}
	Dann lässt sich die Differentialgleichung des gegebenen Problems umschreiben zu 
	\begin{align*}
		-\div\big(A\cdot\nabla u(x,y)\big)+4\cdot\frac{\partial u(x,y)}{\partial y}+u(x,y)=f(x,y)
	\end{align*}
	Nun Multiplizieren wir dies mit einer Testfunktion $v$ und integrieren über $\Omega$:
	\begin{align*}
		\int\limits_\Omega\Big(-\div\big(A\cdot\nabla u(x,y)\big)+4\cdot\frac{\partial u(x,y)}{\partial y}+u(x,y)\Big)\cdot v(x)\d x=\int\limits_\Omega f(x,y)\cdot v(x)\d x
	\end{align*}
\end{lösung}

\subsection{Aufgabe 3.3}
Das Problem 
\begin{align*}
	-\Delta u=1\text{ in }\Omega,\qquad\frac{\partial u}{\partial n}=0\text{ auf }\partial\Omega
\end{align*}
hat keine Lösung. 
Welche Bedingung müsste erfüllt sein, damit eine Lösung existiert?

\begin{proof}
	Lösung siehe
	\url{http://www.math.tu-dresden.de/~sbecher/pdenm/}
\end{proof}

\subsection{Aufgabe 3.4}
Wie lautet eine geeignete schache Formulierung der Randwertaufgabe
\begin{align*}
	-\Delta u+a\cdot\nabla u+\alpha\cdot u\equiv f\text{ in }\Omega,\qquad\frac{\partial u}{\partial n}+\beta\cdot u\equiv g\text{ auf } \partial\Omega
\end{align*}
Geben Sie eine hinreichende Bedingung an, unter denen die eindeutige Lösbarkeit der schwachen Formulierung gesichert ist und beweisen Sie, dass Ihre Annahmen genügen.

\begin{lösung}
	Zuerst werden beide Seiten mit einer Testfunktion $v$ multipliziert. 
	Anschließend wird über $\Omega$ integriert:
	\begin{align*}
		\int\limits_\Omega -\nabla u\cdot v+a\cdot\nabla u\cdot v+\alpha\cdot u\cdot v\d x=\int\limits_\Omega f\cdot v\d x
	\end{align*}
	Mit dem Satz von Gauß 
	\begin{align*}
		\int\limits_\Omega\div(v)\d x=-\int\limits_{\partial\Omega} v\cdot n\d s
	\end{align*}
	und mit
	\begin{align*}
		\Delta u=\div(\nabla u)
	\end{align*}
	erhält man:
	Sei $\lambda:\R^d\to\R$ eine skalare Funktion und $w:\R^d\to\R^d$ Vektorfeld. Dann gilt:
	\begin{align*}
		\div(\lambda w)=(\nabla\lambda)\cdot w+\lambda\cdot\div(w)\\
		\implies
		\lambda\cdot\div(w)=\div(\lambda w)-(\nabla\lambda)\cdot w
	\end{align*}
	Nun wenden wir dies auf unsere PDE an
	\begin{align*}
		\underbrace{\int\limits_\Omega -\laplace u\cdot v}_{=-\int\limits_\Omega\div(\underbrace{\nabla u}_{\hat{=}w})\underbrace{v}_{=\lambda}}+a\cdot\nabla u\cdot v+\alpha\cdot u\cdot v\d x
		&=\int\limits_\Omega f\cdot v \d x
	\end{align*}
	Wir betrachten hier zunächst den ersten Summanden näher
	\begin{align*}
		\int\limits_\Omega -\laplace u\cdot v\d x
		\overset{\text{Def }\laplace}&=
		\int\limits_\Omega - \div(\nabla u) \cdot v \d x \\
 		\overset{\text{part. I.}}&=
 		\int\limits_\Omega\nabla u\cdot\nabla v\d x-\int\limits_{\Omega}\div(\nabla u \cdot v)\\
		\overset{\text{Gauß}}&=
		\int\limits_\Omega\nabla u\cdot\nabla v\d x-\int\limits_{\partial\Omega}\underbrace{(\nabla u\cdot n)}_{=g-\beta\cdot u\text{ laut RB}}\\
		\overset{\text{RB}}&=
		\int\limits_\Omega\nabla u\cdot\nabla v\d x-\int\limits_{\partial\Omega} g\cdot v\d s+\int\limits_{\partial\Omega}\beta\cdot u\cdot v\d s
	\end{align*}
	Setzen wir dies für den ersten Summanden ein und ordnen die resultierende Gleichung etwas um, so erhalten wir
	\begin{align*}
		\underbrace{\int\limits_\Omega\nabla u\cdot\nabla v+a\cdot\nabla u\cdot v+\alpha\cdot u\cdot v\d x+\int\limits_{\partial\Omega}\beta\cdot u\cdot v\d s}_{=:b(u,v)}
		=\underbrace{\int\limits_\Omega f\cdot v\d x+\int\limits_{\partial\Omega} g\cdot v\d s}_{=:l(v)}
	\end{align*}
	mit $u,v\in H^1(\Omega)$. 
	Finde also $u\in H^1(\Omega)$ so, dass
	\begin{align*}
		b(u,v)=l(v)\qquad v\in H^1(\Omega)
	\end{align*} 
	gilt. 
	Wir nutzen Lax-Milgram um zu zeigen, dass eine eindeutige Lösung existiert. 
	Wir überprüfen also nun die Voraussetzungen dafür. 
	Offenbar ist $l$ linear und $b$ bilinear. 
	Zur Stetigkeit von $l$:
	\begin{align*}
		\big|l(v)\big|&\stackrel{\text{C.S.}}{\leq}\Vert f\Vert_{L^2(\Omega)}\cdot\Vert v\Vert_{L^2(\Omega)}+\Vert g\Vert_{L^2(\partial\Omega)}\cdot
		\underbrace{\Vert v\Vert_{L^2(\partial\Omega)}}_{\stackrel{\text{Spur-Theorem}}{\leq} c\cdot\Vert v\Vert_{H^1(\Omega)}}
	\end{align*}
	Damit das gilt, muss $f\in L^2(\Omega)$ und $g\in L^2(\partial\Omega)$ vorausgesetzt werden. 
	Damit ist $l$ stetig.\nl
	Zur Stetigkeit von $b$ (ziemlich analog):
	\begin{align*}
		\qquad\big| b(u,v)\big|\stackrel{\text{C.S.}}{\leq}
		&\Vert\nabla u\Vert_{L^2(\Omega)}\cdot\Vert\nabla v\Vert_{L^2(\Omega)}\\
		&+\Vert a\Vert_{L^\infty(\Omega)}\cdot\Vert\nabla u\Vert_{L^2(\Omega)}\cdot\Vert v\Vert_{L^2(\Omega)} \\
		&+\Vert\alpha\Vert_{L^\infty(\Omega)}\cdot\Vert u\Vert_{L^2(\Omega)}\cdot\Vert u\Vert_{L^2(\Omega)}\\
		&+\Vert\beta\Vert_{L^\infty(\partial\Omega)}\cdot\Vert u\Vert_{L^2(\partial\Omega)}\cdot\Vert v\Vert_{L^2(\partial\Omega)}\\
		&\leq
		\left(1+\Vert a\Vert_{L^\infty(\Omega)}+\Vert\alpha\Vert_{L^\infty(\Omega)}+c^2\cdot\Vert\beta\Vert_{L^\infty(\partial\Omega)}\right)\cdot
		\Vert u\Vert_{H^1(\Omega)}\cdot\Vert v\Vert_{H^1(\Omega)}
	\end{align*}
	Also ist $b$ stetig unter der Annahme, dass $a, \alpha \in L^\infty(\Omega); \beta \in L^\infty(\partial\Omega)$ sind. 
	Das prinzipielle Vorgehen dabei ist immer Cauchy-Schwarz-Ungleichung + Spurungleichung, sowie die Abschätzung der $H^1$ Norm durch ihre Definition als Summe.\nl
	Zeige Koerzivität, also 
	$b(v,v)\geq c\cdot\Vert v\Vert^2_{H^1(\Omega)}=c\cdot\left(\Vert v\Vert^2_{L^2(\Omega)}+\Vert \nabla v\Vert_{L^2(\Omega)}\right)$:
	\begin{align*}
		b(v,v)&=\underbrace{\int\limits_\Omega\nabla v\cdot\nabla v \d x}_{\Vert \nabla v\Vert^2_{L^2(\Omega)}}+\underbrace{\int\limits_\Omega a\cdot\nabla v\cdot v\d x}_{=\int\limits_\Omega a\cdot\frac{1}{2}\cdot\nabla(v^2)}+\int\limits_\Omega\alpha\cdot v\cdot v\d x +\int\limits_{\partial\Omega}\beta\cdot v\cdot v\d s\\
	\end{align*}
	Wende nun wieder partielle Integration von oben mit $\lambda=v^2$ und $w=a$ an:
	\begin{align*}
		\int\limits_\Omega a \cdot \nabla (v^2) \d x
		&=\int\limits_\Omega\div(a\cdot v^2)-\div(a)\cdot v^2\d x\\
		\overset{\text{Gauß}}&=
		\int\limits_{\partial\Omega}(a\cdot n)\cdot(v\cdot v)\d s-\cdot\int\limits\div(a)\cdot(v\cdot v)\d x\\
	\end{align*}
	Einsetzen ergibt
	\begin{align*}
		&b(v,v)=\Vert\nabla v\Vert^2_{L^2(\Omega)}+\int\limits_\Omega\left(\alpha-\frac{1}{2}\cdot\div(a)\right)\cdot(v\cdot v)\d x
		+\int\limits_{\partial\Omega}\left(\beta+\frac{1}{2}\cdot a\cdot n\right)\cdot(v\cdot v)\d s
	\end{align*}
	Wir benötigen also die zusätzlichen Voraussetzungen, dass $\div(a)\in L^\infty(\Omega)$ und $a\cdot u\in L^\infty(\partial\Omega)$.\nl
	Um die Voraussetzungen noch weiter abzuschwächen, können wir Theorem 1.11 aus der Vorlesung verwenden. 
	Die Norm
	\begin{align*}
		\Vertiii{f}:=|f|_{1,2,\Omega}+\big|F(f)\big|+\sqrt{B(f,f)}
	\end{align*}
	ist äquivalent zu $\Vert f\Vert_{1,2,\Omega}$, wenn:
	\begin{enumerate}
		\item $B$ stetig, symmetrisch, bilinear auf $H^1(\Omega)$, $B(v,v)\geq0~\forall v\in V$
		\item $F$ linear, stetig auf $H^1(\Omega)$
		\item $\big|F(1)\big|+\sqrt{B(1,1)}>0$ für $f\equiv1$
	\end{enumerate}
	
	\begin{align*}
		F\equiv 0,\qquad B(v,w):=\int\limits_\Omega\left(\alpha-\frac{1}{2}\cdot\div(a)\right)\cdot(v\cdot q)\d x+\int\limits_{\partial\Omega}\left(\beta+\frac{1}{2}\cdot a\cdot n\right)\cdot(v\cdot w)\d s
	\end{align*}
	Es folgt die Stetigkeit von $B$ wieder aus der Cauchy-Schwarz-Ungleichung und der Spurungleichung. Für $B(v,v)\geq0$ brauchen wir 
	\begin{align*}
		\left(\alpha-\frac{1}{2}\cdot\div(a)\right)\geq 0,\qquad
		\left(\beta-\frac{1}{2}\cdot a\cdot n\right)\geq 0
	\end{align*}
	Für $\big|F(1)\big|+\sqrt{B(1,1)}=\sqrt{B(1,1)}>0$ brauchen wir
	\begin{align*}
		\int\limits_\Omega(\alpha-\frac{1}{2}\div(a)) \d x + \int\limits_{\partial\Omega} (\beta + \frac{1}{2}a\cdot n) \d s > 0
	\end{align*}
\end{lösung}

\subsection{Aufgabe 3.5}
Das Differentialgleichungssystem
\begin{align*}
	-\Delta u_1-c\cdot u_2&\equiv f_1\text{ in }\Omega\\
	-\Delta u_2+c\cdot u_1&\equiv f_2\text{ in }\Omega
\end{align*}
mit $u_1|_{\partial\Omega}=u_2|_{\partial\Omega}=0,c\in L^\infty(\Omega)$ sowie $f_1,f_2\in L^2(\Omega)$ besitzt eine eindeutige schwache Lösung.

\begin{proof}
	Idee: Fasse die beiden Gleichungen als eine auf.\\
	Multipliziere die obere mit Testfunktion $v_1$ und die untere mit Testfunktion $v_2$ und integriere anschließend:
	\begin{align*}
		\int\limits_\Omega(-\Delta u_1-c\cdot u_2)\cdot v_1\d x=\int\limits_\Omega f_1(x)\cdot v_1(x)\d x\\
		\int\limits_\Omega(-\Delta u_2+c\cdot u_1)\cdot v_2\d x=\int\limits_\Omega f_2(x)\cdot v_2(x)\d x\\
		\implies
		\int\limits_\Omega -\Delta u_1\cdot v_1-\Delta u_2\cdot v_2-c\cdot u_2\cdot v_1+c\cdot u_1\cdot v_2\d x=\int\limits_\Omega f_1\cdot v_1+f_2\cdot v_2\d x
	\end{align*}
	Man kann die Gleichungen addieren, weil die Testfunktionen unabhängig voneinander sind.\nl
	Ausführliche Lösung siehe \url{http://www.math.tu-dresden.de/~sbecher/pdenm/}
\end{proof}
