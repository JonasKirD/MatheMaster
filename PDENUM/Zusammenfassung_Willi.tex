\documentclass[12pt]{scrartcl}
\usepackage{palatino,setspace,fancyhdr}
\usepackage[left=10mm,right=10mm,top=25mm,bottom=25mm]{geometry}
\onehalfspacing
\pagestyle{fancy}
\chead{Zusammenfassung PDENM}
\lfoot{Version: \today}
\rfoot{Seite \thepage}
\lhead{}
\rhead{Willi Sontopski}

% This work is licensed under the Creative Commons
% Attribution-NonCommercial-ShareAlike 4.0 International License. To view a copy
% of this license, visit http://creativecommons.org/licenses/by-nc-sa/4.0/ or
% send a letter to Creative Commons, PO Box 1866, Mountain View, CA 94042, USA.

% PACKAGES
\usepackage[english, ngerman]{babel}	% Paket für Sprachselektion, in diesem Fall für deutsches Datum etc
\usepackage[utf8]{inputenc}	% Paket für Umlaute; verwende utf8 Kodierung in TexWorks 
\usepackage[T1]{fontenc} % ö,ü,ä werden richtig kodiert
\usepackage{amsmath} % wichtig für align-Umgebung
\usepackage{amssymb} % wichtig für \mathbb{} usw.
\usepackage{amsthm} % damit kann man eigene Theorem-Umgebungen definieren, proof-Umgebungen, etc.
\usepackage{mathrsfs} % für \mathscr
\usepackage[backref]{hyperref} % Inhaltsverzeichnis und \ref-Befehle werden in der PDF-klickbar
\usepackage{graphicx}
\usepackage{grffile}
\usepackage{setspace} % wichtig für Lesbarkeit. Schöne Zeilenabstände
\usepackage{enumitem} % für custom Liste mit default Buchstaben
\usepackage{ulem} % für bessere Unterstreichung
\usepackage{contour} % für bessere Unterstreichung
\usepackage{epigraph} % für das coole Zitat
\usepackage{float}            % figure-Umgebungen besser positionieren
\usepackage{xfrac}
\usepackage{bbm} %sorgt für Symbol für Indikatorfunktion
\usepackage{color} % bringt Farbe ins Spiel
\usepackage{pdflscape} % damit kann man einzelne Seiten ins Querformat drehen
\usepackage{aligned-overset} % besseres Einrücken, siehe: https://tex.stackexchange.com/questions/257529/overset-and-align-environment-how-to-get-correct-alignment
\usepackage{pgfplots}
	\pgfplotsset{compat=newest}

\usepackage[
    type={CC},
    modifier={by-nc-sa},
    version={4.0},
]{doclicense} % für CC Lizenz-Vermerk

\usepackage{tikz}
  \usetikzlibrary{matrix}
  \usetikzlibrary{cd}
  \usetikzlibrary{babel}
  \usetikzlibrary{calc}
	\usetikzlibrary{positioning}
	\usetikzlibrary{shapes.geometric}
	\usetikzlibrary{fit}
	\usetikzlibrary{arrows}
	
\usepackage{csquotes}
	\MakeOuterQuote{"}

\usepackage{xargs} % for multiple optional args in newcommand
\usepackage{lmodern} % provides a bigger set of font sizes
\usepackage{anyfontsize} % supports fallback scaling for non-existing font size
\usepackage{scrhack} % provides a hack for deprecated float environments used by some libs

% Ich habe gelesen, dass man folgendes Package zuletzt einbinden soll:
\usepackage[english, ngerman, capitalise]{cleveref} % bessere Verweise

% This work is licensed under the Creative Commons
% Attribution-NonCommercial-ShareAlike 4.0 International License. To view a copy
% of this license, visit http://creativecommons.org/licenses/by-nc-sa/4.0/ or
% send a letter to Creative Commons, PO Box 1866, Mountain View, CA 94042, USA.

% THEOREM-ENVIRONMENTS

\newtheoremstyle{mystyle}
  {20pt}   % ABOVESPACE \topsep is default, 20pt looks nice
  {20pt}   % BELOWSPACE \topsep is default, 20pt looks nice
  {\normalfont} % BODYFONT
  {0pt}       % INDENT (empty value is the same as 0pt)
  {\bfseries} % HEADFONT
  {}          % HEADPUNCT (if needed)
  {5pt plus 1pt minus 1pt} % HEADSPACE
	{}          % CUSTOM-HEAD-SPEC
\theoremstyle{mystyle}

% Definitionen der Satz, Lemma... - Umgebungen. Der Zähler von "satz" ist dem "section"-Zähler untergeordnet, alle weiteren Umgebungen bedienen sich des satz-Zählers.
\newtheorem{satz}{Satz}[section]
\newtheorem{lemma}[satz]{Lemma}
\newtheorem{korollar}[satz]{Korollar}
\newtheorem{proposition}[satz]{Proposition}
\newtheorem{beispiel}[satz]{Beispiel}
\newtheorem{definition}[satz]{Definition}
\newtheorem{bemerkungnr}[satz]{Bemerkung}
\newtheorem{theorem}[satz]{Theorem}
\newtheorem{erinnerungnr}[satz]{Erinnerung}
\newtheorem{vermutung}[satz]{Vermutung}

% Bemerkungen, Erinnerungen und Notationshinweise werden ohne Numerierungen dargestellt.
\newtheorem*{bemerkung}{Bemerkung.}
\newtheorem*{erinnerung}{Erinnerung.}
\newtheorem*{notation}{Notation.}
\newtheorem*{aufgabe}{Aufgabe.}
\newtheorem*{lösung}{Lösung.}
\newtheorem*{beisp}{Beispiel.}  %Beispiel ohne Nummerierung
\newtheorem*{defi}{Definition.} %Definition ohne Nummerierung
\newtheorem*{lem}{Lemma.}       %Lemma ohne Nummerierung
\newtheorem*{thm}{Theorem.}     %Theorem ohne Nummerierung
\newtheorem*{konvention}{Konvention.}  


% This work is licensed under the Creative Commons
% Attribution-NonCommercial-ShareAlike 4.0 International License. To view a copy
% of this license, visit http://creativecommons.org/licenses/by-nc-sa/4.0/ or
% send a letter to Creative Commons, PO Box 1866, Mountain View, CA 94042, USA.

% STANDARD SHORTCUTS
%%%%%%%%%%%%%%%%%%%%%%%%%%%%%%%%%%%%%%%%%%%%
\newcommand{\R}{\mathbb{R}}				 % reelle Zahlen
\newcommand{\Rn}{\R^n}					 % der R^n
\newcommand{\N}{\mathbb{N}}				 % natürliche Zahlen
\newcommand{\Q}{\mathbb{Q}}				 % rationale Zahlen
\newcommand{\Z}{\mathbb{Z}}				 % ganze Zahlen
\newcommand{\C}{\mathbb{C}}			   % komplexe Zahlen
\renewcommand{\mit}{\text{ mit }}   % mit
\newcommand{\falls}{\text{falls }} % falls
\renewcommand{\d}{\text{ d}}        % Differential d
\DeclareMathOperator{\tr}{tr} % spur
\DeclareMathOperator{\diag}{diag}
\DeclareMathOperator{\Hom}{Hom}
\DeclareMathOperator{\Span}{span}
\DeclareMathOperator{\im}{im}
\DeclareMathOperator{\SO}{SO}
\newcommand{\ideal}{\trianglelefteq} %Ideal
\newcommand{\properideal}{\mathrel{\ooalign{$\lneqq$\cr\raise.51ex\hbox{$\lhd$}\cr}}} %echtes Ideal
\DeclareMathOperator{\supp}{supp}                 % Träger
\newcommandx{\bracket}[2][1=\cdot, 2=\cdot]{[#1,#2]}
\DeclareMathOperator{\graph}{graph}       % Graph einer Funktion

% ETWAS SPEZIELLERE ZEICHEN
%%%%%%%%%%%%%%%%%%%%%%%%%%%%%%%%%%%%%%%%%%%%
% disjunkte Vereinigung
\newcommand{\bigcupdot}{
	\mathop{\vphantom{\bigcup}\mathpalette\setbigcupdot\cdot}\displaylimits
}
\newcommand{\setbigcupdot}[2]{\ooalign{\hfil$#1\bigcup$\hfil\cr\hfil$#2$\hfil\cr\cr}}
% großes Kreuz
\newcommand*{\bigtimes}{\mathop{\raisebox{-.5ex}{\hbox{\huge{$\times$}}}}} 
% dreifach gestrichene Norm
\newcommand{\Vertiii}[1]{{\left\vert\kern-0.25ex\left\vert\kern-0.25ex\left\vert #1 
    \right\vert\kern-0.25ex\right\vert\kern-0.25ex\right\vert}}
% korrektes argmin und argmax
\DeclareMathOperator*{\argmax}{arg\,max}
\DeclareMathOperator*{\argmin}{arg\,min}

% WHITESPACE COMMANDS
%%%%%%%%%%%%%%%%%%%%%%%%%%%%%%%%%%%%%%%%%%%%
% Zeilenumbruch mit freier Zeile darunter OHNE underfull-hbox-warning
\newcommand{\nl}{\\[\baselineskip]}
% nicht restriktiver newline command
\newcommand{\enter}{$ $\newline} 
% praktischer Tabulator
\newcommand\tab[1][1cm]{\hspace*{#1}}

% TEXT ÜBER ZEICHEN
\newcommand{\stackeq}[1]{\stackrel{#1}{=}} 

% TEXT ÜBER UND UNTER ZEICHEN
\newcommand{\stackrelnew}[3]{\underset{#1}{\overset{#2}{#3}}}

% UNDERLINE (wird nicht mehr genutzt)
% besseres underline 
%\renewcommand{\ULdepth}{1pt}
%\contourlength{0.5pt}
%\newcommand{\ul}[1]{
%	\uline{\phantom{#1}}\llap{\contour{white}{#1}}
%}
\newcommand{\ul}[1]{\underline{#1}} %Umleitung des Commands, da man sich gegen obigen entschieden hat. Dieser erzeugt zu viel Whitespace vor und nach dem Unterstrichenem.

% Commands für Stochastik / Statistik
\newcommand{\A}{\mathcal{A}}
\renewcommand{\P}{\mathbb{P}}
\newcommand{\E}{\mathbb{E}}
\newcommand{\B}{\mathcal{B}} %Borel-Sigma-Algebra
\newcommand{\Var}{\mathbb{V}\text{ar}}
\newcommand{\Cov}{\mathbb{C}\text{ov}} %Kovarianz
\newcommand{\indi}{\mathbbm{1}} % Indikatorfunktion
\renewcommand{\L}{\mathcal{L}} %L_p-Räume

% Verteilungen
\DeclareMathOperator{\Bin}{Bin}       %Binomialverteilung
\newcommand{\Nor}{\mathcal{N}}        %Normalverteilung
\DeclareMathOperator{\Poi}{Poi}       %Poissonverteilung
\DeclareMathOperator{\Exp}{Exp}       %Exponentialverteilung
\DeclareMathOperator{\Cauchy}{Cauchy} %Cauchyverteilung


% This work is licensed under the Creative Commons
% Attribution-NonCommercial-ShareAlike 4.0 International License. To view a copy
% of this license, visit http://creativecommons.org/licenses/by-nc-sa/4.0/ or
% send a letter to Creative Commons, PO Box 1866, Mountain View, CA 94042, USA.

\newcommand{\gdw}{\Leftrightarrow}             % genau dann, wenn


\begin{document}
	\section{Sobolev-Räume}
	$\Omega\subseteq\R^d$ offen und beschränkt.
	Aus dem Divergenz-Satz von Gauß folgt die erste Greensche Formel:
	\begin{align}\label{eqGreen}
		\int\limits_\Omega\nabla u\cdot\nabla v\d x
		&=\int\limits_{\partial\Omega}v\cdot\nabla u\cdot n\d S+\int\limits_\Omega v\cdot\Delta u\d x\quad\mit\\
		\div(\nabla u)&=\laplace u\nonumber
	\end{align}
	$\psi$ heißt \textbf{$\alpha$-te schwache Ableitung} von $\varphi\in L^1(\Omega)\gdw
	%\begin{align*}
		\forall v\in C_0^\infty(\Omega):\int\limits_\Omega\varphi\cdot \underbrace{D^\alpha v}_{\overset{\alpha=1}=v'}\d x
		=\underbrace{(-1)^{|\alpha|}}_{\overset{\alpha=1}{=}-1}\cdot\int\limits_\Omega\psi\cdot v\d x$
	%\end{align*}
	Kurzschreibweise: $\psi=D^\alpha\varphi=\varphi'$; eindeutig bestimmt, stimmt mit klassischer Ableitung überein, $|x|$
	\begin{align*}
		W^{k,p}(\Omega)
		&:=\Big\lbrace\varphi\in L^p(\Omega):D^\alpha\varphi\text{ (schwache Abl.) existiert und }D^\alpha\varphi\in L^p(\Omega)~\forall|\alpha|\leq k\Big\rbrace
		\quad k\in\N_0,p\in[1,\infty)\\
		\Vert\varphi\Vert_{k,p,\Omega}
		&:=\left(\sum\limits_{|\alpha|\leq k}\left\Vert D^\alpha\varphi\right\Vert^p_{L^p}\right)^{\frac{1}{p}}
		=\left(\sum\limits_{|\alpha|\leq k}\int\limits_\Omega\left| D^\alpha\varphi(x)\right|^p\d x\right)^{\frac{1}{p}}\qquad\Big(W^{k,p}(\Omega),\Vert\cdot\Vert_{k,p,\Omega}\Big)\text{ ist BR}\\
		|\varphi|_{k,p,\Omega}
		&:=\left(\sum\limits_{|\alpha|= k}\left\Vert D^\alpha\varphi\right\Vert^p_{L^p}\right)^{\frac{1}{p}}\qquad\text{ ist Halbnorm, also }\big|\varphi|=0\not\implies\varphi=0\\
		\langle \varphi,\psi\rangle_k&:=\sum\limits_{|\alpha|\leq k}\int\limits_\Omega D^\alpha\varphi\cdot D^\alpha\psi\d x\quad\text{Für $p=2$ ist $H^k(\Omega):=W^{k,2}(\Omega)$ mit $\langle\cdot,\cdot\rangle$ Hilbertraum.}
	\end{align*}
	$C^\infty(\overline{\Omega})$ liegt dicht in $W^{k,p}(\Omega)$.
	Stückweise glatte Funkt. sind glatt:
	$\varphi\in W^{k,p}(\Omega)\overset{\varphi|_{\Omega_i}\in C^k(\overline{\Omega_i})}{\Longleftrightarrow}\varphi\in C^{k-1}(\Omega)$\\
	Die Vervollständigung des $C_0^\infty(\Omega)$ bzgl. der Norm $\Vert\cdot\Vert_{k,p,\Omega}$ wird mit $W_0^{k,p}(\Omega)$ bezeichnet.
	$H_0^k(\Omega):=W_0^{k,2}(\Omega)$.\\
	Eine Menge $\Omega$ ist \textbf{Lipschitz-berandet / -Gebiet}, wenn $\partial\Omega$ stückweise als Graph Lipschitz-stetiger ist.\\
	Auf Lipschitz-Gebieten existiert der äußere Normalenvektor fast überall auf $\partial\Omega$.\\
	\textbf{Spursatz:} Sei $\Omega$ Lipschitz-Gebiet. Dann existiert stetiger linearer Op. $\gamma_l:W^{k,p}(\Omega)\rightarrow L^p(\partial\Omega)$ mit\\ 
	$\gamma_l(\varphi)=\frac{\partial^l}{\partial n^l}\varphi|_{\partial\Omega}\quad\forall\varphi\in C^k(\overline{\Omega}),l<k$, also $\gamma_0(\varphi)=\varphi|_{\partial\Omega},~\gamma_1(\varphi)=\frac{\partial}{\partial n}\varphi_{\partial\Omega},\overset{\text{stetig}}{\implies}
	\Vert\gamma_l(\varphi)\Vert_{L^p}\leq c\Vert\varphi\Vert_{W^{k,p}(\Omega)}$\\
	$W_0^{k,p}(\Omega)=\left\lbrace\varphi\in W^{k,p}(\Omega):
		\forall l\in\lbrace0,\ldots,k-1\rbrace:\gamma_l(\varphi)=0\right\rbrace$\\
	$W^{k,p}(\Omega)\stackrel{C}{\hookrightarrow} W^{k-1,p}(\Omega)$, 
	Für $p=2$, $d\in\lbrace2,3\rbrace$, $q<6$:
	$H^2(\Omega)\hookrightarrow C(\overline{\Omega}),~H^1(\Omega)\stackrel{C}{\hookrightarrow} L^q(\Omega)$\\
	\textbf{Poincaré-U:} $\Vert\cdot\Vert_{1,2,\Omega}\cong\big(|\cdot|_{1,2,\Omega}+|\int_\Omega(\cdot)(x)\d x|\big)$ und $\Vert\cdot\Vert_{1,2,\Omega}\cong|\cdot|_{1,2,\Omega}$ auf $\lbrace\varphi\in H^1(\Omega):\int_\Omega u(x)\d x=0\rbrace$\\
	\textbf{Friedrichs-U:} $|\cdot|_{k,p,\Omega}$ ist Norm auf $W_0^{k,p}(\Omega)$ mit $|\cdot|_{k,p,\Omega}\cong\Vert\cdot\Vert_{k,p,\Omega}$, also $\Vert\cdot\Vert_\ast\leq c|\cdot|_{k,p,\Omega},*\in\lbrace(k,p,\Omega),L^p(\Omega)\rbrace$
	
	\section{Abstrakte Variations-Analysis}
	
	\section{Schwache Lösung}
	
	\section{Finite-Elemente-Räume}
	
	\section{Nichtkonforme Methoden}
	
	\section{A posteriori Fehlerschätzer}
	
	\section{Stromlinien-Diffusionsmethode (SDFEM)}
	
	\section{Gemischte Methoden}
\end{document}