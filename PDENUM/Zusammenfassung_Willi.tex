\documentclass[12pt]{scrartcl}
\usepackage{palatino,setspace,fancyhdr}
\usepackage[left=10mm,right=10mm,top=25mm,bottom=25mm]{geometry}
\onehalfspacing
\pagestyle{fancy}
\chead{Zusammenfassung PDENM}
\lfoot{Version: \today}
\rfoot{Seite \thepage}
\lhead{}
\rhead{Willi Sontopski}

% This work is licensed under the Creative Commons
% Attribution-NonCommercial-ShareAlike 4.0 International License. To view a copy
% of this license, visit http://creativecommons.org/licenses/by-nc-sa/4.0/ or
% send a letter to Creative Commons, PO Box 1866, Mountain View, CA 94042, USA.

% PACKAGES
\usepackage[english, ngerman]{babel}	% Paket für Sprachselektion, in diesem Fall für deutsches Datum etc
\usepackage[utf8]{inputenc}	% Paket für Umlaute; verwende utf8 Kodierung in TexWorks 
\usepackage[T1]{fontenc} % ö,ü,ä werden richtig kodiert
\usepackage{amsmath} % wichtig für align-Umgebung
\usepackage{amssymb} % wichtig für \mathbb{} usw.
\usepackage{amsthm} % damit kann man eigene Theorem-Umgebungen definieren, proof-Umgebungen, etc.
\usepackage{mathrsfs} % für \mathscr
\usepackage[backref]{hyperref} % Inhaltsverzeichnis und \ref-Befehle werden in der PDF-klickbar
\usepackage{graphicx}
\usepackage{grffile}
\usepackage{setspace} % wichtig für Lesbarkeit. Schöne Zeilenabstände
\usepackage{enumitem} % für custom Liste mit default Buchstaben
\usepackage{ulem} % für bessere Unterstreichung
\usepackage{contour} % für bessere Unterstreichung
\usepackage{epigraph} % für das coole Zitat
\usepackage{float}            % figure-Umgebungen besser positionieren
\usepackage{xfrac}
\usepackage{bbm} %sorgt für Symbol für Indikatorfunktion
\usepackage{color} % bringt Farbe ins Spiel
\usepackage{pdflscape} % damit kann man einzelne Seiten ins Querformat drehen
\usepackage{aligned-overset} % besseres Einrücken, siehe: https://tex.stackexchange.com/questions/257529/overset-and-align-environment-how-to-get-correct-alignment
\usepackage{pgfplots}
	\pgfplotsset{compat=newest}

\usepackage[
    type={CC},
    modifier={by-nc-sa},
    version={4.0},
]{doclicense} % für CC Lizenz-Vermerk

\usepackage{tikz}
  \usetikzlibrary{matrix}
  \usetikzlibrary{cd}
  \usetikzlibrary{babel}
  \usetikzlibrary{calc}
	\usetikzlibrary{positioning}
	\usetikzlibrary{shapes.geometric}
	\usetikzlibrary{fit}
	\usetikzlibrary{arrows}
	
\usepackage{csquotes}
	\MakeOuterQuote{"}

\usepackage{xargs} % for multiple optional args in newcommand
\usepackage{lmodern} % provides a bigger set of font sizes
\usepackage{anyfontsize} % supports fallback scaling for non-existing font size
\usepackage{scrhack} % provides a hack for deprecated float environments used by some libs

% Ich habe gelesen, dass man folgendes Package zuletzt einbinden soll:
\usepackage[english, ngerman, capitalise]{cleveref} % bessere Verweise

% This work is licensed under the Creative Commons
% Attribution-NonCommercial-ShareAlike 4.0 International License. To view a copy
% of this license, visit http://creativecommons.org/licenses/by-nc-sa/4.0/ or
% send a letter to Creative Commons, PO Box 1866, Mountain View, CA 94042, USA.

% THEOREM-ENVIRONMENTS

\newtheoremstyle{mystyle}
  {20pt}   % ABOVESPACE \topsep is default, 20pt looks nice
  {20pt}   % BELOWSPACE \topsep is default, 20pt looks nice
  {\normalfont} % BODYFONT
  {0pt}       % INDENT (empty value is the same as 0pt)
  {\bfseries} % HEADFONT
  {}          % HEADPUNCT (if needed)
  {5pt plus 1pt minus 1pt} % HEADSPACE
	{}          % CUSTOM-HEAD-SPEC
\theoremstyle{mystyle}

% Definitionen der Satz, Lemma... - Umgebungen. Der Zähler von "satz" ist dem "section"-Zähler untergeordnet, alle weiteren Umgebungen bedienen sich des satz-Zählers.
\newtheorem{satz}{Satz}[section]
\newtheorem{lemma}[satz]{Lemma}
\newtheorem{korollar}[satz]{Korollar}
\newtheorem{proposition}[satz]{Proposition}
\newtheorem{beispiel}[satz]{Beispiel}
\newtheorem{definition}[satz]{Definition}
\newtheorem{bemerkungnr}[satz]{Bemerkung}
\newtheorem{theorem}[satz]{Theorem}
\newtheorem{erinnerungnr}[satz]{Erinnerung}
\newtheorem{vermutung}[satz]{Vermutung}

% Bemerkungen, Erinnerungen und Notationshinweise werden ohne Numerierungen dargestellt.
\newtheorem*{bemerkung}{Bemerkung.}
\newtheorem*{erinnerung}{Erinnerung.}
\newtheorem*{notation}{Notation.}
\newtheorem*{aufgabe}{Aufgabe.}
\newtheorem*{lösung}{Lösung.}
\newtheorem*{beisp}{Beispiel.}  %Beispiel ohne Nummerierung
\newtheorem*{defi}{Definition.} %Definition ohne Nummerierung
\newtheorem*{lem}{Lemma.}       %Lemma ohne Nummerierung
\newtheorem*{thm}{Theorem.}     %Theorem ohne Nummerierung
\newtheorem*{konvention}{Konvention.}  


% This work is licensed under the Creative Commons
% Attribution-NonCommercial-ShareAlike 4.0 International License. To view a copy
% of this license, visit http://creativecommons.org/licenses/by-nc-sa/4.0/ or
% send a letter to Creative Commons, PO Box 1866, Mountain View, CA 94042, USA.

% STANDARD SHORTCUTS
%%%%%%%%%%%%%%%%%%%%%%%%%%%%%%%%%%%%%%%%%%%%
\newcommand{\R}{\mathbb{R}}				 % reelle Zahlen
\newcommand{\Rn}{\R^n}					 % der R^n
\newcommand{\N}{\mathbb{N}}				 % natürliche Zahlen
\newcommand{\Q}{\mathbb{Q}}				 % rationale Zahlen
\newcommand{\Z}{\mathbb{Z}}				 % ganze Zahlen
\newcommand{\C}{\mathbb{C}}			   % komplexe Zahlen
\renewcommand{\mit}{\text{ mit }}   % mit
\newcommand{\falls}{\text{falls }} % falls
\renewcommand{\d}{\text{ d}}        % Differential d
\DeclareMathOperator{\tr}{tr} % spur
\DeclareMathOperator{\diag}{diag}
\DeclareMathOperator{\Hom}{Hom}
\DeclareMathOperator{\Span}{span}
\DeclareMathOperator{\im}{im}
\DeclareMathOperator{\SO}{SO}
\newcommand{\ideal}{\trianglelefteq} %Ideal
\newcommand{\properideal}{\mathrel{\ooalign{$\lneqq$\cr\raise.51ex\hbox{$\lhd$}\cr}}} %echtes Ideal
\DeclareMathOperator{\supp}{supp}                 % Träger
\newcommandx{\bracket}[2][1=\cdot, 2=\cdot]{[#1,#2]}
\DeclareMathOperator{\graph}{graph}       % Graph einer Funktion

% ETWAS SPEZIELLERE ZEICHEN
%%%%%%%%%%%%%%%%%%%%%%%%%%%%%%%%%%%%%%%%%%%%
% disjunkte Vereinigung
\newcommand{\bigcupdot}{
	\mathop{\vphantom{\bigcup}\mathpalette\setbigcupdot\cdot}\displaylimits
}
\newcommand{\setbigcupdot}[2]{\ooalign{\hfil$#1\bigcup$\hfil\cr\hfil$#2$\hfil\cr\cr}}
% großes Kreuz
\newcommand*{\bigtimes}{\mathop{\raisebox{-.5ex}{\hbox{\huge{$\times$}}}}} 
% dreifach gestrichene Norm
\newcommand{\Vertiii}[1]{{\left\vert\kern-0.25ex\left\vert\kern-0.25ex\left\vert #1 
    \right\vert\kern-0.25ex\right\vert\kern-0.25ex\right\vert}}
% korrektes argmin und argmax
\DeclareMathOperator*{\argmax}{arg\,max}
\DeclareMathOperator*{\argmin}{arg\,min}

% WHITESPACE COMMANDS
%%%%%%%%%%%%%%%%%%%%%%%%%%%%%%%%%%%%%%%%%%%%
% Zeilenumbruch mit freier Zeile darunter OHNE underfull-hbox-warning
\newcommand{\nl}{\\[\baselineskip]}
% nicht restriktiver newline command
\newcommand{\enter}{$ $\newline} 
% praktischer Tabulator
\newcommand\tab[1][1cm]{\hspace*{#1}}

% TEXT ÜBER ZEICHEN
\newcommand{\stackeq}[1]{\stackrel{#1}{=}} 

% TEXT ÜBER UND UNTER ZEICHEN
\newcommand{\stackrelnew}[3]{\underset{#1}{\overset{#2}{#3}}}

% UNDERLINE (wird nicht mehr genutzt)
% besseres underline 
%\renewcommand{\ULdepth}{1pt}
%\contourlength{0.5pt}
%\newcommand{\ul}[1]{
%	\uline{\phantom{#1}}\llap{\contour{white}{#1}}
%}
\newcommand{\ul}[1]{\underline{#1}} %Umleitung des Commands, da man sich gegen obigen entschieden hat. Dieser erzeugt zu viel Whitespace vor und nach dem Unterstrichenem.

% Commands für Stochastik / Statistik
\newcommand{\A}{\mathcal{A}}
\renewcommand{\P}{\mathbb{P}}
\newcommand{\E}{\mathbb{E}}
\newcommand{\B}{\mathcal{B}} %Borel-Sigma-Algebra
\newcommand{\Var}{\mathbb{V}\text{ar}}
\newcommand{\Cov}{\mathbb{C}\text{ov}} %Kovarianz
\newcommand{\indi}{\mathbbm{1}} % Indikatorfunktion
\renewcommand{\L}{\mathcal{L}} %L_p-Räume

% Verteilungen
\DeclareMathOperator{\Bin}{Bin}       %Binomialverteilung
\newcommand{\Nor}{\mathcal{N}}        %Normalverteilung
\DeclareMathOperator{\Poi}{Poi}       %Poissonverteilung
\DeclareMathOperator{\Exp}{Exp}       %Exponentialverteilung
\DeclareMathOperator{\Cauchy}{Cauchy} %Cauchyverteilung


% This work is licensed under the Creative Commons
% Attribution-NonCommercial-ShareAlike 4.0 International License. To view a copy
% of this license, visit http://creativecommons.org/licenses/by-nc-sa/4.0/ or
% send a letter to Creative Commons, PO Box 1866, Mountain View, CA 94042, USA.

\newcommand{\gdw}{\Leftrightarrow}             % genau dann, wenn

% This work is licensed under the Creative Commons
% Attribution-NonCommercial-ShareAlike 4.0 International License. To view a copy
% of this license, visit http://creativecommons.org/licenses/by-nc-sa/4.0/ or
% send a letter to Creative Commons, PO Box 1866, Mountain View, CA 94042, USA.
% vim: set noexpandtab:

\newcommand{\T}{\mathcal{T}}       % Triangulierung
\newcommand{\G}{\mathcal{G}}       % Menge




\begin{document}
	\section{Sobolev-Räume}
	
	$\Omega\subseteq\R^d$ offen und beschränkt.
	Aus dem Divergenz-Satz von Gauß folgt die erste Greensche Formel:
	\begin{align}\label{eqGreen}
		\int\limits_\Omega\nabla u\cdot\nabla v\d x
		=\int\limits_{\partial\Omega}v\cdot\nabla u\cdot n\d S-\int\limits_\Omega v\cdot\Delta u\d x\quad\mit\quad
		\div(\nabla u)=\laplace u=\sum\limits_{k=1}^n\frac{\partial^2 f}{\partial x_k^2}
	\end{align}
	$\psi$ heißt \textbf{$\alpha$-te schwache Ableitung} von $\varphi\in L^1(\Omega)\gdw
	%\begin{align*}
		\forall v\in C_0^\infty(\Omega):\int\limits_\Omega\varphi\cdot \underbrace{D^\alpha v}_{\overset{\alpha=1}=v'}\d x
		=\underbrace{(-1)^{|\alpha|}}_{\overset{\alpha=1}{=}-1}\cdot\int\limits_\Omega\psi\cdot v\d x$
	%\end{align*}
	Kurzschreibweise: $\psi=D^\alpha\varphi=\varphi'$; eindeutig bestimmt, stimmt mit klassischer Ableitung überein, $|x|$
	\begin{align*}
		W^{k,p}(\Omega)
		&:=\Big\lbrace\varphi\in L^p(\Omega):D^\alpha\varphi\text{ (schwache Abl.) existiert und }D^\alpha\varphi\in L^p(\Omega)~\forall|\alpha|\leq k\Big\rbrace
		\quad k\in\N_0,p\in[1,\infty)\\
		\Vert\varphi\Vert_{k,p,\Omega}
		&:=\left(\sum\limits_{|\alpha|\leq k}\left\Vert D^\alpha\varphi\right\Vert^p_{L^p}\right)^{\frac{1}{p}}
		=\left(\sum\limits_{|\alpha|\leq k}\int\limits_\Omega\left| D^\alpha\varphi(x)\right|^p\d x\right)^{\frac{1}{p}};~\Big(W^{k,p}(\Omega),\Vert\cdot\Vert_{k,p,\Omega}\Big)\text{BR};\Vert\cdot\Vert_{k,p,\Omega}\leq\Vert\cdot\Vert_{k+1,p,\Omega}\\
		|\varphi|_{k,p,\Omega}
		&:=\left(\sum\limits_{|\alpha|= k}\left\Vert D^\alpha\varphi\right\Vert^p_{L^p}\right)^{\frac{1}{p}}\qquad\text{ ist Halbnorm, also }\big|\varphi|=0\not\implies\varphi=0\\
		\langle \varphi,\psi\rangle_k&:=\sum\limits_{|\alpha|\leq k}\int\limits_\Omega D^\alpha\varphi\cdot D^\alpha\psi\d x\quad\text{Für $p=2$ ist $H^k(\Omega):=W^{k,2}(\Omega)$ mit $\langle\cdot,\cdot\rangle$ Hilbertraum.}
	\end{align*}
	$(W^{0,p}(\Omega),\Vert\cdot\Vert_{0,p,\Omega})=(L^p(\Omega),\Vert\cdot\Vert_{L^p(\Omega)})$.\\
	$C^\infty(\overline{\Omega})$ liegt dicht in $W^{k,p}(\Omega)$.
	Stückweise glatte Funkt. sind glatt:
	$\varphi\in W^{k,p}(\Omega)\overset{\varphi|_{\Omega_i}\in C^k(\overline{\Omega_i})}{\Longleftrightarrow}\varphi\in C^{k-1}(\Omega)$\\
	Die Vervollständigung des $C_0^\infty(\Omega)$ bzgl. der Norm $\Vert\cdot\Vert_{k,p,\Omega}$ wird mit $W_0^{k,p}(\Omega)$ bezeichnet.
	$H_0^k(\Omega):=W_0^{k,2}(\Omega)$.\\
	Eine Menge $\Omega$ ist \textbf{Lipschitz-berandet / -Gebiet}, wenn $\partial\Omega$ stückweise als Graph Lipschitz-stetiger ist.\\
	Auf Lipschitz-Gebieten existiert der äußere Normalenvektor fast überall auf $\partial\Omega$.\\
	\textbf{Spursatz:} Sei $\Omega$ Lipschitz-Gebiet. Dann existiert stetiger linearer Op. $\gamma_l:W^{k,p}(\Omega)\rightarrow L^p(\partial\Omega)$ mit\\ 
	$\gamma_l(\varphi)=\frac{\partial^l}{\partial n^l}\varphi|_{\partial\Omega}\quad\forall\varphi\in C^k(\overline{\Omega}),l<k$, also $\gamma_0(\varphi)=\varphi|_{\partial\Omega},~\gamma_1(\varphi)=\frac{\partial}{\partial n}\varphi_{\partial\Omega},\overset{\text{stetig}}{\implies}
	\Vert\gamma_l(\varphi)\Vert_{L^p}\leq c\Vert\varphi\Vert_{W^{k,p}(\Omega)}$\\
	$W_0^{k,p}(\Omega)=\left\lbrace\varphi\in W^{k,p}(\Omega):
		\forall l\in\lbrace0,\ldots,k-1\rbrace:\gamma_l(\varphi)=0\right\rbrace$\\
	$W^{k,p}(\Omega)\stackrel{C}{\hookrightarrow} W^{k-1,p}(\Omega)$, 
	Für $p=2$, $d\in\lbrace2,3\rbrace$, $q<6$:
	$H^2(\Omega)\hookrightarrow C(\overline{\Omega}),~H^1(\Omega)\stackrel{C}{\hookrightarrow} L^q(\Omega)$\\
	\textbf{Poincaré-U:} $\Vert\cdot\Vert_{1,2,\Omega}\cong\big(|\cdot|_{1,2,\Omega}+|\int_\Omega(\cdot)(x)\d x|\big)$ und $\Vert\cdot\Vert_{1,2,\Omega}\cong|\cdot|_{1,2,\Omega}$ auf $\lbrace\varphi\in H^1(\Omega):\int_\Omega \varphi(x)\d x=0\rbrace$\\
	\textbf{Friedrichs-U:} $|\cdot|_{k,p,\Omega}$ ist Norm auf $W_0^{k,p}(\Omega)$ mit $|\cdot|_{k,p,\Omega}\cong\Vert\cdot\Vert_{k,p,\Omega}$, also $\Vert\cdot\Vert_\ast\leq c|\cdot|_{k,p,\Omega},*\in\lbrace(k,p,\Omega),L^p(\Omega)\rbrace$
	
	\section{Abstrakte Variations-Analysis}
	
	\textbf{Poisson:} $-\laplace u=f$ auf $\Omega$, $u=0$ auf $\partial\Omega=:\Gamma$; Schwache Formulierung:\\
	$l(v):=
	\int\limits_\Omega f\cdot v\d x
	\overset{\text{Poi}}=
	\int\limits_\Omega-\laplace u\cdot v\d x
	\overset{\eqref{eqGreen}}=
	\int\limits_\Omega\nabla u\cdot\nabla v\d x-\int\limits_{\partial\Omega}\frac{\partial u}{\partial n}v\d x=\int\limits_\Omega\nabla u\cdot\nabla v\d x:=a(u,v),~u,v\in H_0^1(\Omega)$\\
	$a$ ist stetig auf $V$, d.h. $\exists M>0:\forall u,v\in V:\big|a(u,v)\big|\leq M\cdot\Vert u\Vert_V\cdot\Vert v\Vert_V$ und koerziv / $V$-elliptisch, d.h. $\exists\alpha>0:\forall v\in V:a(v,v)\geq\alpha\cdot\Vert v\Vert^2_V$\\
	\textbf{CS:} $|\int_\Omega\nabla u\cdot\nabla v\d x|\leq\vert u\vert_{1,2,\Omega}\cdot\vert v\vert_{1,2,\Omega}~\forall u,v\in W_0^{1,2}(\Omega)$ und $\left|\int\limits_\Omega u\cdot v\d x\right|\leq\Vert u\Vert_{0,2,\Omega}\cdot\Vert v\Vert_{0,2,\Omega}~\forall u,v\in W^{0,2}(\Omega)$\\
	\textbf{Lax-Milgram:} $V$ HR, $a$ stetige Bilinearform, $b$ stetige Linearform $\implies\exists!u\in V\colon\forall v\in V\colon a(u,v)=l(v)$.\\
	\textit{Beweis.} Trafo dualer Operator $Au:V\to\R,u\mapsto a(u,v)\leadsto A$ stetig und lineare auf $V\leadsto$ Konstruiere Kontraktion $\leadsto$ B. Fixpunksatz $\leadsto\square$\\
	\underline{Idee:} Approximation des Variationsproblems auf endlich dimensionalem Unterraum $V_h\subseteq V$.
	Auf $V_h$ gibt es wegen Lax-Milgram auch eindeutige Lösung.
	Das Variationsproblem auf $V_h$ lässt sich stets als LGS umschreiben und damit lösen:
	\begin{align*}
		\Big(a(u_n,v_h)=l(v_h)\qquad\forall v_h\in V_h\Big)\overset{\text{Basis}}&\Longleftrightarrow
		\Big(a(u_n,\varphi_i)=l(\varphi_i)\qquad\forall i\in\lbrace1,\ldots,n\rbrace\Big)\\
		u_n\overset{\text{Basis}}&=\sum\limits_{j=1}^n u_j\cdot\varphi_j &\exists&u_1,\ldots,u_n\in\R\\
		\implies\sum\limits_{j=1}^n a(\varphi_j,\varphi_i)\cdot u_j
		&=a\left(\sum\limits_{j=1}^n u_j\cdot\varphi_j,\varphi_i\right)	
		\overset{\text{Lin}}=l(\varphi_i)&\forall& i\in\lbrace1,\ldots,n\rbrace\\
		\overset{\text{LGS}}\implies
		&\left.\begin{array}{ll}
			A:=(a_{i,j})_{i,j=1,\ldots,n} &a_{i,j}:=a\big(\varphi_j,\varphi_i\big)\\
			b:=(b_i)_{i=1,\ldots,n} &b_i:=l(\varphi_i)\\
			u:=(u_j)_{j=1,\ldots,n}
		\end{array}\right\rbrace A\cdot u=b
	\end{align*}
	\textbf{Céas-Lemma:} $V$ HR, $V\subseteq V_h$ UVR, $a$ $M$-stetige $\alpha$-koerzive Linearform auf $V$ und $l$ stetige Linearform $\implies\exists C=\frac{M}{\alpha}>0:\Vert u-u_h\Vert_V\leq C\cdot\inf\limits_{v_h\in V_h}\Vert u-v_h\Vert_V$
	"Approximation $u_h$ von $u$ höchstens $\frac{M}{\alpha}$ schlechter als beste Approximation für $u$ im Raum $V_h$, quasi-optimal."\\
	\textit{Beweis.} stetige minus diskretes Problem ergibt \textbf{Galerkin Orthogonalität} 
	$a(u-u_h,v_h)=0\qquad v_h\in V_h$
	\begin{align*}
		\alpha\cdot\Vert u-u_h\Vert^2_V
		\overset{\text{koerziv}}&\leq
		a(u-u_h,u-u_h)\\
		&=a(u-u_h,u-v_v+v_h- u_h)\\
		\overset{\text{bilin}}&=
		a(u-u_h,u-v_h)+\underbrace{a(u-u_h,\underbrace{v_h-u_h}_{\in V_h})}_{\overset{\text{GO}}=0}\\
		\overset{\text{stetig}}&{\leq}
		M\cdot\Vert u-u_h\Vert_V\cdot\Vert u-v_h\Vert_V\\
		\implies
		\Vert u-u_h\Vert_V&\leq\frac{M}{\alpha}\cdot\Vert u-v_h\Vert_V\qquad\forall v_h\in V_h\\
		\implies
		\Vert u-u_h\Vert_V&\leq \frac{M}{\alpha}\cdot\inf\limits_{v_h\in V_h}\Vert u-v_h\Vert_V
	\end{align*}
	
	\section{Schwache Lösung}
	
	Hier: skalare, lineare PDE zweiter Ordnung\\
	homogene DB ist $u=0$ auf $\partial\Omega$, Neumann-Randbedingung ist $\frac{\partial u}{\partial n}=g$ auf $\partial\Omega,g\in L^2$, Mischformen\\
	Die \textbf{Konvektions-Diffusionsgl.} mit homo-Neumann-RB ist eindeutig schwach lösbar für $\alpha-\frac{1}{2}\div(a)\geq0$:
	\begin{align*}
	-\sum\limits_{i,j=1}^d\frac{\partial}{\partial x_i}\left(A_{i,j}(x)\cdot\frac{\partial u}{\partial x_j}\right)+\sum\limits_{i=1}^d a_i(x)\cdot\frac{\partial u}{\partial x_i}+\alpha(x)\cdot u&=f(x)\text{ in }\Omega\\
	\overset{\text{weak}}{\rightleftarrows}
	\int\limits_\Omega\nabla u^T\cdot A\cdot\nabla v+a\cdot\nabla u\cdot v+\alpha\cdot u\cdot v\d x&=\int\limits_\Omega f\cdot v\d x
	\end{align*}
	
	\section{Finite-Elemente-Räume}

	Für $\Omega=(0,1)\subseteq\R$, $k\in\N_0$ und $m\in\N_0$ und Zerlegung $\mathcal{T}_n:=\big\lbrace I_j:0\leq j\leq n\big\rbrace,~I_j:=\big(t_j,t_{j+1}\big)$, $h:=\max\limits_{0\leq j\leq n} h_j,h_j:=t_{j+1}-t_j$ definiere
	\begin{align*}
		S_h^{k,-1}&:=\Big\lbrace\varphi:[0,1]\to\R:\varphi\big|_{I_j}\in P_k,~0\leq j\leq n\Big\rbrace
		\quad\text{stückweise Grad-$k$-Polynome}\\
		S_h^{k,m}&:=S_h^{k,-1}\cap C^m\big([0,1]\big)\overset{\text{Glätte}}{\subseteq} H^{m+1}(\Omega)
		\quad\text{stückweise Grad-$k$-Polynome $m$-glatt}\\
		S_{h,0}^{k,m}&:=\Big\lbrace\varphi\in S_h^{k,m}:\varphi(0)=\varphi(1)=0\Big\rbrace
		\quad\ldots\text{ mit kompaktem Träger}
	\end{align*}
	Auch hier kann man das Problem wieder diskretisieren und durch Wahl einer Basis auf ein LGS gelangen (\textbf{Stiffness-Matrix)}. Approximationseigenschaft von 1DFEM: $\inf\limits_{v_h\in S_{h,0}^{k,0}}\big|u-v_h\big|_{1,2}\leq h^k\cdot|u|_{k+1,2}~\forall u$\\%\in H^{k+1}(\Omega)\cap H^{1}_0(\Omega)$
	2D-FEM: Sei $\Omega\subseteq\R^2$ beschränktes Polygon.
	Dies wird mittels \textbf{Triangulation} in offene disjunkte Teildreiecke $\T=\lbrace K_1,\ldots,K_N\rbrace$ zerlegt, $\overline{\Omega}=\bigcup_{i=1}^N\overline{K_i}$. 
	Die Zerlegung ist \textbf{zulässig}, falls $\overline{K_i}\cap\overline{K_j}$ leer, ein Eckpunkt oder eine gemeinsame Kante.
	Der Raum der stetigen stückweise linearen Funktionen ist\\
	$V_h:=\left\lbrace
	v\in C(\overline{\Omega}) : v|_{K_i}\in P_1(K_i)~\forall i=1,\ldots,N,\\
	v|_{\partial\Omega}=0\right\rbrace\subseteq H^1_0(\Omega)$\\
	Finites-Element ist $(K,V,\Sigma)$ mit $K\subseteq\R^d$ nichtleer, offen, beschränkt und Lipschitz-berandet; \\$V=\lbrace f:K\to\R\rbrace$ mit $\dim(V)=m<\infty$ Funktionenraum der \textbf{Formfunktionen} (z. B. Polynome); $\Sigma=\lbrace N_1,\ldots,N_m:V\to\R\rbrace$ mit \textbf{Nodal-Funktionalen / DOFs} $N_i$ (Knotenvariablen).
	Ist $\lbrace\varphi_1,\ldots,\varphi_m\rbrace$ Basis von $V$, so ist $\Sigma$ die Basis des Dualraums $V^\ast$, also $\forall i\in\lbrace 1,\ldots,m\rbrace:\exists! N_i\colon V\to\R:\forall j\in\lbrace1,\ldots,m\rbrace: N_i(\varphi_j)=\delta_{i,j}$\\
	Dies ist äquivalent zur \textbf{$V$-Unisolvenz}, d.h. $\forall\alpha_1,\ldots,\alpha_m\in\R:\exists! v\in V:N_1(v)=\alpha_1,\ldots,N_m(v)=\alpha_m$\\
	$\Sigma$ ist unisolvent $\gdw M:=(m_{i,j})$ invertierbar mit $m_{i,j}:=N_i(\varphi_i)$\\
	
	
	
	Familie $\lbrace\T_h\rbrace_h$ heißt \textbf{form-regulär} $:\gdw\exists\sigma>0:\forall h:\forall K\in\T_h:h_K\leq\sigma\cdot \rho_K$
	
	\section{Nichtkonforme Methoden}
	
	Nichtkonform bedeutet $V_h\not\in V$, konform bedeutet $V_h\subseteq V$.
	Besser für Probleme höherer Ordnung, sonst Probleme mit Einschränkungen.
	
	\section{A posteriori Fehlerschätzer}
	
	\section{Stromlinien-Diffusionsmethode (SDFEM)}
	
	Ziel: Stabileres Verfahren konstruieren, welches für Finite-Elemente beliebiger Ordnung genutzt werden kann mit höherer Konvergenzordnung
	
	\section{Gemischte Methoden}
	
	Problem: Es ist oftmals schwer Lösungen durch konforme Finite-Elemente-Räume zu approximieren.\\
	\textbf{Satz.} Für eine stetige Bilinearform $b:Q\times V\to\R$ mit zugehörigen Operatoren
	$B\colon V\to Q^\ast$ und $B^\ast\colon Q\to V^\ast$ sind die folgenden Bedingungen äquivalent:
	\begin{enumerate}[label=(\roman*)]
		\item $\inf\limits_{q\in Q\setminus\lbrace0\rbrace}\sup\limits_{v\in V\setminus\lbrace0\rbrace}\frac{b(q,v)}{\Vert v\Vert_V\cdot\Vert q\Vert_Q}\geq\beta$, die \textbf{inf-sup-Bedingung / LBB-Bedingung} ist 
		\item $\begin{aligned}
			B\colon W^\perp\to Q^\ast
		\end{aligned}$ ist ein Isomorphismus mit $\Vert B v\Vert_{Q^\ast}\geq\beta\cdot\Vert v\Vert_V\qquad\forall v\in W^\perp$
		\item $\begin{aligned}
			B^\ast\colon Q\to\big\lbrace g\in V^\ast:\langle g,v\rangle=0~\forall v\in W\big\rbrace
		\end{aligned}$ ist ein Isomorphismus mit $\big\Vert B^\ast q\big\Vert_{V^\ast}\geq\beta\cdot\Vert q\Vert_Q\qquad\forall q\in Q$
	\end{enumerate}
	\textbf{Theorem.} Seien $Q,V$ HR, $a\colon V\times V\to\R$ stetig, koerzive Bilinearform und $b\colon V\to\R$ stetige Bilinearform ???. Dann: LBB $\implies a(u,v)=f(v)$ hat für jedes $f\in V^\ast$ eindeutige Lösung\\
	\textbf{Lemma Fortin:} Angenommen stetige LBB. Dann: diskrete LBB $\Leftrightarrow\exists$ Familie von Interpolationsoperatoren $\lbrace I_h:V\to V_h\rbrace_h$ mit $b(q_h,v)=b(q_h,I_h v)\forall q_h\in Q_h,\forall v\in V$ und $\Vert I_h v\Vert_V\leq c\Vert v\Vert_V\forall v\in V$. Wähle meist Integralmittel als \textbf{Fortin-Operator}.
	
\end{document}