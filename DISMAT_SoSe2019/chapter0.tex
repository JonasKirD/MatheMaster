% !TEX root = DISM.tex
% This work is licensed under the Creative Commons
% Attribution-NonCommercial-ShareAlike 4.0 International License. To view a copy
% of this license, visit http://creativecommons.org/licenses/by-nc-sa/4.0/ or
% send a letter to Creative Commons, PO Box 1866, Mountain View, CA 94042, USA.
\chapter{Set Theory}
\section{Well-founded Orders}

\begin{definition}
    If 
    \begin{itemize}
        \item $X$ is a set
        \item $(X, < \subseteq X \times X)$ a binary relation
        \item $<$ is irreflexive: there is no $x \in X$, st. $x < x$
        \item $<$ is transitive: for all $x,y,z \in X$
        $$ x < y \text{ and } y < z \implies x < z.$$
    \end{itemize}
    Then we call $(X, <)$ a \define{strict partial order}.
\end{definition}

\begin{definition}
    If 
    \begin{itemize}
        \item $X$ is a set
        \item $Y \subseteq X$
        \item $y \in Y$.
    \end{itemize}
    Then 
    \begin{itemize}
        \item $y$ is a \define{smallest element} if $y \leq y'$ for every $y' \in Y$.
        \item $y$ is a \define{minimal element} if $y' < y$ for no $y' \in Y$.
        \item we define \define{biggest element} and \define{maximal element} in a similar way.
    \end{itemize}
\end{definition}

\begin{definition}
    If 
    \begin{itemize}
        \item $(X, <)$ is a strict partial order on $X$
        \item every subset $Y \subseteq X$ contains a minimal element.
    \end{itemize}
    Then we call $(X, <)$ \define{well-founded}. If furthermore $<$ is total we say $(X , <)$ is a \define{well-order}.
\end{definition}

\begin{remark}
    If $(X, <)$ is a strict partial order on $X$.
    Then 
    \begin{enumerate}[label=(\roman*)]
        \item $(X, <)$ is well-founded $\iff$ there is no decreasing chain in sequence in $X$
        \item $(X, <)$ is a well-order $\iff$ every non-empty subset of $X$ contains a smallest element.
    \end{enumerate}
\end{remark}

\begin{example}
    \begin{itemize}
        \item the usual order $<$ on $\N$ is a well-order
        \item the usual order $<$ on $\Z$ is a well-order
        \item the usual order $<$ on $\set{q \in \Q \mid q \geq 0}$ is not a well-order
    \end{itemize}
\end{example}

\section{Ordinal Numbers}

\begin{definition}
    If
    \begin{itemize}
        \item $X$ is a set
        \item for all $x \in X$
        $$ y \in x \implies y \in X \text{ (or equivalently $x \in X \implies x \subseteq X$)}.$$
    \end{itemize}     
    Then $X$ is called \define{transitive}.
\end{definition}

\begin{definition}
    If
    \begin{itemize}
        \item $(X , \in)$ is a well-order
        \item $X$ is transitive.
    \end{itemize}
    Then we call $X$ an \define{ordinal}.
\end{definition}

\begin{proposition}\label{prop:1}
    If $\alpha, \beta$ are ordinals.\\
    Then 
    \begin{enumerate}[label=(\arabic*)]
        \item $\emptyset$ is an ordinal
        \item if $\alpha \ne \emptyset$, then $\emptyset \in \alpha$
        \item $\alpha \notin \alpha$
        \item if $x \in \alpha$, then $x = \set{y \in \alpha \mid y \in x}$
        \item if $x \in \alpha$, then $x$ is an ordinal
        \item $\beta \subseteq \alpha$ $\iff$ $\beta \in \alpha$ or $\beta = \alpha$
        \item $s(\alpha) := \alpha \cup \set{\alpha}$.
    \end{enumerate}
\end{proposition}
\begin{proof}
    \emph{(1):} clear\\
    \emph{(2):} $\alpha \ne \emptyset \implies \alpha$ has a minimal element $x \in \alpha$.
    If $x = \emptyset$, then there is nothing to show. Suppose $x \ne \emptyset$, but then there is $y \in x$, whence
    $ y\in \alpha$, which contradicts the minimality of $x$.\\
    \emph{(3):} clear, since $\in$ is a strict order.\\
    \emph{(4):} this is a direct consequence of the transitivity of $\alpha$.\\
    \emph{(5):}
    \begin{itemize}[label=-]
        \item transitivity\\
        Let $z \in x$ and $y \in z$. Since $x \in \alpha$ and $z \in x$ $\implies z \in \alpha$ $\implies y \in \alpha$.\\
        As $\in$ is total on $\alpha$, it is
        $$ y \in x \text{ or } x \in y \text{ or } x=y. $$
        So if $x \in y \in z \in x$, we get $ x \in x$ which cannot be. The case of $x = y$ is similar. Hence we get $y \in x$.

        \item $\in$ is a well-order on $x$.
        $\in$ restricted to $x$ defines a total order, because $x \subseteq \alpha$. Let $y \subseteq x$, then $y \subseteq \alpha$.
        Hence $y$ has a minimal element.
    \end{itemize}
    \emph{(6):} One direction is a direct consequence of the transitivity of $\alpha$. So, suppose $\beta \subsetneq \alpha$. 
    Then $\alpha - \beta \ne \emptyset$, hence there is $x \in \alpha - \beta$ minimal.\\
    \emph{Claim.} $x = \beta$.
    \emph{$\subseteq$:} $y \in x \implies y \in \alpha \text{ and } y \notin \beta$ this contradicts to $x$ is minimal.\\
    \emph{$\supseteq$:} Suppose $y \in \beta - x$, since $\beta \subseteq \alpha, y \in \alpha$. Since $\in$ is total
    $$ \underbrace{y \in x}_{\text{impossible by assumption}} \text{ or } \underbrace{x \in y}_{\text{contradicts to $y \in \beta$ hence $x \in \beta$}} \text{ or } \underbrace{x = y}_{\text{contradiction, since $y \in \beta$, but $x \notin \beta$}}.$$
    \emph{(7):} 
    \begin{itemize}[label=-]
        \item $s(\alpha)$ is transitive.
        \begin{align*}
            x \in s(\alpha) & \iff x \in \alpha \cup \set{\alpha}\\
                & \iff x \in \alpha \text{ or } x = \alpha\\
                & \implies x \subseteq \alpha \text{ or } x = \alpha\\
                & \implies x \subseteq \alpha \subseteq s(\alpha)
        \end{align*}
        \item $\in$ is a well-order on $s(\alpha)$\\
        Clear.
    \end{itemize}
\end{proof}

\begin{notation}
    By induction on $n \in \N$, we define an ordinal $\underline{n}$:
    \begin{itemize}
        \item $\underline{0} := \emptyset$
        \item if $\underline{n}$ has been defined
        $$ \underline{n+1} := s(\underline{n}) = \underline{n} \cup \set{\underline{n}}.$$
    \end{itemize}
\end{notation}

\begin{remark}
    $$\underline{0} < \underline{1} < \ldots < \underline{n}.$$
    Usually we will identify $n$ and $\underline{n}$.
\end{remark}

\begin{definition}
    $$ \omega := \bigcup_{n \in \N}\underline{n} \text{ (which is an ordinal)}.$$
\end{definition}

\begin{definition}
    If $\alpha$ is an ordinal.\\
    Then 
    \begin{itemize}
        \item $\alpha$ is called a \define{successor} $:\iff$ there is some ordinal $\beta$, st. $\alpha = s(\beta)$.
        \item $\alpha$ is called a \define{limit ordinal} $:\iff$ $\alpha \ne \emptyset$ and $\alpha$ is not a successor.
        \item $\alpha$ is called \define{finite} $:\iff$ $\alpha$ is not a limit ordinal and none of its elements is a limit ordinal.
    \end{itemize}
\end{definition}

\begin{proposition}
    If $X \ne \emptyset$ is a set of ordinals.\\
    Then 
    \begin{enumerate}[label=(\alph*)]
        \item $\bigcap_{\alpha \in X} \alpha$ is an ordinal;
        \item $\bigcap_{\alpha \in X} \alpha$ is the smallest element of $X$.
    \end{enumerate}
\end{proposition}
\begin{proof} \emph{(a):}
    \begin{itemize}[label=-]
        \item $\bigcap_{\alpha \in X} \alpha$ is transitive (exercise)
        \item $\bigcap_{\alpha \in X} \alpha$ is well-ordered
         \begin{itemize}[label=-]
             \item total order: $\bigcap_{\alpha \in X} \alpha \subseteq \alpha$
             \item well-founded: similar.
         \end{itemize}
    \end{itemize}
    \emph{(b):} Contra positive. Suppose 
    $$ \beta = \bigcap_{\alpha \in X} \alpha \notin X. $$
    Since $\beta \subseteq \alpha$ for all $\alpha \in X$, by \cref{prop:1}.(5) and $\beta \ne \alpha$, $\beta \in \alpha$ for all $\alpha \in X$.
\end{proof}

\begin{theorem}
    If $\alpha, \beta$ ordinals.\\
    Then exactly one of the following holds
    $$\alpha \in \beta \text{ or } \beta \in \alpha \text{ or } \alpha = \beta.$$
\end{theorem}
\begin{proof}
    \begin{align*}
        & X = \set{\alpha, \beta} \text{ is a set of ordinals}\\
        \implies & \alpha \cap \beta \text{ is an ordinal and } \alpha \cap \beta \subseteq \alpha, \beta\\
        \implies & \alpha = \alpha \cap \beta \text{ or } \beta = \alpha \cap \beta\\
        \implies & \alpha \subseteq \beta \text{ or } \beta \subseteq \alpha\\
        \implies & \alpha \in \beta \text{ or } \alpha = \beta \text{ or } \beta \in \alpha
    \end{align*}
\end{proof}

\begin{notation}
    $\alpha < \beta$ means $\alpha \in \beta$.\\
    $\alpha \leq \beta$ means $\alpha \subseteq \beta$.
\end{notation}

\begin{exercise}
    If $\alpha$ is an ordinal.\\
    Then 
    \begin{enumerate}[label=(\roman*)]
        \item $\alpha$ is finite $\iff$ $\alpha < \omega$
        \item $\omega$ is the smallest limit ordinal.
    \end{enumerate}
\end{exercise}

\begin{proposition}
    If 
    \begin{itemize}
        \item $\lambda$ is an ordinal
        \item $\lambda$ is non-empty.
    \end{itemize}
    Then the following are equivalent:
    \begin{enumerate}[label=(\roman*)]
        \item $\lambda$ is a limit ordinal
        \item $\lambda = \bigcup\limits_{\alpha < \lambda} \alpha$.
    \end{enumerate}
\end{proposition}
\begin{proof}
    \emph{$(i) \implies (ii)$.} Let $\beta = \bigcup\limits_{\alpha < \lambda} \alpha$, which is an ordinal.\\
    To show: $\beta = \lambda.$\\
    ($\subseteq$) $\gamma \in \beta$, so there is $\alpha < \lambda$ such that $\gamma \in \alpha$. 
    By transitivity of $\lambda$, $\gamma \in \alpha \in \lambda \implies \gamma \in \lambda$.\\
    ($\supseteq$) From above, $\beta \subseteq \lambda$. Suppose for a contradiction $\beta \in \lambda$.
    Then $s(\beta) = \beta \cup \set{\beta} \subseteq \lambda$. Since $\lambda$ is limit $\lambda \ne s(\beta)$, 
    this implies $s(\beta) \in \lambda (s(\beta) < \lambda)$ is a contradiction, since 
    $$ \beta \in s(\beta) \subseteq \bigcup_{\alpha < \lambda} \alpha = \beta \implies \beta \in \beta. $$
    \emph{$(i) \implies (i)$.} Contra position. Suppose $\lambda = s(\gamma)$ for some ordinal $\gamma$.
    $$ \bigcup_{\alpha < \lambda} \alpha = \bigcup_{\alpha < s(\gamma)} \alpha = \bigcup_{\alpha \leq \gamma} \alpha
        = \bigcup_{\alpha \subseteq \gamma} \alpha = \gamma < s(\gamma) = \lambda.$$
\end{proof}

\begin{exercise}
    $\lambda$ is a limit ordinal $\iff$ there is a set of ordinals $A$, 
    st. for each $\alpha \in A$ there is a $\beta \in A$, st. $\alpha < \beta$ and 
    $$ \lambda = \bigcup_{\alpha \in A} \alpha.$$
\end{exercise}

\begin{lemma}
    If 
    \begin{itemize}
        \item $\alpha, \beta$ are ordinals
        \item $f: \alpha \to \beta$ is an strictly increasing map ($x<y \implies f(x) < f(y)$).
    \end{itemize}
    Then for all $\gamma \in \alpha$
    $$ \gamma \leq f(\gamma).$$
    In particular 
    \begin{itemize}
        \item $\alpha \leq \beta$
        \item $f$ is an isomorphism $ \implies \alpha = \beta$ and $f$ is the identify.
    \end{itemize}
\end{lemma}
\begin{proof}
    Consider the set
    $$ X = \set{\gamma \in \alpha \mid f(\gamma) < \gamma} \subseteq \gamma.$$
    If $X = \emptyset$ we are done. Let's suppose $X \ne \emptyset$ and let $\gamma \in X$ be minimal.
    Then $f(\gamma) < \gamma$, so since $\alpha$ is transitive, $f(\alpha) \in \alpha$.
    Then $f(f(\gamma)) < f(\gamma)$, this contradicts the minimality of $\gamma$, since $f(\gamma) \in X$ and $f(\gamma) < \gamma$.    
\end{proof}

\begin{theorem}\label{thm:1}
    If $X$ is well-ordered.\\
    Then $X$ is isomorphic (as a partial order) to an ordinal. Furthermore the ordinal and the isomorphism is unique.
\end{theorem}

\begin{theorem}[Transfinite induction]
    If
    \begin{itemize}
        \item $P$ is a property of ordinals
        \item $\emptyset$ satisfies $P$
        \item if $\alpha$ satisfies $P$, then $s(\alpha)$ also satisfies $P$
        \item if $\alpha$ is a limit ordinal and for all $\beta < \alpha$, $\beta$ satisfies $P$, then
        $\alpha$ satisfies $P$.
    \end{itemize}
    Then all ordinals satisfy $P$.
\end{theorem}

\begin{definition}
    If $\alpha, \beta$ are ordinals.
    Then 
    \begin{enumerate}[label=\arabic*)]
        \item $\alpha + \beta$ is defined by transfinite induction on $\beta$
        \begin{itemize}
            \item $\alpha + \emptyset : = \alpha$
            \item Assuming $\alpha + \beta$ has been defined
            $$\alpha + s(\beta) := s(\alpha + \beta)$$
            \item If $\beta$ is a limit ordinal, and $\alpha + \gamma$ has been defined for all $\gamma < \beta$
            $$ \alpha + \beta := \bigcup_{\gamma < \beta} \alpha + \gamma. $$
        \end{itemize}
    \end{enumerate}
\end{definition}

\section{Axiom of Choice}
\begin{recall}[Axiom of Choice]
    If 
    \begin{itemize}
        \item $I$ is a set
        \item $(X_i)_{i\in I}$ is a family of sets
        \item for all $i \in I$ $X_i \ne \emptyset$.
    \end{itemize}
    Then the axiom of choice states, that
    $$ \prod_{i \in I} X_i \ne \emptyset.$$
\end{recall}

\begin{recall}[Zorn's Lemma]
    If 
    \begin{itemize}
        \item $X$ is a set
        \item $\leq$ is a partial order on $X$
        \item $X \ne \emptyset$
        \item every chain $Y \subseteq X$ has an upper bound.
    \end{itemize}
    Then Zorn's lemma states, that $X$ has a maximal element.
\end{recall}

\begin{recall}[Zermelo's Theorem]
    Every set can be well-ordered.
\end{recall}

\begin{recall}
    Axiom of choice $\iff$ Zorn's lemma $\iff$ Zermelo's theorem.
\end{recall}

\section{Cardinal numbers}

\begin{definition}
    If $A,B$ are sets.\\
    Then 
    \begin{itemize}
        \item $A$ and $B$ are called \define{equinumerous} $:\iff$ there is a bijection $f:A \to B$.
        \item $A$ is \define{subnumerous then $B$} $:\iff$ there is an injection $g:A \to B$.
    \end{itemize}
\end{definition}

\begin{notation}
    If $A,B$ are equinumerous, we write $A \simeq B$.\\
    If $A$ is subnumerous then $B$, we write $A \preceq B$.
\end{notation}

\begin{remark}
    $\simeq$ is a equivalence relation.
\end{remark}

\begin{definition}
    If $A$ is a set.\\
    Then $A$ is \define{countable} $:\iff$ $A \preceq \N$.
\end{definition}

\begin{theorem}[Cantor]
    If $A$ is a set.\\
    Then $A \prec \mathcal{P}(A)$ ($A \preceq \mathcal{P}(A)$ but not $A \simeq \mathcal{P}(A)$).
\end{theorem}
\begin{proof}
    Let $f: A \to \mathcal{P}(A)$ be a function. Consider the set
    $$ B = \set{x \in A \mid \notin f(x)} \subseteq A.$$
    If $f$ is surjective, there is $x \in A$ st. $f(x) = B$.
    Then 
    $$ x \in f(x) = B \iff x \notin f(x), $$
    since this is a contradiction, no map $f$ is surjective. In particular $A \not \simeq \mathcal{P}(A)$.
    On the other hand
    $$\begin{cases}
        f: A \to \mathcal{P}(A)\\
        a \mapsto \set{a}
    \end{cases} \text{ is injective so } A \preceq \mathcal{P}(A).$$
\end{proof}

\begin{definition}
    If $\lambda$ is an ordinal.\\
    Then $\lambda$ is called a \define{cardinal} $:\iff$ $\lambda$ is not equinumerous to a smaller ordinal.
\end{definition}

\begin{theorem}[Cantor-Schröder-Bernstein]
    If
    \begin{itemize}
        \item $X,Y$ are sets
        \item $X \preceq Y$
        \item $Y \preceq X$.
    \end{itemize}
    Then $X \simeq Y$.
\end{theorem}

\begin{proposition}
    If $X$ is a set.\\
    Then $X$ is equinumerous to a unique cardinal, denoted by $\abs{X}$.
\end{proposition}
\begin{proof}
    By Zermelo's theorem, there is a well-order $\prec$ on $X$. By \labelcref{thm:1}, $(X, \prec)$ is order-isomorphic to $(\alpha, \in)$
    where $\alpha$ is an ordinal. In particular, $X \simeq \alpha$.\\
    Consider the set $$ Y = \set{\beta \in s(\alpha) \mid \beta \simeq \alpha}.$$
    So $Y \ne \emptyset$, since $\alpha \in Y$. Since $s(\alpha)$ is an ordinal, let $\lambda$ be the minimal element of $Y$.\\
    \emph{Claim.} $\lambda$ is an ordinal.

    Suppose not. Hence there is $\gamma < \lambda$, st. $\gamma \simeq \lambda$. 
    But $\lambda \simeq \alpha$, so $\gamma \simeq \alpha$, which contradicts to the minimality of $\lambda$.\\
    For uniqueness, if $X \simeq \kappa$ for $\kappa$ a cardinal, then $\lambda \simeq \kappa$, which implies $\lambda = \kappa$ by definition of cardinal.
\end{proof}
