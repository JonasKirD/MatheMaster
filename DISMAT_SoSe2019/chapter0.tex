% !TEX root = DISM.tex
% This work is licensed under the Creative Commons
% Attribution-NonCommercial-ShareAlike 4.0 International License. To view a copy
% of this license, visit http://creativecommons.org/licenses/by-nc-sa/4.0/ or
% send a letter to Creative Commons, PO Box 1866, Mountain View, CA 94042, USA.

\chapter{Set Theory}
\begin{definition}
    If $A,B$ are sets.\\
    Then 
    \begin{itemize}
        \item $A$ and $B$ are called \define{equinumerous} $:\iff$ there is a bijection $f:A \to B$.
        \item $A$ is \define{subnumerous then $B$} $:\iff$ there is an injection $g:A \to B$.
    \end{itemize}
\end{definition}

\begin{notation}
    If $A,B$ are equinumerous, we write $A \simeq B$.\\
    If $A$ is subnumerous then $B$, we write $A \preceq B$.
\end{notation}

\begin{remark}
    $\simeq$ is a equivalence relation.
\end{remark}

\begin{definition}
    If $A$ is a set.\\
    Then $A$ is \define{countable} $:\iff$ $A \preceq \N$.
\end{definition}

\begin{theorem}[Cantor]
    If $A$ is a set.\\
    Then $A \prec \mathcal{P}(A)$ ($A \preceq \mathcal{P}(A)$ but not $A \simeq \mathcal{P}(A)$).
\end{theorem}
\begin{proof}
    Let $f: A \to \mathcal{P}(A)$ be a function. Consider the set
    $$ B = \set{x \in A \mid \notin f(x)} \subseteq A.$$
    If $f$ is surjective, there is $x \in A$ s.t. $f(x) = B$.
    Then 
    $$ x \in f(x) = B \iff x \notin f(x), $$
    since this is a contradiction, no map $f$ is surjective. In particular $A \not \simeq \mathcal{P}(A)$.
    On the other hand
    $$\begin{cases}
        f: A \to \mathcal{P}(A)\\
        a \mapsto \set{a}
    \end{cases} \text{ is injective so } A \preceq \mathcal{P}(A).$$
\end{proof}

\begin{definition}
    If $\lambda$ is an ordinal.\\
    Then $\lambda$ is called a \define{cardinal} $:\iff$ $\lambda$ is not equinumerous to a smaller ordinal.
\end{definition}

\begin{theorem}[Cantor-Bernstein]
    If
    \begin{itemize}
        \item $X,Y$ are sets
        \item $X \preceq Y$
        \item $Y \preceq X$.
    \end{itemize}
    Then $X \simeq Y$.
\end{theorem}

\begin{notation}
    By induction on $n \in \N$, we define an ordinal $\underline{n}$:
    \begin{itemize}
        \item $\underline{0} := \emptyset$
        \item if $\underline{n}$ has been defined
        $$ \underline{n+1} := s(\underline{n}) = \underline{n} \cup \set{\underline{n}}.$$
    \end{itemize}
\end{notation}

\begin{remark}
    $$\underline{0} < \underline{1} < \ldots < \underline{n}.$$
    We let 
    $$ \omega := \bigcup_{n \in \N}\underline{n} \text{ (which is an ordinal)}.$$
    Usually we will identify $n$ and $\underline{n}$.
\end{remark}

\begin{definition}
    If $\alpha$ is an ordinal.\\
    Then 
    \begin{itemize}
        \item $\alpha$ is called a \define{successor} $:\iff$ there is some ordinal $\beta$, s.t. $\alpha = s(\beta)$.
        \item $\alpha$ is called a \define{limit ordinal} $:\iff$ $\alpha \ne \emptyset$ and $\alpha$ is not a successor.
        \item $\alpha$ is called \define{finite} $:\iff$ $\alpha$ is not a limit ordinal and none of its elements is a limit ordinal.
    \end{itemize}
\end{definition}

\begin{exercise}
    If $\alpha$ is an ordinal.\\
    Then 
    \begin{enumerate}[label=(\roman*)]
        \item $\alpha$ is finite $\iff$ $\alpha < \omega$
        \item $\omega$ is the smallest limit ordinal.
    \end{enumerate}
\end{exercise}

\begin{proposition}
    If 
    \begin{itemize}
        \item $\lambda$ is an ordinal
        \item $\lambda$ is non-empty.
    \end{itemize}
    Then the following are equivalent:
    \begin{enumerate}[label=(\roman*)]
        \item $\lambda$ is a limit ordinal
        \item $\lambda = \bigcup\limits_{\alpha < \lambda} \alpha$.
    \end{enumerate}
\end{proposition}
\begin{proof}
    \emph{$(i) \implies (ii)$.} Let $\beta = \bigcup\limits_{\alpha < \lambda} \alpha$, which is an ordinal.\\
    To show: $\beta = \lambda.$\\
    ($\subseteq$) $\gamma \in \beta$, so there is $\alpha < \lambda$ such that $\gamma \in \alpha$. 
    By transitivity of $\lambda$, $\gamma \in \alpha \in \lambda \implies \gamma \in \lambda$.\\
    ($\supseteq$) From above, $\beta \subseteq \lambda$. Suppose for a contradiction $\beta \in \lambda$.
    Then $s(\beta) = \beta \cup \set{\beta} \subseteq \lambda$. Since $\lambda$ is limit $\lambda \ne s(\beta)$, 
    this implies $s(\beta) \in \lambda (s(\beta) < \lambda)$ is a contradiction, since 
    $$ \beta \in s(\beta) \subseteq \bigcup_{\alpha < \lambda} \alpha = \beta \implies \beta \in \beta. $$
    \emph{$(i) \implies (i)$.} Contra position. Suppose $\lambda = s(\gamma)$ for some ordinal $\gamma$.
    $$ \bigcup_{\alpha < \lambda} \alpha = \bigcup_{\alpha < s(\gamma)} \alpha = \bigcup_{\alpha \leq \gamma} \alpha
        = \bigcup_{\alpha \subseteq \gamma} \alpha = \gamma < s(\gamma) = \lambda.$$
\end{proof}

\begin{exercise}
    $\lambda$ is a limit ordinal $\iff$ there is a set of ordinals $A$, 
    s.t. for each $\alpha \in A$ there is a $\beta \in A$, s.t. $\alpha < \beta$ and 
    $$ \lambda = \bigcup_{\alpha \in A} \alpha.$$
\end{exercise}

\begin{lemma}
    If 
    \begin{itemize}
        \item $\alpha, \beta$ are ordinals
        \item $f: \alpha \to \beta$ is an strictly increasing map ($x<y \implies f(x) < f(y)$).
    \end{itemize}
    Then for all $\gamma \in \alpha$
    $$ \gamma \leq f(\gamma).$$
    In particular 
    \begin{itemize}
        \item $\alpha \leq \beta$
        \item $f$ is an isomorphism $ \implies \alpha = \beta$ and $f$ is the identify.
    \end{itemize}
\end{lemma}
\begin{proof}
    Consider the set
    $$ X = \set{\gamma \in \alpha \mid f(\gamma) < \gamma} \subseteq \gamma.$$
    If $X = \emptyset$ we are done. Let's suppose $X \ne \emptyset$ and let $\gamma \in X$ be minimal.
    Then $f(\gamma) < \gamma$, so since $\alpha$ is transitive, $f(\alpha) \in \alpha$.
    Then $f(f(\gamma)) < f(\gamma)$, this contradicts the minimality of $\gamma$, since $f(\gamma) \in X$ and $f(\gamma) < \gamma$.    
\end{proof}

\begin{theorem}\label{thm:1}
    If $X$ is well-ordered.\\
    Then $X$ is isomorphic (as a partial order) to an ordinal. Furthermore the ordinal and the isomorphism is unique.
\end{theorem}

\begin{recall}
    The axiom of choice:\\
    If 
    \begin{itemize}
        \item $I$ is a set
        \item $(X_i)_{i\in I}$ is a family of sets
        \item for all $i \in I$ $X_i \ne \emptyset$.
    \end{itemize}
    Then the axiom of choice states, that
    $$ \prod_{i \in I} X_i \ne \emptyset.$$
\end{recall}

\begin{recall}
    The Zorn's lemma:\\
    If 
    \begin{itemize}
        \item $X$ is a set
        \item $\leq$ is a partial order on $X$
        \item $X \ne \emptyset$
        \item every chain $Y \subseteq X$ has an upper bound.
    \end{itemize}
    Then Zorn's lemma states, that $X$ has a maximal element.
\end{recall}

\begin{recall}
    Zermelo's theorem: Every set can be well-ordered.\\
    Theorem: Axiom of choice $\iff$ Zorn's lemma $\iff$ Zermelo's theorem.
\end{recall}

\begin{proposition}
    If $X$ is a set.\\
    Then $X$ is equinumerous to a unique cardinal.
\end{proposition}
\begin{proof}
    By Zermelo's theorem, there is a well-order $\prec$ on $X$. By \labelcref{thm:1}, $(X, \prec)$ is order-isomorphic to $(\alpha, \in)$
    where $\alpha$ is an ordinal. In particular, $X \simeq \alpha$.\\
    Consider the set $$ Y = \set{\beta \in s(\alpha) \mid \beta \simeq \alpha}.$$
    So $Y \ne \emptyset$, since $\alpha \in Y$. Since $s(\alpha)$ is an ordinal, let $\lambda$ be the minimal element of $Y$.\\
    \emph{Claim.} $\lambda$ is an ordinal.

    Suppose not. Hence there is $\gamma < \lambda$, s.t. $\gamma \simeq \lambda$. 
    But $\lambda \simeq \alpha$, so $\gamma \simeq \alpha$, which contradicts to the minimality of $\lambda$.\\
    For uniqueness, if $X \simeq \kappa$ for $\kappa$ a cardinal, then $\lambda \simeq \kappa$, which implies $\lambda = \kappa$ by definition of cardinal.
\end{proof}

\begin{theorem}[Transfinite induction]
    If
    \begin{itemize}
        \item $P$ is a property of cardinals
        \item $\emptyset$ satisfies $P$.
    \end{itemize}
    Then 
    \begin{enumerate}[label=\arabic*)]
        \item if $\alpha$ satisfies $P$, then $s(\alpha)$ also satisfies $P$
        \item if $\alpha$ is a limit ordinal and for all $\beta < \alpha$, $\beta$ satisfies $P$, then
        $\alpha$ satisfies $P$.
    \end{enumerate}
\end{theorem}

\begin{definition}
    If $\alpha, \beta$ are ordinals.
    Then 
    \begin{enumerate}[label=\arabic*)]
        \item $\alpha + \beta$ is defined by transfinite induction on $\beta$
        \begin{itemize}
            \item $\alpha + \emptyset : = \alpha$
            \item Assuming $\alpha + \beta$ has been defined
            $$\alpha + s(\beta) := s(\alpha + \beta)$$
            \item If $\beta$ is a limit ordinal, and $\alpha + \beta$ has been defined for all $\gamma < \beta$
            $$ \alpha + \beta := \bigcup_{\gamma < \beta} \alpha + \gamma. $$
        \end{itemize}
    \end{enumerate}
\end{definition}