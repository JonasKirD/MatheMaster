\section{Multimesh concept}

\begin{example}
	Navier-Stokes-Equation:
	\begin{align*}
		\partial_t u + u \cdot \nabla u &= \nabla p + \nu \laplace u \\
		\nabla \cdot u = 0 
	\end{align*}
	with $u = (u_1,u_2,u_3)^T$ as velocity, $p$ as pressure and $\nu$ as viscosity and proper intial and boundary conditions
\end{example}

\begin{example}
	Hahn-Hilliard-Equation(phase field moduls)\\
	(e.g. seperation of two liquids)
	\begin{align*}
	\partial_t u &= \laplace u \\
	\mu &= -\varepsilon \laplace u + \frac{1}{\varepsilon}w'(u) 
	\end{align*}
	with $u$ as concentration(phase variable), $\mu$ as chemical potential and $w(u) = \frac{1}{4} (1-u^2)^2$ as the double-well potential and a small parameter $\varepsilon$.
	%TODO add picture
\end{example}
\textbf{eventuell fehler in der obigen Gleichung}
\begin{example}
	for illustration , biharmonic equation
	\begin{align*}
	\laplace^2 u &= 0 \tin \Omega \\
	u &= 0 \ton \partial\Omega\\
	\frac{\partial u}{\partial n} &=0 \ton \partial \Omega
	\end{align*}
	classical solution: $u \in C^4(\Omega)\cap C^1(\bar{\Omega})$\nl
	do operator splitting and rewrite as a system:
	\begin{align*}
		\begin{rcases}
			-\laplace u + v &=0\\
			\laplace v &= 0
		\end{rcases} \tin \Omega
	\end{align*}
	variational/weak formulation: Find $(u,v) \in H^1_0(\Omega)\times H^1(\Omega)$
	\begin{align*}
		\int \limits_{\Omega} \nabla u \cdot \nabla \varphi + \int \limits_{\Omega} v\varphi &=0 \qquad \forall \varphi \in H^1\\
		\int \limits_{\Omega} \nabla v \cdot \nabla \psi &=0 \qquad \forall \psi \in H^1_0
	\end{align*}
	assume $\T^0_h, \T^1_h$ are different partitions of $\Omega$
	\begin{align*}
		V^0_h &= \left\{ v_h \in H^1: v_h|_T \in P_{\alpha},\ \forall T \in \T^0_h  \right\}\\
		V^1_h &= \left\{ v_h \in H^1_0: v_h|_T \in P_{\alpha},\ \forall T \in \T^1_h  \right\}
	\end{align*}
	$\implies$ discrete setting
	\begin{align*}
		\int \limits_{\Omega} \nabla u_h \cdot \nabla \varphi + \int \limits_{\Omega} v_h\varphi &=0 \qquad \forall \varphi \in V^0_h\\
		\int \limits_{\Omega} \nabla v_h \cdot \nabla \psi &=0 \qquad \forall \psi \in V^1_h
	\end{align*}
	Ansatz: 
	\begin{align*}
		u_h = \sum  \limits_{i=1}^{n} u_i\varphi_i,\quad v_h= \sum \limits_{i=1}^{m} v_i\psi_i
	\end{align*}
	with $\{\varphi_i,1\leq i \leq n \}$,$\{ \psi_i,\ 1 \leq i \leq m \}$ as basis of $V^0_h$ and $V^1_h$.
	\begin{align*}
		\sum \limits_{j=1}^n u_j \left( \sum \limits_{T \in \T^0_h} \ \int \limits_T \nabla \varphi_j \cdot \nabla \varphi_i \right) + \sum \limits_{j=1}^m v_j \left( \sum \limits_{T \in \T^0_h\cup \T^1_h} \ \int \limits_T \nabla \psi_j  \varphi_i \right) &=0 \quad \forall i = 1, \dots,n\\
		\sum \limits_{j=1}^m v_j \left( \sum \limits_{T \in \T^1_h} \ \int \limits_T \nabla \psi_j \cdot \nabla \psi_i \right) &=0 \quad \forall i = 1, \dots,m
	\end{align*}
	The problem is in $\int \limits_T \nabla \psi_j  \varphi_i$. There is a need to use the union of two partitions. For this we need the following assumptions on meshes:\nl
	any element $T^0 \in \T^0_h$ is either a subelement of an element $T^1 \in \T^1_h$ or vice versa.
	%TODO add picture
	This is fullfilled if e.g. bisection is used starting from the same coarse mesh.\nl
	$\implies \T^0_h \cup \T^1_h$ is union of locally finest simplices. The integral is typically computed by using local basisfunctions.\nl
	$\varphi_{i,j}$: $j$-th local basisfunction on $T_i \in \T^0_h$\\
	$\psi_{i,j}$: $j$-th local basisfunction on $T_i \in \T^1_h$\nl
	there are two cases: integral is evaluated on element of $\T^0_h$ or $\T^1_0$.
	\begin{align*}
		\int \limits_{T_i  \in \T^0_h} \psi_{k,l} \varphi_{i,j}
	\end{align*}
	for so,e $l,j$ with elemtn $T_k \in \T^1_h$ with $T_i \subset T_k$. $\implies$ $\psi_{k,l}$ is not a local basisfunction on $T_i$.\nl
	Idea: define $\psi_{k,l}$ as linear combination of local basisfunctions of $T_i$.
	\begin{align*}
		\int \limits_{T \in \T^0_h} \psi_{k,l}\varphi_{i,j} = \int \limits_{T \in \T^0_h} \sum \limits_m \left( c_{k,m} \varphi_{i,m} \right)\varphi_{i,j}
	\end{align*}
	with coefficients $c_{k,m}$. For the other case: 
	\begin{align*}
	\int \limits_{T \in \T^1_h} \psi_{k,l}\varphi_{i,j} = \int \limits_{T \in \T^0_h} \sum \limits_m  \psi_{k,l} \left( c_{k,m} \psi_{i,m} \right)
	\end{align*}
	For the assembly step we use a mesh traverse algorithm which creates requested elements on demand, traverses two meshes in parallel and create union in virtual way.
	% TODO add picture
	% TODO add second picture 
	($\varphi = \psi_0 + \frac{1}{2}\psi_1$)\\
	element matrix $M_{T'}$:(somewhere in this example the dimension of the basis got changed...) 
	\begin{align*}
		M_{T'}&= 
		\begin{pmatrix}
		\int \limits_{T'} \psi_0\varphi_0 & \dots & \int \limits_{T'} \psi_0\varphi_n \\
		\vdots 					 		  & 	  & \vdots\\
		\int \limits_{T'} \psi_n\varphi_0 & \dots & \int \limits_{T'} \psi_n\varphi_n
		\end{pmatrix}\\
		&=
		\begin{pmatrix}
		\int \limits_{T'} \sum \limits_i \left( c_{0i} \varphi_i \right)\varphi_0 & \dots & \int \limits_{T'} \sum \limits_i\left( c_{0i} \varphi_i \right)\varphi_n \\
		\vdots 					 		  & 	  & \vdots\\
		\int \limits_{T'} \sum \limits_i\left( c_{ni} \varphi_i \right)\varphi_0 & \dots & \int \limits_{T'} \sum \limits_i\left( c_{ni} \varphi_i \right)\varphi_n
		\end{pmatrix}\\
		&= C 
		\begin{pmatrix}
		\int \limits_{T'} \varphi_0\varphi_0 & \dots & \int \limits_{T'} \varphi_0\varphi_n \\
		\vdots 					 		  & 	  & \vdots\\
		\int \limits_{T'} \varphi_n\varphi_0 & \dots & \int \limits_{T'} \varphi_n\varphi_n
		\end{pmatrix}
	\end{align*}
	the other case leads to $M_{T'} = (\dots) C^T$.\\
	(PhD thesis Sigi Ling 2016 )
\end{example}
