% This work is licensed under the Creative Commons
% Attribution-NonCommercial-ShareAlike 4.0 International License. To view a copy
% of this license, visit http://creativecommons.org/licenses/by-nc-sa/4.0/ or
% send a letter to Creative Commons, PO Box 1866, Mountain View, CA 94042, USA.

\chapter{Algebraische Funktionenkörper}
\section{Transzendente Körpererweiterungen}
Sei $K$ ein Körper und $F|K$ eine Körpererweiterung.

\begin{definition}
    Der \textbf{rationale Funktionenkörper} über $K$ ist
    $$ K(T) := \text{Quot}(K[T]). $$
\end{definition}

\begin{bemerkungnr}
    Ist $t \in F$ transzendent über $K$, so ist $K(t) \cong K(T)$.
    
    Umgekehrt ist $T$ als Element von $K(T)$ transzendent über $K$.
\end{bemerkungnr}

\begin{definition}
    $F$ ist ein \textbf{algebraischer Funktionenkörper einer Variablen} über $K$, 
    wenn es $t \in F$ transzendent über $K$ gibt mit $[F:K(t)] < \infty$.
\end{definition}

\begin{bemerkungnr}
    \begin{enumerate}[label=\alph*)]
        \item Ein Funktionenkörper über $K$ ist also $K$-isomorph zu einer endlichen Erweiterung des $K(T)$.
        
        \item Ist $F$ ein Funktionenkörper über $K$ und $E|F$ eine endliche Körpererweiterung von $F$, 
        so ist auch $E$ ein Funktionenkörper über $K$.

        \item Ist $F$ ein Funktionenkörper über $K$ und $K \subseteq L \subseteq F$ ein Zwischenkörper mit 
        $L|K$ algebraisch, so ist $F$ auch Funktionenkörper über $L$.
    \end{enumerate}
\end{bemerkungnr}

\begin{beispiel}
    \begin{enumerate}[label=\alph*)]
        \item $\Q(T)(\sqrt{-1}) = \Q(\sqrt{-1})(T)$\\
        ist ein rationaler Funktionenkörper über $\Q(\sqrt{-1})$. 
        
        \item $\Q(T)(\sqrt{T^2+1})$ ist ein Funktionenkörper über $\Q$.

        \item $\Q(T)(\sqrt{T+1}) = \Q(\sqrt{T+1})$ ist ein rationaler Funktionenkörper über $\Q$ 
    \end{enumerate}
\end{beispiel}

\begin{definition}
    Ist $F$ Funktionenkörper über $K$, 
    so ist der relative algebraische Abschluss $\widetilde{K}$ von $K$ in $F$ der 
    \textbf{Konstantenkörper} von $F$, d.h.
    $$ \widetilde{K} := \{\alpha \in F \| \alpha \text{ algebraisch  über } K\}.$$
\end{definition}

\begin{beispiel}\label{beispiel:1.7}
    Der Konstantenkörper des rationalen Funktionenkörpers $K(T)$ ist $K$:

    Ist $f(T) = \frac{g(T)}{h(T)} \in K(T) \setminus K$, mit $g,h \in K[T]$, algebraisch über $K$,
    so existiert $p(X) \in K[X]$ mit $p(f(T)) = 0$. Dann gilt
    \begin{align*}
        p(f(T)) &= \sum\limits_{i=0}^n a_i f(T)^i = 0\\
        & \implies \sum\limits_{i=0}^n a_i \frac{g(T)^i}{h(T)^i} = 0\\
        & \implies \sum\limits_{i=0}^n a_i g(T)^i h(T)^{n-i} = 0 \\
        & \implies T \text{ algebraisch über } K \text{\Lightning zu $T$ transzendent über $K$}
    \end{align*}
    wobei $a_i \in K$ die Koeffizienten von $p$ sind.
\end{beispiel}

\begin{erinnerung}
    Ist $E|F$ eine algebraische Körpererweiterung und $R$ ein Ring. So gilt
    $$ F \subseteq R \subseteq E \implies R \text{ ist Körper.} $$
\end{erinnerung}
\begin{proof}
    \begin{align*}
        \alpha \in R & \stackrel{\alpha \text{ alg.}}{\implies} \exists a_0,\ldots, a_{n-1} \in F: \alpha^n + a_{n-1}\alpha^{n-1}+\ldots+a_0 = 0\\
        & \implies \alpha^{-1} = -\frac{1}{a_0}\left(\alpha^{n-1} + a_{n-1}\alpha^{n-2} + \ldots + a_1\right) \in F[\alpha] \subseteq R
    \end{align*}
\end{proof}

\begin{bemerkung}
    Es folgt also insbesondere für $\alpha$ algebraisch über $F$:
    $$ F[\alpha] = F(\alpha). $$
\end{bemerkung}

\begin{lemma}\label{lemma:finiteExp}
    Ist $L|K$ eine endliche Körpererweiterung,
    so ist $$ [L(T):K(T)] = [L:K]. $$
\end{lemma}
\begin{proof}
    Sei $x_1, \ldots, x_n$ eine Basis von $L$, d.h. 
    $$ L = \bigoplus_{i=1}^n Kx_i. $$
    Setze $ M := \sum\limits_{i=1}^n K(T)x_i$. Da $1 \in L$ gibt es $\lambda_1, \ldots \lambda_n \in K$, sodass
    $$ 1 = \sum_{i=1}^n \lambda_i x_i 
        \implies \forall f \in K(T)\colon f=\sum_{i=1}^n f\lambda_ix_i 
        \implies K(T) \subsetneqq \sum_{i=1}^n K(T)x_i. $$
    Also gilt
    $$ K(T) \subsetneqq \sum_{i=1}^n K(T)x_i \subseteq L(T). $$
    \begin{itemize}
        \item \textit{$M$ ist Teilring von $L(T)$:}
        Offenbar enthält $M$ die $0$ und die $1$. Seien $x,y \in M$. Es gilt
        \begin{align*}
            x\cdot y &= \left(\sum_{i=1}^n \lambda_i x_i \right)\cdot\left(\sum_{i=1}^n \mu_i x_i\right) 
                = \sum_{i,j=1}^n \lambda_i \mu_j \underbrace{x_i \cdot x_j}_{\in L}
                =\sum_{i,j=1}^n \lambda_i \mu_j \sum_{k=1}^n \kappa_{i,j,k} x_k\\
                &= \sum_{k=1}^n \underbrace{\left(\sum_{i,j=1}^n \lambda_i \mu_j \kappa_{i,j,k}\right)}_{\in K(T)} x_k \in M\\
            x + y &= \left(\sum_{i=1}^n \lambda_i x_i \right)+\left(\sum_{i=1}^n \mu_i x_i\right)
                = \sum_{i=1}^n \underbrace{(\lambda_i + \mu_i)}_{\in K(T)}x_i \in M
        \end{align*}

        \item \textit{$L(T)|K(T)$ ist algebraisch:} Aus $L(T) = K(L)(T) = K(T)(L)$ folgt die Behauptung, 
        da $L$ algebraisch über $K$ und damit auch über $K(T)$.

        \item \textit{$M$ ist Körper:} Da $L(T)$ algebraisch über $K(T)$ und $K(T) \subseteq M \subseteq L(T)$ 
        folgt aus der Erinnerung die Behauptung.
    \end{itemize}

    Da $M$ ein Körper ist und $L \subseteq M$ sowie $K(T) \subseteq M$ folgt aus $L(T) = K(T)(L)$
    $$ L(T) \subseteq M \implies M = L(T) \implies [L(T):K(T)] \leq [L:K].$$

    Sei also $x_1,...,x_n \in L$ $K$-linear unabhängig. Wir zeigen nun, dass dann
    $x_1,...,x_n \in L(T)$ $K(T)$-linear unabhängig ist.
    
    Angenommen $$ \sum\limits_{i=1}^n f_i(T)x_i = 0$$ mit $f_i(T) \in K(T)$, nicht alle 0.
    O.E. seien $f_i(T) \in K[T]$ und es gibt keinen nicht-konstanten Teiler aller $f_i$.
    Dann folgt es existiert $i$ und $\tau \in K$ mit $f_i(\tau) \neq 0$. 
    Damit gilt also $$ \sum\limits_{i=1}^n f_i(\tau)x_i = 0.$$
    Da $x_1,...,x_n$ $K$-linear unabhängig folgt aber $f_i(\tau) = 0$ für alle $i$.

\end{proof}

\begin{satz}\label{satz:1.9}
    Ist $F$ ein Funktionenkörper über $K$ mit Konstantenkörper $\widetilde{K}$,
    so ist $\widetilde{K}|K$ endlich.
\end{satz}
\begin{proof}
    Ist $t\in F$ transzendent über $K$ mit $F|K(t)$ endlich,
    so ist wegen \cref{lemma:finiteExp} 
    $$ [\widetilde{K}:K] = [\widetilde{K}(t):K(t)] \leq [F:K(t)] < \infty.$$
\end{proof}

\begin{definition}
    \begin{enumerate}[label=\arabic*)]
        \item $a_1,...,a_n \in F$ \textbf{algebraisch unabhängig} über $K$,\\
        wenn für alle $f \in K[X_1,...,X_n]$:
        $$ f(a_1,...,a_n) = 0 \implies f=0. $$

        \item $(a_i)_{i \in I}$ Familie in $F$ ist \textbf{algebraisch unabhängig},
        wenn für alle endlichen Teilmengen $J \subseteq I$:
        $ (a_i)_{i \in J}$ algebraisch unabhängig über $K$.
    \end{enumerate}
\end{definition}

\begin{beispiel}
    $a$ ist algebraisch unabhängig über $K$ genau dann, wenn $a$ ist transzendent über $K$.
\end{beispiel}

\begin{definition}
    $\X = (x_i)_{i \in I}$ ist \textbf{Transzendenzbasis} von $F|K$, 
    wenn $\X$ algebraisch unabhängig über $K$ ist und $F|K(\X)$ algebraisch.
\end{definition}

\begin{satz}
    $\X = (x_i)_{i \in I}$ ist genau dann Transzendenzbasis von $F|K$,
    wenn $\X$ maximal algebraisch unabhängig in $F$ über $K$ ist.
    
    Insbesondere existiert ein Transzendenzbasis von $F|K$.
\end{satz}

\begin{proof}
    \begin{itemize}[label=]
        \item \underline{$\Leftarrow$:} $x \in F \setminus K(\X) \implies \X \cup \{x\}$ ist algebraisch abhängig, 
        d.h es existieren $i_1,...,i_n$ und $f \in K[X_1,...,X_n, X] \setminus \{0\}$ mit 
        $$ f(x_{i_1},...,x_{i_n},x) = 0. $$
        Da $x_{i_1},...,x_{i_n}$ algebraisch unabhängig ist $\deg_X f > 0$ 
        und somit ist $x$ algebraisch über $K(x_{i_1},...,x_{i_n})$.
        
        \item \underline{$\Rightarrow$:} $x \in F \setminus \X \implies x$ algebraisch über $K(\X)$,
        d.h. es gibt $0 \neq f \in K(\X)[X]$ mit $f(x) = 0$. 
        $\implies$ es existiert $0\neq g \in K[X_1,...,X_n,X]$ und $i_1,...,i_n \in I$ mit
        $g(x_{i_1},...,x_{i_n},x) = 0$ somit sind $x_{i_1},...,x_{i_n},x$ algebraisch abhängig.
    \end{itemize}
\end{proof}

\begin{lemma}[Austauschlemma]\label{lemma:exchange}
    Ist $\Y=(y_j)_{j \in J}$ mit $F|K(\Y)$ algebraisch und \\
    $\X = (x_i)_{i \in I}$ algebraisch unabhängig,
    so existiert $J_0 \subseteq J$ mit $\X \cup (y_j)_{j\in J_0}$ Transzendenzbasis von $F|K$.
\end{lemma}
\begin{proof}
    Nach dem Lemma von Zorn existiert $J_0 \subseteq J$ maximal mit 
    $\X^{'} := \X \cup (y_j)_{j \in J_0}$ algebraisch unabhängig.
    Für jedes $j \in J$ ist dann $y_j$ algebraisch über $K(\X^{'})$\\
    $\implies K(\X\cup \Y)$ algebraisch über $K(\X^{'})$
    Somit ist $\X^{'}$ Transzendenzbasis.
\end{proof}

\begin{satz}[Steinitz, 1910]
    Je zwei Transzendenzbasen von $F|K$ haben die gleiche Kardinalität.
\end{satz}

\begin{proof}[Beweis nur für den endlichen Fall]
    Seien $\X=(x_i)_{i \in I}$ und $\Y = (y_j)_{j \in J}$ Transzendenzbasen von $F|K$.\\
    Wegen Symmetrie ist nur zu zeigen
    $$ |I| \leq |J|. $$
    Induktion nach $n = |J|$:
    \begin{itemize}[label=]
        \item \underline{$n=0$:} klar, $F|K$ algebraisch.
        \item  \underline{$n-1 \to n$:} $F|K$ ist nicht algebraisch 
        $\implies |I| > 0$, O.E. $1 \in I$. Nach \cref{lemma:exchange} existiert $J_0 \subseteq J$ mit
        $\{x_1\} \cup (y_j)_{j\in J_0}$ Transzendenzbasis. 
        Da $\Y$ maximal algebraisch unabhängig ist, ist $|J_0| \leq |J| -1$.
        Sowohl $\X$ als auch $\{x_1\} \cup (y_j)_{j \in J_0}$ sind Transzendenzbasen von $F|K$, die $x_1$ enthalten.
        Somit sind sowohl $\X^{'}:= (x_i)_{i \in I\setminus \{1\}}$ als auch $(y_j)_{j \in J_0}$ Transzendenzbasen von $F|K(x_1)$.
        \begin{align*}
            &\implies |I \setminus\{1\}| \leq |J_0| \\
            &\implies |I| -1 \leq |J| -1 \\
            &\implies |I| \leq |J|            
        \end{align*}
    \end{itemize}
\end{proof}

\begin{definition}
    Der \textbf{Transzendenzgrad} von $F|K$, 
    $$ \trdeg(F|K) $$ ist die Mächtigkeit einer Transzendenzbasis.
\end{definition}

\begin{korollar}
    Sind $K \subseteq L \subseteq F$ Körper, so ist 
    $$ \trdeg(F|K) = \trdeg(L|K) + \trdeg(F|L). $$
\end{korollar}
\begin{proof}
    Ist $\mathfrak{X}$ Transzendenzbasis von $L|K$ und $\mathcal{Y}$ Transzendenzbasis von $F|L$, dann folgt
    $ \mathfrak{X} \cup \mathcal{Y}$ ist Transzendenzbasis von $F|K$.
    $$\begin{tikzcd}
        & K(\mathfrak{X} \cup \mathcal{Y})(L) & \\
        K(\mathfrak{X} \cup \mathcal{Y}) \rar[dash]{algebraisch} & L(\mathcal{Y}) \uar[dash] \rar[dash]{algebraisch} & F \\
        K(\mathfrak{X}) \uar[dash] \rar[dash]{algebraisch}& L \uar[dash] & \\
        \mathfrak{X} \uar[dash] & &
    \end{tikzcd}$$
\end{proof}

\begin{satz}
    $F$ ist genau dann Funktionenkörper über $K$, wenn $F|K$ endlich erzeugt ist und 
    $$ \trdeg(F|K) = 1. $$
\end{satz}
\begin{proof}
    \begin{itemize}[label=]
        \item \underline{$\implies$:}
        \begin{align*}
            \left[F:K(t)\right] = n < \infty \implies \trdeg(F|K) & = \trdeg(F|K(t)) + \trdeg(K(t)|K)\\
            & = 0 + 1 = 1,
        \end{align*}
        denn $F|K(t)$ ist endlich, also algebraisch.
        Sei $b_1,...,b_n$ eine Transzendenzbasis von $F|K(t)$, $\implies F=K(t, b_1,...,b_n)$ ist endlich erzeugt über $K$. 
        \item \underline{$\Longleftarrow$:}
        $t$ sei Transzendenzbasis von $F|K$, daraus folgt $F|K(t)$ algebraisch.
        Aus $F|K$ endlich erzeugt, folgt $F|K(t)$ endlich erzeugt.
        Aus $F|K(t)$ ist endlich erzeugt und algebraisch folgt $F|K(t)$ endlich.
    \end{itemize}
\end{proof}

\begin{korollar}
    Ist $F$ ein Funktionenkörper über $K$ und $t \in F$, so gilt:
    $$ t \text{ ist transzendent über } K \iff F|K(t) \text{ endlich}.$$ 
\end{korollar}
\begin{korollar}\label{korollar:1.20}
    Ist $F$ ein Funktionenkörper über $K$ und $K \subseteq E \subseteq F$ ein Zwischenkörper, so ist entweder
    \begin{enumerate}[label=(\arabic*)]
        \item $E|K$ algebraisch oder
        \item $E|K$ ein Funktionenkörper über $K$.
    \end{enumerate}
    In beiden Fällen ist $F|K$ endlich erzeugt.
\end{korollar}
\begin{proof}
    \begin{itemize}[label=]
        \item \underline{$E|K$ algebraisch:}
        folgt aus \cref{satz:1.9}
        \item \underline{$E|K$ transzendent:}
        \begin{align*}
            t \in E \text{ transzendent über } K \implies [E:K(t)] \leq [F:K(t)] < \infty\\
            \trdeg(E|K) = \trdeg(F|K) - \trdeg(F|E) = 1
        \end{align*}
    \end{itemize}
\end{proof}

\begin{bemerkungnr}
    So ist zum Beispiel jedes $$ K \subsetneqq E \subseteq K(T)$$
    nach \cref{beispiel:1.7} transzendent und nach \cref{korollar:1.20} ist $E$ Funktionenkörper
    über $K$. Als Korollar aus dem Satz von Riemann-Hurwitz werden wir unter anderem erhalten:
\end{bemerkungnr}
\begin{theorem}[Lüroth, 1875]
    Ist $K \subsetneqq E \subseteq K(T)$ ein Zwischenkörper, so ist $E$ wieder ein rationaler Funktionenkörper über $K$,
    d.h.
    $$ E \cong_K K(T). $$
\end{theorem}

\section{Funktionenkörper algebraischer Kurven}
Sei $K$ ein Körper und $\K \supseteq K$ ein algebraisch abgeschlossener Körper.

\begin{definition}
    Eine \textbf{irreduzible, ebene, affine, algebraische Kurve} über $K$ ist eine Menge der Form
    $$ C = \mathcal{V}(f) := \{P \in \K^2 \mid f(P) = 0\} \subseteq \K^2,$$
    mit $f \in K[X,Y]$ irreduzibel.
\end{definition}

\begin{beispiel}\label{beispiel:2.2}
    $K = \Q$ und $\K = \C$.
    \begin{enumerate}[label=\alph*)]
        \item Kreis, $C = \mathcal{V}(X^2 + Y^2 - 1)$
        \item Neil'sche Parabel, $C = \mathcal{V}(Y^2 - X^3)$
        \item Elliptische Kurve, $C = \mathcal{V}(Y^2 - X^3 + X)$
    \end{enumerate}
\end{beispiel}

\begin{definition}
    Sei $C = \mathcal{V}(f)$ eine irreduzible, ebene, affine, algebraische Kurve. 
    Dann ist $$ K[C] := \sfrac{K[X,Y]}{(f)} $$ der \textbf{Koordinatenring von $C$} und
    $$ K(C) := \Quot(K[C]) $$ der \textbf{Funktionenkörper von $C$}.
\end{definition}

\begin{bemerkungnr}
    Der Koordinatenring beschreibt die polynomialen Funktionen auf $C$ und ist unabhängig von der Wahl von $f$.
    Insbesondere gibt es zu jedem $P \in C$ einen Homomorphismus
    $$ \begin{cases}
        K[C] \to \K \\
        \phi \mapsto \phi(P).
    \end{cases}$$
    Der Ring $K[C]$ ist nullteilerfrei, da $(f)$ prim, $K(C)$ ist also wohldefiniert und beschreibt die 
    "rationalen Funktionen"\ auf $C$, die in fast allen Punkten auf $C$ definiert sind.
\end{bemerkungnr}

\begin{satz}
    Ist $C = \mathcal{V}(f)$ eine irreduzible, ebene, affine, algebraische Kurve über $K$, so ist $K(C)$ ein
    algebraischer Funktionenkörper über $K$.

    Ist $\Char(K) = 0$, so gilt auch die Umkehrung, d.h. für jeden Funktionenkörper $F$ über $K$ von der Form
    $$F \cong_K K(C),$$ ist $C$ eine irreduzible, ebene, affine, algebraische Kurve über $K$.
\end{satz}
\begin{proof}
    \begin{align*}
        &K[C] = \sfrac{K[X,Y]}{(f)} = K[x,y], \\
        &x = X + (f), y = Y + (f)\\
        \implies & K(C) = K(x,y) \text{ und } f(x,y) = 0
    \end{align*}
    Ohne Einschränkung $f \notin K[X] \implies$ $y$ ist algebraisch über $K(x)$,
    $x$ ist transzendent über $K$, denn
    $$ K[X] \hookrightarrow K[X,Y] \to K[x,y] $$
    ist injektiv. Somit $K(C)$ Funktionenkörper über $K$.

    Ist $\Char(K) = 0$ so gilt der Satz vom primitiven Element:
    Ist $E|K$ endlich, so ist $E = F(\alpha)$ für ein $\alpha \in E$.
    Wähle $x \in F$ transzendent über $K$.
    \begin{align*}
        \implies & K(x) \text{ endlich}\\
        \implies & \text{es existiert } y \in F \text{ mit } F=K(x,y).
    \end{align*} 
    Wähle $0 \neq f \in K(x)[Y]$ mit $f(x,y) =0$, o.E. $f \in K[X,Y]$ irreduzibel.
    Daraus folgt $F = K(x,y)=K(C)$ mit $C = \mathcal{V}(f)$.
\end{proof}

\begin{bemerkungnr}
    In der algebraischen Geometrie hat jede irreduzible Kurve $C$ über $K$ (also jede irreduzible
    eindimensionale quasiprojektive $K-$Varietät) einen Funktionenkörper, der ein Funktionenkörper
    über $K$ ist. 
    
    Allgemein: Funktionenkörper in $\dim C + 1$ Variablen.
\end{bemerkungnr}

\begin{theorem}
    Ist $F$ ein algebraischer Funktionenkörper über $K$, so gibt es eine irreduzible projektive Kurve
    $C$ mit $K(C) \cong_K F$.
    
    Ist $\Char(K)=0$, so gibt es sogar so ein $C$ die glatt ist, und $C$ ist dann bis auf Isomorphie eindeutig bestimmt.
\end{theorem}

\begin{definition}
    Sei $C = \mathcal{V}(f)$ eine irreduzible, ebene, affine, algebraische Kurve. Der Punkt $P \in C$ ist
    \textbf{glatt}(oder regulär), wenn
    $$ \frac{\partial}{\partial X}f(P) \neq 0 \text{ oder } \frac{\partial}{\partial Y} f(P) \neq 0.$$
    Die Kurve ist \textbf{glatt}, wenn jeder Punkt $P \in C$ glatt ist.
\end{definition}

\begin{beispiel}
    \cref{beispiel:2.2} a) und c) sind glatt

    \cref{beispiel:2.2} b) ist überall glatt außer in $(0,0)$, dort singulär
\end{beispiel}

\begin{definition}
    Sei $C\subseteq \K^2$ eine irreduzible affine ebene Kurve, $P \in C$. 
    Der \textbf{lokale Ring} von $P$ auf $C$ ist
    $$ \mathcal{O}_{C,P} := \left\{\frac{g}{h}\mid g,h\in K[C], h(P) \ne0\right\}.$$
\end{definition}

\begin{bemerkungnr}
    $\mathcal{O}_{C,P}$ ist ein lokaler Ring, genauer
    $$ \mathcal{O}_{C,P} = K[C]_{m_P},$$
    wobei $m_P$ der Kern der Abbildung $\varphi \mapsto \varphi(P)$ also
    $$ m_P := \mathcal{I}(P). $$
    Insbesondere ist $\mathcal{O}_{C,P}$ noethersch und $\dim \mathcal{O}_{C,P} = 1$.
\end{bemerkungnr}

\begin{satz}
    Ist $C$ eine irreduzible algebraische Kurve und $P \in C$, so ist $P$ genau dann glatt, wenn 
    $\mathcal{O}_{C,P}$ ein Hauptidealring ist.
\end{satz}

\begin{definition}
    Ein \textbf{diskreter Bewertungsring} von $F$ ist ein lokaler Hauptidealring $R \subseteq F$ mit $\Quot(R) = F$.
\end{definition}

\begin{bemerkungnr}
    \begin{enumerate}[label=\alph*)]
        \item Wie jeder lokale Ring hat ein diskreter Bewertungsring einen \textbf{Restklassenkörper}, nämlich
        $$ \kappa_R := \sfrac{R}{m}, $$
        wobei $m$ das maximale Ideal von $R$ ist.

        \item Erinnerung:
        $$\begin{tikzcd}
            & euklidisch \dar& \\
            eindimensional & \lar Hauptidealring \dar \rar & noethersch\\
            & faktoriell \rar & ganz abgeschlossen
        \end{tikzcd}$$
    \end{enumerate}
\end{bemerkungnr}

\begin{theorem}
    Sei $K = \K$. Ist $C$ eine glatte, irreduzible, projektive Kurve, so liefert
    $ P \mapsto \mathcal{O}_{C,P}$ eine Bijektion zwischen Punkten von $C$ und diskreten
    Bewertungsringen $K \subseteq R \subsetneqq K(C)$.
\end{theorem}

\begin{satz}
    Ist $R$ ein diskreter Bewertungsring von $F$ mit maximalem Ideal $m = (\pi)$, so lässt sich jedes Element $x \in F^\times$ 
    eindeutig schreiben als
    $$ x = u \cdot \pi^{v_\pi(x)}$$
    mit $u \in R^\times$ und $v_\pi(x) \in \Z$.
\end{satz}
\begin{proof}
    $R$ ist faktoriell und $m$ ist das einzige Primideal $\ne (0)$. $\pi$ ist also bis auf Einheiten das einzige Primelement.
\end{proof}

\begin{definition}
    Eine \textbf{diskrete Bewertung} auf $F$ ist eine Abbildung
    $$ v: F \to \Z \cup \{\infty\},$$
    die für $x,y \in F$ erfüllt:
    \begin{enumerate}[label=(\arabic*)]
        \item $v(x) = \infty \implies x = 0$
        \item $v(xy) = v(x) + v(y)$
        \item $v(x+y) \geq \min\{v(x), v(y)\}$.
    \end{enumerate}
    (Beachte $n + \infty = \infty + n = \infty + \infty = \infty$)
\end{definition}

\begin{satz}
    Ist $R$ ein diskreter Bewertungsring von $F$ mit maximalem Ideal $m = (\pi)$, so ist $v_\pi$ eine 
    diskrete Bewertung auf $F$ (mit $v_\pi(0) := \infty$).
    Umgekehrt, wenn $v$ eine diskrete Bewertung auf $F$, so ist
    $$ \mathcal{O}_v := \{x \in F \mid v(x) \geq 0\}$$
    ein diskreter Bewertungsring von $F$ und
    $$ m_v := \{x \in F \mid v(x) > 0\} $$
    das maximale Ideal von $\mathcal{O}_v$.
\end{satz}
\begin{proof}
    \underline{$\mathcal{O}_v$ ist ein Ring:} $\checkmark$\\
    \underline{$m_v$ ist Ideal von $\mathcal{O}_v$ und $\mathcal{O}_v^\times = \mathcal{O}_v\setminus m_v$:}$\checkmark$\\
    \underline{$I \ideal \mathcal{O}_v \implies I=(a)$ für $a \in \mathcal{O}_v$ mit $v(a) = \min\{v(x) \mid x \in I\}$:}
    \begin{align*}
        x \in I & \implies v(a) \leq v(x) \implies v\left(\frac{x}{a}\right) \geq 0\\
        & \implies \frac{x}{a} \in \mathcal{O}_v \implies x = \frac{x}{a}\cdot a \in \mathcal{O}_v a = (a)
    \end{align*}
\end{proof}

\begin{lemma}
    Ist $R$ diskreter Bewertungsring von $F$, so ist $R$ ein maximaler echter Teilring von $F$, d.h. ist
    $R \subsetneqq S \subseteq F$ ein Ring, so ist $S = F$.
\end{lemma}
\begin{proof}
    Ist $x \in S \setminus R$, so ist $x = u \cdot \pi^k$ mit $u \in R^\times$, $k< 0$.
    Sei nun $y \in F^\times$ beliebig. Schreibe
    $$ y = w \cdot \pi^l$$
    mit $w \in R^\times, l \in \Z$.

    Ist \underline{$l \geq 0$}, so ist $y \in R \subseteq S$.

    Ist \underline{$l < 0$}, so ist $yx^l=\underbrace{wu^l}_{\in R^\times} \cdot \overbrace{\pi^{l+kl}}^{\geq 0}\in R$
    somit $y \in R[x] \subseteq S$.
\end{proof}

\begin{beispiel}
    Sei $F = K(T) = K(\mathbb{P}^1)$. Da $K[T]$ faktoriell ist liefert jedes irreduzible normierte $p\in K[T]$ eine 
    diskrete Bewertung $v_p$ mit zugehörigem Bewertungsring
    $$ \mathcal{O}_p := \mathcal{O}_{v_p} = K[T]_{(p)}$$
    mit Restklassenkörper
    $$ \kappa_ := \kappa_{\mathcal{O}_{v_p}} = \sfrac{K[T]}{(p)}.$$
    Desweiteren ist $\mathcal{O}_\infty := K[T^{-1}]_{(T^{-1})}$ ein diskreter Bewertungsring von $F$ mit zugehöriger diskreter Bewertung
    $$v_\infty\left(\frac{f}{g}\right) = \deg(g) - \deg(f) \qquad (f,g \in K[T]) .$$
    Das sind schon alle auf $\A^1 = \P^1 \cup \{\infty\}$.
\end{beispiel}

\begin{satz}
    Die diskreten Bewertungsringe $K \subseteq R \subsetneqq K(T)$ von $K(T)$ sind genau die $\mathcal{O}_p$ mit 
    $p \in K[T]$ normiert irreduzibel, sowie $\mathcal{O}_\infty$.
\end{satz}
\begin{proof}
    Sei $m$ das maximale Ideal von $R$.\\
    \underline{$T\in R$:} Dann ist $K[T] \subseteq R$ und $\mathfrak{p} := m \cap K[T]$ ist ein Primideal von $K[T]$ mit
    $K[T]_\mathfrak{p} \subseteq R$. Insbesondere ist $\mathfrak{p} \ne (0)$. Somit ist $\mathfrak{p} = (p)$ für ein irreduzibeles
    normiertes $p \in K[T]$. Nach Lemma 2.19, angewandt auf den diskreten Bewertungsring $\mathcal{O}_p = K[T]_\mathfrak{p}$,
    folgt $\mathcal{O}_p = R$.\\
    \underline{$T\notin R$:} Dann ist $T^{-1} \in R$, sogar $T^{-1} \in m$, dann
    $$m=(\pi) \implies T = u \cdot \pi^k \implies k < 0 \implies T^{-1} = u^{-1}\cdot \pi^{-k} \in R.$$
    Somit ist $K[T^{-1}] \subseteq R$ und $m \cap K[T^{-1}] = K[T^{-1}]T^{-1}$. Wie im ersten Fall folgert man daraus, dass
    $$ \mathcal{O}_\infty = K[T^-1]_{K[T^{-1}]T^{-1}} = R.$$
\end{proof}

\begin{bemerkungnr}
    Es bezeichne $S(F|K)$ die Menge der diskreten Bewertungsringe $K \subseteq R \subsetneqq F$ von $F$.
    Der Restklassenkörper $\kappa_R$ eines Elements von $S(F|K)$ ist stets eine Erweiterung von $K$.
    Im Fall $K = \K$ werden wir sehen, dass $\kappa_R = K$. Jedes $\varphi \in F$ liefert somit eine "rationale Funktion"
    $$ S(F|K) \to \K \cup \{\infty\}. $$
\end{bemerkungnr}

\section{Riemannsch Flächen}
\begin{definition}
    Eine \textbf{Riemannsche Fläche} ist eine zusammenhängende eindimensionale komplexe Mannigfaltigkeit.
    (\textit{Im rellen glatte zwei dimensionale Mannigfaltigkeit.})
\end{definition}

\begin{beispiel}
    \begin{enumerate}[label=\alph*.)]
        \item $\mathbb{D} = \{t \in \C \mid |z| < 1\}$.
        \item Ist $C$ eine glatte Kurve über $K = \K = \C$, so trägt $C = C(\C)$("Menge der komplexen Punkte")
        die Struktur einer Riemannschen Fläche.

        Z.B. $$ \P^1(\C) = \widehat{\C} = \C \cup \{\infty\}, \text{ die Riemannsche Zahlenspähre.}$$
    \end{enumerate}
\end{beispiel}

\begin{definition}
    Sei $X$ eine Riemannsche Fläche.
    \begin{align*}
        \mathcal{H} &:= \text{ Ring der holomorphen Funktionen auf }X\\
        \mathcal{M} &:= \text{ Körper der meromorphen Funktionen auf }X \\
        &\textit{ "Quotient 2er holomorpher Funktionen"}
    \end{align*}
\end{definition}

\begin{beispiel}
    \begin{align*}
        \mathcal{H}(\mathbb{D}) & \text{"="} \left\{\sum\limits_{i=0}^\infty a_i T^i \mid a_i \in \C, 
            \sum|a_i|\tau^i < \infty \quad \forall \tau < 1\right\}\\
        \mathcal{H}(\widehat{\C}) & = \C, \text{ nach Liouville}\\
        \mathcal{M}(\widehat{\C}) & = \C(T), \text{ da } \widehat{\C} \text{ kompakt ist und }
            \mathcal{H}(\widehat{\C}) = \C
    \end{align*}
\end{beispiel}

\begin{theorem}[Riemannscher Existenzsatz]
    Ist $X$ eine kompakte Riemannsche Mannigfaltigkeit, so ist $\mathcal{M}(X)$ ein algebraischer Funktionenkörper über $\C$.
\end{theorem}

\begin{bemerkungnr}
    \begin{enumerate}[label=\alph*.)]
        \item Umgekehrt ist jeder Funktionenkörper über $\C$ isomorph zu einem Körper der Form $\mathcal{M}(X)$,
        nämlich für ein $X = C(\C)$, siehe 2.7
        \item Vergessen wir die komplexe Struktur auf $X$, so ist $X$ eine orientierbare, zusammenhängende topologische Fläche.
    \end{enumerate}
\end{bemerkungnr}

\begin{definition}
    Das \textbf{Geschlecht} $g_X$ einer orientierbaren zusammenhängenden topologischen Fläche $X$ ist die maximale Anzahl von
    möglichen Schnitten entlang disjunkter, einfach geschlossener Kurven, sodass die Fläche nach dem Schnittvorgang noch zusammenhängend ist.
\end{definition}

\begin{bemerkungnr}
    Es gilt $\chi_X = 2 - 2g_X$, wobei
    $$ \chi_X = \sum\limits_{i=0}^\infty (-1)^i \dim_\R (H_i, \R)$$
    die \textbf{Eulercharakteristik} von $X$ ist.

    Für orientierbare kompakte Flächen gilt 
    $$H_0(X) = \Z, H_1(X) = \pi_1(X)^{ab}, H_2(X) = \Z,$$
    woraus folgt
    $$ g_X = \frac{1}{2} \rang \pi_1(X)^{ab}.$$
\end{bemerkungnr}

\begin{beispiel}
    Ist die Fläche homöomorph zu einem simplizialen Komplex mit $E$ Ecken, $K$ Kanten und $F$ Flächen,
    so ist
    $$ \chi_X = E - K + F.$$
    Spezialfall: Eulerscher Polyedersatz

    Für jedes Polyeder $X \subseteq \R^3$ ist 
    $$ E - K + F = 2.$$
\end{beispiel}

\begin{beispiel}
    \begin{itemize}
        \item Sphäre
        $$g_{S^2} = 0 \implies \chi_{S^2} = 2 $$

        \item Torus
        $$g_{T^2} = 1 \implies \chi_{T^2} = 0 $$

        \item Summe zweier Tori
        $$ g = 2 \implies \chi = -2$$
    \end{itemize}
\end{beispiel}

\begin{theorem}[Klassifikation der Flächen]
    Zwei orientierbare kompakte zusammenhängende Flächen sind genau dann homöomorph, wenn sie das gleiche Geschlecht haben.
\end{theorem}

\begin{bemerkungnr}
    Eine nichtkonstante holomorphe Abbildung $f: Y \to X$ zwischen Riemannschen Flächen liefert über fast allen Punkten von $X$ eine
    \textbf{topologische Überlagerung}(d.h. für fast alle $x \in X$ gibt es offene Umgebungen $x \in U \subseteq X$ für die
    $f^{-1}(U)=\bigsqcup U_i$ mit $f|_{U_i}: U_i \to U$ Homöomorphismus $\forall i$).

    Haben fast alle Punkte von $X$ genau $n$ Urbilder, so hat $f$ \textbf{Grad} $n$, und die Punkte mit weniger als $n$ Urbilder
    heißen \textbf{Verzweigungspunkte} von $f$.

    Die Abbildung $f: Y \to X$ induziert eine Einbettung
    $$ \begin{cases}
            \mathcal{M}(X) \hookrightarrow \mathcal{M}(Y)\\
            g \mapsto g \circ f.
        \end{cases}$$
\end{bemerkungnr}

\begin{beispiel}
    $$f : \widehat{\C} \to \widehat{\C}, z \mapsto z^l, l \in \N$$
    besitzt Verzweigungspunkte bei $0$ und $\infty$.
\end{beispiel}

\begin{bemerkungnr}
    Lokal um einen Punkt $y \in Y$ ist jede holomorphe Abbildung $f: Y \to X$ von der Form
    $$ z \mapsto z^{e_y}, e_y \in \Z_{\geq 0}.$$
    Hat $f$ Grad $n$, so ist für $x \in X$:
    $$ \sum\limits_{y \in f^{-1}(x)} e_y = n. $$
    Insbesondere ist $x$ ein Verzweigungspunkt $\iff e_y > 1$ für ein $y \in f^{-1}(x)$.
\end{bemerkungnr}

\begin{satz}
    Ist $F$ ein Funktionenkörper über $K$ und $E|F$ endlich, so wird durch $R \mapsto R \cap F$ eine Abbildung
    $S(E|K) \to S(F|K)$ definiert.
\end{satz}
\begin{proof}
    Ist $v$ die diskrete Bewertung zu $R$, so ist $v|_F$ auch eine diskrete Bewertung und 
    $\mathcal{O}_{v|_F}$ ist genau 
    $$ \mathcal{O}_v \cap F = R \cap F.$$
    Wäre $R\cap F = F$, so folgt aus $E|F$ algebraisch und $R$ ganzabgeschlossen, schon, dass $E \subseteq R$,
    was im Widerspruch zu $R \in S(E|K)$ steht.
\end{proof}

\begin{bemerkungnr}
    Wir werden jedem Funktionenkörper ein Geschlecht zuordnen und algebraische Analoga zweier fundamentaler Sätze über
    Riemannsche Flächen beweisen:

    Der \textbf{Satz von Riemann-Roch}, beschreibt die maximale Anzahl linear unabhängiger meromorpher Funktionen auf $X$
    mit vorgegebenen Null- und Polstellen.

    Der \textbf{Satz von Riemann-Hurwitz} beschreibt bei einer holomorphen Abbildung $f: Y \to X$ das Geschlecht von $Y$ in
    Abhängigkeit vom Geschlecht von $X$ und dem Verzweigungsverhalten von $f$.
\end{bemerkungnr}

\begin{bemerkungnr}
    Die topologischen Überlagerungen von $X$ werden durch die Fundamentalgruppe $\pi_1(X)$ beschrieben.

    Da zum Beispiel die Sphäre $\widehat{\C} = \P^1(\widehat{\C})$ einfach zusammenhängend ist,
    d.h. $\pi_1(X) = 0$, ist jede holomorphe Abbildung $f: Y \to \widehat{\C}$ vom Grad $n > 1$ verzweigt,
    d.h. hat mindestens eine Faser mit weniger als $n$ Punkten.
\end{bemerkungnr}

Wir werden aus Riemann-Hurwitz ableiten:
\begin{theorem}
    Ist $K$ algebraisch abgeschlossen und $[F:K(T)] = n > 1$ so ist $F|K(T)$ verzweigt, d.h. die Abbildung
    $S(F|K) \to S(K(T)|K)$ hat mindestens eine Faser mit weniger als $n$ Elementen.    
\end{theorem}

\begin{bemerkungnr}
    Ist $F$ ein Zahlkörper, d.h. $[F:\Q] = n < \infty$, so ist der ganze Abschluss $\mathcal{O}_F$ von $\Z$ in $F$ ein Dedekindring,
    d.h. noethersch, ganzabgeschlossen und eindimensional.\\
    In einem Dedekindring $R$ hat man eine eindeutige Zerlegung in Primideale:
    \begin{align*}
        I \ideal R \implies I &= \mathfrak{p}_1^{e_1} \ldots \mathfrak{p}_r^{e_r} \text{ mit } \mathfrak{p}_1 \ldots \mathfrak{p}_r \in \Spec(\mathcal{O}_F)\\
        &  \text{ paarweise verschieden und } e_i \geq 1. 
    \end{align*}
\end{bemerkungnr}

\begin{theorem}[Minkowski 1910]
    Ist $[F:\Q] = n > 1$, so ist $F|\Q$ verzweigt, d.h. es gibt mindestens eine Primzahl $p$, für die
    $$ \mathcal{O}_{Fp} = \mathfrak{p}_1^{e_1} \ldots \mathfrak{p}_r^{e_r}$$
\end{theorem}

\begin{bemerkungnr}
    "Primstellen" $S(F)$ eines Zahlkörpers $F$\\
    \underline{endliche Stellen:} Jedes $(0) \ne \mathfrak{p} \in \Spec(\mathcal{O}_F)$ liefert eine diskrete Bewertung:

    $(\mathcal{O}_F)_\mathfrak{p}$ ist ein diskreter Bewertungsring. Für $x \in F^\times$ ist 
    $$ (\mathcal{O}_F)_\mathfrak{p}\cdot x = (\mathfrak{p}(\mathcal{O}_F)_\mathfrak{p})^{v_\mathfrak{p}(x)}.$$
    \underline{unendliche Stellen:} Einbettungen $\sigma: F \hookrightarrow \C$.
\end{bemerkungnr}

\section{Globale Körper}
Sei $q = p^k$ eine Primpotenz.

\begin{erinnerungnr}
    Zu jeder Primpotenz $p^k$ gibt es bis auf Isomorphie genau einen Körper $\F_{p^k}$ mit $p^k$ Elementen, und zu jedem $d$ hat
    $\F_q$ genau eine Erweiterung $\F_{q^d}$ vom Grad $d$ in einem fixierten algebraischen Abschluss.
\end{erinnerungnr}

\begin{satz}
    Sowohl $A = \Z$ als auch $A = \F_q[T]$ erfüllen:
    \begin{enumerate}[label=(\alph*)]
        \item $A$ ist ein Hauptidealring
        \item $A^\times$ ist endlich
        \item $A$ hat unendlich viele Primelemente
        \item Für jedes $0 \ne x \in A$ ist $\|x\|:= |\sfrac{A}{(x)}| < \infty$
    \end{enumerate}
\end{satz}
\begin{proof}
    (d) $0 \ne f \in \F_q[T] \implies \|f\| = | \sfrac{\F_q[T]}{(f)} | = q^{\deg f}$.
\end{proof}

\begin{bemerkungnr}
    Ein Vertretersystem der Primelemente modulo Einheiten ist
    \begin{itemize}
        \item $\Z=A$: positive Primzahlen
        \item $\F_q[T]=A$: die normierten irreduziblen Polynome
    \end{itemize}
\end{bemerkungnr}

\begin{theorem}[Primzahlsatz]
    $$ |\{p \in \N \mid p \text{ prim }, p \leq x\}| \sim \frac{x}{\log x},\quad x \to \infty.$$
\end{theorem}

\begin{theorem}[Primpolynomsatz]
    $$|\{P \in \F_q[T] \mid P \text{ normiert irreduzibel }, \deg P = n\}| \sim \frac{q^n}{n},\quad q^n \to \infty.$$
\end{theorem}

\begin{bemerkungnr}
    Setzt man $x := q^n$, dann sieht man es steht auf der rechten Seite $$ \frac{x}{\log_q x}.$$
\end{bemerkungnr}
\begin{proof}
    Ist $P \in \F_q[T]$ normiert irreduzibel vom Grad $n$, so hat $P$ genau $n$ viele Nullstellen in $\overline{\F_q}$.
    Umgekehrt ist jedes $\alpha \in \F_{q^n}$ Nullstelle eines normierten irreduziblen Polynoms, nämlich
    $$ P := \MinPol(\alpha|\F_q) \in \F_q[T]$$
    vom Grad
    $$ d := \deg (\alpha|\F_q) = [\F_q(\alpha): \F_q]|[\F_{q^n}:\F_q] = n.$$
    Mit $a_d := |\{P \in \F_q[T] \mid P \text{ normiert irreduzibel, } \deg P = d\}|$ ist also
    $$ \sum_{d|n}d a_d = q^n .$$
    Insbesondere $a_d \leq \frac{q^n}{d} \implies$
    \begin{align*}
        \implies \frac{a_n}{\sfrac{q^n}{n}} & = 1 - \sum_{\substack{d|n\\ d\ne n}} d a_d \frac{1}{q^n}\\
        & \geq 1 - \sum_{\substack{d|n\\ d\ne n}} q^{d-n}\\
        & \geq 1 - \sum_{i=0}^{\lfloor\frac{1}{2}\rfloor} q^{\lfloor\frac{1}{2}\rfloor-n} \geq 1 - \frac{n}{2}q^{-\frac{n}{2}}\\
        & \to 1 \text{ für }q^n \to \infty.
    \end{align*}
\end{proof}

\begin{definition}
    Die \textbf{Dedekindsche Zeta-Funktion} eines Zahlkörpers $F$ ist die analytische Fortsetzung auf $\C \setminus \{1\}$
    von
    $$ \zeta_F(s) := \sum_{(0) \ne \mathfrak{q}\ideal \O_F} | \sfrac{\O_F}{\mathfrak{q}}|^{-s}$$
    für $\Re(s) > 1$.
\end{definition}

\begin{beispiel}
    Für $F=\Q$ ist
    $$ \zeta_\Q(s) = \zeta(s) = \sum_{n=1}^\infty \frac{1}{n^s}, \text{ für } \Re(s) > 1,$$
    die Riemannsche Zeta-Funktion.
\end{beispiel}

\begin{vermutung}[Riemannsche Vermutung, Riemann für $F=\Q$]
    Alle Nullstellen von $\zeta_F$ mit $0 \leq \Re(s) \leq 1$ liegen auf der Geraden $\Re(s)=\frac{1}{2}$.
\end{vermutung}

\begin{bemerkungnr}
    Ist $F|\F_q(T)$ endlich, so ist der ganze Abschluss von $\F_q[T]$ in $F$ ein Dedekindring $\O_F$, der genau wie der 
    Ganzheitsring eines Zahlkörpers nur endliche Restklassenringe besitzt.\\
    Die Lokalisierungen $\O_\mathfrak{p} := (\O_F)_\mathfrak{p}$, $\mathfrak{p} \in \Spec \O_F$, sind diskrete Bewertungsringe,
    und auf diese Weise erhält man fast alle $\O \in S(F|\F_q)$, nämlich alle außer denen mit
    $$ \O \cap \F_q(T) = \O_\infty. $$
\end{bemerkungnr}

\begin{definition}
    Ist $F|\F_q(T)$ endlich, so ist die \textbf{Zeta-Funktion von $F$} die analytische Fortsetzung von
    $$ \zeta_F(s) := \sum_{(0) \ne \mathfrak{q}\ideal \O_F}|\sfrac{\O_F}{\mathfrak{q}}|^{-s}, \Re(s) > 1.$$
\end{definition}

\begin{theorem}[Weil 1949]
    Sei $F|\F_q(T)$ endlich. Alle Nullstellen $s$ von $\zeta_F$ mit $0 \leq \Re(s) \leq 1$
    liegen auf der Geraden $\Re(s) = \frac{1}{2}$.
\end{theorem}

\begin{korollar}[Hasse-Weil-Schranke]
    Ist $C$ eine projektive Kurve über $\F_q$, die über $\overline{\F_q}$ irreduzibel ist und vom Geschlecht $g$, so ist
    $$ ||C(\F_q)|-(q+1)| \leq 2 g \sqrt{q}.$$
\end{korollar}

\begin{korollar}
    Ist $f \in \Z[x,y]$ vom Grad $d$, irreduzibel in $\overline{\F_q}[X,Y]$ so ist
    $$ |\{(x,y) \in \F_q^2 \mid f(x,y)=0\}| = q + O_d(q^{\sfrac{1}{2}}).$$
    Insbesondere gilt: ist $f$ irreduzibel in $\C[X,Y]$, so hat $f$ Nullstellen in $\F_q[X,Y]$ für alle $q$ 
    groß genug.
\end{korollar}

\begin{bemerkungnr}
    Man fast
    \begin{enumerate}[label=(\alph*)]
        \item Zahlkörper
        \item Funktionenkörper über endlichen Körpern
    \end{enumerate}
    unter dem Namen \textbf{globale Körper} zusammen.
\end{bemerkungnr}
