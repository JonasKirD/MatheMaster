% This work is licensed under the Creative Commons
% Attribution-NonCommercial-ShareAlike 4.0 International License. To view a copy
% of this license, visit http://creativecommons.org/licenses/by-nc-sa/4.0/ or
% send a letter to Creative Commons, PO Box 1866, Mountain View, CA 94042, USA.

Folgende Tabelle zeigt die Übersicht über die Voraussetzungen der verschiedenen\\ Argmin-Theoreme:\nl
\begin{tabular}{c|c|c|c}
	& $C(\R)$ & $C(I)$ mit $I$ kompakt & $C_c(\R)$ konvex \\
	\hline
		\makecell{$f,f_n$\\ Fkt.} & 
		\makecell{$\tau\in A(f)$ wohlsepariert\\ $A(f_n)\neq\emptyset\forall n\geq N_0\in\N$\\ $f_n\to f$ glm.} & 
		\makecell{$\tau\in A(f)$ eindeutig\\ $f_n\to f$ glm.} & 
		\makecell{$\tau\in A(f)$ eindeutig\\ $A(f_n)\neq\emptyset~\forall n\geq N_0\in\N$\\ $f_n\to f$ pktw. $D\overset{\text{dicht}}{\subseteq}\R^d$}\\
	\hline
		\makecell{$M,M_n$\\ SP} & 
		\makecell{$\tau\in A(M)$ wohlsep. f.s.\\ $A(f_n)\neq\emptyset$ f.s. \\ $M_n\to M$ glm. f.s.} & 
		\makecell{$\tau\in A(f)$ eindeutig f.s.\\ $M_n\to M$ glm. f.s.} & 
		\makecell{$\tau\in A(M)$ eindeutig f.s.\\ $A(f_n)\neq\emptyset$ f.s.\\ $M_n(t)\to M(t)$ pktw. f.s.}\\
	\hline
		\makecell{$Z_n,Z$\\ SP} & 
		\makecell{$\sigma\in A(Z)$ eindeutig f.s.\\ $Z_n\overset{\L}{\to}Z$\\ $(\sigma_n)_{n\in\N}$ stoch. besch.} & 
		------ & 
		\makecell{$\sigma\in A(Z)$ eindeutig f.s.\\ $Z_n\overset{\fd}{\to}Z$ \\ $A(Z_n)\neq\emptyset$}\\
	\end{tabular}\nl
	Hierbei folgt für die letzte Zeile stets Konvergenz in Verteilung des Argmins, bei den anderen Zellen nur fast sichere Konvergenz.
	Als Faustregel kann man sich auch merken, dass bei stochastischen Prozessen alle Voraussetzungen nur $\P$-fast sicher gelten müssen.