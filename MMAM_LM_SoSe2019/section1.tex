% This work is licensed under the Creative Commons
% Attribution-NonCommercial-ShareAlike 4.0 International License. To view a copy
% of this license, visit http://creativecommons.org/licenses/by-nc-sa/4.0/ or
% send a letter to Creative Commons, PO Box 1866, Mountain View, CA 94042, USA.

\section{Einführung}
%TODO Literaturverzeichnis
Viele Gesetzmäßigkeiten aus der Erfahrungswelt sind (zumindest näherungsweise) linearer Art.
Typischerweise gibt es aber zufällige Schwankungen bzw. Störungen.

\begin{beispiel}\label{beispiel:1.1}
	Körpergröße (cm) und -gewicht (kg) von Männern\\
	\begin{tabular}{c|cccccccc}
		Größe $x$ & 172 & 174 & 180 & 182 & 186 & 188 & 190 & 181\\ 
		\hline
		Gewicht $Y$ & 56 & 59 & 65 & 76 & 80 & 85 & 93 & 73
		% Fergers Gewicht war dreistellig.
	\end{tabular}
	
	%TODO Plot points
	\begin{figure}[H]
		\begin{center}
			% This work is licensed under the Creative Commons
% Attribution-NonCommercial-ShareAlike 4.0 International License. To view a copy
% of this license, visit http://creativecommons.org/licenses/by-nc-sa/4.0/ or
% send a letter to Creative Commons, PO Box 1866, Mountain View, CA 94042, USA.

\begin{tikzpicture}[x=5mm,y=1mm]
	\draw[thin,->] (168,50) -- (193,50) node[right] {$x$};
	\draw[thin,->] (168,50) -- (168,100) node[above] {$y$};

	\foreach \x [count=\xi starting from 170] in {5,10,15,20}{
    	%\draw (\x,2pt) -- (\x,-2pt);
    %\draw (2pt,\x) -- (-2pt,\x);
    %\ifodd\xi
    %    \node[anchor=north] at (\x,0) {$\x$};
    %    \node[anchor=east] at (0,\x) {$\x$};
    %\fi
	}
	
	\foreach \x [count=\xi starting from 50] in {10,20,30,40,50}{
    %	\draw (2pt,\x) -- (-2pt,\x);
    %\ifodd\xi
    %    \node[anchor=north] at (\x,0) {$\x$};
    %    \node[anchor=east] at (0,\x) {$\x$};
    %\fi
	}

	\foreach \Point in {(172,56), (174,59), (180,65), (182,76), (186,80), (188,85), (190, 93), (181,73)}
	{
   		\node at \Point {\textbullet};
	}


	% to ensure that the points are being properly centered:
	%\draw [dotted, gray] (-4,-6) grid (5,5);
	%\node [red] at (3,2.5) {\textbullet};
	%\node [blue] at (3,-2.5) {$\circ$};
\end{tikzpicture}
			\caption{Plot der Körpergrößen und -gewichte}
			\label{Abb:beispiel1.1}
		\end{center}
	\end{figure}

	Die \define{Punktwolke (scatter plot)} weist auf eine \betone{lineare} Beziehung hin:
	\begin{align*}
		Y=a+b\mal x+\varepsilon,
	\end{align*}
	wobei $\varepsilon$ eine zufällige Schwankung und $a,b$ \betone{unbekannte Parameter} des Modells.
\end{beispiel}

\begin{beispiel}(Ohmsches Gesetz)\label{beispiel:1.2}\enter
	$U$ ist Spannung, $I$ ist Stromstärke, $\rho$ ist unbekannter Widerstand
	\begin{align*}
		U=\rho\mal I+\varepsilon
	\end{align*}
\end{beispiel}

\begin{beispiel}\label{beispiel:1.3}
	Die Anhaltestrecke $Y$ eines Autos lässt sich zerlegen in die während der Reaktionszeit zurückgelegte Strecke (Reaktionsweg) und ein Bremsweg.
	Aus physikalischen Gesetzen erwarten wir Proportionalität zur Reaktionsweg und Geschwindigkeit $x$ sowie zwischen Bremsweg und dem \betone{Quadrat} von $x$:
	\begin{align*}
		Y&=a\mal x+b\mal x^2+\varepsilon
	\end{align*}
	Beachte: Keine Linearität in $x$, wohl aber in $(a,b)$!
\end{beispiel}

Allgemein lässt sich das Ergebnis $Y$ eines Zufallsexperimentes (ZE) häufig in der Form
\begin{align*}
	Y&=\beta_1\mal x_1+\ldots+\beta_p\mal x_p+\varepsilon
\end{align*}
darstellen, wobei die $\beta_i$ gewisse (unbekannte) Konstanten sind, die $x_i$ gewisse (bekannte) \define{Einflussgrößen (inputs / Dosen / designs)} und $\varepsilon$ ein stochastischer Fehler (zentriert).
Bei $n$-maliger Durchführung (Wiederholung) des ZE erhält man:
\begin{align}\label{eq:1.1}
	Y_i=\beta_1\mal x_{i,1}+\ldots+\beta_p\mal x_{i,p}+\varepsilon_i,\qquad\forall i\in\set{1,\ldots,n}
\end{align}
mit $\E[\varepsilon_i]=0$.
Setze
\begin{align*}
	Y:=
	\begin{pmatrix}
		Y_1\\
		\vdots\\
		Y_n
	\end{pmatrix}
\end{align*}
$Y$ heißt \define{Beobachtungsvektor / Datenvektor / response}.
Formal ist
\begin{align*}
	Y\colon(\Omega,\A,\P)\to\big(\R^n,\B(\R^n)\big)
\end{align*}
messbar.
Der sogenannte \define{Parametervektor} ist
\begin{align*}
	\beta:=\begin{pmatrix}
		\beta_1\\
		\vdots\\
		\beta_p
	\end{pmatrix}
\end{align*}
Außerdem ist
\begin{align*}
	X:=\big(x_{i,j}\big)_{\begin{subarray}{c}
		1\leq i\leq n\\
		1\leq j\leq p
	\end{subarray}}
\end{align*}
eine $(n\times p)$-Matrix der Inputs, die sogenannte \define{Design-Matrix}.
\begin{align*}
	\varepsilon:=\begin{pmatrix}
		\varepsilon_1\\
		\vdots\\
		\varepsilon_n
	\end{pmatrix},\qquad
	\varepsilon\colon:(\Omega,\A)\to\big(\R^n,\B(\R^n)\big)
\end{align*}
ist der \define{Fehlervektor}.
Dann gilt: %TODO fix labels
\begin{align}\label{eq:1.2}%\tag{1.2}
	\eqref{eq:1.1}
	\iff 
	Y=X\mal\beta+\varepsilon
\end{align}
Die ist die Grundgleichung des \define{linearen Modells (LM)}.
Zwei Grundannahmen:
\begin{align}\label{eq:1.3}
	\E[\varepsilon_i]&=0\qquad\forall i\in\set{1,\ldots,n}\\
	n&\geq p\qquad\text{(Mindestens so viele Daten wie Modellparameter)}
	\label{eq:1.4}
\end{align}
Beachte: \eqref{eq:1.2} heißt \define{lineares} Modell wegen seiner Linearität in $\beta$, z.B. 
\begin{align*}
	Y_i&=\beta_1\mal x_i+\beta_2\mal\log(x_i)+\beta_3\mal\exp(x_i)+\varepsilon_i
	\qquad\forall i\in\set{1,\ldots,n}
\end{align*}
ist lineares Modesll, nicht aber z.B.
\begin{align*}
	Y_i&=\beta_1\mal x_1+\beta_2\mal\exp(\beta_x\mal x_i)+\varepsilon_i
	\qquad\forall i\in\set{1,\ldots,n}
\end{align*}
Untersuchung des LM \eqref{eq:1.2} erfordert Methoden der linearen Algebra.



