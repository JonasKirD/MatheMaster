% This work is licensed under the Creative Commons
% Attribution-NonCommercial-ShareAlike 4.0 International License. To view a copy
% of this license, visit http://creativecommons.org/licenses/by-nc-sa/4.0/ or
% send a letter to Creative Commons, PO Box 1866, Mountain View, CA 94042, USA.

\setcounter{chapter}{1}
\section{Elemente der Linearen Algebra}
Ziel: Bereitstellung einiger Hilfsmittel aus der linearen Algebra.
Literatur:
\begin{enumerate}[label=(\arabic*)]
	\item G. Fischer (1997) \emph{Lineare Algebra}
	\item M. Koecher (1997) \emph{Lineare Al...}
\end{enumerate}
%TODO Add Reference

Im gesamten Abschnitt ist
\begin{align*}
	\R^n:=\set{\begin{pmatrix}
		x_1\\
		\vdots\\
		x_n
	\end{pmatrix}
	:x_i\in\R\quad\forall 1\leq i\leq n}\qquad\forall n\in\N
\end{align*}
der gewöhnliche \define{euklidische Raum} versehen mit dem kanonischen Skalarprodukt
\begin{align*}
	\scaProd{x}{y}:=\sum\limits_{i=1}^n x_i\mal y_i\qquad\forall x,y\in\R^n
\end{align*}
und zugehöriger \define{euklidische Norm}
\begin{align*}
	\norm{x}:=\sqrt{\scaProd{x}{x}}
	=\sqrt{\sum\limits_{i=1}^n x_i^2}
	\qquad\forall x\in\R
\end{align*}

\begin{definition}\label{def2.1}\
	\begin{enumerate}[label=(\arabic*)]
		\item $x,y\in\R^n$ heißen \define{orthogonal}, in Zeichen
		\begin{align*}
			x\perp y:\iff\scaProd{x}{y}=0
		\end{align*}
		\item Sei $U\subseteq\R^n$ und $x\in\R^n$. Dann setze
		\begin{align*}
			x\perp U:\iff\forall u\in U: x\perp u
		\end{align*}
		\item Das \define{orthogonale Komplement} von $U\subseteq\R^n$ ist
		\begin{align*}
			U^\perp:=\set{x\in\R^n:x\perp U}
		\end{align*}
	\end{enumerate}
\end{definition}

%TODO Chapter-Section-Fix, ähnlich wie letztes Semester
%TODO Nummern im ersten Abschnitt sind 1.1 bis 1.4

%TODO fix counters
\begin{satz}\label{satz2.1}\
	\begin{enumerate}[label=(\arabic*)]
		\item $\begin{aligned}
			x\perp y\implies \norm{x+y}^2=\norm{x}^2+\norm{y}^2
		\end{aligned}$ (Pythagoras)
		\item $U^\perp$ ist Untervektorraum (UVR / UR) von $\R^n$
		\item Falls $U\subseteq\R^n$ UVR von $\R^n$ ist, gilt:
		\begin{enumerate}[label=(\roman*)]
			\item $\begin{aligned}
				U\cap U^\perp=\set{0}
			\end{aligned}$
			\item $\begin{aligned}
				\big(U^\perp)^\perp=U
			\end{aligned}$
		\end{enumerate}
	\end{enumerate}
\end{satz}

\begin{proof} %TODO Add references
	\betone{Zeige (1):}
	\begin{align*}
		\norm{x+y}^2
		&=\scaProd{x+y}{x+y}
		=\underbrace{\scaProd{x}{x}}_{=\norm{x}^2}+2\mal\underbrace{\scaProd{x}{y}}_{=0}+\underbrace{\scaProd{y}{y}}_{=\norm{y}^2}
	\end{align*}
	\betone{Zeige (2):} Seien $x,y\in U^\perp$ und seien $\alpha,\beta\in\R$
	\begin{align*}
		\scaProd{\alpha\mal x+\beta\mal y}{u}
		\overset{\Lin}&{=}
		\alpha\mal\underbrace{\scaProd{x}{u}}_{=0}+\beta\mal\underbrace{\scaProd{y}{u}}_{=0}\qquad\forall u\in U\\
		&\implies\alpha\mal x+\beta\mal y\in U^\perp
	\end{align*}
	\betone{Zeige (3)(i):}
	Da $U$ und $U^\perp$ Vektorräume sind, gilt $0\in U,U^\perp$ und somit $U\cap U^\perp\supseteq\set{0}$.
	\begin{align*}
		x\in U\cap U^\perp\implies\scaProd{x}{x}=0
		\implies\norm{x}=0
		\implies x=0
	\end{align*}
	\betone{Zeige (3)(ii):} 
	Zeige "$\supseteq$":
	Sei $x\in U$ und $y\in U^\perp$. Zeige
	\begin{align}\label{eq:ProofSatz2.1Stern}\tag{$*$}
		\scaProd{x}{y}=0
	\end{align}
	Da $y\in U^\perp$, gilt $\scaProd{y}{u}=0$ für alle $u\in U$, also insbesondere für $u=x\in U$.
	Somit folgt \eqref{eq:ProofSatz2.1Stern}.\\
	Teil "$\subseteq$" siehe Koecher, Seite 160.
\end{proof}




