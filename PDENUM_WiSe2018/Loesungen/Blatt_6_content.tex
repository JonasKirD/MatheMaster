% This work is licensed under the Creative Commons
% Attribution-NonCommercial-ShareAlike 4.0 International License. To view a copy
% of this license, visit http://creativecommons.org/licenses/by-nc-sa/4.0/ or
% send a letter to Creative Commons, PO Box 1866, Mountain View, CA 94042, USA.
% vim: set noexpandtab:

\section{Aufgabenblatt 6}
Hier nur die Lösungen. Aufgabenstellungen siehe Vorlesungswebsite.

\subsection{Aufgabe 6.1}
Wozu brauchen wir Quasi-Interpolation?
%TODO Antwort ergänzen

\subsubsection{Aufgabe 6.1 (a)}
%TODO Evtl. Skizze einfügen

\begin{align*}
	R_h v&=\sum\limits_{Z\in N_h}\Big(\pi_Z v)\varphi_Z\\
	\pi_Z:&=\frac{1}{|w_z|}\int\limits_{w_z} v
\end{align*}
%TODO hier fehlt evtl. was
\begin{align*}
	&=\underbrace{\pi_{(0,0)}v}_{=\frac{1}{|\Omega|}\int\limits_\Omega v}\underbrace{\varphi_{(0,0)}}_{=v}\\
	\overset{\text{Symmetrie}}&=
	\frac{1}{2\cdot 2}8\underbrace{\int\limits_{\tilde{T}}v\d(x,y)}_{=\int\limits_{x=0}^1\int\limits_{y=0}^x 1-x\d y \d x=\frac{1}{6}}\\
	&=\frac{2}{6}v=\frac{1}{3}v
\end{align*}

Originale Clément-Interpolation: "Summe über alle Knoten"
\begin{align*}
	\frac{1}{w_z^2}\int\limits_{w_z^2} v\d x
	=\frac{1}{1\cdot 1}2\cdot\underbrace{\int\limits_{\tilde{T}}v\d(x,y)}_{=\frac{1}{6}}\\
	\implies R_h v=\frac{1}{3}\underbrace{\sum\limits_{z\in l_2^1}\varphi_z}_{\equiv1}
\end{align*}

\subsubsection{Aufgabe 6.1 (b)}
Scott-Zhang-Approximation
%TODO Hier könnte man eine Skizze einfügen
\begin{align*}
	\int\limits_{E_z}\psi_{z,j}(a)\varphi_{z,m}(s)\d s
	&=\delta_{j,k}\qquad\forall j,k\in\lbrace1,2\rbrace\\
	\int\limits_{E_z}\psi_z(s)\varphi_z(s)\d s
	&=\left\lbrace\begin{array}{cl}
		1, &\falls z=z'\\
		0, &\falls z'\in N_h,z'\neq z
	\end{array}\right.\\
	SZ_h v(x)
	\sum\limits_{Z\in N_h}\varphi_z(x)\int\limits_{E_z}\psi_z(s)v(s)\d s
\end{align*}
%$N_h$ ist die Knotenmenge
$v\in V_h$ (Finite-Elemente-Raum)
\begin{align*}
	v(x)=\sum\limits_{z\in N_h} v_i(z)\varphi_z(x)
\end{align*}
Also
\begin{align*}
	\int\limits_{E_z}\psi_z(s)v(s)\d s
	=\sum\limits_{z'\in N_h} v_i(z')\int\limits_{E_z}\psi_z(s)\varphi_{z'}(s)\d s
	=v(z)\\
	\implies
	SZ_h v(x)=v(x)
\end{align*}

$SZ_h v\in V_h\cap H_0^1(\Omega)$ falls $v\in H_0^1(\Omega)$\\
$SZ_h v=v$ falls $v\in V_h$

\begin{align*}
	\left.\begin{array}{l}
			\int\limits_0^1\overbrace{\psi_z(s)}^{a+bs}(s)\d s=0\\
		\int\limits_0^1\psi_z(s)(1-s)\d s=1
	\end{array}\right\rbrace\qquad
	\psi_z(s)=2(2-3s)
\end{align*}

\subsection{Aufgabe 6.2}
\subsubsection{Aufgabe 6.2 (a)}
\begin{align*}
	b_T&=\left\lbrace\begin{array}{cl}
		27, &\falls \lambda_{T,1}\lambda_{T,2}\lambda_{T,3},x\in T\\
		0, &\sonst
	\end{array}\right.\\
	\lambda_{T_i}&\geq 0\qquad x\in T\qquad 0\leq b_T\\
	\lambda_{T_i}&0\frac{1}{3}\qquad i=1,2,3\\
	b_T(x)&=27\cdot\left(\frac{1}{3}\right)^3=1\\
	x\in\partial T&\implies b_T(x)
\end{align*}
Zweite Funktion:
\begin{align*}
	b_E=\left\lbrace\begin{array}{cl}
		4, &\falls\lambda_{T,1}\lambda_{T,2},e\in\text{edge}\\
		0, &\sonst
	\end{array}\right.\\
	x\in\partial\omega\implies b_E=0\qquad 0\leq b_E\leq 1\\
	x=\frac{1}{2}n_1+\frac{1}{2}n_2\implies b_E(x)=1
\end{align*}

%\subsubsection{Aufgabe 6.2 (b)}
%TODO 
%\subsubsection*{Aufgabe 6.2 (b) (i)}
%TODO 

%\subsubsection*{Aufgabe 6.2 (b) (ii)}
%TODO 

%\subsubsection*{Aufgabe 6.2 (b) (iii)}
%TODO 
%\subsection{Aufgabe 6.3}
%\subsubsection{Aufgabe 6.3 (a)}
%TODO 
%\subsubsection{Aufgabe 6.3 (b)}
%TODO 
%\subsection{Aufgabe (Zusatz)}
%\subsubsection{Aufgabe (Zusatz) (a)}
%TODO 
%\subsubsection{Aufgabe (Zusatz) (b)}
%TODO 
