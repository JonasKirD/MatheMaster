\chapter{The non-zero structure of the matrix}%
\label{cha:The non-zero structure of the matrix}

%TODO eine neue section hätte es hier auch gemacht. Das neue chapter ist zu viel

A matrix $A \in \R^{n \times n}$ is called \underline{sparse matrix}, if the number of non-zero entries $(a_{i,j}\neq 0)$ is $\ll n^2$. Otherwise it is called a \underline{full matrix}. 

\begin{definition}
\label{thm:sparsetypes}
A (sparse) matrix is called a \underline{banded matrix}  with bandwidth $M$ if
\[
a_{i,j}\neq 0 \text{ only if } i-m_{l} \leq j \leq i + m_{n}
.\] 
with $m_{l},m_{n} \in \N_{>0}$ the lower and upper bandwidth and $M:=m_{l} + m_{n} + 1$.
\end{definition}

What is the bandwidth of $\laplace_{h}^{(5)}$? (using lexicographical ordering)

\begin{align*}
\begin{bmatrix}
	x & x &   &   & \\
	x & x & x &   & \\
	  & x & x & x & \\
	  &   & x & x & \\
	  &   &   &   & \ddots
\end{bmatrix}
\begin{bmatrix}
	x &   &   &   & \\
	  & \ddots &   &   & \\
	  &   & \ddots &   & \\
	  &   &   & \ddots & \\
	  &   &   &   & x
\end{bmatrix} \\
\underbrace{ 
\begin{bmatrix}
	x &   &   &   & \\
	  & \ddots &   &   & \\
	  &   & \ddots &   & \\
	  &   &   & \ddots & \\
	  &   &   &   & x
\end{bmatrix}}_{N_{x}+2}
\underbrace{ 
\begin{bmatrix}
	x & x &   &   & \\
	x & x & x &   & \\
	  & x & x & x & \\
	  &   & x & x & \\
	  &   &   &   & \ddots
\end{bmatrix}}_{A_0}
\end{align*}

The bandwidth of $A_0$ is 3.
We can deduce the other constants with this information:
\begin{itemize}
	\item $m_{l}=m_{n}=N_{x}+2 (\text{ or } N_{y}+2)$
	\item the boundary width of $A$ is $\O(n^{\frac{1}{2}})$ in 2D.
\end{itemize}

What is the number of non-zeros in $\laplace_{h}^{(5)} $?

\[
	nnz(A) \approx 5 \cdot n = 5(N_{x} +2)(N_{y} +2)
.\] 

\section{Finite Difference Refinement}%
\label{sec:Finite Difference Refinement}

Denote by $\Omega_{h}$ the mesh of grid points of width $h$, i.e. 
\[
	(x_{i}, y_{j}) = (ih, jh)
\] 
and by $\Omega_{H}$ the mesh of grid points with width $H$. Assume here $H=2h$.
%TODO plot grid mit H und h

Given a discrete solution $u^{h}$ at $\Omega_{h}$, how to transfer the values to a coarser/finer mesh?

Answer: Use Interpolation!

The transfer from fine mesh $\Omega_{h}$ to coarse mesh $\Omega_{H}$ is called "\underline{Restriction}", denoted by $R_{h}^{H}$ and the transfer from coarse to fine mesh is called "Prolongation" denoted by $P_{H}^{h}$.
\[
R_{h}^{H}: \Omega_{h} \rightarrow \Omega_{H}, \quad
P_{H}^{h}: \Omega_{H} \rightarrow \Omega_{h}
.\] 

\begin{enumerate}[label=\Alph{enumi})]
	\item Prolongation:
		\begin{enumerate}[label=\underline{\arabic{enumi}D}]
			\item The simplest operator is defined by polynomial (linear) interpolation
		%TODO plot 1D interpolation Linie.
		\begin{figure}[ht!]
			\begin{center}
				\includegraphics[width=0.5\textwidth]{pics/}
			\end{center}
			\caption{1D}
			\label{fig:prolongation1}
		\end{figure}
		\begin{align*}
			u_{2j}^{h} &= u_{j}^{H} \qquad\qquad\qquad \text{ for } j=0, \ldots, \frac{N+1}{2} \\
			u_{2j+1}^{h} &= \frac{1}{2}(u_{j}^{H}+u_{j+1}^{H})
		\end{align*}
		In matrix form, we obtain
		\[
		u^{h}= \frac{1}{2}\begin{bmatrix}
			1& & & & \\
			2& & & & \\
			1&1& & & \\
			 &2& & & \\
			 &1&1& & \\
			 & &2& & \\
			 & &1& & \\
			 & & &\ddots& \\
			 & & & &1 \\
			 & & & &2 \\
			 & & & &1
		\end{bmatrix}u^{H}
		.\] 
		Because of the specific coefficients, the interpolation is denoted in a stencil form by
		\[
			P_{H}^{h} \overset{\wedge}{=} \frac{1}{2}
			\left]
			\begin{matrix}
				1 & 2 & 1	
			\end{matrix}
			\right[
		.\] 

	\item The interpolation can be done in each coordinate direction.
		We get a tensor-product of the two 1D rules
		\begin{align*}
			u_{2i, 2j}^{h} &= u_{i,j}^{H} \qquad\qquad\qquad i=0, \ldots, \frac{N_{x}+1}{2} \\
			u_{2i+1, 2j}^{h} &= \frac{1}{2}(u_{i,j}^{H}+u_{i+1,j}^{H}) \qquad\qquad\qquad j=0, \ldots, \frac{N_{y}+1}{2} \\
			u_{2i,2j+1}^{h} &= \frac{1}{2}(u_{i,j}^{H}+u_{i,j+1}^{H}) \\
			u_{2i+1, 2j+1}^{h} &= \frac{1}{4}(u_{i,j}^{H}+ u_{i+1,j}^{H}+ u_{i,j+1}^{H}+ u_{i+1,j+1}^{H})
		\end{align*}
		In stencil form we get
		\[
			P_{H}^{h} \overset{wedge}{=} \frac{1}{4}
			\left]
			\begin{matrix}
				1 & 2 & 1 \\
				2 & 4 & 2 \\
				1 & 2 & 1 
			\end{matrix}
			\right[
		.\] 
		Note: if you interpret the 1D stencil as a row vector $p^{T}$, the 2D stencil is just the outer product 
		\[
		p\cdot p^{T}
		.\] 
		Note: higher order interpolation rules are possibly by incorporating more coarse grid rules.
		\end{enumerate}
		

\end{enumerate}

