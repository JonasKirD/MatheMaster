\chapter{The non-zero structure of the matrix}%
\label{cha:The non-zero structure of the matrix}

%TODO eine neue section hätte es hier auch gemacht. Das neue chapter ist zu viel

A matrix $A \in \R^{n \times n}$ is called \underline{sparse matrix}, if the number of non-zero entries $(a_{i,j}\neq 0)$ is $\ll n^2$. Otherwise it is called a \underline{full matrix}. 

\begin{definition}
\label{thm:sparsetypes}
A (sparse) matrix is called a \underline{banded matrix}  with bandwidth $M$ if
\[
a_{i,j}\neq 0 \text{ only if } i-m_{l} \leq j \leq i + m_{n}
.\] 
with $m_{l},m_{n} \in \N_{>0}$ the lower and upper bandwidth and $M:=m_{l} + m_{n} + 1$.
\end{definition}

What is the bandwidth of $\laplace_{h}^{(5)}$? (using lexicographical ordering)

\begin{align*}
\begin{bmatrix}
	x & x &   &   & \\
	x & x & x &   & \\
	  & x & x & x & \\
	  &   & x & x & \\
	  &   &   &   & \ddots
\end{bmatrix}
\begin{bmatrix}
	x &   &   &   & \\
	  & \ddots &   &   & \\
	  &   & \ddots &   & \\
	  &   &   & \ddots & \\
	  &   &   &   & x
\end{bmatrix} \\
\underbrace{ 
\begin{bmatrix}
	x &   &   &   & \\
	  & \ddots &   &   & \\
	  &   & \ddots &   & \\
	  &   &   & \ddots & \\
	  &   &   &   & x
\end{bmatrix}}_{N_{x}+2}
\underbrace{ 
\begin{bmatrix}
	x & x &   &   & \\
	x & x & x &   & \\
	  & x & x & x & \\
	  &   & x & x & \\
	  &   &   &   & \ddots
\end{bmatrix}}_{A_0}
\end{align*}

The bandwidth of $A_0$ is 3.
We can deduce the other constants with this information:
\begin{itemize}
	\item $m_{l}=m_{n}=N_{x}+2 (\text{ or } N_{y}+2)$
	\item the boundary width of $A$ is $\O(n^{\frac{1}{2}})$ in 2D.
\end{itemize}

What is the number of non-zeros in $\laplace_{h}^{(5)} $?

\[
	nnz(A) \approx 5 \cdot n = 5(N_{x} +2)(N_{y} +2)
.\] 

\section{Finite Difference Refinement}%
\label{sec:Finite Difference Refinement}

Denote by $\Omega_{h}$ the mesh of grid points of width $h$, i.e. 
\[
	(x_{i}, y_{j}) = (ih, jh)
\] 
and by $\Omega_{H}$ the mesh of grid points with width $H$. Assume here $H=2h$.

\begin{figure}[H]
	\center
\begin{tikzpicture}[scale=1]
		
		\def \xone{0};
		\def \yone{0};		
		\def \h{2};		
		% draw coordinate system
		
		%define lines Ax means first line
		\coordinate (A) at (\xone,\yone);

        \foreach \i in {0,1,2,3,4}
		{
			\draw ( $(A) + (\i*\h/4,0) $)-- ++(0,\h);
			\draw ( $(A) + (0,\i*\h/4) $)-- ++(\h,0);
			\foreach \j in {0,1,2,3,4}
			{
				\filldraw ($(A) + (\i*\h/4,\j*\h/4)$) circle (1pt);
			}
		}
        \foreach \i in {0,2,4}
		{
			\foreach \j in {0,2,4}
			{
				\filldraw ($(A) + (\i*\h/4,\j*\h/4)$) circle (2pt);
			}
		}

        %\fill[black,font=\footnotesize] (A) ++(1,0) node[below] {$x_{1}$}
		%								(A) ++(0,1) node[left] {$x_{2}$}
		%								(A) ++(1.4,0.5) node[below] {$n$};
                                        
\end{tikzpicture}

\caption{a coarse and a fine grid}
\label{ch_1_grid_h_H}
\end{figure}


Given a discrete solution $u^{h}$ at $\Omega_{h}$, how to transfer the values to a coarser/finer mesh?

Answer: Use Interpolation!

The transfer from fine mesh $\Omega_{h}$ to coarse mesh $\Omega_{H}$ is called "\underline{Restriction}", denoted by $R_{h}^{H}$ and the transfer from coarse to fine mesh is called "Prolongation" denoted by $P_{H}^{h}$.
\[
R_{h}^{H}: \Omega_{h} \rightarrow \Omega_{H}, \quad
P_{H}^{h}: \Omega_{H} \rightarrow \Omega_{h}
.\] 

\begin{enumerate}[label=\Alph{enumi})]
	\item Prolongation:
		\begin{enumerate}[label=\underline{\arabic{enumi}D}]
			\item The simplest operator is defined by polynomial (linear) interpolation
		%TODO plot 1D interpolation Linie.
		\begin{figure}[H]
	\center
\begin{tikzpicture}[scale=1.5]
		
		\def \xone{0};
		\def \yone{0};		
		\def \h{4};		
		
		%define lines Ax means first line
		\coordinate (A) at (\xone,\yone);
		\coordinate (B) at ($(A) + (1*\h/3,0)$);
		\coordinate (C) at ($(A) + (2*\h/3,0)$);
		\coordinate (D) at ($(A) + (\h,0)$);

		\draw  (A) -- ++(\h,0);
		\draw  (B) -- ++(0,\h/5);
		\draw  (D) -- ++(0,\h/2);
		\draw[dotted]  ( $(B) +(0,\h/5)$) -- ( $(D) +(0,\h/2)$);

		\filldraw (A) circle (1pt);
		\filldraw (C) circle (1pt);
		\filldraw (B) circle (2pt);
		\filldraw (D) circle (2pt);

		%indize coarse mesh
		\node at (B) [below] {$i$};
		\node at (D) [below] {$i+1$};

		% indizes fine mesh
		\node at ( $(B) + (0, -0.5) $) [below] {$2i$};
		\node at ( $(C) + (0, -0.5) $) [below] {$2i+1$};
		\node at ( $(D) + (0, -0.5) $) [below] {$2(i+1)$};
                                        
\end{tikzpicture}

\caption{picture of the stencil in a 2D grid}
\label{ch_1_grid_stencil_2D}
\end{figure}

		%\begin{figure}[ht!]
		%	\begin{center}
				%\includegraphics[width=0.5\textwidth]{pics/}
		%	\end{center}
		%	\caption{1D}
		%	\label{fig:prolongation1}
		%\end{figure}
		\begin{align*}
			u_{2j}^{h} &= u_{j}^{H}  &\text{ for } j=0, \ldots, \frac{N+1}{2} \\
			u_{2j+1}^{h} &= \frac{1}{2}(u_{j}^{H}+u_{j+1}^{H})
		\end{align*}
		In matrix form, we obtain
		\[
		u^{h}= \frac{1}{2}\begin{bmatrix}
			1& & & & \\
			2& & & & \\
			1&1& & & \\
			 &2& & & \\
			 &1&1& & \\
			 & &2& & \\
			 & &1& & \\
			 & & &\ddots& \\
			 & & & &1 \\
			 & & & &2 \\
			 & & & &1
		\end{bmatrix}u^{H}
		.\] 
		Because of the specific coefficients, the interpolation is denoted in a stencil form by
		\[
			P_{H}^{h} \overset{\wedge}{=} \frac{1}{2}
			\left]
			\begin{matrix}
				1 & 2 & 1	
			\end{matrix}
			\right[
		.\] 

	\item The interpolation can be done in each coordinate direction.
		We get a tensor-product of the two 1D rules
		\begin{align*}
			u_{2i, 2j}^{h} &= u_{i,j}^{H} &i=0, \ldots, \frac{N_{x}+1}{2} \\
			u_{2i+1, 2j}^{h} &= \frac{1}{2}(u_{i,j}^{H}+u_{i+1,j}^{H}) &j=0, \ldots, \frac{N_{y}+1}{2} \\
			u_{2i,2j+1}^{h} &= \frac{1}{2}(u_{i,j}^{H}+u_{i,j+1}^{H}) \\
			u_{2i+1, 2j+1}^{h} &= \frac{1}{4}(u_{i,j}^{H}+ u_{i+1,j}^{H}+ u_{i,j+1}^{H}+ u_{i+1,j+1}^{H})
		\end{align*}
		In stencil form we get
		\[
			P_{H}^{h} \overset{wedge}{=} \frac{1}{4}
			\left]
			\begin{matrix}
				1 & 2 & 1 \\
				2 & 4 & 2 \\
				1 & 2 & 1 
			\end{matrix}
			\right[
		.\] 
		\begin{figure}[H]
	\center
\begin{tikzpicture}[scale=1]
		
		\def \xone{0};
		\def \yone{0};		
		\def \h{2};		
		% draw coordinate system
		
		%define lines Ax means first line
		\coordinate (A) at (\xone,\yone);

        \foreach \i in {0,1,2,3,4}
		{
			\draw ( $(A) + (\i*\h/4,0) $)-- ++(0,\h);
			\draw ( $(A) + (0,\i*\h/4) $)-- ++(\h,0);
			\foreach \j in {0,1,2,3,4}
			{
				\filldraw ($(A) + (\i*\h/4,\j*\h/4)$) circle (1pt);
			}
		}
        \foreach \i in {0,2,4}
		{
			\foreach \j in {0,2,4}
			{
				\filldraw ($(A) + (\i*\h/4,\j*\h/4)$) circle (2pt);
			}
		}
		\draw[green!40] ($(A) + ( \h/2,\h/2)$) -- ($(A) + (\h,\h)$);
		\draw[green!40,thick] ($(A) + ( \h,\h/2)$) -- ($(A) + (\h/2,\h)$);
                                        
\end{tikzpicture}

\caption{picture of the stencil in a 2D grid}
\label{ch_1_grid_stencil_2D}
\end{figure}

		Note: if you interpret the 1D stencil as a row vector $p^{T}$, the 2D stencil is just the outer product 
		\[
		p\cdot p^{T}
		.\] 
		Note: higher order interpolation rules are possibly by incorporating more coarse grid rules.
		\end{enumerate}
		

\end{enumerate}

