% !TEX root = NLANA.tex
% This work is licensed under the Creative Commons
% Attribution-NonCommercial-ShareAlike 4.0 International License. To view a copy
% of this license, visit http://creativecommons.org/licenses/by-nc-sa/4.0/ or
% send a letter to Creative Commons, PO Box 1866, Mountain View, CA 94042, USA.

\chapter{Large Sparse Linear Systems}
\section{Model problems and discretization}%
\label{sec:Modelproblems and discretization}


\begin{equation}\label{eq:eq_1}\tag{1}
	\frac{\partial^2 u}{\partial x_1^2} + \frac{\partial^2 u}{\partial x_2^2} =: \laplace u = f(x) 
\end{equation}
with $\Omega \subset \R^2$ bounded, open domain and 
$ x =
\begin{pmatrix}
x_1 \\
x_2
\end{pmatrix}
\in \Omega
$
HIER SKIZZE + n outer normal on $\partial \Omega$
with boundary conditions
\[
\alpha u + \beta \frac{\partial u}{\partial n} = g \qquad \text{ on } \partial \Omega
.\] 

If \begin{itemize}
	\item $ \beta = 0$, we get a Dirichlet problem.
	\item $ \alpha \neq 0$, we get a Neumann problem.
	\item $\alpha = 0 \text{ and } \beta = 1$, we have
		\begin{enumerate}
			\item Since $u = \text{const}$ solves \href{eq:eq_1}{(1)} % TODO FIX THIS
				for $f=0 \text{ and } g = 0$, the solution to \href{eq:eq_1}{(1)} is unique up to a constant
			\item Integrating \href{eq:eq_1}{(1)} over $\Omega$ and Green's formula yield
				\[
				- \int_{\partial\Omega} \frac{\partial u}{\partial n} = - \int_{\Omega} \laplace u = \int_{\Omega} f
				.\] 
				This means, we get a compatibility condition
				\[
				\int_{\partial \Omega} g + \int_{\Omega}f = 0
				.\] 
		\end{enumerate}
\end{itemize} 

Another variant of \href{eq:eq_1}{(1)} is
\[
	Lu := \nabla(A\nabla u)
.\] 
where A is a positive definite matrix.

\begin{equation} \label{eq:eq_2}\tag{2}
LU = f \qquad \in \Omega \text{ + boundary condition}
\end{equation}

\section{Discretization with finite differences}%
\label{sec:Discretization with finite differences}

The basic idea is:

\begin{itemize}
	\item local approximation of partial derivatives
	\item derived by low order Taylor series
\end{itemize}

\begin{itemize}
	\item \underline{(1D-case):} 
		\[
			u'(x) \approx \frac{u(x+h)-u(x)}{h} = \delta^{+}u(x) \qquad \text{forward difference}
		.\] 
		For functions $u \in C^{4}$ in a neighbourhood of $x$, we get by Taylor's formula:
		\begin{equation} \label{eq:eq_3} \tag{\ast}
			u(x+h) = u(x) + h u'(x) + \frac{h^{2}}{2} u''(x) + \frac{h^{3}}{6}u'''(x) + \frac{h^{4}}{24}u''''(\xi_{+})	
		\end{equation}
		for some $\xi_{+} \in (x, x+h)$. Rearranging of the equation gives
		\[
			u'(x) = \frac{u(x+h)-u(x)}{h}-\frac{h}{2}u''(x) + \O(h^{2})
		.\] 
		Now we plug this in in \href{eq:eq_4}{($\ast$)} and replace $h$ by $-h$ to get
		\begin{equation} \label{eq:eq_4} \tag{\ast \ast}
			u(x-h) = u(x) - hu'(x) + \frac{h^{2}}{2}u''(x)- \frac{h^{3}}{6} u'''(x) + \frac{h^{4}}{24}u''''(\xi _{-})
		\end{equation}
		For some appropriate $\xi _{-} \in (x-h, x)$.
		Adding up \href{eq:eq_3}{($\ast$)} and \href{eq:eq_4}{($\ast \ast)$} yields
		\[
			u''(x)= \frac{u(x+h) - 2u(x) + u(x-h)}{h^{2}} + \frac{h^{2}}{12}u''''(\xi )
		.\] for some $\xi  \in [\xi _{-}, \xi _{+}]$

		This is called the central difference approximation of the second order derivative.
		
		Let
		\[
			u'(x) \approx \frac{u(x)-u(x-h)}{h} = \delta^{-}u(x) \qquad \text{backward difference}
		.\] Then $u''(x) \approx \delta^{-}\delta^{+}u(x)$.

		For the elliptic operator $L:=\partial_{x} \Big(a(x)\partial_{x}\Big)$ we get a second order accurate formula by evaluating $a(x)$ inside the intervals $(x-h, x)$ and $(x, x+h)$
		\begin{align*}
			\partial_{x}(a(x) \partial_{x}u) &=
			\delta^{+}(a(x- \frac{h}{2} \delta^{-} u) + \O(h^{2}) \\
											 &\approx \frac{a(x+\frac{h}{2})(u(x+h)-u(x))-a(x-\frac{h}{2})(u(x)-u(x-h))}{h^{2}}
		\end{align*}
		with $a(x \pm \frac{h}{2})$ wither evaluated directly or by the average
		\[
			a(x \pm \frac{h}{2}) \approx \frac{1}{2}(a(x \pm h) - a(x))
		.\] 
		
	\item \underline{(2D \& 3D cases):} 
		The laplacian is the sum of all second derivatives
		\[
			\laplace = \partial x_1^2 + \partial x_2^2 (+\partial x_3^2)
		.\] 
		With (possibly) different step width $h$ in each coordinate direction we get
		\begin{align*}
			\laplace u(x) &= \frac{u(x_1 + h_1, x_2) - 2u(x_1, x_2) + u(x_1 -h_1, x_2)}{h_1^2}
						  &+ \frac{u(x_1, x_2 + h_2) - 2u(x_1, x_2) + u(x_1, x_2 -h_2)}{h_2^2}
		\end{align*}
		but for $h_1 = h_2 = h$ we get
		\[
			\laplace u(x) \approx \frac{1}{h^2}\left[u(x_1+ h, x_2) + u(x_1-h, x_2) + u(x_1, x_2 -h) + u(x_1, x_2 -h) -4u(x_1, x_2)\right]
		.\] 
		Denoting the forward/backward difference formulas
		% TODO stimmt das letzt Wort?
		in the direction i by $\delta_{i}^{+}$ and $\delta_{i}^{-}$ we can write
		\[
			\laplace u(x) \approx \sum_{i=1}^{2}{\delta_{i}^{+}\delta_{i}^{-}u(x)}=:\laplace_{h}^{(5)}u(x)
		.\] 
		The formula can be sketched as a stencil, the so called " 5-point stencil "
		% TODO hier Skizze
		where the values in the nodes correspond to the coefficients in the formula.
		Other possible stencils are:
		\begin{itemize}
			\item 5-point-stencil, $2^{\text{nd}}$ order accurate
				%TODO Skizze
			\item 9-point-stencil,  $2^{\text{nd}}$ order accurate and even $6^{\text{th}}$ order accurate for harmonic funtions
				%TODO Skizze
		\end{itemize}
\end{itemize}

\section{Finite difference on a grid}%
\label{sec:Finite difference on a grid}
Let $\Omega =(0, X_{E}) \times (0,Y_{E})$ and subdivide each interval into $N_{x}+1 / N_{y} + 1$ subintervals.

%Skizze grid here und irgendein umgekehrtes cases environment Ding

\[
\left.
	\begin{array}{c}
	N_{x} = 1 \\
	N_{y} = 2
\end{array}
\right\} \qquad
h_{x} = \frac{x_{E}}{N_{x}+1}, \quad h_{y}= \frac{y_{E}}{N_{y}+1}
.\] 

Each node (vertex) in this grid is assigned an index tuple
\[
	(x,y) = (ih_{x}, jh_{y}) \stackeq{\wedge} (i,j)
.\] 
for $i \in \{0,1, \ldots , N_{x}+1\}, j \in  \{0,1, \ldots , N_{y}+1\}$

We denote the value at the node $(i,j)$ by
\[
	u(x,y)=u(ih_{x},jh_{y})=:u_{i,j}
.\] 
This results in the discrete Laplace operator $(h=h_{x}=h_{y})$
\[
	\laplace_{h}^{(5)}u_{i,j}=\frac{1}{2}(u_{i+1,j}+u_{i-1,j} +u_{i,j+1} - 4u_{i,j})
.\] 





