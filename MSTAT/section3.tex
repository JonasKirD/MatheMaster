% This work is licensed under the Creative Commons
% Attribution-NonCommercial-ShareAlike 4.0 International License. To view a copy
% of this license, visit http://creativecommons.org/licenses/by-nc-sa/4.0/ or
% send a letter to Creative Commons, PO Box 1866, Mountain View, CA 94042, USA.

\section{Zufallsvariablen in metrischen Räumen} %3
\begin{definition}\label{def3.1}
	Die durch den metrischen Raum $(\S,d)$ induzierte $\sigma$-Algebra  
	\begin{align*}
		\B(\S):=\sigma\big(\G(\S)\big).
	\end{align*}
	heißt \textbf{Borel-$\sigma$-Algebra}.
	Elemente $B\in\B(\S)$ heißen \textbf{Borel-Mengen} in $\S$.\\
	Beachte: $\B(\S)=\B_d(\S)$ hängt i. A. von der Metrik $d$ ab.
\end{definition}

\begin{lemma}\label{Lemma3.2}
	Es gilt:
	\begin{enumerate}[label=(\arabic*)]
		\item 
		$\begin{aligned}
			\B(\S)=\sigma\big(\mathcal{F}(\S)\big)
		\end{aligned}$
		\item Ist $\begin{aligned}
			f:(\S,d)\to(\S',d)
		\end{aligned}$ stetig, so ist $f$ $\B_d(\S)$-$\B_d(\S')$-messbar.
		\item Sei $\mathcal{G}_0$ abzählbare Basis von $\mathcal{G}(\S)$. Dann gilt:
		\begin{align*}
			\sigma(\mathcal{G}_0)=\B(\S)
		\end{align*}
	\end{enumerate}
\end{lemma}

\begin{proof}\enter
	\underline{Zu (1), zeige ``$\subseteq$'':}
	Sei $G\in\G(\S)$. Dann gilt:
	\begin{align*}
		G^C\in\mathcal{F}(\S)\subseteq\sigma\big(\mathcal{F}(\S)\big)
		\implies
		G=\big(G^C\big)^C\in\mathcal{F}(\S)
	\end{align*}
	da $\sigma\big(\mathcal{F}(\S)\big)$ Komplement-stabil ist. Also folgt
	\begin{align*}
		\mathcal{G}\subseteq\sigma(\mathcal{F})
		\implies\sigma(\mathcal{G})\subseteq\sigma(\mathcal{F})
	\end{align*}
	\underline{Zu (1), zeige ``$\supseteq$'':} Analog.\nl
	\underline{Zeige (2):}
	\begin{align*}
		f^{-1}\big(\B_{d'}(\S')\big)&=f^{-1}\Big(\sigma\big(\mathcal{G}(\S')\big)\Big)\\
		&=\sigma\Big(\underbrace{f^{-1}\big(\mathcal{G}(\S')\big)}_{\stackrel{f\text{ stetig}}{\subseteq}\mathcal{G}(S)}\Big)\\
		&\subseteq\sigma\big(\mathcal{G}(\S)\big)\\
		&=\B(\S)
	\end{align*}
	\underline{Zu (3), zeige ``$\subseteq$'':}\\
	Klar wegen $\mathcal{G}_0\subseteq\mathcal{G}(\S)$ und $\sigma$ monoton.\\
	\underline{Zu (3), zeige ``$\supseteq$'':} Sei $G\in\mathcal{G}$. Dann:
	\begin{align*}
		G&=\bigcup\limits_{i\in\N} G_i\mit\text{geeigneten }G_i\in\mathcal{G}_0\subseteq\sigma(\mathcal{G}_0)\\
		&\implies
		G\in\sigma(\mathcal{G}_0)
	\end{align*}
	Aus der Stabilität unter Vereinigungen folgt die Behauptung:
	\begin{align*}
		G=\bigcup\limits_{i\in\N} G_i\subseteq\sigma\big(\G_0\big)=\B(\S)
	\end{align*}
\end{proof}

\begin{satz}\label{satz3.3}
	Sei $(\S,d)$ separabler metrischer Raum. Dann gilt:
	\begin{align*}
		\B_{d\times d}(\S\times\S)=\B(S)\otimes\B(\S)
	\end{align*}
\end{satz}

\begin{proof}
	Seien
	\begin{align*}
		&\pi_1:\S\times\S\to\S,\qquad \pi_1(x,y):=x\qquad\forall(x,y)\in\S\times S\\
		&\pi_2:\S\times\S\to\S,\qquad \pi_2(x,y):=y\qquad\forall(x,y)\in\S\times S
	\end{align*}
	die \textbf{Projektionsabbildungen}. Dann gilt
	\begin{align*}
		\B(\S)\otimes\B(\S) 
		\overset{\text{Def}}&=
		\sigma(\pi_1,\pi_2)\\
		\overset{\text{Def}}&=
		\sigma\Big(\pi_1^{-1}\big(\sigma(\mathcal{G})\big)\cup\pi_2^{-1}\big(\sigma(\mathcal{G})\big)\Big)\\
		\overset{(+)}&=
		\sigma\Big(\sigma\big(\pi_1^{-1}(\mathcal{G})\big)\cup\sigma\big(\pi_2^{-1}(\mathcal{G})\big)\Big)\\
		&=
		\sigma\Big(\pi_1^{-1}(\mathcal{G})\big)\cup\pi_2^{-1}(\mathcal{G})\Big)\\
		&=\sigma\Big(\big\lbrace G\times S,S\times G':G,G'\in\mathcal{G}\big\rbrace\Big)\\
		&=
		\sigma\Big(\big\lbrace \overbrace{G\times G'}^{=(G\times S)\cap(S\times G')}:G,G'1\in\mathcal{G}\big\rbrace\Big)\\
		\overset{\text{($\ast$)}}&=
		\sigma\left(\left\lbrace\bigcup\limits_{\begin{subarray}{c}
			G\in\mathcal{O}\\
			G'\in\mathcal{O}'
		\end{subarray}}G\times G':\mathcal{O},\mathcal{O}'\subseteq\mathcal{G}\right\rbrace\right)\\
		\overset{2.10~(3)}&=
		\sigma\Big(\mathcal{G}(\S\times\S)\Big)\\
		\overset{\text{Def}}&=
		\B(\S\times\S)
	\end{align*}
	Zum Nachweis von (+):\\
	Zeige ``$\supseteq$'': Setze
	\begin{align*}
		\xi&:=
		\underbrace{\sigma\big(\pi_1^{-1}(\mathcal{G})\big)}_{\supseteq \pi_1^{-1}(\mathcal{G})}\cup\underbrace{\sigma\big(\pi_2^{-1}(\mathcal{G})\big)}_{\pi_2^{-1}(\mathcal{G})}\\
		&\supseteq
		\pi_1^{-1}(\mathcal{G})\cup\pi_2^{-1}(\mathcal{G})\\
		&=:\mathcal{H}\\
		&\implies\sigma(\xi)\supseteq\sigma(\mathcal{H})
	\end{align*}
	Zeige ``$\subseteq$'': Es gilt
	\begin{align*}
		&\pi_1^{-1}(\mathcal{G})\subseteq\big(\pi_1^{-1}(\mathcal{G})\cup\pi_2^{-1}(\mathcal{G})\big)=\mathcal{H}\\
		&\implies
		\sigma\big(\pi_1^{-1}(\mathcal{G})\big)\subseteq\sigma(\mathcal{H})\text{ und analog }\\
		&\implies
		\sigma\big(\pi_2^{-1}(\mathcal{G})\big)\subseteq\sigma(\mathcal{H})\\
		&\implies
		\xi=\underbrace{\sigma\big(\pi_1^{-1}(\mathcal{G})\big)}_{\subseteq\sigma(\mathcal{H})}\cup\underbrace{\sigma\big(\pi_2^{-1}(\mathcal{G})\big)}_{\subseteq\sigma(\mathcal{H})}\subseteq\sigma(\mathcal{H})\\
		&\implies
		\sigma(\xi)\subseteq\sigma(\mathcal{H})
	\end{align*}

	Bleibt Nachweis von ($\ast$):\\
	``$\subseteq$'': ist klar (gilt auch ohne Separabilität)\\
	``$\supseteq$'': Gemäß Satz \ref{satz2.9} existiert abzählbare Basis $\mathcal{G}_0$  von $\mathcal{G}$. Sei
	\begin{align*}
		G^\ast&=\bigcup\limits_{\begin{subarray}{c}
			G\in\mathcal{O}\\
			G'\in\mathcal{O}'
		\end{subarray}}G\times G'\text{ und }\mathcal{O},\mathcal{O}'\subseteq\mathcal{G}\\
		&\stackeq{(!)}
		\bigcup\limits_{\begin{subarray}{c}
			G,G'\text{ offen}\\
			G,G'\subseteq G^\ast
		\end{subarray}}
		G\times G'\\
		&\stackeq{(!)}
		\bigcup\limits_{\begin{subarray}{c}
			G_0,G_0'\in\mathcal{G}_0\\
			G\times G_0'\subseteq G^\ast
		\end{subarray}}
		G_0\times G_0'\\
		&=\text{ abzählbare Vereinigung, da $\mathcal{G}_0$ abzählbare Basis }\\
		&\implies
		G^\ast\in\sigma\Big(\big\lbrace G\times G':G,G'\in\mathcal{G}\big\rbrace\Big)
	\end{align*}
\end{proof}

\begin{definition}\label{def3.4}
	Sei $(\Omega,\A)$ ein Messraum.
	Eine Abbildung
	$X:\Omega\to\S$, die $\A$-$\B(\S)$-messbar ist, heißt \textbf{Zufallsvariable (ZV)} in den metrischen Raum $(\S,d)$ über $(\Omega,\A)$.\nl
	Sei $\P$ ein Wahrscheinlichkeitsmaß auf $(\Omega,\A)$, also $(\Omega,\A,\P)$ ein Wahrscheinlichkeitsraum. 
	Das Bildmaß
	\begin{align*}
		\P\circ X^{-1}&:=:\P_X:=:\L(X):=:\L(X~|~\P)\\
		(\P\circ X^{-1})(B)&:=\P\left(X^{-1}(B)\right)=\P\Big(\big\lbrace\omega\in\Omega:X(\omega)\in B\big\rbrace\Big)
		=: \P[X\in B]
		\qquad\forall B\in\B(\S)
	\end{align*}
	heißt \textbf{Verteilung} von $X$ unter $\P$.
\end{definition}

\begin{satz}\label{Satz3.5}
	Sei $(\S,d)$ separabler metrischer Raum und seien $X,Y$ Zufallsvariablen in $(\S,d)$ über $(\Omega,\A)$.\\
	Dann ist $d(X,Y)$ eine reelle Zufallsvariable.
\end{satz}

\begin{proof}
	\begin{align*}
		X,Y:(\Omega,\A)\to(\S,\B(\S))\text{ sind messbar }\\
		\stackrel{\text{MINT}}{\Longleftrightarrow}
		(X,Y):(\Omega,\A)\to\big(\S\times\S,\underbrace{\B(\S)\otimes\B(\S)}_{\stackeq{\ref{satz3.3}}\B(\S\times\S)}\big)\text{ ist messbar}\\
	\end{align*}
	Jede Metrik ist bekanntlich stetig, also auch
	\begin{align*}
		d:\big(\S\times\S,\G(\S\times\S)\big)\to\R.
	\end{align*}
	Dann folgt aus Lemma \ref{Lemma3.2}, dass
	\begin{align*}
		d:\B(\S\times\S)\to\B(\R)
	\end{align*}
	messbar ist.
	Damit folgt die Behauptung, denn $d(X,Y)=d\circ(X,Y)$ ist messbar als Komposition von messbaren Abbildungen.
\end{proof}

\subsection{Fast sichere Konvergenz} %NoNumber
\begin{definition}\label{def3.6}
	Seien $X,X_n,n\in\N$ Zufallsvariablen in $(\S,d)$ über $(\Omega,\A,\P)$. 
	Dann:
	\begin{align*}
		X_n\stackrel{n\to\infty}{\longrightarrow} X\quad\P\text{-fast sicher }:\Longleftrightarrow
		\P\Big(\underbrace{\big\lbrace\omega\in\Omega:d\big(X_n(\omega),X(\omega)\big)\stackrel{n\to\infty}{\longrightarrow} 0\big\rbrace}_{=:M}\Big)=1
	\end{align*}
	Beachte: Die Definition von Konvergenz mengentheoretisch aufgeschrieben (Schnitt $\hat{=}$ ``für alle''; Vereinigung $\hat{=}$ ``Es gibt''):
	\begin{align*}
		\bigcap\limits_{0<\varepsilon\in\Q}
		\bigcup\limits_{m\in\N}
		\bigcap\limits_{n\geq m}
		\big\lbrace\underbrace{d(X_n,X)}_{=:\xi_n}<\varepsilon\big\rbrace\stackrel{\ref{Satz3.5}}{\in}\A\\
		\text{denn }\xi_n^{-1}\big((-\infty,\varepsilon)\big)\in\A
	\end{align*}
\end{definition}

Die bekannten Regeln (Ergebnisse) für \underline{reelle} Zufallsvariablen lassen sich mühelos verallgemeinern. Dazu z. B.:

\begin{satz}\label{Satz3.7}
	\begin{align*}
		X_n
		\stackrel{n\to\infty}{\longrightarrow}
		X\quad\P\text{-fast sicher }\wedge
		X_n
		\stackrel{n\to\infty}{\longrightarrow}
		X'\quad\P\text{-fast sicher }
		\implies
		X=X'\quad\P\text{-fast sicher}
	\end{align*}
\end{satz}

\begin{proof}
	\begin{align*}
		\lbrace X\neq X'\rbrace
		&\subseteq\lbrace X_n
		\stackrel{n\to\infty}{\not\longrightarrow}
		X\rbrace
		\cup\lbrace X_n
		\stackrel{n\to\infty}{\not\longrightarrow}
		X'\rbrace\\
		&\implies
		\P[X_n\not\to X]+\P[X_n\not\to X']=0+0\\
		&\implies
		\P[X\neq X']=0\\
		&\implies\P\big(X=X'\big)=1-\underbrace{\P\big(X\neq X'\big)}_{=0}=1
	\end{align*}
\end{proof}

\begin{satz}\label{Satz3.8}
	Seien $X,X_n,n\in\N$ Zufallsvariablen im metrischen Raum $(\S,d)$ und sei\\
	$f:(\S,d)\to(\S',d')$ messbar und stetig in $X$ $\P$-fast sicher.
	Dann gilt:
	\begin{align*}
		X_n
		\stackrel{n\to\infty}{\longrightarrow}
		X\quad\P\text{-fast sicher }
		\implies
		f(X_n)
		\stackrel{n\to\infty}{\longrightarrow}
		f(X)\quad\P\text{-fast sicher}
	\end{align*}
\end{satz}

\begin{proof}
	\begin{align*}
		\lbrace X_n
		\stackrel{n\to\infty}{\longrightarrow}
		X\rbrace\cap\lbrace f\text{ stetig in }X\rbrace\rbrace
		\stackrel{\text{Folgen-Stetigkeit}}{\subseteq}
		\lbrace f(X_n)
		\stackrel{n\to\infty}{\longrightarrow}
		f(X)\rbrace
	\end{align*}
	Mit
	\begin{align*}
		\big(\forall i\in\N:\P(E_i)=1\big)\implies\P\left(\bigcap\limits_{i\in\N} E_i\right)=1
	\end{align*}
	folgt
	\begin{align*}
		1=\P\Big(\Big\lbrace X_n\overset{n\to\infty}{\longrightarrow}X\text{ und $f$ stetig in }X\Big\rbrace\Big)\leq\P\Big(\Big\lbrace f(X_n)\overset{n\to\infty}{\longrightarrow} f(X)\Big\rbrace\Big)\\
		\implies \P\Big(\Big\lbrace f(X_n)\overset{n\to\infty}{\longrightarrow} f(X)\Big\rbrace\Big)
	\end{align*}
\end{proof}

\begin{satz}[Konvergenz-Kriterium]\label{Satz3.9}
	\begin{align*}
		X_n
		\stackrel{n\to\infty}{\longrightarrow}
		X\quad\P\text{-fast sicher}
		\Longleftrightarrow
		\forall\varepsilon>0:\limn\P\left(\sup\limits_{m\geq n} d(X_m,X)>\varepsilon\right)=0
	\end{align*}
\end{satz}

\begin{proof}
	Man ersetze im Beweis für den Fall reeller Zufallsvariablen $|X_n-X|$ durch $d(X_n,X)$. 
	Und beachte, dass alle Schlussfolgerungen bestehen bleiben.
\end{proof}

Ein sehr nützliches Kriterium ist Folgendes:
\begin{satz}\label{Satz3.10}
	\begin{align*}
		\sum\limits_{n\in\N_{\geq1}}\P\big(d(X_n,X)>\varepsilon\big)<\infty\qquad\forall\varepsilon>0
		\implies
		X_n
		\stackrel{n\to\infty}{\longrightarrow}
		X\quad\P\text{-fast sicher}
	\end{align*}
\end{satz}

\begin{proof}
	Setze
	\begin{align*}
		A_n(\varepsilon):=\big\lbrace d(X_n,X)>\varepsilon\big\rbrace\stackrel{\ref{Satz3.5}}{\in}\A
	\end{align*}
	Dann folgt aus dem \textit{ersten Borel-Cantelli-Lemma}:
	\begin{align*}
		&\P\left(\limsup\limits_{n\to\infty} A_n(\varepsilon)\right)=0\qquad\forall\varepsilon>0
	\end{align*}
		Mit
	\begin{align*}
		\liminf\limits_{n\to\infty}\big(A_n(\varepsilon)\big)
		\stackeq{\text{Def}}
		\bigcup\limits_{m\in\N}\bigcap\limits_{n\geq m}\big(A_n(\varepsilon)\big)^C
		=
		\bigcup\limits_{m\in\N}\bigcap\limits_{n\geq m}\big\lbrace d(X_n,X)\leq\varepsilon\big\rbrace
	\end{align*}
	folgt dann
	\begin{align*}
		1=\P\left(\left(\limsup\limits_{n\to\infty} A_n(\varepsilon)\right)^C\right)
		=\P\left(\liminf\limits_{n\to\infty}\big(A_n(\varepsilon)\big)\right)\qquad\forall\varepsilon>0
	\end{align*}
	Da Abzählbare Durchschnitte von Eins-Mengen (also Mengen mit $\P$-Maß 1) wieder Eins-Mengen sind, folgt schließlich:
	\begin{align*}
		\P\Bigg(\underbrace{
			\bigcap\limits_{0<\varepsilon\in\Q}\bigcup\limits_{m\in\N}\bigcap\limits_{n\geq m}\big\lbrace d(X_n,X)\leq\varepsilon\big\rbrace
		}_{\lbrace X_n\to X\rbrace=\lbrace d(X_n,X)\to0\rbrace}\Bigg)=1
	\end{align*}
\end{proof}

Weitere Eigenschaften der fast sicheren Konvergenz von Zufallsvariablen in metrischen Räumen finden sich z. B. in \textit{Wahrscheinlichkeitstheorie} von Gäussler und Stute (1977), Kapitel 8.2.

\subsection{Stochastische Konvergenz} %NoNumber
\begin{definition}\label{def3.11}
	$X_n$ \textbf{konvergiert stochastisch gegen $X$}, i.Z.:
	\begin{align*}
		X_n
		\stackrelnew{n\to\infty}{\P}{\longrightarrow}
		X:\Longleftrightarrow\forall\varepsilon>0:
		\P\Big(\big\lbrace d(X_n,X)>\varepsilon\big\rbrace\Big)
		\stackrel{n\to\infty}{\longrightarrow}
		0
	\end{align*}
\end{definition}

\begin{satz}\label{Satz3.12}
	\begin{align*}
		X_n
		\stackrel{n\to\infty}{\longrightarrow}
		X\quad\P\text{-fast sicher }
		\implies X_n
		\stackrelnew{n\to\infty}{\P}{\longrightarrow}
		X
	\end{align*}
\end{satz}

\begin{proof}
	\begin{align*}
		\forall\varepsilon>0:
		0\leq\P\big(d(X_n,X)>\varepsilon\big)
		\leq\P\left(\sup\limits_{m\geq n}d(X_m,X)>\varepsilon\right)
		\stackrelnew{n\to\infty}{\P}{\longrightarrow}
		0
	\end{align*}
	gemäß Satz \ref{Satz3.9}.
\end{proof}

Die Umkehrung von Satz \ref{Satz3.12} gilt i. A. \underline{nicht}, aber es gilt das folgende Teilfolgenkriterium:

\begin{satz}[Teilfolgenkriterium für stochastische Konvergenz]\label{satz3.13}\enter
	Folgende Aussagen sind äquivalent:
	\begin{enumerate}[label=(\arabic*)]
		\item $\begin{aligned}
			X_n
			\stackrelnew{n\to\infty}{\P}{\longrightarrow}
			X
		\end{aligned}$
		\item Zu jeder Teilfolge (TF) $(X_{n'})$ von $(X_n)_{n\in\N}$ existiert eine Teilfolge $(X_{n''})$ von $(X_{n'})$ derart, dass $X_{n''}
		\stackrel{n''\to\infty}{\longrightarrow} X$ $\P$-fast sicher.
	\end{enumerate}
\end{satz}

\begin{proof}
	Wie im Reellen.
\end{proof}

Mit dem Teilfolgenkriterium lassen sich Rechenregeln für fast sichere Konvergenz auf stochastische Konvergenz übertragen.

\begin{korollar}\label{Korollar3.14}\
	\begin{enumerate}[label=(\arabic*)]
		\item $\begin{aligned}
			X_n
			\stackrelnew{n\to\infty}{\P}{\longrightarrow}
			X\wedge X_n
			\stackrelnew{n\to\infty}{\P}{\longrightarrow}
			X'
			\implies X=X'\quad\P\text{-fast sicher}
		\end{aligned}$
		\item $\begin{aligned}
			X_n
			\stackrelnew{n\to\infty}{\P}{\longrightarrow}
			X\text{ in }(\S,d),~f:(\S,d)\to(\S',d')\text{ messbar mit $f$ stetig in $X$ $\P$-fast sicher }
		\end{aligned}$
		\begin{align*}
			\implies f(X_n)
			\stackrelnew{n\to\infty}{\P}{\longrightarrow}
			f(X)
		\end{align*}
	\end{enumerate}
\end{korollar}

\begin{proof}
	\underline{Zeige (1):}
	\begin{align*}
		X_n
		\stackrelnew{n\to\infty}{\P}{\longrightarrow}
		X
		\stackrel{\ref{satz3.13}}{\implies}
		\exists\text{ TF }(X_{n'})\subseteq(X_n)_{n\in\N}\mit X_{n'}
		\stackrel{n'\to\infty}{\longrightarrow}
		X\text{ fast sicher}
	\end{align*}
	Zu $(X_{n'})$ existiert (wegen $X_n\stackrelnew{n\to\infty}{\P}{\longrightarrow} X'$ und Satz \ref{satz3.3}) eine Teilfolge
	$(X_{n''})\subseteq(X_{n'})$ mit
	\begin{align*}
		X_{n''}
		\stackrel{n\to\infty}{\longrightarrow} X'\text{ fast sicher}
		\stackrel{\ref{Satz3.7}}{\implies}
		X=X'\text{ fast sicher }
	\end{align*}

	\underline{Zeige (2):} Zur Übung.
\end{proof}

\subsection{Konvergenz in Produkträumen} %noNumber
Seien $(\S,d)$ und $(\S',d')$ separable metrische Räume. Dann ist auch $(\S\times\S',d\times d')$ ein separabler metrischer Raum. 
Dies folgt z. B. aus dem \textit{Satz von der koordinatenweise Konvergenz}:

\begin{align}\label{eq3.1KoordinatenweissKonvergenz}\tag{3.1}
	\big(a_n,a_n'\big)
	\stackrelnew{d\times d'}{n\to\infty}{\longrightarrow}
	(a,a')
	\Longleftrightarrow
	(a_n)
	\stackrelnew{d}{n\to\infty}{\longrightarrow}
	a
	\wedge
	(a_n')
	\stackrelnew{d'}{n\to\infty}{\longrightarrow}
	(a')
\end{align}

Es ``stochastische Versionen'' dieses Satzes.

\begin{satz}\label{satz3.15}\
	\begin{enumerate}[label=(\arabic*)]
		\item $\begin{aligned}
			(X_n,X_n')
			\stackrelnew{}{n\to\infty}{\longrightarrow}
			(X,X')~\P\text{-f.s.}
			\Longleftrightarrow
			X_n
			\stackrelnew{}{n\to\infty}{\longrightarrow}
			X~\P\text{-f.s. }\wedge
			X_n'
			\stackrelnew{}{n\to\infty}{\longrightarrow}
			X'~\P\text{-f.s.}
		\end{aligned}$
		\item $\begin{aligned}
			(X_n,X_n')
			\stackrelnew{n\to\infty}{\P}{\longrightarrow}
			(X,X')
			\Longleftrightarrow
			X_n
			\stackrelnew{n\to\infty}{\P}{\longrightarrow}
			X\wedge
			X_n'
			\stackrelnew{n\to\infty}{\P}{\longrightarrow}
			X'
		\end{aligned}$
	\end{enumerate}
\end{satz}

\begin{proof}
	\underline{Zu (1):}
	\begin{align*}
		(1)\text{, linke Seite }
		&\stackrel{\eqref{eq3.1KoordinatenweissKonvergenz}}{\Longleftrightarrow}
		X_n\to X,~X_n'\to X'\quad\P\text{-fast sicher}\\
		&\stackrel{\cap\text{ Eins-Mengen}}{\Longleftrightarrow}
		(1),\text{ rechte Seite}
	\end{align*}

	\underline{Zu (2):}
	\begin{align*}
		(2)\text{, linke Seite}
		&\stackrel{\eqref{eq3.1KoordinatenweissKonvergenz}}{\Longleftrightarrow}
		\forall\text{ TF }(X_{n'},X_{n'}')\subseteq(X_n,X_n'):\\
		&\qquad\exists\text{ TTF }(X_{n''},X_{n''}')\subseteq(X_{n'},X_{n'}'):
		(X_{n''},X_{n''}')\stackrel{n\to\infty}{\longrightarrow}\text{ f.s.}
	\end{align*}
	Also wegen Teil (1) mit
	\begin{align*}
		X_{n''}\to X\text{ f.s. und }X_{n''}'\to X'\text{ f.s.}
	\end{align*}
	Somit:
	\begin{align*}
		\forall\text{ TF }(X_{n'})\subseteq(X_n):\exists\text{ TTF }(X_{n''})\subseteq(X_{n'}):X_{n''}\to X\text{ f.s. }
		&\stackrel{\eqref{eq3.1KoordinatenweissKonvergenz}}{\Longleftrightarrow}
		X_n\stackrelnew{n\to\infty}{\P}{\longrightarrow} X
	\end{align*}
	Und es gilt analog: $X_n'\to X'$.
\end{proof}

\subsection{Gleichheit in Verteilung} %NoNumber
\begin{definition}\label{def3.16}
	Zufallsvariablen $X,Y$ in $(\S,d)$ über $(\Omega,\A,\P)$ heißen \textbf{gleich in Verteilung}, in Zeichen
	\begin{align*}
		X\stackeq{\mathcal{L}} Y:\Longleftrightarrow \P\circ X^{-1}=\P\circ Y^{-1}
	\end{align*}
\end{definition}

\begin{bemerkung} %noNomber
	Definition \ref{def3.16} kann erweitert werden auf Zufallsvariablen\\ $X:(\Omega,\A,\P)\to(\S,d)$ und $Y:(\tilde{\Omega},\tilde{\A},\tilde{\P})\to(\S,d)$ durch
	\begin{align*}
		X\stackeq{\mathcal{L}} Y:\Longleftrightarrow\P\circ X^{-1}=\tilde{\P}\circ Y^{-1}
	\end{align*}
\end{bemerkung}

Charakterisierung von Verteilungsgleichheit in folgendem Satz:

\begin{satz}\label{satz3.17}\
	Es gilt:
	\begin{enumerate}[label=(\arabic*)]
		\item Seien $\P,Q$ Wahrscheinlichkeitsmaße auf $\B(\S)$. Dann gilt:
		\begin{align*}
			\P\equiv Q\Longleftrightarrow
			\int\limits f\d\P=\int\limits f\d Q\qquad\forall f\in C^b(\S)\text{ glm. stetig}
		\end{align*}
		\item $\begin{aligned}
			X\stackeq{\mathcal{L}} Y\Longleftrightarrow
			\E\big[f(X)\big]=\E\big[f(Y)\big]\qquad\forall f\in C^b(\S)\text{ glm. stetig}
		\end{aligned}$
	\end{enumerate}
\end{satz}

\begin{proof}
	\underline{Zu (1) zeige ``$\implies$'':} Klar.\nl
	\underline{Zu (1) zeige ``$\Longleftarrow$'':}
	\begin{align*}
		\B(\S)\stackeq{\ref{Lemma3.2}(1)}\sigma\big(\mathcal{F}(\S)\big)
	\end{align*}
	und $\F$ ist Durchschnittsstabil. Wegen dem Maßeindeutigkeitssatz reicht es zu zeigen:
	\begin{align*}
		\P(F)=Q(F)\qquad\forall F\in\mathcal{F}(\S)
	\end{align*}
	Sei $F\subseteq\S$ abgeschlossen. Setze
	\begin{align*}
		f_k(x):=\varphi\Big(k\cdot d(x,F)\Big)
	\end{align*}
	(vgl. Satz \ref{Satz2.4}).
	Aus dem Lemma \ref{lemma2.3} folgt, dass die $f_k$ beschränkt und gleichmäßig stetig sind mit $f_k\stackrel{k\to\infty}{\downarrow}\indi_F$. 
	Also gilt:
	\begin{align*}
		\P(F)
		\overset{\text{Def}}&=
		\int\limits\indi_F\d\P\\
		&=\int\limits\lim\limits_{k\to\infty} f_k\d\P\\
		\overset{\text{Mono-Konv}}&=
		\lim\limits_{k\to\infty}\int\limits f_k\d\P\\
		\overset{\text{Vor}}&=
		\lim\limits_{k\to\infty}\int\limits f_k\d Q\\
		\overset{\text{Mono-Konv}}&=
		\int\limits\lim\limits_{k\to\infty} f_k\d Q\\
		&=\int\limits\indi_F\d Q\\
		\overset{\text{Def}}&=
		Q(F)
	\end{align*}
	Da $F$ beliebig war, folgt die Behauptung.\nl
	\underline{Zu (2):} folgt aus (1) mit dem Transformationssatz \eqref{eqTrafo}:
	\begin{align*}
		X\stackeq{\mathcal{L}} Y
		&\stackrel{\ref{def3.16}}{\Longleftrightarrow}
		\P\circ X^{-1}=\P\circ Y^{-1}\\
		&\stackrel{(1)}{\Longleftrightarrow}
		\int\limits_{\S} f\d(\P\circ X^{-1})=\int\limits_{\S} f\d(\P\circ Y^{-1})
		&\forall f\in C^b(\S)\text{ glm. stetig}\\
		\overset{\eqref{eqProof3.17}}&\Longleftrightarrow
		\E\big[f(X)\big]=\E\big[f(Y)\big]
		&\forall f\in C^b(\S)\text{ glm. stetig}\\
	\end{align*}
	Verwende dafür:
	\begin{align}\label{eqProof3.17}\tag{$\ast$}
		\int\limits_{\S} f\d(\P\circ X^{-1})
		\stackeq{\text{Trafo}}
		\int\limits_{\Omega}\underbrace{f\circ X}_{=:f(X)}\d\P
		\stackeq{\text{Def}}
		\E\big[f(X)\big]
	\end{align}
\end{proof}
