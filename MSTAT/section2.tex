% This work is licensed under the Creative Commons
% Attribution-NonCommercial-ShareAlike 4.0 International License. To view a copy
% of this license, visit http://creativecommons.org/licenses/by-nc-sa/4.0/ or
% send a letter to Creative Commons, PO Box 1866, Mountain View, CA 94042, USA.

\section{Konzepte aus metrischen Räumen} %2
Sei $(\mathcal{S},d)$ metrischer Raum.

\begin{beispiel}[Supremums-Metrik] %2.1
	\begin{align*}
		\mathcal{S}=C([0,1]):=\big\lbrace f:[0,1]\to\R: f\text{ stetig}\big\rbrace\\
		d(f,g):=\sup\limits_{t\in[0,1]}\big|f(t)-g(t)\big|,\qquad\forall f,g\in C([0,1])
	\end{align*}
\end{beispiel}

\begin{definition}\
	\begin{enumerate}[label={(\arabic*)}]
		\item Für $x\in\mathcal{S},~r>0$ ist
		\begin{align*}
			B(x,r):=B_d(x,r):=\lbrace y\in\mathcal{S}:d(x,y)<r\rbrace
		\end{align*}
		die offene Kugel um Mittelpunkt $x$ und Radius $r$.
		\item Sei $A\subseteq\mathcal{S}$. Dann:
		\begin{align*}
			\stackrel{\circ}{A}&:=\inner(A):=\text{ das Innere von }A\\
			\overline{A}&:=\text{ Abschluss von }A\\
			\partial A&:=\overline{A}\cap\overline{A^C}=\overline{A}\setminus\stackrel{\circ}{A}\text{ ist der Rand von }A\\
			A^C&:=\mathcal{S}\setminus A
		\end{align*}
		\item \begin{align*}
			\mathcal{G}:=\mathcal{G}(\mathcal{S})&:=\big\lbrace G\subseteq\mathcal{S}: G\text{ ist offen bzgl. }d\big\rbrace\\
			&=\big\lbrace G\subseteq\mathcal{S}:\forall x\in G:\exists r>0:B_d(x,r)\subseteq G\big\rbrace
		\end{align*}
		ist die durch $d$ induzierte Topologie.
		\begin{align*}
			\mathcal{F}:=\mathcal{F}(\mathcal{S}):=\big\lbrace F\subseteq\mathcal{S}:F\text{ ist abgeschlossen}\big\rbrace
		\end{align*}
		\item Sei $\emptyset\neq A\subseteq\mathcal{S},~x\in\mathcal{S}$. Dann ist
		\begin{align*}
			d(x,A):=\inf\lbrace d(x,a):a\in A\rbrace\geq0
		\end{align*}
		der Abstand von $x$ zu $A$.
		\item $C(\mathcal{S}):=\lbrace f:S\to\R:f\text{ stetig}\rbrace$
		\begin{align*}
			C^b(\mathcal{S}):=\lbrace f\in C(\mathcal{S}):f\text{ beschränkt}\rbrace\\
			\Vert f\Vert:=\Vert f\Vert_\infty:=\sup\limits_{x\in\mathcal{S}}|f(x)|
		\end{align*}
	\end{enumerate}
\end{definition}

\begin{lemma}\label{lemma2.3}\ %2.3
	\begin{enumerate}[label={(\arabic*)}]
		\item $\begin{aligned}
			x\in\overline{A}\Longleftrightarrow d(x,A)=0
		\end{aligned}$
		\item $\begin{aligned}
			\big| d(x,A)-d(y,A)\big|\leq d(x,y)\qquad\forall x,y\in\mathcal{S}
		\end{aligned}$
		\item $\begin{aligned}
			d(\cdot, A):\mathcal{S}\to\R,\qquad x\mapsto d(x,A)
		\end{aligned}$ ist gleichmäßig stetig ($A\neq\emptyset$).
	\end{enumerate}
\end{lemma}

\begin{proof}
	\underline{Zeige (1) ``$\Rightarrow$'':} Sei $x\in\overline{A}$. Dann gilt:
	\begin{align*}
		&\forall\varepsilon>0:\exists a\in A: d(x,a)<\varepsilon\\
		&\implies d(x,A)\leq d(x,a)<\varepsilon~\forall\varepsilon>0\\
		&\stackrel{\varepsilon\to0}{\implies}
		d(x,A)=0
	\end{align*}
	\underline{Zeige (1) ``$\Leftarrow$'':}
	Sei $d(x,A)=0$. Dann folgt aus der Infimumseigenschaft:
	\begin{align*}
		&\forall\varepsilon>0:\exists a\in A;0\leq d(x,a)\leq0+\varepsilon=\varepsilon\\
		&\implies x\in\overline{A}
	\end{align*}
	\underline{Zeige (2):} Seien $x,y\in\mathcal{S}$. Dann gilt:
	\begin{align*}
		&d(x,a)
		\stackrel{\Delta\text{Ungl}}{\leq}
		d(x,y)+d(y,a)\qquad\forall a\in A\\
		&\implies
		d(x,A)\leq d(x,y)+d(y,A)\implies d(x,A)-d(y,A)\leq d(x,y)
	\end{align*}
	Vertauschen von $x$ und $y$ liefert:
	\begin{align*}
		d(y,A)-d(x,A)\leq d(y,x)=d(x,y)\implies\text{ Behauptung}
	\end{align*}
	\underline{Zeige (3):} Folgt aus (2) da, die Funktion $d(\cdot,A)$ Lipschitz-stetig und damit gleichmäßig stetig ist.
\end{proof}

\begin{satz}\label{Satz2.4} %2.4
	Zu $A\subseteq\mathcal{S}$ und $\varepsilon>0$ existiert ein gleichmäßig stetige Funktion 
	\begin{align*}
		f:\mathcal{S}\to[0,1]\text{ mit der Eigenschaft} f(x)=\left\lbrace\begin{array}{cl}
			1, & \falls x\in A\\
			0, & \falls d(x,A)\geq\varepsilon
		\end{array}\right.
	\end{align*}
\end{satz}

\begin{proof}
	Setze
	\begin{align*}
		\varphi:\R\to[0,1],\qquad \varphi(t):=\left\lbrace\begin{array}{cl}
			1 , & \falls t\leq0\\
			1-t, & \falls 0<1<1\\
			0, & \falls t\geq1
		\end{array}\right.
	\end{align*}
	Dann ist $\varphi$ gleichmäßig stetig auf $\R$. Sei
	\begin{align*}
		f(x):=\varphi\left(\frac{1}{\varepsilon}\cdot d(x,A)\right)\qquad\forall x\in\S
	\end{align*}
	Dann hat dieses $f$ die gewünschte Eigenschaft wegen Lemma \ref{lemma2.3}.
\end{proof}

\begin{definition} %2.5
	Ein metrischer Raum $(\S,d)$ heißt \textbf{separabel}
	\begin{align*}
		&:\Longleftrightarrow\exists\text{ abzählbares } S_0\subseteq\S:\S\subseteq\overline{S_0}\\
		&\Longleftrightarrow\exists\text{ abzählbares } S_0\subseteq\S:\S=\overline{S_0}\\
		&\Longleftrightarrow\exists\text{ abzählbares } S_0\subseteq\S:S_0\text{ liegt dicht in }\S
	\end{align*}
\end{definition}

\begin{beispiel}\label{beisp2.6} %2.6
	$C([0,1])$ mit Supremums-Metrik ist separabal.
	
	\begin{proof}
		\begin{align*}
			S_0:=\big\lbrace P:P\text{ ist Polynom mit \underline{rationalen} Koeffizienten}\big\rbrace
		\end{align*}
		$S_0$ ist abzählbar. Aus dem \textit{Approximationssatz von Weierstraß} und der Dichtheit von $\Q$ folgt die Behauptung.
	\end{proof}
\end{beispiel}

\begin{definition} %2.7
	$\G_0\subseteq\G$ heißt \textbf{Basis} von $\G:\Longleftrightarrow\forall G\in\G:G$ ist Vereinigung von Mengen aus $\G_0$,
	so genannte \textbf{$\G_0$-Mengen}.
\end{definition}

\begin{beispiel} %2.8
	Die Menge
	\begin{align*}
		\big\lbrace B(x,r):x\in\S,0<r\in\Q\big\rbrace
	\end{align*}
	ist Basis von $\G$, denn:
	\begin{proof}
		Sei $G\in\G$. Dann gilt:
		\begin{align*}
			&\forall x\in G:\exists 0<r_x\in\Q:B(x,r_x)\subseteq G\\
			&\implies
			G=\bigcup\limits_{x\in G}\underbrace{\lbrace x\rbrace}_{\subseteq B(x,r_x)}\subseteq\bigcup\limits_{x\in G} \underbrace{B(x,r_x)}_{\subseteq G}\subseteq G\implies G=\bigcup\limits_{x\in G} \underbrace{B(x,r_x)}_{\in\G_0}
		\end{align*}
	\end{proof}
\end{beispiel}

\begin{satz}\label{satz2.9}
	$\S$ separabel $\Longleftrightarrow\G$ hat abzählbare Basis
\end{satz}

\begin{proof}
	\underline{Zeige ``$\Rightarrow$'':}\\
	Sei $S_0\subseteq\S$ abzählbar und dicht in $\S$. Zeige:
	\begin{align*}
		\G_0:=\big\lbrace B(x,r):x\in S_0,0<r\in\Q\big\rbrace\subseteq\G\text{ ist Basis.}
	\end{align*}
	Sei also $G$ offen. Dann folgt aus Beispiel 2.8:
	\begin{align}\label{proof2.9Sternchen}\tag{$\ast$}
		G=\bigcup\limits_{x\in G} B(x,r_x),\qquad 0<r_x\in\Q,\forall x\in G
	\end{align}
	Da $\overline{S_0}=\S$ gilt:
	\begin{align*}
		&\forall x\in G:\exists y_x\in S_0: d(x,y_x)<\frac{r_x}{2}\\
		&\implies d(x,y)
		\stackrel{\Delta\text{Ungl}}{\leq}
		d(x,y_x)+d(y_x,x)< \underbrace{\frac{r_x}{2}+\frac{r_x}{2}}_{=r_x}\qquad\forall y\in B\left(y_x,\frac{r_x}{2}\right)\\
		&\implies B\left(y_x,\frac{r_x}{2}\right)\subseteq B(x,r_x)\qquad\forall x\in G\\
		&\implies G\stackrel{\eqref{proof2.9Sternchen}}{\supseteq}
		\bigcup\limits_{x\in G}\underbrace{B\left(y_x,\frac{r_x}{2}\right)}_{\supseteq\lbrace x\rbrace}
		\supseteq\bigcup\limits_{x\in G}\lbrace x\rbrace=G\\
		&\implies G=\bigcup	\limits_{x\in G}\underbrace{B\left(y_x,\frac{r_x}{2}\right)}_{\in\G_0}
	\end{align*}
	Also ist $\G_0$ einen Basis. Da $S_0$ abzählbar ist $\G_0$ abzählbar.\nl
	\underline{Zeige ``$\Leftarrow$'':}\\
	Sei $\G_0$ abzählbare Basis von $\G$ und sei o.B.d.A. $\emptyset\notin\G_0$. Wähle für jedes $G\in\G_0$ ein $x_G\in G$ fest aus. Setze
	\begin{align*}
		S_0:=\lbrace x_G:G\in\G_0\rbrace.
	\end{align*}
	$S_0$ ist auch abzählbar. Bleibt Dichtheit zu zeigen.\\
	Sei $x\in\S$ und $\varepsilon>0$. Da $B(x,\varepsilon)$ offen und $\G_0$ Basis, gilt: 
	\begin{align*}
		&\exists\G_{x,\varepsilon}\subseteq\G_0\mit B(x,\varepsilon)=\bigcup\limits_{G\in\G_{x,\varepsilon}} G\\
		&\implies G\subseteq B(x,\varepsilon)\qquad\forall G\in\G_{x\varepsilon}
	\end{align*}
	Wähle ein $G$ von diesen aus. Dann gilt:
	\begin{align*}
		x_G\in G\subseteq B(x,\varepsilon)
		\implies x_G\in B(x,\varepsilon)
		\implies d(\underbrace{x_G}_{\in S_0},x)<\varepsilon
	\end{align*}
\end{proof}

\begin{satz}\label{Satz2.10} %2.10
	Seien $(\S,d)$ und $(\S',d')$ metrische Räume.
	\begin{enumerate}[label={(\arabic*)}]
		\item Auf $\S\times\S'$ sind Metriken definiert durch
		\begin{align*}
			d_1\Big((x,x'),(y,y')\Big)&:=\left( \big(d(x,y)\big)^2+\big(d'(x',y')\big)^2\right)^{\frac{1}{2}} &\forall(x,x'),(y,y')\in \S\times\S'\\
			d_2\Big((x,x'),(y,y')\Big)&:=\max \left\lbrace d(x,y),d'(x',y')\right\rbrace &\forall(x,x'),(y,y')\in \S\times\S'\\
			d_3\Big((x,x'),(y,y')\Big)&:=d(x,y)+d'(x',y') &\forall(x,x'),(y,y')\in \S\times\S'
		\end{align*}
		\item Die Metriken $d_1,d_2,d_3$ induzieren dieselbe Topologie $\mathcal{G}(\S\times \S')$ auf $\S\times\S'$, 
		die sogenannte \textbf{Produkttopologie} von $\mathcal{G}(\S)$ und $\mathcal{G}(\S')$.
		\item $\begin{aligned}
			\mathcal{G}(\S\times\S')=\left\lbrace\bigcup\limits_{\begin{subarray}{c}
				G\in\mathcal{O}\\ G'\in\mathcal{O}'
			\end{subarray}}G\times G':\mathcal{O}\subseteq\mathcal{G}(\S),\mathcal{O}'\subseteq\mathcal{G}(\S')\right\rbrace
		\end{aligned}$\\
		d.h.
		\begin{align*}
			\big\lbrace G\times G':G\in\mathcal{G}(\S),G'\in\mathcal{G}(\S')\big\rbrace
		\end{align*}
		bildet eine Basis von $\mathcal{G}(S\times\S')$.
	\end{enumerate}
\end{satz}

\begin{proof}\enter
	\underline{Zu (1):} Überprüfung der Eigenschaften einer Metrik (zur Übung).\\
	\underline{Zu (2):} Punktweise gelten die Beziehungen:
	\begin{align*}
		d_2\leq d_1\leq\sqrt{2}\cdot d_2,\qquad
		\frac{1}{\sqrt{2}}\cdot d_3\leq d_1\leq d_3,\qquad
		d_2\leq d_3\leq 2\cdot d_2
	\end{align*}
	Beachte beim Nachweis, dass die $d_i$'s als Metriken größer Null sind. Aus obigen Beziehungen folgt u. a.:
	\begin{align*}
		B_{d_2}\left(x,\frac{r}{\sqrt{2}}\right)\subseteq B_{d_1}(x,r)
	\end{align*}
	denn:
	\begin{align*}
		r>\sqrt{2}\cdot d_2(y,x)\geq d_1(y,x)
	\end{align*}
	\underline{Zu (3), zeige ``$\subseteq$'':}\\
	Sei $G^\ast\in\mathcal{G}(\S\times\S')$. Dann gilt:
	\begin{align*}
		\forall x^\ast=(x,y)\in G^\ast:\exists r=r_{x^\ast}>0:
		G^\ast=\bigcup\limits_{x^\ast\in G^\ast} B\big(x^\ast,r_{x^\ast}\big)
	\end{align*}
	Wegen Teil (2) sei o.B.d.A. $\S^\ast:=\S\times\S'$ versehen mit der Metrik $d_2$. Dann gilt:
	\begin{align*}
		B_{d_2}\big(x^\ast,r_{x^\ast}\big)&=\Big\lbrace(y,y')\in \S\times\S':\max\big\lbrace d(x,y),d'(x',y')\big\rbrace<r_{x^\ast}\Big\rbrace\\
		&=\Big\lbrace(y,y')\in\S\times\S':d(x,y)<r_{x^\ast}\wedge d'(x',y')<r_{x^\ast}\Big\rbrace\\
		&= \underbrace{B_d\big(x,r_{x^\ast}\big)}_{\in\mathcal{G}(\S)}\times \underbrace{B_{d'}\big(x', r_{x^\ast}\big)}_{\in\mathcal{G}(\S')}
	\end{align*}
	\underline{Zu (3), zeige ``$\supseteq$'':}\\
	Sei zunächst $G\times G'\mit G,G'$ offen und $x^\ast=(x,x')\in G\times G'$. Also ist $x\in G$ und $x'\in G'$ und somit
	\begin{align*}
		\exists r,r'>0:B_d(x,r)\subseteq G\wedge B_{d'}(x',r')\subseteq G'
	\end{align*}
	Setze $r^\ast:=\min\lbrace r,r'\rbrace>0$. Damit folgt

	\begin{align*}
		B_{d_2}\big( x^\ast,r^\ast\big)&\subseteq B_d(x,r)\times B_{d'}\big(x',r'\big)\\
		&\subseteq
		G\times G'=G^\ast\\
		&\implies
		G\times G'\in\mathcal{G}(\S\times\S')\\
		&\implies
		\bigcup\limits_{\begin{subarray}{c} G\in\mathcal{O}\\G'\in\mathcal{O}'\end{subarray}}G\times G'\subseteq\mathcal{G}(\S\times\S')
		\qquad\forall\mathcal{O}\subseteq\mathcal{G}(\S),\mathcal{O}'\subseteq\mathcal{G}(\S')
	\end{align*}
	da die Produkttopologie vereinigungsstabil ist.
\end{proof}

\begin{defi}
	Die Metriken $d_1,d_2,d_3$ heißen \textbf{Produktmetriken}.
	Daher alternative\\ Schreibweise $d\times d'$, also z. B. $d\times d':=\max\lbrace d,d'\rbrace$ usw.
\end{defi}

\begin{bemerkungnr} %2.11
	Analog lassen sich Produktmetriken für \underline{endlich viele} metrische Räume $(S_i,d_i)_{i\in\lbrace1,\ldots,k\rbrace}$ definieren, z. B.
	\begin{align*}
		d_1\times\ldots\times d_k:=\left(\sum\limits_{i=1}^k d_i^2\right)^{\frac{1}{2}},
	\end{align*}
	die wiederum dieselbe Produkttopologie induzieren.
\end{bemerkungnr}