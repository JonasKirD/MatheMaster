% This work is licensed under the Creative Commons
% Attribution-NonCommercial-ShareAlike 4.0 International License. To view a copy
% of this license, visit http://creativecommons.org/licenses/by-nc-sa/4.0/ or
% send a letter to Creative Commons, PO Box 1866, Mountain View, CA 94042, USA.

\newcommand{\directoryPrefix}{../latex/} % Je nach Ordnertiefe muss dieser Command angepasst werden. Bei Fragen mich anschreiben.
\input{\directoryPrefix templates}
\TemplateSummary{Willi Sontopski}{MSTAT}

\begin{document}
	\section{Der Median}
	\begin{itemize}
		\item $m$ Median $\gdw m=\argmin(\E[|X-t|])\gdw F(m-)\leq\frac{1}{2}\leq F(m)\gdw \E[|X-t|]\geq\E[|X-m|]~\forall t\in\R$
		\item Trafo: $\int\limits_\Omega g(X(\omega))~\P(\d\omega)
		=\int\limits_\R g(x)~\P_X(\d x)
		=\int\limits_\R g(x)\cdot f_X(x)\ds x$
		\item Median i.A. \betone{nicht} eindeutig, wähle $m:=F^{-1}(1/2)\mit F^{-1}(u):=\inf\lbrace x\in\R:F(x)\geq u\rbrace$ \define{verallg. Inverse / Quantilfunktion}; $Y$ stetig, aber nicht diffbar; Minimierung über Ableiten unmöglich
		\item empir. Median $\hat{m_n}:=\argmin\limits_{t\in\R}Y_n(t)\mit Y_n(t):=\int_\R|x-t|~F_n(\d x)=\frac{1}{n}\sum\limits_{i=1}^n|x_i-t|$
		\item SGGZ sagt: $Y_n(t)\overset{n\to\infty}{\longrightarrow}Y(t)$ f.s. $\forall t\in\R$.
	\end{itemize}
	
	\section{Konzepte aus metrischen Räumen}
	\begin{itemize}
		\item $\S$ metrischer Raum; $\F$ Menge der offenen Teilmengen und $\G$ Menge der geschlossenen Teilmengen
		\item "Bump-Function": $\forall A\subseteq\S,\forall\varepsilon:\exists f_A:\S\to[0,1]$ glm. stetig s.d. $f_A\approx\indi_A$ ($f_A(x)=0$ für $d(x,A)\geq\varepsilon$)
		\item MR \define{separabel} $:\gdw\exists S_0\subseteq\S$ abzählbar s.d $S_0=\S\gdw\exists S_0\subseteq\S$ abzählbar mit $S_0$ liegt dicht in $\S$
		\item $\G_0\subseteq\G$ \define{Basis} $:\gdw\forall G\in\G:G$ ist $\bigcup$ von Mengen aus $\G_0$; $\S$ separabel $\gdw\G(\S)$ hat abzählbare Basis
		\item \define{Produktmetriken}: $d_1\times d_2:\in\big\lbrace \sqrt{d_1^2+d_2^2}, d_1+d_2,\max(d_1,d_2)\big\rbrace$; sind äquivalent, erzeugen selbe Topo
	\end{itemize}
	
	\section{Zufallsvariablen in metrischen Räumen}
	\begin{itemize}
		\item Borel-$\sigma$-Algbra auf $\S$ ist $\B(\S):=\sigma(\G(\S))=\sigma(\F(\S))\overset{\G_0\text{ abz- Basis}}{=}\sigma(\G_0)$, hängt. i.A. von $d$ ab.
		\item $\B_{d_1\times d_2}(\S\times\S)=\B_{d_1}(\S_1)\otimes\B_{d_2}(\S_2)$ für separable MR
		\item \define{Zufallsvariable} ist $X\colon\Omega\to\S$ die $\A$-$\B(\S)$-messbar ist ($(\Omega,\A)$ Messraum, $(\S,d)$ MR).
		\item \define{Verteilung} von $X$ unter $\P$ ($(\Omega,\A,\P)$ WR) ist
		$(\P\circ X^{-1})(B):=\P[X\in B]~\forall B\in\B(\S)$
		\item $(\S,d)$ separabel, $X,Y$ ZV $\implies d(X,Y)$ ist reelle ZV (jede Metrik stetig)
		\item $X_n\stackrel{n\to\infty}{\longrightarrow} X~\P\text{ f.s.}
		:\gdw\P\big(\big\lbrace\omega\in\Omega:d\big(X_n(\omega),X(\omega)\big)\stackrel{n\to\infty}{\longrightarrow} 0\big\rbrace\big)=1\gdw d(X_n,X)\overset{n\to\infty}{=}0$ f.s.
		\item $X_n\overset{n\to\infty}{\longrightarrow}X$ f.s. und $f$ messbar und stetig in $X\implies f(X_n)\overset{n\to\infty}{\longrightarrow}f(X)$ f.s.
		\item 
		$X_n
		\stackrelnew{n\to\infty}{\P}{\longrightarrow}
		X:\gdw\forall\varepsilon>0:
		\P\Big(\big\lbrace d(X_n,X)>\varepsilon\big\rbrace\Big)
		\stackrel{n\to\infty}{\longrightarrow}
		0$
		\define{stochastische K. / K. in Wahrscheinlichkeit}
		\item Konvergenz f.s. $\implies$ Konvergenz in W.; Umkehrung nicht wegen "wandernden Hüten"
		\item $X_n\overset{\P}{\longrightarrow}X\gdw$ Zu jeder TF $X_n'$ existiert TTF $(X_n'')$ s.d. $X_{n''}\overset{\text{f.s.}}{\longrightarrow}X$ f.s.
		\item $X_n\overset{\P}{\longrightarrow}X$ und $f$ messbar und stetig in $X\implies f(X_n)\overset{\P}{\longrightarrow}f(X)$
		\item $(\S_1,d_1),(\S_,d_2)$ separabel $\implies$ Produktraum $(\S_1\times\S_2,d_1\times d_2)$ separabel
		\item Für beide Konvergenzarten gilt koordinatenweise Konvergenz.
		\item $X,Y$ \define{gleich in Verteilung}, i.Z. $X\overset{\L}{=}Y:\gdw \P\circ X^{-1}=\P\circ Y^{-1}\gdw\E[f(X)]=\E[f(Y)]~\forall f\in C^b(\S)$ glm
		\item $\P_1=\P_2\gdw\int f\d\P_1=\int f\d\P_2~\forall f\in C^b(\S)$ glm. stetig (gilt wegen Bump functions)
	\end{itemize}
	
	\section{Verteilungskonvergenz von Zufallsvariablen in metrischen Räumen}
	
	\begin{itemize}
		\item $\P_n\stackrelnew{w}{n\to\infty}{\longrightarrow} \P
			:\gdw
			\int\limits f\d \P_n\stackrel{n\to\infty}{\longrightarrow}\int\limits f\d \P~\forall f\in C^b(\S)$
			\define{schwache Konvergenz von Maßen}
		\item $X_n\stackrel{\mathcal{L}}{\longrightarrow} X\text{ in }(\S,d)
			:\gdw
			\P\circ X_n^{-1}\stackrelnew{w}{n\to\infty}{\longrightarrow}\P\circ X^{-1}$
			\define{Konvergenz in Verteilung}
		
		\item $\P_n\overset{\text{w}}{\longrightarrow}\P\gdw F_n\rightharpoonup F:\gdw F_n(x)\overset{n\to\infty}{\longrightarrow} F(x)~\forall x\in C_F$ (Stetigkeitsstelle) \define{schwache Konvergenz von Verteilungsfunktionen} ($\P_n$ zu $F_N$ assoziiert)
		\item $X_n\stackrel{\L}{\longrightarrow} X
			\stackrel{\text{Def}}{\Longleftrightarrow}
			\underbrace{\P\circ X_n^{-1}}_{\hat{=}P_n}
			\stackrelnew{w}{}{\longrightarrow} \underbrace{\P\circ X^{-1}}_{\hat{=}P}
			\stackrel{}{\Longleftrightarrow}
			\underbrace{\P(X_n\leq x)}_{\hat{=}F_n(x)}
			\stackrel{n\to\infty}{\longrightarrow}
			\underbrace{\P(X\leq x)}_{\hat{=}F(x)}$ für alle Stetigkeitsstellen $x$
	\end{itemize}
	
\begin{minipage}{0.49\textwidth}
	\define{Portmanteau-Theorem 1}: Äquivalent:
		\begin{enumerate}[label=(\arabic*)]
		\item $\begin{aligned}
			\P_n\stackrelnew{w}{}{\longrightarrow} \P
		\end{aligned}$
		\item $\begin{aligned}
			\int\limits f\d \P_n\stackrel{}{\longrightarrow}\int\limits f\d \P~\forall f\in C^b(\S)\text{ glm. stetig}
		\end{aligned}$
		\item $\begin{aligned}
			\limsup\limits_{n\to\infty} \P_n(F)\leq \P(F)~\forall F\in\F(\S)
		\end{aligned}$
		\item $\begin{aligned}
			\liminf\limits_{n\to\infty} \P_n(G)\geq \P(G)~\forall G\in\G(\S)
		\end{aligned}$
		\item $\begin{aligned}
			\limn \P_n(B)=\P(B)~\forall B\in\B(\S)
		\end{aligned}$\\ $\mit \P(\partial B)=0$, also \textbf{$\P$-randlos}.
	\end{enumerate}
\end{minipage}
\begin{minipage}{0.49\textwidth}
	\define{Portmanteau 2}: Äquivalent:
		\begin{enumerate}[label=(\arabic*)]
		\item $\begin{aligned}
			X_n\stackrel{\L}{\longrightarrow} X\text{ in }(\S,d)
		\end{aligned}$
		\item $\begin{aligned}
			\E\big[f(X_n)\big]\stackrel{n\to\infty}{\longrightarrow}\E\big[f(X)\big]~\forall f\in C^b(\S)
		\end{aligned}$ glm stetig
		\item $\begin{aligned}
			\limsup\limits_{n\to\infty}\P(X_n\in F)\leq\P(X\in F)~\forall F\in\F
		\end{aligned}$
		\item $\begin{aligned}
			\liminf\limits_{n\to\infty}\P(X_n\in G)\geq\P(X\in G)~\forall G\in\G
		\end{aligned}$
		\item $\begin{aligned}
			\P(X_n\in B)\stackrel{n\to\infty}{\longrightarrow}\P(X\in B)~\forall B\in\B(\S)
		\end{aligned}$ $\mit\P(X\in\partial B)=0$
	\end{enumerate}
\end{minipage}

\begin{itemize}
	\item Portmanteau 1: (1) $\Rightarrow$ (2): Def, (2) $\Rightarrow$ (3): Bump fs;
	Portmanteau 2 folgt aus 1 mit $P_n:=\P\circ X_n^{-1}$ und $P:=\P\circ X^{-1}$.
	\item \define{CMT}: Für $h\colon(\S,d)\to(\S',d')$ $\B(\S)$-$\B(\S')$-messbar gilt ($D_h$ Menge der Unstetigkeitsstellen von $h$):
	\begin{enumerate}[label=(\arabic*)]
		\item $\begin{aligned}
			P_n\stackrelnew{w}{}{\longrightarrow} P\wedge P(D_h)=0
			\implies P_n\circ h^{-1}\stackrelnew{w}{}{\longrightarrow} P\circ h^{-1}
		\end{aligned}$
		\item $\begin{aligned}
			X_n\stackrel{\L}{\longrightarrow}X\text{ in }(\S,d)\wedge\P(X\in D_h)=0
			\implies h(X_n)\stackrel{\L}{\longrightarrow} h(X)\text{ in }(\S',d') 
		\end{aligned}$ ("$h$ stetig in $X$")
	\end{enumerate}
	\item \textit{Beweis.} (2) folgt direkt aus (1) nach Definition.
	Nutze Portmanteau 1, ((1) $\gdw$ (3)):\\
	$\limsup\limits_{n\to\infty} \P_n\circ h^{-1}(F)
		=\limsup\limits_{n\to\infty} \P_n\big(\underbrace{h^{-1}(F)}_{\subseteq \overline{h^{-1}(F)}}\big)
		\leq \limsup\limits_{n\to\infty} \P_n\big(\underbrace{\overline{h^{-1}(F)}}_{\in\F(\S)}\big)
		\overset{\text{Portm}}{\leq}
		\P\big(\overline{h^{-1}(F)}\big)\\
		\stackrel{(*)}{\leq}
		\P\Big(h^{-1}(F)\cup D_h\Big)
		\leq \P\Big(h^{-1}(F)\Big)+\underbrace{\P(D_h)}_{\overset{\Vor}{=}0}
		=\P\circ h^{-1}(F)
		$
		Bei $(*)$: Sei $x\in\overline{h^{-1}(F)}$. Falls $x\in D_h$ trivial, $x\not\in D_h\leadsto h$ stetig in $x$.
		Sei $(x_n)_{n\in\N}\subseteq\S\mit x_n\longrightarrow x$ (existiert nach Def des Abschlusses) $\leadsto h(x_n)=h(x)\overset{!}{\in} \overline{F}=F\leadsto x\in h^{-1}(F)~\square$   TF-Prinzip für schwache Konvergenz:
	\item $\begin{aligned}
			Q_n\stackrelnew{w}{}{\longrightarrow} Q
			\Longleftrightarrow
			\text{Jede TF }(Q_{n'})\subseteq(Q_n)_{n\in\N}\text{ enthält TF }(Q_{n''})\subseteq(Q_{n'}):Q_{n''}\stackrelnew{w}{}{\longrightarrow} Q
		\end{aligned}$
	\item $\begin{aligned}
			X_n\stackrel{\L}{\longrightarrow} X\Longleftrightarrow\text{Jede TF }(X_{n'})\subseteq(X_n)_{n\in\N}\text{ enthält TF }(X_{n''})\subseteq(X_{n'}):X_{n''}\stackrel{\L}{\longrightarrow} X
		\end{aligned}$
	\item $X_n\stackrel{\P}{\longrightarrow} X\implies X_n\stackrel{\L}{\longrightarrow} X$; Umkehrung i.A. nicht, nur falls $X$ f.s. konstant
	\item \define{Cramér}: Seien $(X_n)_{n\in\N},(Y_n)_{n\in\N}$ Folgen im separablen metrischen Raum $(\S,d)$, die \textbf{stochastisch äquivalent sind}, d.h. $d(X_n,Y_n)\stackrel{\P}{\longrightarrow}0$
	Dann gilt:
	$X_n\stackrel{\L}{\longrightarrow} X\Longleftrightarrow Y_n\stackrel{\L}{\longrightarrow} X$
	\item \define{Cramér-Slutsky} (Koordinatenweise Konvergenz): $(\S,d),(\S',d')$ separable MR, $X_n\overset{\L}{\longrightarrow}X$ in $(\S,d)$, $Y_n\overset{\L}{\longrightarrow}Y$ in $(\S',d')$ und  $Y$ f.s. konstant. Dann gilt: $(X_n,Y_n)\overset{\L}{\longrightarrow}(X,Y)$ in $(\S\times\S',d\times d')$. (folgt aus Cramér)
	\item $\begin{aligned}
			\P_n\stackrelnew{\omega}{}{\longrightarrow} \P 
		\end{aligned}\gdw\begin{aligned}
			\P_n(A_1\times A_2)\stackrel{n\to\infty}{\longrightarrow} \P(A_1\times A_2)~\forall A_i\in\B(\S_i)~\P_i\text{-randlos mit } i=1,2
		\end{aligned}$ für $(\S,d)$ separabel
	\item $\big(\P_n^{(1)}\otimes\P_n^{(2)}\big)\stackrelnew{w}{}{\longrightarrow}\P^{(1)}\otimes\P^{(2)}\Longleftrightarrow
	\begin{aligned}
			\P_n^{(1)}\stackrelnew{w}{}{\longrightarrow}\P^{(1)}\wedge
			\P_n^{(2)}\stackrelnew{w}{}{\longrightarrow}\P^{(2)}
		\end{aligned}$ im separablem Produktraum $\S=\S_1\times\S_2$
	\item $X_n\stackrel{\L}{\longrightarrow} X\text{ in }\S_1\wedge Y_n\stackrel{\L}{\longrightarrow} Y\text{ in }\S_2
		\Longleftrightarrow (X_n,Y_n)\stackrel{\L}{\longrightarrow}(X,Y)\text{ in }\S_1\times\S_2$ für $\S=\S_1\times\S_2$ separabel, $X_n$ und $Y_n$ sind unabhängig für alle $n\in\N$; $X$ und $Y$ sind unabhängig
	\item Sei $(X_n)_{n\in\N}$ iid mit $\E[X_i]=:\mu$ und $\sigma^2:=\Var(X_i)\in(0,\infty)$. 
	Dann (SGGZ + ZGWS + CMT): $\sqrt{n}\cdot(\overline{X}-\mu)\overset{\L}{\longrightarrow}\Nor(0,\sigma^2)$ mit $\overline{X}$ emp. Erwartung;
	$\big(\overline{X}_n,S_n^2\big)\stackrel{n\to\infty}{\longrightarrow}\big(\mu,\sigma^2\big)\text{ in }\R^2$ f.s.
	\item Also: $(\overline{X}_n)_{n\in\N}$ und $(S_n^2)_{n\in\N}$ \textbf{asymptotisch normal}, d.h.
	$\sqrt{n}\cdot\big(\overline{X}_n-\mu\big)\stackrel{\L}{\longrightarrow}\mathcal{N}(0,\sigma^2),\qquad
	\sqrt{n}\cdot\big(S_n^2-\sigma^2\big)\stackrel{\L}{\longrightarrow}\mathcal{N}(0,\tau^2)$
\end{itemize}

	\section{Verteilungskonvergenz in \texorpdfstring{$\R^d$}{Rd}}
	\define{CF} $\varphi_X(t):=\E[\exp(\ii\cdot\langle t,X\rangle)]$; Eindeutigkeitssatz: $X\overset{\L}{=}Y\gdw\varphi_X=\varphi_Y$;\\
	Stetigkeitssatz (SSS): $X_n\overset{\L}{\longrightarrow}X$ in $\R^d\gdw\forall t\in\R^d:\varphi_{X_n}(t)\overset{n\to\infty}{\longrightarrow}\varphi_X(t)$\\
	\define{Cramér-Wold-Device}: $X_n\overset{\L}{\longrightarrow}X$ in $\R^d\Longleftrightarrow\langle t,X_n\rangle\overset{\L}{\longrightarrow}\langle t,X\rangle$ in $\R$ für alle $t\in\R^d$\\
	\textit{Beweis.} "(1) $\Rightarrow$ (2)" folgt aus CMT, da $x\mapsto\langle x,t\rangle$ stetig.\\
	"(2) $\Rightarrow$ (1)":
	$	\varphi_{X_n}(t)
		\stackeq{\text{Def}}\E\Big[\exp\big(i\cdot\langle t,X_n\rangle\cdot 1\big)\Big]
		\stackeq{\text{Def}}\varphi_{\langle t,X_n\rangle}(1)
		\stackrelnew{\text{SSS+(2)}}{n\to\infty}{\longrightarrow}\underbrace{\varphi_{\langle t,X\rangle}(1)}_{=\varphi_X(t)}
		\leadsto\varphi_{X_n}\stackrel{n\to\infty}{\longrightarrow}\varphi_X
		\stackrel{\text{SSS}}{\implies}(1)$

	\section{Der multivariate zentrale Grenzwertsatz (ZGWS) für Dreiecksschemata}
	\begin{itemize}
		\item \define{$\triangle$-Schema} ist $\lbrace X_{n,k}:1\leq k\leq n,n\in\N\rbrace$ mit $X_{n,k}$ unabhängige reelle ZV.
		\item \define{Lindeberg-ZGWS}: $\triangle$ mit $\E[X_{n,k}]=0$, $\sigma_{n,k}^2:=\E[X_{n,k}^2]<\infty$ und $s_n^2:=\sum\limits_{k=1}^n\sigma_{n,k}^2=\Var\Big(\sum\limits_{k=1}^n X_{n,k}\Big)$ und \define{LB} 
		$\sum\limits_{k=1}^n\E\Big[X_{n,k}^2\cdot\indi_{\lbrace|X_{n,k}|>\varepsilon\rbrace}\Big]\overset{n\to\infty}{\longrightarrow}0~\forall\varepsilon>0$ und $s_n^2\overset{n\to\infty}{\longrightarrow}\sigma^2\in(0,\infty)$ Dann: $\sum\limits_{k=1}^n X_{n,k}\overset{\L}{\longrightarrow}\Nor(0,\sigma^2)$
		\item multivariater Fall: Sei $\lbrace X_{n,k}:k\leq n,n\in\N\big\rbrace$ ein $\Delta$-Schema von ZV
		$X_{n,k}=\left(X_{n,k}^{(1)},\ldots,X_{n,k}^{(d)}\right)\text{ in }\R^d$
		Es gelte die \textbf{zeilenweise Unabhängigkeit}:
		$X_{n,1},\ldots,X_{n,n}\text{ sind unabhängig}\qquad\forall n\in\N$
		(Also die Vektoren seien unabhängig. Daraus folgt nicht, dass deren Komponenten unabhängig sind.)
		Gelte auch 
		$\E\big[X_{n,k}\big]:=\left(\E\left[X_{n,k}^{(j)}\right]\right)_{1\leq j\leq d}=0:=(0,\ldots,0)
		~\forall k,n\in\N$ und
		$\E\left[\left(X_{n,k}^{(j)}\right)\right]<\infty
		~\forall 1\leq j\leq d,\forall n,k\in\N$
		(Kovarianzmatrix?)
		Gelte LB
		$\sum\limits_{k=1}^n\E\Big[\Vert X_{n,k}\Vert^2\cdot\indi_{\big\lbrace\Vert X_{n,k}\Vert>\varepsilon\big\rbrace}\Big]\stackrel{n\to\infty}{\longrightarrow}0\qquad\forall\varepsilon>0$
		und Normierungsbedingung
		$\sum\limits_{k=1}^n\Cov\left(X_{n,k}\right)\stackrelnew{\text{komponentenweise}}{n\to\infty}{\longrightarrow}\Gamma\mit\Gamma\in\R^{d\times d}\text{ positiv definit}
		$
		Dann MZGWS:
		$\sum\limits_{k=1}^n X_{n,k}\stackrel{\L}{\longrightarrow}\mathcal{N}_d(0,\Gamma)\text{ in }\R^d$
		\item Koro ($d=1$ ist klassischer ZGWS):
		Sei $(X_i)_{i\in\N}$ iid in $\R^d$ mit
		$\E\Big[(X_1^{(j)})^2\Big]<\infty~\forall 1\leq j\leq d
		;\\
		\mu:=\E\big[X_1\big]=\left(\E\left(X_1^{(1)}\right),\ldots,\E\left(X_1^{(d)}\right)\right)\in\R^d,~
		\Gamma:=\Cov(X_1)
		=\left(\Cov\left(X_1^{(i)},X_1^{(j)}\right)\right)_{i,j=1}^d\\
		=\left(\E\left[\left(X_1^{(i)}-\mu_i\right)\cdot\left(X_1^{(j)}-\mu_j\right)\right]\right)_{i,j=1}^d\text{ positiv definit}
		$
		Dann:
		$
		\frac{1}{\sqrt{n}}\cdot\sum\limits_{i=1}^n(X_i-\mu)\stackrel{\L}{\longrightarrow}\mathcal{N}_d(0,\Gamma)
		$
	\end{itemize}
	
	\section{Verteilungskonvergenz im Raum stetiger Funktionen}
	\begin{itemize}
		\item $\B(C)
			=\sigma\Big(\pi_t:t\in I\Big)=\sigma\Big(\pi_T:T\subseteq I,T\text{ endlich}\Big)
			=\sigma\Big(\big\lbrace \pi_t^{-1}(B):t\in I,B\in\B(\R)\big\rbrace\Big)$\\
		"kleinste $\sigma$-Algebra, sodass alle $\pi_t$ messbar sind"
		\item $X\colon I\to\R,\qquad t\mapsto X(t,\omega)$ heißt \define{Pfad} für jedes $\omega\in\Omega$
		\item $X\colon\Omega\to\R,\qquad\omega\mapsto X(t,\omega)$ ist eine einzelne ZV für jedes $t\in I$
		\item $X\colon\Omega\to (I\to\R),\qquad \omega\mapsto(t\mapsto X(t,\omega))$ heißt \define{Pfadabbildung}
		\item Falls alle Pfade stetig (also $X$ ein \define{stetiger stochastischer Prozess} ist) sind, gilt $I\to\R=C(I)$.
		\item Man kann einen (stetigen) stochastischen Prozess mit seiner Pfadabbildung identifizieren: 
		$\big\lbrace X(t)\mid t\in I, X(t)\colon\Omega\to\R\big\rbrace
			\cong X\colon\Omega\to C(I)
		$
		\item $	X\text{ ist }A\text{-}\B(C)\text{-messbar}\Longleftrightarrow\forall t\in I:\pi_t\circ X\text { ist }\A\text{-}\B(R)\text{-messbar}$;
		Alle Pfadabbildungen $X_t:\Omega\to C(I)$ eines stetigen SP sind $\A$-$\B(S)$-messbar.
		\item Maße $P=Q\gdw\forall T\subseteq I$ endlich$:P\circ\pi_T^{-1}=Q\circ\pi_T^{-1}$;
		$X\overset{\L}{=}Y\gdw\pi_T(X)\overset{\L}{=}\pi_T(Y)$
		\item \define{endlich dim. Randverteilungen (fidis)}: $\pi_T\circ X:\Omega\to\R^d$ ist Zufallsvektor, genauer:\\
		$\pi_T\circ X\colon\Omega\to C\to\R_k,~
		\pi_T\big(X(\omega)\big)=\big(X(\omega)(t_1),\ldots,X(\omega)(t_k)\big)\forall\omega\in\Omega,\forall T=\big\lbrace t_1,\ldots,t_k\rbrace\subseteq I$
		\item \define{Stetigkeits- / Oszillationsmodul}
	$\omega(f,\delta):=\sup\limits\lbrace |f(s)-f(t)|:s,t\in I\mit |s-t|\leq\delta\rbrace$;
	$f\in C(I)\gdw\omega(f,\delta)\overset{\delta\to0}{\longrightarrow}0$
		\item $X_n\overset{\fd}{\longrightarrow}X:\gdw\pi_T\circ X_n\overset{\L}{\longrightarrow}\pi_T\circ X~\forall T\subseteq I$ endlich \define{Konvergenz der fidis}
		\item $X_n\overset{\fd}{\longrightarrow}X$ und 
	$\lim\limits_{k\to\infty}\limsup\limits_{n\to\infty}\P\Big(\omega\big(X_n,\delta_k\big)>\varepsilon\Big)=0~\forall\varepsilon>0$
		für eine Folge $(\delta_k)_{k\in\N}\subseteq(0,\infty)\mit\delta_k\downarrow0$ Dann gilt:
	$X_n\stackrel{\L}{\longrightarrow}X\text{ in }(C,d)$
		\item \define{Momentenkriterium von Kolmo}: $X_n\stackrelnew{\text{fd}}{}{\longrightarrow} X$ und $\exists\gamma>0,\exists\alpha>1$ und $F\colon I\to\R$ stetig + monoton wachsende mit
		$\E\Big[|X_n(s)-X_n(t)|^\gamma\Big]\leq\big(F(s)-F(t)\big)^\alpha~\forall s>t,s,t\in I$ Dann: 
	$X_n\stackrel{\L}{\longrightarrow} X\text{ in }\big(C(I),d\big)$
		\item Konvergenz von SP $\Leftarrow$ Konvergenz der fidis + "Straffheit" (Momentenkriterium); bei konvexen SP reicht Konvergenz der fidis (Straffheit automatisch erfüllt)
		\item \define{BB}: $I=[0,b]$ und $B:=\big\lbrace B(t):=B(t,\omega)\big\rbrace$ stetiger SP mit $B(0)=B(0,\omega)=0$, unabhängige Zuwächse:
		$B(t_i)-B(t_{i-1}),~1\leq i\leq r~\forall~ 0=:t_0\leq t_1<\ldots<t_r\leq b$;
		Normalverteile Zuwächse $0\leq s<t\leq b\implies B(t)-B(s)\sim\mathcal{N}(0,t-s)$; BB existiert nach Lévy; Verteilung einer BB eindeutig bestimmt
		\item \define{Donsker}: Sei $(\xi_i)_{i\in\N}$ iid mit $\E[\xi_1]=0$ und $\Var(\xi_1)=1$, $S_k:=\sum\limits_{i=1}^k\xi_i$ (RW!) und $X_n(t)$ der Polygonzug durch die Punkte $\big(\frac{k}{n},\frac{S_k}{\sqrt{n}}\big)_{0\leq k\leq b\cdot n}$ Dann: $X_n\overset{\L}{\longrightarrow}$ BB in $(C([0,b]),d)$. (folgt aus Mom. von Kolmo);\\
		Standardisierung $\hat{\xi}:=\frac{\xi_i-\mu}{\sigma}$ kann helfen; BB hängt \betone{nicht} von Verteilung ab
		\item \define{Change-Point-Problem:} $X_{1,n},\ldots,X_{n,n},n\in\N$ unabhängig mit 
$\left\lbrace\begin{array}{cl}
		X_{i,n}\text{ i.i.d.}\sim(\mu,\sigma^2), &\falls 1\leq i\leq\tau_n\\
		X_{i,n}\text{ i.i.d.}\sim(\nu,\tau^2), &\falls \tau_n< i\leq n
	\end{array}\right.$
wobei $\tau_n\in\lbrace1,\ldots,n\rbrace$ der \ul{unbekannte} \define{Change-point} "$\triangle$-Schema, damit change-point mitwandern kann"
		\item \define{Brownsche Brücke} ist SP $B_0(t):=B(t)-t\cdot B(1)\qquad\forall t\in[0,1]$ für BB auf $[0,1]$ ($B_0(0)=1$ und $B_0(1)=0$)
		\item Der bisherige Fall $C(I)$ mit $I=[a,b]$ deckt $C(\R)$ nicht ab.
	$C(\R)$ ist vollständiger separabler MR; $\B_d(C(\R))=\sigma(\pi_t:t\in\R)=\sigma(\pi_T:T\subseteq\R$ endlich) (analog);
	$X$ $\A$-$\B_d(C)$-messbar $\gdw\forall t\in\R:\pi_t\circ X$ $\A$-$\B(\R)$-messbar
		\item Für $X,Y$ ZV in $(C(\R),d)$ gilt: $X\overset{\L}{=}Y\gdw\pi_T\circ X\overset{\L}{=}\pi_T\circ Y\gdw\big(X(t_1),\ldots,X(t_k)\big)\stackeq{\L}\big(Y(t_1),\ldots,Y(t_k)\big)~\forall t_j$
		\item $X_n\stackrel{\L}{\longrightarrow}X\text{ in }\big(C(\R),d\big)
		\Longleftrightarrow\forall j\in\N:
		X_n|_{I_j}\stackrel{\L}{\longrightarrow} X|_{I_j}\text{ in }\big(C(I_j),d_j\big)\mit I_j:=[-j,j]$
	\end{itemize}

	\section{Argmin-Theoreme in \texorpdfstring{$C(\R)$}{C(R)}} %8
	\begin{itemize}
		\item Wann überträgt sich die Konvergenz (f.s. oder in Verteilung) von stetigen stochastischen Prozessen auf deren Minimalstellen?
		\item Für $f\in C(\R)$: $A(f):=\argmin(f):=\big\lbrace t\in\R:f(t)=\inf\limits_{s\in\R}f(s)\big\rbrace$ Menge aller Min-Stellen;\\
		$\tau\in A(f)$ \define{wohl-separiert} $:\gdw\inf\lbrace f(t):|t-\tau|\geq\varepsilon\rbrace>f(\tau)~\forall 0<\varepsilon\in\Q\Rightarrow\tau$ eindeutig (siehe Bild)
		\item 8.3: $f,f_n,n\in\N$ aus $C(\R)$, $\tau_n\in A(f_n)\neq\emptyset~\forall n_0\in\N$ und $\tau\in A(f)$ wohlsepariert und $\Vert f_n-f\Vert_\infty\overset{n\to\infty}{\longrightarrow}0$ Dann: $\tau_n\overset{n\to\infty}{\longrightarrow}\tau$; lässt sich übertragen auf offene und kompakte Intervalle (da muss $\tau$ nur eindeutig sein)
		\item 8.4: $f,f_n,n\in\N$ aus $C([a,b])$. Dann: $A(f_n)\neq\emptyset$ (da kompakt) und:
		Falls $\lbrace\tau\rbrace=A(f)$ und $\Vert f_n-f\Vert_\infty\overset{n\to\infty}{\longrightarrow}0$ so gilt für \underline{jede} Auswahl $\tau_n\in A(f_n):\tau_n\overset{n\to\infty}{\longrightarrow}\tau$
		\item 8.5: $M,M_n,n\in\N$ SP mit Pfaden in $C(\R)$ (d.h. $M\colon \R\to\R$ stetig), $\tau(\omega)\in A\big(M(\cdot,\omega)\big)$%\overset{\Def}{=}			\set{t\in R:\inf\limits_{s\in\R}M(s,\omega)=M(t,\omega)}$ 
		f.s. für ZV $\tau\colon\Omega\to\R$; $\inf\limits\big\lbrace M(t):|t-\tau|\geq\varepsilon\big\rbrace>M(\tau)\text{ f.s.}\qquad\forall 0<\varepsilon\in\Q$; $\norm[\big]{M_n-M}_\infty\overset{n\to\infty}{\longrightarrow}0$ f.s.;
	$\forall(\tau_n)_{n\in\N}$ von ZV mit $\tau_n\in A(M_n)$ f.s.:
	Dann: $\tau_n\overset{n\to\infty}{\longrightarrow}\tau$ f.s.
	\textit{Beweis.} Abzählbare Schnitte von 4 Einsmengen + 8.3 $\square$
		\item 8.6: $M$,$M_n$, $n\in\N$ SP mit Pfaden in $C(I)$, $I$ kompakt, $\exists\tau$ ZV eindeutige Minstelle f.s.; $\norm{M_n-M}_\infty$ f.s.; $(\tau_n)_{n\in\N}$ mit $\tau_n\in A(M_n)$ f.s beliebig.
		Dann: $\tau_n\overset{n\to\infty}{\longrightarrow}\tau$ (folgt aus 8.4)
		\item 8.7: Exponential-Familie, $\Theta\subseteq\R$ kompakt. Dann MLE stark konsistent (nach 8.6), also $\hat{\theta}_n\overset{n\to\infty}{\longrightarrow}\theta_0$ f.s.
		\item $(\sigma_n)_{n\in\N}$ \define{stochastisch beschränkt} $:\gdw\lim\limits_{t\to\infty}\limsup\limits_{n\to\infty}\P(|\sigma_n|>j)=0$
		\item CPP: Schätzer $\hat{\tau}_n:=\argmax\limits_{0\leq k\leq n}|S_k|$ mit $S_k:=\sum\limits_{i=1}^k(X_i-\overline{X}_n)$ ($S_n=0$)
	\end{itemize}
	
	\section{Verteilungskonvergenz im Raum konvexer Funktionen}
	\begin{itemize}
		\item $O\subseteq\R$ \define{konvex} $:\gdw x,y\in O\Rightarrow\forall\lambda\in(0,1):\lambda x+(1-\lambda)y\in O$ (Intervall);
		$f\colon O\to\R$ \define{konvex} $:\gdw f(\lambda x+(1-\lambda)y)\leq\lambda f(x)+(1-\lambda) f(y)~\forall x,y\in O,\lambda\in(0,1)$;
		$C_c(O):=\set{f\colon O\to\R:f\text{ konvex}}$ (konvex $\Rightarrow$ stetig);
		SP $X=\lbrace X(t):t\in O)$ mit Pfaden in $C_c(O)$ heißt \define{konvex}
		\item $C_c(O)$ separabel;
		$\B_d\big(C_c(O)\big)=\sigma\big(\pi_t:t\in O\big)=\sigma\big(\pi_T:T\subseteq O\text{ endlich}\big)$
		\item $X\overset{\L}{=}Y\gdw\pi_T\circ X=\pi_T\circ Y$ (fidis gleich)
		\item $f_n$ konvex, $f\colon O\to\R$, $D\subseteq O$ dicht, $f_n(x)\overset{n\to\infty}{\longrightarrow}f(x)~\forall x\in D$ Dann: Konvergenz $\forall x\in O$, $f$ konvex und $\norm{f_n-f}_\infty\overset{n\to\infty}{\longrightarrow}0~\forall K\in O$ kompakt
		\item  $X_n,n\in\N$ konvexe SP und sei $X$ stetiger SP auf $\R$.
		Dann: $X_n\overset{\text{fd}}{\longrightarrow}X\gdw X_n\overset{\L}{\longrightarrow} X\text{ in }\big(C(\R),d\big)\Rightarrow X$ konvexer SP
		\textit{Beweis.}
		(1) $\Rightarrow$ (2): Portmanteau; (2) $\Rightarrow$ (1): CMT $\square$
		\item \define{Subspace}: $Z_n\overset{\L}{\longrightarrow} Z\text{ in }(\S,d)
		\Longleftrightarrow
		Z_n\overset{\L}{\longrightarrow} Z\text{ in }(U,d)$
		\item Koro: $X_n$ konvexer SP: $X_n\overset{\text{fd}}{\longrightarrow}X\implies
		X_n\overset{\L}{\longrightarrow} X\text{ in }\big(C_c(\R),d\big)$ (folgt aus Subspace-Lemma + vorherigem)
	\end{itemize}
	
	\section{Argmin-Theoreme in \texorpdfstring{$C_c(\R)$}{CcR}}
	\begin{itemize}
		\item 10.2: Sei $D\subseteq\R^d$ dicht in $\R^d$ (z.B. $D=\Q^d$) und seien $f,f_n,n\in\N$ konvexe Funktionen auf $\R^d$, wobei $f$ eine eindeutige Minimalstelle $\tau$ besitze ($A(f)=\lbrace\tau\rbrace$).
		Seien $\tau_n\in A(f_n)\neq\emptyset$ $\forall n\geq N_0\in\N$.
		Dann: $\Big(f_n(t)\overset{n\to\infty}{\longrightarrow} f(t)\qquad\forall t\in D\Big)\implies \tau_n\overset{n\to\infty}{\longrightarrow}\tau$
		\item 10.3: Seien $M,M_n,n\in\N$ konvexe SP auf $\R^d$ mit
		$A(M)=\lbrace\tau\rbrace$ f.s. für eine Zufallsvariable $\tau$;
		$M_n(t)\overset{n\to\infty}{\longrightarrow} M(t)~\P\text{-f.s.}~\forall t\in\R^d$; Seien $\tau_n$ ZV mit $\tau_n\in A(M_n)$ f.s. $\forall n\in\N$.
	Dann: $\tau_n\overset{n\to\infty}{\longrightarrow}\tau~\P\text{-f.s.}\text{ in }\R^d$ (folgt aus 10.2 über Einsmengen)
		\item 10.4: $Z,Z_n,n\in\N$ konvexe SP auf $\R$ mit $A(Z)=\lbrace\sigma\rbrace$ f.s. für eine ZV $\sigma$ und $Z_n\overset{\fd}{\longrightarrow}Z$.
		Dann gilt für jede Auswahl $\sigma_n\in A(Z_n)$: $\argmin(Z_n)=\sigma_n\overset{\L}{\longrightarrow}\sigma$
		\item Sei $U_n,n\in\N$ reelle ZV mit $\E[U_n]\overset{n\to\infty}{\longrightarrow}\mu$ und $\Var(U_n)\overset{n\to\infty}{\longrightarrow}0$. Dann: $U_n\overset{\P}{\longrightarrow}\mu$.
	\end{itemize}
	
	\section{Rekapitulation des Einführungskapitels}
	Der Median ist $m:=\argmin\limits_{t\in\R}Y(t)$, $Y\in C(\R)$, $Y(t):=\E[|X-t|]\overset{\text{Trafo}}{=}\int_\R|x-t|~(F\d x)$, Schätzer\\ $Y_n(t):=\int_\R|x-t|~(F_n\d x)\overset{\text{emp}}=\sum\limits_{i=1}^n|X_i-t|\underset{\text{SGGZ}}{\overset{n\to\infty}{\longrightarrow}}\E[|X_1-t|]~\forall t\in\R$\\
	$M:=Y$ und $M_n:=Y_n$ lassen sich als SP auffassen, sind sogar konvex. Aus Argmin-Theorem 10.3 folgt \underline{sofort} die Konvergenz des $\argmin$'s, ohne weitere Voraussetzungen an die starke Konsistenz des Medians (abgesehen Eindeutigkeit der Minimalstelle): $\hat{m}_n\overset{n\to\infty}{\longrightarrow} m$\\
	Um die Verteilungskonvergenz des Medians zu zeigen, betrachte den \define{reskalierten Prozess}\\ $Z_n(t):=a_n\cdot\sqrt{n}\cdot\Big(M_n\big(m+\frac{t}{a_n}\big)-M_n(m)\Big)$.
	Dann gilt: $a_n(\hat{m}_n-m)=\argmin\limits_{t\in\R}Z_n(t)$ und Konvergenz der fidis $Z_n\overset{\fd}{\longrightarrow}Z$
	
	% This work is licensed under the Creative Commons
% Attribution-NonCommercial-ShareAlike 4.0 International License. To view a copy
% of this license, visit http://creativecommons.org/licenses/by-nc-sa/4.0/ or
% send a letter to Creative Commons, PO Box 1866, Mountain View, CA 94042, USA.

Folgende Tabelle zeigt die Übersicht über die Voraussetzungen der verschiedenen\\ Argmin-Theoreme:\nl
\begin{tabular}{c|c|c|c}
	& $C(\R)$ & $C(I)$ mit $I$ kompakt & $C_c(\R)$ konvex \\
	\hline
		\makecell{$f,f_n$\\ Fkt.} & 
		\makecell{$\tau\in A(f)$ wohlsepariert\\ $A(f_n)\neq\emptyset\forall n\geq N_0\in\N$\\ $f_n\to f$ glm.} & 
		\makecell{$\tau\in A(f)$ eindeutig\\ $f_n\to f$ glm.} & 
		\makecell{$\tau\in A(f)$ eindeutig\\ $A(f_n)\neq\emptyset~\forall n\geq N_0\in\N$\\ $f_n\to f$ pktw. $D\overset{\text{dicht}}{\subseteq}\R^d$}\\
	\hline
		\makecell{$M,M_n$\\ SP} & 
		\makecell{$\tau\in A(M)$ wohlsep. f.s.\\ $A(f_n)\neq\emptyset$ f.s. \\ $M_n\to M$ glm. f.s.} & 
		\makecell{$\tau\in A(f)$ eindeutig f.s.\\ $M_n\to M$ glm. f.s.} & 
		\makecell{$\tau\in A(M)$ eindeutig f.s.\\ $A(f_n)\neq\emptyset$ f.s.\\ $M_n(t)\to M(t)$ pktw. f.s.}\\
	\hline
		\makecell{$Z_n,Z$\\ SP} & 
		\makecell{$\sigma\in A(Z)$ eindeutig f.s.\\ $Z_n\overset{\L}{\to}Z$\\ $(\sigma_n)_{n\in\N}$ stoch. besch.} & 
		------ & 
		\makecell{$\sigma\in A(Z)$ eindeutig f.s.\\ $Z_n\overset{\fd}{\to}Z$ \\ $A(Z_n)\neq\emptyset$}\\
	\end{tabular}\nl
	Hierbei folgt für die letzte Zeile stets Konvergenz in Verteilung des Argmins, bei den anderen Zellen nur fast sichere Konvergenz.
	Als Faustregel kann man sich auch merken, dass bei stochastischen Prozessen alle Voraussetzungen nur $\P$-fast sicher gelten müssen.
	
\end{document}