\section{Lineare Optimierung}
ganzzahlige Lineare Programme:
\begin{equation*}
	\min_{x} c^Tx \text{ subject to } \quad Ax = b, x \geq 0, x \in \Z^n
\end{equation*}
Zusätzliche obere Schranken für die Variablen sind auch möglich.
Beispielsweise wäre durch $0 \leq x \leq 1$, $x \in \Z^n$ ein binäres Problem gegeben.
\subsection{Optimalitätsbedingungen für kontinuierliche LPs}
\begin{itemize}
	\item LP in Normalform
		\begin{equation}\label{nf}
			\min_{x} c^Tx \text{ subject to } Ax = b, x\geq 0\tag{P}
		\end{equation}
	\item Die zulässige Menge ist ein Polyeder:
		\begin{equation*}
			P = \set{ x | Ax = b ,x \geq 0 }
		\end{equation*}
  \item Problem konvex (d.\,h\ Zielfunktion und zulässige Menge sind konvex) $\to$	lokale Minima = globale Minima
  \item Das dazu \define{duale Problem} ist gegeben durch
		\begin{equation*}\label{dual_cont}
			\max_{y}b^Ty \text{ subject to } A^Ty \leq c, y \in \R^n \tag{D}
		\end{equation*}
\end{itemize}
\begin{lemma}[Schwache Dualität] \label{thm:schwache_dualitat}
		Es gilt für alle $x$ zulässig für \eqref{nf} und alle $y$ zulässig für \eqref{dual_cont}:
		\begin{equation*}
			c^Tx \geq b^Ty
		\end{equation*}
		Bei Gleichheit gilt außerdem: $x$ löst \eqref{nf}, $y$ löst \eqref{dual_cont}.
\end{lemma} % end lemma Schwache Dualität
\begin{lemma}[Starke Dualität] \label{thm:starke_dualitat}
  Es sind äquivalent
  \begin{enumerate}
    \item \eqref{nf} hat eine Lösung $x^*$.
    \item \eqref{dual_cont} hat eine Lösung $y^*$.
    \item $(x^*,y^*)$ lösen das \define{KKT-System}
      \begin{align*}
        Ax^* &= b , \ x^* \geq 0 & \text{(primale Zulässigkeit)}\\
        A^{T} y^* &\leq c & \text{(duale Zulässigkeit)}\\
        \forall i = 1,\dots , n: x_{i}&= 0 \text{ oder } \left(A^{T}y^*  \right)_{i} = c_{i} & \text{ (Komplementaritätsbedingungen)}
      \end{align*}
      Dann gilt auch, dass die \define{Dualitätslücke} gleich $0$ ist:
      \begin{equation*}
        b^{T}\underbrace{y^* = (\underbrace{x^*}_{\geq 0})^{T}}_{\text{primal Zul.}} \underbrace{\underbrace{A^{T} y^*}_{\leq c }}_{\text{dual Z.}} \stackrel{\text{Kompbed.}}= \left(x^*\right)^{T} c
      \end{equation*}
    \end{enumerate}
\end{lemma} % end lemma Starke Dualität
\paragraph{Lösungsansätze für lineare Programme}
\begin{itemize}
  \item Innere-Punkte-Verfahren. (polynomielle Laufzeit)
  \item Ellipsoid-Verfahren. (polynomielle Laufzeit)
  \item Symplex-Verfahren (theoretisch exponentielle Laufzeit, aber in der Praxis gut)
\end{itemize}
\paragraph{Ecken und Lösungen}
\begin{itemize}
  \item $\hat{x}$ ist Ecke von $P$ $\iff$ $ \hat{x}$ kann nicht als echte Konvexkombination von Punkten aus $P$ geschrieben werden
		$\iff$ Sei
		\begin{equation*}
			I = \set*{i = 1 ,\dots, n | \hat{x}_{i} \geq 0 } = \supp(\hat{x})
		\end{equation*}
		Dann gilt $\rang(A_{\cdot,I})=|I|$.
	\item Hat das LP \eqref{nf} eine Lösung, so gibt es auch eine Ecklösung.
	\item Es gibt nur endlich viele Ecken.
	\item Idee des Simplex-Verfahrens: Ecken von $P$ in sinnvoller Reihenfolge absuchen.\\
		\underline{Annahme:} $A \in \R^{m \times n }$ hat \betone{vollen Zeilenrang} $m$. \\
		Sei $\hat{x}$ eine Ecke von $P$ und $I = \set*{i \mid \hat{x}_{i}>0}$ \\
		$\implies$ rang($A_{\cdot,I}   )= |I| \leq m$ \\
    $\implies$ Es gibt $B \subseteq \set*{1,\dots,n}$ mit $I \subseteq B, |B| = m $ und $A_{B}\coloneqq A_{\cdot,B}$ ist regulär. (Finde $B$ durch Basisergänzung, da $A$ $m$ linear unabhängige Spalten hat.)
	\item Basen\\
		Sei $B \subset \set*{1,\dots,n}$ mit $|B|=m$ und $N = \{1,\dots,n\}\setminus B$
    \begin{itemize}[label=$\to$] % warum eigentlich ein Pfeil?
      \item $B$ heißt \define{Basis}, falls $A_{B}$ regulär.
      \item $B$ heißt \define{primal zulässige Basis}, falls
        \begin{equation*}
          x_{B}=A_{B}^{-1}b \geq 0
        \end{equation*} $\implies$ Mit $x_{N} = 0$ ist $(x_{B},x_{N})$ zulässig für \eqref{nf} und eine Ecke von $P$.
      \item $B$ heißt \define{dual zulässige Basis}, falls
        \begin{equation*}
          A_{N}^TA_{B}^{-T}c_{b} \leq c_{N} \implies y \text{ ist zulässig für \eqref{dual_cont}}
        \end{equation*}
      \item $\left(x_{B},x_{N} \right)$ mit $x_{B} = A^{-1}_{B},\ x_{N}= 0$ heißt \define{Basislösung}. Ist $x_{B} > 0 $ so heißt die Basislösung \define{nichtdegeneriert}.
        (Im degenerierten Fall kann $B$ unterbestimmt sein, man bleibt u.\,U.\ in der gleichen Ecke.)
    \end{itemize}
\end{itemize}
%
\subsection{Wiederholung Simplex-Verfahren}
Annahme: $A \in R^{m \times n}$ hat Rang $m$. (Ansonsten können so lange linear von den anderen Zeilen abhängige Zeilen gelöscht werden, bis $A$ vollen Zeilenrang $m$ hat.)

Sei $B$ eine primal zulässige Basis mit Basislösung
\begin{align*}
	\hat{x}_{N}&=0,\\
	\hat{x}_{B}&= A_{B}^{-1}b
\end{align*}
Für alle $x \in P$ gilt:
\begin{align*}
	Ax&=
	\begin{pmatrix}
		A_{B} & A_{N}
	\end{pmatrix}
	\begin{pmatrix}
		x_{B} \\ x_{N}
	\end{pmatrix}
	\\
	  & A_{B}x_{B} + A_{N}x_{N}= b
\end{align*}
$\implies x_{B} = A_{B}^{-1}b - A_{B}^{-1}A_{N}x_{N} \to$ Zielfunktion vereinfachen
\begin{align*}
	c^Tx&= c_{B}^Tx_{B} + c_{N}^Tx_{N} = c_{B}^T \underbrace{A_{B}^{-1}b}_{= \hat{x}_{B}} - c_{B}^TA_{B}^{-1}A_{N}x_{N} + c_{N}^Tx_{N}\\
		&= c^{T}_{B} \hat{x}_{B} +( \underbrace{ c_{N} - A_{N}^{T} \underbrace{A_{B}^{-T}c_{B}}_{=y} }_{=: z_{N}, \text{ reduzierte Kosten}} )^{T} x_{N}
\end{align*}

\begin{itemize}
	\item Gilt $z_{N} \geq 0$: $\hat{x}$ löst LP (mit dualer Variable $y$)
	\item Andernfalls: Wähle $j\in \N$ mit $z_{j}<0$.

		Frage: Wie groß kann $x_{j}$ gewählt werden, ohne $P$ zu verlassen?

		Wir wählen das neue $x_{N}$ als $\gamma e_{j}$ mit zu bestimmenden $\gamma$.
		\begin{itemize}
      \item[$\implies$] neues $x_{B}
        = \hat{x}_{B} - A_{B}^{-1}A_{N}\gamma e_{j}
        = \hat{x}_{B} - \gamma \underbrace{A_{B}^{-1}A_{\cdot,j}}_{\eqqcolon w}
        \overset{!}{\geq} 0$
			\item Wähle $\gamma = \min \set{ \frac{\hat{x}_{i}}{w_{i}} \mid w_{i} > 0 }$.
      \item Wenn $\gamma = + \infty$, so ist \eqref{nf} unbeschränkt.
			\item Andernfalls wähle $i \in B$ mit $\gamma = \frac{\hat{x}_{i}}{w_{i}}$
		\end{itemize}
  \item Dann ist die neue Ecke $x_B^+$
		\begin{align*}
			\hat{x}_{B}-\gamma_{w}&= x_{B}^+\\
			x_{j}^+& = \gamma\\
			x_{k}^+ &= 0 \quad \text{ für alle anderen } k\\
			B^+ &= (B \setminus \set*{i})\cup \set*{j}\\
			N^+ &= (N \setminus \set*{j})\cup \set*{i}
			% B& = \set*{1,3,5} \\
			% B& = (3,1,5)
		\end{align*}
    Pro Iteration müssen zwei Gleichungssysteme gelöst werden. Das ist sehr teuer. Da sich immer nur $1$ Spalte ändert, kann dies allerdings stark vereinfacht werden durch ein Update der LR-Zerlegung. Das geht, indem in der Implementierung die Elemente von $B$, $B^+$ eine Reihenfolge haben und man ersetzt immer $i$ an seiner Stelle durch $j$.
\end{itemize}
\paragraph{Startbasis}
Gesucht als Startbasis ist eine primal zulässige Basis für
\begin{equation*}
	\min_{x} c^Tx \text{ subject to } Ax = b , x \geq 0
\end{equation*}
Dafür gibt es die Simplex-Phase 1: Löse das Hilfsproblem
\begin{align*}
	\min_{x,s} \sum_{i=1}^{m} s_{i} \text{ subject to }
	\begin{pmatrix}
		A& I
	\end{pmatrix}
	\begin{pmatrix}
		x \\ s
	\end{pmatrix}
	, (x,s)& \geq 0
\end{align*}
Gilt in einer Lösung ($x^0$), dass $s^* =0$, so ist das $x^*$ zulässig für \eqref{nf} und die Basis des Hilfsproblems \enquote{ist} eine primal zulässige Basis für \eqref{nf}.

Eine primal zulässige Startbasis für das Hilfsproblem ist (trivialerweise) gegeben durch
\begin{equation*}
	x = 0, s = b \geq 0, \text{ Basismatrix } I
\end{equation*}
Tableau-Methode für %TODO: was sollen uns diese Zeilen sagen?
\begin{equation*}
	\min_{x}c^Tx \text{ subject to } Ax \leq b , x \geq 0
\end{equation*}
\section{Duales Simplex-Verfahren}
\begin{align}
  \text{primales LP} && \min_{x}c^{T} x \text{ subject to }Ax = b, x \geq 0 \tag{P} \label{primalLP} \\
  \text{duales LP} && \max_{y}b^{T} y \text{ subject to }A^{T} y \leq c, y \geq 0 \tag{D}\label{dualLP}
\end{align}
Wir nehmen wieder an, dass $A \in \R^{m \times n}$ Rang $m$ hat.

Für eine Basis $B$ ist wie bisher $x_{B}= A^{-1}_{B}b, x_{N}=0$.

\underline{Idee:} Wende Simplex-Verfahren auf \eqref{dualLP} an.
Dafür führe \eqref{dualLP} in die duale Normalform:
\begin{equation}\label{dualnf}
	\max_{y} b^{T} y \text{ subject to }
	\begin{pmatrix}
		A^{T} & I
	\end{pmatrix}
	\begin{pmatrix}
	y\\z
	\end{pmatrix}
	= c, \quad z \geq 0, \quad y \in \R^m \tag{D'}
\end{equation}
Eine Basis von \eqref{dualnf}: $n \{ \klammern[\Big]{ \underbrace{A^T}_m \underbrace I_n} \eqqcolon D
% \begin{pmatrix}
% 	\underbrace{A^{T}}_{m} \underbrace{I}_{n}
% \end{pmatrix}
\in R^{n \times(m +n)}$
sei $H \subset \set{1,\dots ,n+m }$ mit $|H|=n$ und $D_{\cdot,H}\in \R^{n \times n}$ regulär.
\begin{theorem}
	Entweder \eqref{dualLP} ist unzulässig oder unbeschränkt oder es gibt eine optimale Basislösung $(y^*,z^*)$ so dass die zugehörige basis von der Form ist:
	\begin{equation*}
		H_{\text{opt}}=\underbrace{ \set{1,..,m }}_{\text{ganzes $A$ in Basismatrix}}\cup (N+m)
	\end{equation*}
	mit $N \subset \set{1,\dots ,n }, |N|= n-m$ ($N$ zeigt an, welche Spalten von $I$ gebraucht werden)
	$\implies$ Basen für \eqref{dualLP} werden i.A. durch $N$ beschrieben, nicht durch $H$.
\end{theorem}
Welche Basis $H$ bzw. $N$ liefert zulässige Basislösung für \eqref{dualLP}?
% irgendwas kompiliert hier nicht
\begin{align*}
	\begin{pmatrix}
		A^{T}  & I
	\end{pmatrix}
	\begin{pmatrix}
	y\\z
	\end{pmatrix}
	&=c, \quad z \geq 0\\
	&{}\iff A_{B}^{T} y + z_{B} \geq c_{B}, \quad z_{B} \geq 0\\
	&{}\hspace{1.1cm} A_{N}^{T} y + z_{N} = c_{N}, \quad z_{N} \geq 0 \\
	&{}\iff y = A_{B}^{-T}(c_{B}-z_{B})\\
	&{}\hspace{1.1cm} z_{N}= c_{N} - A_{N}^{T}y = c_{N}-A_{N}^{T} A_{B}^{-T}(c_{B}-z_{B}) ,\quad z_{B}\geq 0 ,z_{N} \geq 0
\end{align*}
$(y,z_{B},z_{N})$ Basislösung $\to$ Nichtbasisvariablen $z_{B} =0$ \\
$\implies$
\begin{equation*}
	y = A_{B}^{-T}c_{B}, \quad z_{N}= c_{N}- A_{N}^{T} \underbrace{A_{B}^{-T}c_{B}}_{{=y}}\geq 0
\end{equation*}
\subsection{Duales Simplex-Verfahren}
Initialisierung: Dual zulässige Basis $B$, d.h. $B \subset \set{1,\dots ,n }$ mit $|B|=m$, $A_{B}$ regulär und $z_{B}=0,y=A_{B}^{-T}c_{B}, z_{N}= c_{N}-A_{N}^{T}y \geq 0$
\begin{enumerate}%[label = Schritt \arabic)]
	\item Bestimme zugehörige komplementäre duale Variable
		\begin{align*}
			x_{N}&=0 \quad (\to \text{komplementär }z_{N})\\
			x_{B}&= A_{B}^{-1} b\quad ( \to A(x_{B},x_{N})^{T} =b)
		\end{align*}
	\item Gilt $x_{B} \geq 0 $: \textbf{Stop}.\\
		$(y,z_{B},z_{N}),(x_{B},x_{N})$ KKT-Punkt $\implies$ Lösung gefunden.\\
		Andernfalls wähle $ i \in B$ mit $x_{i}<0$ ($\to z_{i}>0$ verkleinert dualen Zielfunktionswert)
	\item Löse $A_{B}^{T} w = e_{i}$, $\alpha_{N} = -A_{N}^{T} w$
	\item Schrittweitenbestimmung:\\
    Gilt $\alpha_{N} \leq 0 $ :\textbf{Stop}, \eqref{dualLP} ist unbeschränkt ($\to$ \eqref{nf} unzulässig).\\
		Andernfalls berechne $\gamma = \min \set{ \frac{z_{i}}{\alpha_{i}}|j \in N , \alpha_{j} >0 }$ und wähle $j \in N$ mit $\gamma = \frac{z_{j}}{\alpha_{j}}$
	\item Update:
		\begin{align*}
			z_{N}^+ &= z_{N} - \gamma \alpha_{N} \quad (\to z_{N} \geq 0, z_{j}^+ =0)\\
			z_{i}^+ &= \gamma\\
			y^+ &= y -\gamma w\\
			N^+ &= (N \backslash \set{ j }) \cup \set{ i }\\
			B^+ &= (B \backslash \set{ i }) \cup \set{ j }
		\end{align*}
\end{enumerate}

\subsection{(primales) Simplex-Verfahren für LPs mit Schranken}
Betrachte
\begin{equation*}
	\min_{x} c^{T} x \text{ subject to } Ax = b ,\quad l \leq x \leq u
\end{equation*}
mit $l \in \left(\R \cup \set{ - \infty } \right)^n,u \in \left(\R \cup \set{ +\infty } \right)^n $

\underline{Ziel:} möglichst nahe an der Form
\begin{equation*}
	\min_{x} c^{T} x \text{ subject to } Ax = b ,\quad x \geq 0
\end{equation*}
kommen.\\
\underline{mögliche Modifikationen:}
\begin{itemize}
	\item Schlupfvariablen + Variablensplit $x = x^+ -x^-, x^+,x^- \geq0 $ (Nachteil: erhöht die Anzahl der Variablen und linearen Gleichungen)
	\item Fall: $x_{i} \geq l_{i}$ mit $l_{i}\in \R$ : Substituiere $\hat{x}_{i} := x_{i}- l_{i}\geq 0$
	\item Fall: $x_{i} \leq u_{i}$ mit $u_{i}\in \R$ : Substituiere  $\hat{x}_{i} := u_{i}- x_{i}\geq 0$
	\item Fall: $l_{i} \leq x_{i} \leq u_{i}$ mit $l_{i},u_{i} \in \R$ : Substituiere $0 \leq \hat{x}_{i} := x_{i}- l_{i} \leq u_{i}-l_{i}$
	\item Fall $-\infty \leq x_{i} \leq \infty$ : Freie Variable: entweder Variablensplit $x_{i}= x_{i}^+ - x_{i}^-$ oder: $x_{i}$ als freie Variable im Simplex-Verfahren berücksichtigen( wie $y$ im dualen Simplex)
\end{itemize}
Angenommen, es gibt keine freien Variablen $\implies$ oBdA LP der Form
\begin{equation*}
	\min_{x} c^{T} x \text{ subject to } Ax = b, \quad 0 \leq x \leq u
\end{equation*}
mit $u \in [0, + \infty]^n$.\nl
Bisher im primalen Simplex-Verfahren:\\
Basis $B$, Basisvariablen $x_{B}= A_{B}^{-1} b \geq 0$, Nichtbasisvariablen $x_{N}=0$ \\
\underline{Idee:} Teile die Nichtbasisvariablen in
\begin{align*}
	N_{l} &= \set{ i \in N \ | \ x_{i} = 0  } \implies x_{N_{l}}=0\\
	N_{u} &= \set{ i \in N \ | \ x_{i} = u_{i}  } \implies x_{N_{u}}=u_{N_{u}}
\end{align*}
Wie im Simplex-Verfahren:
\begin{align*}
	A_{B}x_{B} + A_{N_{l}}x_{N_{l}} + A_{N_{u}} x_{N_{u}} = b \implies x_{B} = A_{B}^{-1} (b - A_{N_{l}} x_{N_{l}} - A_{N_{l}}x_{N_{l}})\\
	c^{T} x = c_{B}^{T} x_{B} + c_{N_{l}}^{T} x_{N_{l}}+c_{N_{u}}^{T} x_{N_{u}} = c_{B}^{T} A_{B}^{-1} b + z_{N_{l}}^{T} x_{N_{l}} + z_{N_{u}}^{T} x_{N_{u}}
\end{align*}
Identifiziere $i \in N_{l}$ mit $z_{i} <0 $ oder $ i\in N_{u}$ mit $z_{i}> 0$.\\
Bei Schrittweiten beachten: $x_{B} = \gamma w \in [0,u_{B}]$ und
\begin{equation*}
	x_{i}^+ = \begin{cases}
	 \gamma \in [0,u_{i}]\\
	 u_{i}-\gamma
\end{cases}
\end{equation*}
Basisupdate: neuer Fall $B^+ = B$, $i$ wechselt von $N_{l}$ nach $N_{u}$ oder umgekehrt.

